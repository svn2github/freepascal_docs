% \begin{meta-comment}
%
% $Id: syntax.dtx,v 1.1 2000/07/13 09:10:21 michael Exp $
%
% Syntax typesetting package for LaTeX 2e
%
% (c) 1996 Mark Wooding
%
%----- Revision history -----------------------------------------------------
%
% $Log: syntax.dtx,v $
% Revision 1.1  2000/07/13 09:10:21  michael
% + Initial import
%
% Revision 1.1  1998/09/21 10:19:01  michael
% Initial implementation
%
% Revision 1.9  1996/11/28 00:19:10  mdw
% Added abbreviations for syntax diagram constructions.  These have been
% getting on my nerves for too long now...
%
% Revision 1.8  1996/11/19 21:02:15  mdw
% Entered into RCS
%
%
% \end{meta-comment}
%
% \begin{meta-comment} <general public licence>
%%
%% syntax package -- typesetting syntax descriptions
%% Copyright (c) 1996 Mark Wooding
%%
%% This program is free software; you can redistribute it and/or modify
%% it under the terms of the GNU General Public License as published by
%% the Free Software Foundation; either version 2 of the License, or
%% (at your option) any later version.
%%
%% This program is distributed in the hope that it will be useful,
%% but WITHOUT ANY WARRANTY; without even the implied warranty of
%% MERCHANTABILITY or FITNESS FOR A PARTICULAR PURPOSE.  See the
%% GNU General Public License for more details.
%%
%% You should have received a copy of the GNU General Public License
%% along with this program; if not, write to the Free Software
%% Foundation, Inc., 675 Mass Ave, Cambridge, MA 02139, USA.
%%
% \end{meta-comment}
%
% \begin{meta-comment} <Package preamble>
%<+package>\NeedsTeXFormat{LaTeX2e}
%<+package>\ProvidesPackage{syntax}
%<+package>                [1996/05/17 1.9 Syntax typesetting (MDW)]
% \end{meta-comment}
%
% \CheckSum{1465}
%% \CharacterTable
%%  {Upper-case    \A\B\C\D\E\F\G\H\I\J\K\L\M\N\O\P\Q\R\S\T\U\V\W\X\Y\Z
%%   Lower-case    \a\b\c\d\e\f\g\h\i\j\k\l\m\n\o\p\q\r\s\t\u\v\w\x\y\z
%%   Digits        \0\1\2\3\4\5\6\7\8\9
%%   Exclamation   \!     Double quote  \"     Hash (number) \#
%%   Dollar        \$     Percent       \%     Ampersand     \&
%%   Acute accent  \'     Left paren    \(     Right paren   \)
%%   Asterisk      \*     Plus          \+     Comma         \,
%%   Minus         \-     Point         \.     Solidus       \/
%%   Colon         \:     Semicolon     \;     Less than     \<
%%   Equals        \=     Greater than  \>     Question mark \?
%%   Commercial at \@     Left bracket  \[     Backslash     \\
%%   Right bracket \]     Circumflex    \^     Underscore    \_
%%   Grave accent  \`     Left brace    \{     Vertical bar  \|
%%   Right brace   \}     Tilde         \~}
%%
%
% \begin{meta-comment} <driver>
%
%<*driver>
%
% This hacking will remember the old default underscore character.  Even if
% T1 fonts are being used, it will get the grotty version.  Why is it that
% all of the encoding handling ends up looking like this?
%
\expandafter\let\expandafter\oldus\csname?\string\textunderscore\endcsname
%
% \begin{meta-comment}
%
% $Id: mdwtools.tex,v 1.1 2000/07/13 09:10:21 michael Exp $
%
% Common declarations for mdwtools.dtx files
%
% (c) 1996 Mark Wooding
%
%----- Revision history -----------------------------------------------------
%
% $Log: mdwtools.tex,v $
% Revision 1.1  2000/07/13 09:10:21  michael
% + Initial import
%
% Revision 1.1  1998/09/21 10:19:01  michael
% Initial implementation
%
% Revision 1.4  1996/11/19 20:55:55  mdw
% Entered into RCS
%
%
% \end{meta-comment}
%
% \begin{meta-comment} <general public licence>
%%
%% mdwtools common declarations
%% Copyright (c) 1996 Mark Wooding
%%
%% This program is free software; you can redistribute it and/or modify
%% it under the terms of the GNU General Public License as published by
%% the Free Software Foundation; either version 2 of the License, or
%% (at your option) any later version.
%%
%% This program is distributed in the hope that it will be useful,
%% but WITHOUT ANY WARRANTY; without even the implied warranty of
%% MERCHANTABILITY or FITNESS FOR A PARTICULAR PURPOSE.  See the
%% GNU General Public License for more details.
%%
%% You should have received a copy of the GNU General Public License
%% along with this program; if not, write to the Free Software
%% Foundation, Inc., 675 Mass Ave, Cambridge, MA 02139, USA.
%%
% \end{meta-comment}
%
% \begin{meta-comment} <file preamble>
%<*mdwtools>
\ProvidesFile{mdwtools.tex}
             [1996/05/10 1.4 Shared definitions for mdwtools .dtx files]
%</mdwtools>
% \end{meta-comment}
%
% \CheckSum{668}
%% \CharacterTable
%%  {Upper-case    \A\B\C\D\E\F\G\H\I\J\K\L\M\N\O\P\Q\R\S\T\U\V\W\X\Y\Z
%%   Lower-case    \a\b\c\d\e\f\g\h\i\j\k\l\m\n\o\p\q\r\s\t\u\v\w\x\y\z
%%   Digits        \0\1\2\3\4\5\6\7\8\9
%%   Exclamation   \!     Double quote  \"     Hash (number) \#
%%   Dollar        \$     Percent       \%     Ampersand     \&
%%   Acute accent  \'     Left paren    \(     Right paren   \)
%%   Asterisk      \*     Plus          \+     Comma         \,
%%   Minus         \-     Point         \.     Solidus       \/
%%   Colon         \:     Semicolon     \;     Less than     \<
%%   Equals        \=     Greater than  \>     Question mark \?
%%   Commercial at \@     Left bracket  \[     Backslash     \\
%%   Right bracket \]     Circumflex    \^     Underscore    \_
%%   Grave accent  \`     Left brace    \{     Vertical bar  \|
%%   Right brace   \}     Tilde         \~}
%%
%
% \section{Introduction and user guide}
%
% This file is really rather strange; it gets |\input| by other package
% documentation files to set up most of the environmental gubbins for them.
% It handles almost everything, like loading a document class, finding any
% packages, and building and formatting the title.
%
% It also offers an opportunity for users to customise my nice documentation,
% by using a |mdwtools.cfg| file (not included).
%
%
% \subsection{Declarations}
%
% A typical documentation file contains something like
% \begin{listinglist} \listingsize \obeylines
% |% \begin{meta-comment}
%
% $Id: mdwtools.tex,v 1.1 2000/07/13 09:10:21 michael Exp $
%
% Common declarations for mdwtools.dtx files
%
% (c) 1996 Mark Wooding
%
%----- Revision history -----------------------------------------------------
%
% $Log: mdwtools.tex,v $
% Revision 1.1  2000/07/13 09:10:21  michael
% + Initial import
%
% Revision 1.1  1998/09/21 10:19:01  michael
% Initial implementation
%
% Revision 1.4  1996/11/19 20:55:55  mdw
% Entered into RCS
%
%
% \end{meta-comment}
%
% \begin{meta-comment} <general public licence>
%%
%% mdwtools common declarations
%% Copyright (c) 1996 Mark Wooding
%%
%% This program is free software; you can redistribute it and/or modify
%% it under the terms of the GNU General Public License as published by
%% the Free Software Foundation; either version 2 of the License, or
%% (at your option) any later version.
%%
%% This program is distributed in the hope that it will be useful,
%% but WITHOUT ANY WARRANTY; without even the implied warranty of
%% MERCHANTABILITY or FITNESS FOR A PARTICULAR PURPOSE.  See the
%% GNU General Public License for more details.
%%
%% You should have received a copy of the GNU General Public License
%% along with this program; if not, write to the Free Software
%% Foundation, Inc., 675 Mass Ave, Cambridge, MA 02139, USA.
%%
% \end{meta-comment}
%
% \begin{meta-comment} <file preamble>
%<*mdwtools>
\ProvidesFile{mdwtools.tex}
             [1996/05/10 1.4 Shared definitions for mdwtools .dtx files]
%</mdwtools>
% \end{meta-comment}
%
% \CheckSum{668}
%% \CharacterTable
%%  {Upper-case    \A\B\C\D\E\F\G\H\I\J\K\L\M\N\O\P\Q\R\S\T\U\V\W\X\Y\Z
%%   Lower-case    \a\b\c\d\e\f\g\h\i\j\k\l\m\n\o\p\q\r\s\t\u\v\w\x\y\z
%%   Digits        \0\1\2\3\4\5\6\7\8\9
%%   Exclamation   \!     Double quote  \"     Hash (number) \#
%%   Dollar        \$     Percent       \%     Ampersand     \&
%%   Acute accent  \'     Left paren    \(     Right paren   \)
%%   Asterisk      \*     Plus          \+     Comma         \,
%%   Minus         \-     Point         \.     Solidus       \/
%%   Colon         \:     Semicolon     \;     Less than     \<
%%   Equals        \=     Greater than  \>     Question mark \?
%%   Commercial at \@     Left bracket  \[     Backslash     \\
%%   Right bracket \]     Circumflex    \^     Underscore    \_
%%   Grave accent  \`     Left brace    \{     Vertical bar  \|
%%   Right brace   \}     Tilde         \~}
%%
%
% \section{Introduction and user guide}
%
% This file is really rather strange; it gets |\input| by other package
% documentation files to set up most of the environmental gubbins for them.
% It handles almost everything, like loading a document class, finding any
% packages, and building and formatting the title.
%
% It also offers an opportunity for users to customise my nice documentation,
% by using a |mdwtools.cfg| file (not included).
%
%
% \subsection{Declarations}
%
% A typical documentation file contains something like
% \begin{listinglist} \listingsize \obeylines
% |% \begin{meta-comment}
%
% $Id: mdwtools.tex,v 1.1 2000/07/13 09:10:21 michael Exp $
%
% Common declarations for mdwtools.dtx files
%
% (c) 1996 Mark Wooding
%
%----- Revision history -----------------------------------------------------
%
% $Log: mdwtools.tex,v $
% Revision 1.1  2000/07/13 09:10:21  michael
% + Initial import
%
% Revision 1.1  1998/09/21 10:19:01  michael
% Initial implementation
%
% Revision 1.4  1996/11/19 20:55:55  mdw
% Entered into RCS
%
%
% \end{meta-comment}
%
% \begin{meta-comment} <general public licence>
%%
%% mdwtools common declarations
%% Copyright (c) 1996 Mark Wooding
%%
%% This program is free software; you can redistribute it and/or modify
%% it under the terms of the GNU General Public License as published by
%% the Free Software Foundation; either version 2 of the License, or
%% (at your option) any later version.
%%
%% This program is distributed in the hope that it will be useful,
%% but WITHOUT ANY WARRANTY; without even the implied warranty of
%% MERCHANTABILITY or FITNESS FOR A PARTICULAR PURPOSE.  See the
%% GNU General Public License for more details.
%%
%% You should have received a copy of the GNU General Public License
%% along with this program; if not, write to the Free Software
%% Foundation, Inc., 675 Mass Ave, Cambridge, MA 02139, USA.
%%
% \end{meta-comment}
%
% \begin{meta-comment} <file preamble>
%<*mdwtools>
\ProvidesFile{mdwtools.tex}
             [1996/05/10 1.4 Shared definitions for mdwtools .dtx files]
%</mdwtools>
% \end{meta-comment}
%
% \CheckSum{668}
%% \CharacterTable
%%  {Upper-case    \A\B\C\D\E\F\G\H\I\J\K\L\M\N\O\P\Q\R\S\T\U\V\W\X\Y\Z
%%   Lower-case    \a\b\c\d\e\f\g\h\i\j\k\l\m\n\o\p\q\r\s\t\u\v\w\x\y\z
%%   Digits        \0\1\2\3\4\5\6\7\8\9
%%   Exclamation   \!     Double quote  \"     Hash (number) \#
%%   Dollar        \$     Percent       \%     Ampersand     \&
%%   Acute accent  \'     Left paren    \(     Right paren   \)
%%   Asterisk      \*     Plus          \+     Comma         \,
%%   Minus         \-     Point         \.     Solidus       \/
%%   Colon         \:     Semicolon     \;     Less than     \<
%%   Equals        \=     Greater than  \>     Question mark \?
%%   Commercial at \@     Left bracket  \[     Backslash     \\
%%   Right bracket \]     Circumflex    \^     Underscore    \_
%%   Grave accent  \`     Left brace    \{     Vertical bar  \|
%%   Right brace   \}     Tilde         \~}
%%
%
% \section{Introduction and user guide}
%
% This file is really rather strange; it gets |\input| by other package
% documentation files to set up most of the environmental gubbins for them.
% It handles almost everything, like loading a document class, finding any
% packages, and building and formatting the title.
%
% It also offers an opportunity for users to customise my nice documentation,
% by using a |mdwtools.cfg| file (not included).
%
%
% \subsection{Declarations}
%
% A typical documentation file contains something like
% \begin{listinglist} \listingsize \obeylines
% |\input{mdwtools}|
% \<declarations>
% |\mdwdoc|
% \end{listinglist}
% The initial |\input| reads in this file and sets up the various commands
% which may be needed.  The final |\mdwdoc| actually starts the document,
% inserting a title (which is automatically generated), a table of
% contents etc., and reads the documentation file in (using the |\DocInput|
% command from the \package{doc} package.
%
% \subsubsection{Describing packages}
%
% \DescribeMacro{\describespackage}
% \DescribeMacro{\describesclass}
% \DescribeMacro{\describesfile}
% \DescribeMacro{\describesfile*}
% The most important declarations are those which declare what the
% documentation describes.  Saying \syntax{"\\describespackage{<package>}"}
% loads the \<package> (if necessary) and adds it to the auto-generated
% title, along with a footnote containing version information.  Similarly,
% |\describesclass| adds a document class name to the title (without loading
% it -- the document itself must do this, with the |\documentclass| command).
% For files which aren't packages or classes, use the |\describesfile| or
% |\describesfile*| command (the $*$-version won't |\input| the file, which
% is handy for files like |mdwtools.tex|, which are already input).
%
% \DescribeMacro{\author}
% \DescribeMacro{\date}
% \DescribeMacro{\title}
% The |\author|, |\date| and |\title| declarations work slightly differently
% to normal -- they ensure that only the \emph{first} declaration has an
% effect.  (Don't you play with |\author|, please, unless you're using this
% program to document your own packages.)  Using |\title| suppresses the
% automatic title generation.
%
% \DescribeMacro{\docdate}
% The default date is worked out from the version string of the package or
% document class whose name is the same as that of the documentation file.
% You can choose a different `main' file by saying
% \syntax{"\\docdate{"<file>"}"}.
%
% \subsubsection{Contents handling}
%
% \DescribeMacro{\addcontents}
% A documentation file always has a table of contents.  Other
% contents-like lists can be added by saying
% \syntax{"\\addcontents{"<extension>"}{"<command>"}"}.  The \<extension>
% is the file extension of the contents file (e.g., \lit{lot} for the
% list of tables); the \<command> is the command to actually typeset the
% contents file (e.g., |\listoftables|).
%
% \subsubsection{Other declarations}
%
% \DescribeMacro{\implementation}
% The \package{doc} package wants you to say
% \syntax{"\\StopEventually{"<stuff>"}"}' before describing the package
% implementation.  Using |mdwtools.tex|, you just say |\implementation|, and
% everything works.  It will automatically read in the licence text (from
% |gpl.tex|, and wraps some other things up.
%
% 
% \subsection{Other commands}
%
% The |mdwtools.tex| file includes the \package{syntax} and \package{sverb}
% packages so that they can be used in documentation files.  It also defines
% some trivial commands of its own.
%
% \DescribeMacro{\<}
% Saying \syntax{"\\<"<text>">" is the same as "\\synt{"<text>"}"}; this
% is a simple abbreviation.
%
% \DescribeMacro{\smallf}
% Saying \syntax{"\\smallf" <number>"/"<number>} typesets a little fraction,
% like this: \smallf 3/4.  It's useful when you want to say that the default
% value of a length is 2 \smallf 1/2\,pt, or something like that.
%
%
% \subsection{Customisation}
%
% You can customise the way that the package documentation looks by writing
% a file called |mdwtools.cfg|.  You can redefine various commands (before
% they're defined here, even; |mdwtools.tex| checks most of the commands that
% it defines to make sure they haven't been defined already.
%
% \DescribeMacro{\indexing}
% If you don't want the prompt about whether to generate index files, you
% can define the |\indexing| command to either \lit{y} or \lit{n}.  I'd
% recommend that you use |\providecommand| for this, to allow further
% customisation from the command line.
%
% \DescribeMacro{\mdwdateformat}
% If you don't like my date format (maybe you're American or something),
% you can redefine the |\mdwdateformat| command.  It takes three arguments:
% the year, month and date, as numbers; it should expand to something which
% typesets the date nicely.  The default format gives something like
% `10 May 1996'.  You can produce something rather more exotic, like
% `10\textsuperscript{th} May \textsc{\romannumeral 1996}' by saying
%\begin{listing}
%\newcommand{\mdwdateformat}[3]{%
%  \number#3\textsuperscript{\numsuffix{#3}}\ %
%  \monthname{#2}\ %
%  \textsc{\romannumeral #1}%
%}
%\end{listing}
% \DescribeMacro{\monthname}
% \DescribeMacro{\numsuffix}
% Saying \syntax{"\\monthname{"<number>"}"} expands to the name of the
% numbered month (which can be useful when doing date formats).  Saying
% \syntax{"\\numsuffix{"<number>"}"} will expand to the appropriate suffix
% (`th' or `rd' or whatever) for the \<number>.  You'll have to superscript
% it yourself, if this is what you want to do.  Putting the year number
% in roman numerals is just pretentious |;-)|.
%
% \DescribeMacro{\mdwhook}
% After all the declarations in |mdwtools.tex|, the command |\mdwhook| is
% executed, if it exists.  This can be set up by the configuration file
% to do whatever you want.
%
% There are lots of other things you can play with; you should look at the
% implementation section to see what's possible.
%
% \implementation
%
% \section{Implementation}
%
%    \begin{macrocode}
%<*mdwtools>
%    \end{macrocode}
%
% The first thing is that I'm not a \LaTeX\ package or anything official
% like that, so I must enable `|@|' as a letter by hand.
%
%    \begin{macrocode}
\makeatletter
%    \end{macrocode}
%
% Now input the user's configuration file, if it exists.  This is fairly
% simple stuff.
%
%    \begin{macrocode}
\@input{mdwtools.cfg}
%    \end{macrocode}
%
% Well, that's the easy bit done.
%
%
% \subsection{Initialisation}
%
% Obviously the first thing to do is to obtain a document class.  Obviously,
% it would be silly to do this if a document class has already been loaded,
% either by the package documentation or by the configuration file.
%
% The only way I can think of for finding out if a document class is already
% loaded is by seeing if the |\documentclass| command has been redefined
% to raise an error.  This isn't too hard, really.
%
%    \begin{macrocode}
\ifx\documentclass\@twoclasseserror\else
  \documentclass[a4paper]{ltxdoc}
  \ifx\doneclasses\mdw@undefined\else\doneclasses\fi
\fi
%    \end{macrocode}
%
% As part of my standard environment, I'll load some of my more useful
% packages.  If they're already loaded (possibly with different options),
% I'll not try to load them again.
%
%    \begin{macrocode}
\@ifpackageloaded{doc}{}{\usepackage{doc}}
\@ifpackageloaded{syntax}{}{\usepackage[rounded]{syntax}}
\@ifpackageloaded{sverb}{}{\usepackage{sverb}}
%    \end{macrocode}
%
%
% \subsection{Some macros for interaction}
%
% I like the \LaTeX\ star-boxes, although it's a pain having to cope with
% \TeX's space-handling rules.  I'll define a new typing-out macro which
% makes spaces more significant, and has a $*$-version which doesn't put
% a newline on the end, and interacts prettily with |\read|.
%
% First of all, I need to make spaces active, so I can define things about
% active spaces.
%
%    \begin{macrocode}
\begingroup\obeyspaces
%    \end{macrocode}
%
% Now to define the main macro.  This is easy stuff.  Spaces must be
% carefully rationed here, though.
%
% I'll start a group, make spaces active, and make spaces expand to ordinary
% space-like spaces.  Then I'll look for a star, and pass either |\message|
% (which doesn't start a newline, and interacts with |\read| well) or
% |\immediate\write 16| which does a normal write well.
%
%    \begin{macrocode}
\gdef\mdwtype{%
\begingroup\catcode`\ \active\let \space%
\@ifstar{\mdwtype@i{\message}}{\mdwtype@i{\immediate\write\sixt@@n}}%
}
\endgroup
%    \end{macrocode}
%
% Now for the easy bit.  I have the thing to do, and the thing to do it to,
% so do that and end the group.
%
%    \begin{macrocode}
\def\mdwtype@i#1#2{#1{#2}\endgroup}
%    \end{macrocode}
%
%
% \subsection{Decide on indexing}
%
% A configuration file can decide on indexing by defining the |\indexing|
% macro to either \lit{y} or \lit{n}.  If it's not set, then I'll prompt
% the user.
%
% First of all, I want a switch to say whether I'm indexing.
%
%    \begin{macrocode}
\newif\ifcreateindex
%    \end{macrocode}
%
% Right: now I need to decide how to make progress.  If the macro's not set,
% then I want to set it, and start a row of stars.
%
%    \begin{macrocode}
\ifx\indexing\@@undefined
  \mdwtype{*****************************}
  \def\indexing{?}
\fi
%    \end{macrocode}
%
% Now enter a loop, asking the user whether to do indexing, until I get
% a sensible answer.
%
%    \begin{macrocode}
\loop
  \@tempswafalse
  \if y\indexing\@tempswatrue\createindextrue\fi
  \if Y\indexing\@tempswatrue\createindextrue\fi
  \if n\indexing\@tempswatrue\createindexfalse\fi
  \if N\indexing\@tempswatrue\createindexfalse\fi
  \if@tempswa\else
  \mdwtype*{* Create index files? (y/n) *}
  \read\sixt@@n to\indexing%
\repeat
%    \end{macrocode}
%
% Now, based on the results of that, display a message about the indexing.
%
%    \begin{macrocode}
\mdwtype{*****************************}
\ifcreateindex
  \mdwtype{* Creating index files      *}
  \mdwtype{* This may take some time   *}
\else
  \mdwtype{* Not creating index files  *}
\fi
\mdwtype{*****************************}
%    \end{macrocode}
%
% Now I can play with the indexing commands of the \package{doc} package
% to do whatever it is that the user wants.
%
%    \begin{macrocode}
\ifcreateindex
  \CodelineIndex
  \EnableCrossrefs
\else
  \CodelineNumbered
  \DisableCrossrefs
\fi
%    \end{macrocode}
%
% And register lots of plain \TeX\ things which shouldn't be indexed.
% This contains lots of |\if|\dots\ things which don't fit nicely in
% conditionals, which is a shame.  Still, it doesn't matter that much,
% really.
%
%    \begin{macrocode}
\DoNotIndex{\def,\long,\edef,\xdef,\gdef,\let,\global}
\DoNotIndex{\if,\ifnum,\ifdim,\ifcat,\ifmmode,\ifvmode,\ifhmode,%
            \iftrue,\iffalse,\ifvoid,\ifx,\ifeof,\ifcase,\else,\or,\fi}
\DoNotIndex{\box,\copy,\setbox,\unvbox,\unhbox,\hbox,%
            \vbox,\vtop,\vcenter}
\DoNotIndex{\@empty,\immediate,\write}
\DoNotIndex{\egroup,\bgroup,\expandafter,\begingroup,\endgroup}
\DoNotIndex{\divide,\advance,\multiply,\count,\dimen}
\DoNotIndex{\relax,\space,\string}
\DoNotIndex{\csname,\endcsname,\@spaces,\openin,\openout,%
            \closein,\closeout}
\DoNotIndex{\catcode,\endinput}
\DoNotIndex{\jobname,\message,\read,\the,\m@ne,\noexpand}
\DoNotIndex{\hsize,\vsize,\hskip,\vskip,\kern,\hfil,\hfill,\hss}
\DoNotIndex{\m@ne,\z@,\z@skip,\@ne,\tw@,\p@}
\DoNotIndex{\dp,\wd,\ht,\vss,\unskip}
%    \end{macrocode}
%
% Last bit of indexing stuff, for now: I'll typeset the index in two columns
% (the default is three, which makes them too narrow for my tastes).
%
%    \begin{macrocode}
\setcounter{IndexColumns}{2}
%    \end{macrocode}
%
%
% \subsection{Selectively defining things}
%
% I don't want to tread on anyone's toes if they redefine any of these
% commands and things in a configuration file.  The following definitions
% are fairly evil, but should do the job OK.
%
% \begin{macro}{\@gobbledef}
%
% This macro eats the following |\def|inition, leaving not a trace behind.
%
%    \begin{macrocode}
\def\@gobbledef#1#{\@gobble}
%    \end{macrocode}
%
% \end{macro}
%
% \begin{macro}{\tdef}
% \begin{macro}{\tlet}
%
% The |\tdef| command is a sort of `tentative' definition -- it's like
% |\def| if the control sequence named doesn't already have a definition.
% |\tlet| does the same thing with |\let|.
%
%    \begin{macrocode}
\def\tdef#1{
  \ifx#1\@@undefined%
    \expandafter\def\expandafter#1%
  \else%
    \expandafter\@gobbledef%
  \fi%
}
\def\tlet#1#2{\ifx#1\@@undefined\let#1=#2\fi}
%    \end{macrocode}
%
% \end{macro}
% \end{macro}
%
%
% \subsection{General markup things}
%
% Now for some really simple things.  I'll define how to typeset package
% names and environment names (both in the sans serif font, for now).
%
%    \begin{macrocode}
\tlet\package\textsf
\tlet\env\textsf
%    \end{macrocode}
%
% I'll define the |\<|\dots|>| shortcut for syntax items suggested in the
% \package{syntax} package.
%
%    \begin{macrocode}
\tdef\<#1>{\synt{#1}}
%    \end{macrocode}
%
% And because it's used in a few places (mainly for typesetting lengths),
% here's a command for typesetting fractions in text.
%
%    \begin{macrocode}
\tdef\smallf#1/#2{\ensuremath{^{#1}\!/\!_{#2}}}
%    \end{macrocode}
%
%
% \subsection{A table environment}
%
% \begin{environment}{tab}
%
% Most of the packages don't use the (obviously perfect) \package{mdwtab}
% package, because it's big, and takes a while to load.  Here's an
% environment for typesetting centred tables.  The first (optional) argument
% is some declarations to perform.  The mandatory argument is the table
% preamble (obviously).
%
%    \begin{macrocode}
\@ifundefined{tab}{%
  \newenvironment{tab}[2][\relax]{%
    \par\vskip2ex%
    \centering%
    #1%
    \begin{tabular}{#2}%
  }{%
    \end{tabular}%
    \par\vskip2ex%
  }
}{}
%    \end{macrocode}
%
% \end{environment}
%
%
% \subsection{Commenting out of stuff}
%
% \begin{environment}{meta-comment}
%
% Using |\iffalse|\dots|\fi| isn't much fun.  I'll define a gobbling
% environment using the \package{sverb} stuff.
%
%    \begin{macrocode}
\ignoreenv{meta-comment}
%    \end{macrocode}
%
% \end{environment}
%
%
% \subsection{Float handling}
%
% This gubbins will try to avoid float pages as much as possible, and (with
% any luck) encourage floats to be put on the same pages as text.
%
%    \begin{macrocode}
\def\textfraction{0.1}
\def\topfraction{0.9}
\def\bottomfraction{0.9}
\def\floatpagefraction{0.7}
%    \end{macrocode}
%
% Now redefine the default float-placement parameters to allow `here' floats.
%
%    \begin{macrocode}
\def\fps@figure{htbp}
\def\fps@table{htbp}
%    \end{macrocode}
%
%
% \subsection{Other bits of parameter tweaking}
%
% Make \env{grammar} environments look pretty, by indenting the left hand
% sides by a large amount.
%
%    \begin{macrocode}
\grammarindent1in
%    \end{macrocode}
%
% I don't like being told by \TeX\ that my paragraphs are hard to linebreak:
% I know this already.  This lot should shut \TeX\ up about most problems.
%
%    \begin{macrocode}
\sloppy
\hbadness\@M
\hfuzz10\p@
%    \end{macrocode}
%
% Also make \TeX\ shut up in the index.  The \package{multicol} package
% irritatingly plays with |\hbadness|.  This is the best hook I could find
% for playing with this setting.
%
%    \begin{macrocode}
\expandafter\def\expandafter\IndexParms\expandafter{%
  \IndexParms%
  \hbadness\@M%
}
%    \end{macrocode}
%
% The other thing I really don't like is `Marginpar moved' warnings.  This
% will get rid of them, and lots of other \LaTeX\ warnings at the same time.
%
%    \begin{macrocode}
\let\@latex@warning@no@line\@gobble
%    \end{macrocode}
%
% Put some extra space between table rows, please.
%
%    \begin{macrocode}
\def\arraystretch{1.2}
%    \end{macrocode}
%
% Most of the code is at guard level one, so typeset that in upright text.
%
%    \begin{macrocode}
\setcounter{StandardModuleDepth}{1}
%    \end{macrocode}
%
%
% \subsection{Contents handling}
%
% I use at least one contents file (the main table of contents) although
% I may want more.  I'll keep a list of contents files which I need to
% handle.
%
% There are two things I need to do to contents files here:
% \begin{itemize}
% \item I must typeset the table of contents at the beginning of the
%       document; and
% \item I want to typeset tables of contents in two columns (using the
%       \package{multicol} package).
% \end{itemize}
%
% The list consists of items of the form
% \syntax{"\\do{"<extension>"}{"<command>"}"}, where \<extension> is the
% file extension of the contents file, and \<command> is the command to
% typeset it.
%
% \begin{macro}{\docontents}
%
% This is where I keep the list of contents files.  I'll initialise it to
% just do the standard contents table.
%
%    \begin{macrocode}
\def\docontents{\do{toc}{\tableofcontents}}
%    \end{macrocode}
%
% \end{macro}
%
% \begin{macro}{\addcontents}
%
% By saying \syntax{"\\addcontents{"<extension>"}{"<command>"}"}, a document
% can register a new table of contents which gets given the two-column
% treatment properly.  This is really easy to implement.
%
%    \begin{macrocode}
\def\addcontents#1#2{%
  \toks@\expandafter{\docontents\do{#1}{#2}}%
  \edef\docontents{\the\toks@}%
}
%    \end{macrocode}
%
% \end{macro}
%
%
% \subsection{Finishing it all off}
%
% \begin{macro}{\finalstuff}
%
% The |\finalstuff| macro is a hook for doing things at the end of the
% document.  Currently, it inputs the licence agreement as an appendix.
%
%    \begin{macrocode}
\tdef\finalstuff{\appendix\part*{Appendix}\input{gpl}}
%    \end{macrocode}
%
% \end{macro}
%
% \begin{macro}{\implementation}
%
% The |\implementation| macro starts typesetting the implementation of
% the package(s).  If we're not doing the implementation, it just does
% this lot and ends the input file.
%
% I define a macro with arguments inside the |\StopEventually|, which causes
% problems, since the code gets put through an extra level of |\def|fing
% depending on whether the implementation stuff gets typeset or not.  I'll
% store the code I want to do in a separate macro.
%
%    \begin{macrocode}
\def\implementation{\StopEventually{\attheend}}
%    \end{macrocode}
%
% Now for the actual activity.  First, I'll do the |\finalstuff|.  Then, if
% \package{doc}'s managed to find the \package{multicol} package, I'll add
% the end of the environment to the end of each contents file in the list.
% Finally, I'll read the index in from its formatted |.ind| file.
%
%    \begin{macrocode}
\tdef\attheend{%
  \finalstuff%
  \ifhave@multicol%
    \def\do##1##2{\addtocontents{##1}{\protect\end{multicols}}}%
    \docontents%
  \fi%
  \PrintIndex%
}
%    \end{macrocode}
%
% \end{macro}
%
%
% \subsection{File version information}
%
% \begin{macro}{\mdwpkginfo}
%
% For setting up the automatic titles, I'll need to be able to work out
% file versions and things.  This macro will, given a file name, extract
% from \LaTeX\ the version information and format it into a sensible string.
%
% First of all, I'll put the original string (direct from the
% |\Provides|\dots\ command).  Then I'll pass it to another macro which can
% parse up the string into its various bits, along with the original
% filename.
%
%    \begin{macrocode}
\def\mdwpkginfo#1{%
  \edef\@tempa{\csname ver@#1\endcsname}%
  \expandafter\mdwpkginfo@i\@tempa\@@#1\@@%
}
%    \end{macrocode}
%
% Now for the real business.  I'll store the string I build in macros called
% \syntax{"\\"<filename>"date", "\\"<filename>"version" and
% "\\"<filename>"info"}, which store the file's date, version and
% `information string' respectively.  (Note that the file extension isn't
% included in the name.)
%
% This is mainly just tedious playing with |\expandafter|.  The date format
% is defined by a separate macro, which can be modified from the
% configuration file.
%
%    \begin{macrocode}
\def\mdwpkginfo@i#1/#2/#3 #4 #5\@@#6.#7\@@{%
  \expandafter\def\csname #6date\endcsname%
    {\protect\mdwdateformat{#1}{#2}{#3}}%
  \expandafter\def\csname #6version\endcsname{#4}%
  \expandafter\def\csname #6info\endcsname{#5}%
}
%    \end{macrocode}
%
% \end{macro}
%
% \begin{macro}{\mdwdateformat}
%
% Given three arguments, a year, a month and a date (all numeric), build a
% pretty date string.  This is fairly simple really.
%
%    \begin{macrocode}
\tdef\mdwdateformat#1#2#3{\number#3\ \monthname{#2}\ \number#1}
\def\monthname#1{%
  \ifcase#1\or%
     January\or February\or March\or April\or May\or June\or%
     July\or August\or September\or October\or November\or December%
  \fi%
}
\def\numsuffix#1{%
  \ifnum#1=1 st\else%
  \ifnum#1=2 nd\else%
  \ifnum#1=3 rd\else%
  \ifnum#1=21 st\else%
  \ifnum#1=22 nd\else%
  \ifnum#1=23 rd\else%
  \ifnum#1=31 st\else%
  th%
  \fi\fi\fi\fi\fi\fi\fi%
}
%    \end{macrocode}
%
% \end{macro}
%
% \begin{macro}{\mdwfileinfo}
%
% Saying \syntax{"\\mdwfileinfo{"<file-name>"}{"<info>"}"} extracts the
% wanted item of \<info> from the version information for file \<file-name>.
%
%    \begin{macrocode}
\def\mdwfileinfo#1#2{\mdwfileinfo@i{#2}#1.\@@}
\def\mdwfileinfo@i#1#2.#3\@@{\csname#2#1\endcsname}
%    \end{macrocode}
%
% \end{macro}
%
%
% \subsection{List handling}
%
% There are several other lists I need to build.  These macros will do
% the necessary stuff.
%
% \begin{macro}{\mdw@ifitem}
%
% The macro \syntax{"\\mdw@ifitem"<item>"\\in"<list>"{"<true-text>"}"^^A
% "{"<false-text>"}"} does \<true-text> if the \<item> matches any item in
% the \<list>; otherwise it does \<false-text>.
%
%    \begin{macrocode}
\def\mdw@ifitem#1\in#2{%
  \@tempswafalse%
  \def\@tempa{#1}%
  \def\do##1{\def\@tempb{##1}\ifx\@tempa\@tempb\@tempswatrue\fi}%
  #2%
  \if@tempswa\expandafter\@firstoftwo\else\expandafter\@secondoftwo\fi%
}
%    \end{macrocode}
%
% \end{macro}
%
% \begin{macro}{\mdw@append}
%
% Saying \syntax{"\\mdw@append"<item>"\\to"<list>} adds the given \<item>
% to the end of the given \<list>.
%
%    \begin{macrocode}
\def\mdw@append#1\to#2{%
  \toks@{\do{#1}}%
  \toks\tw@\expandafter{#2}%
  \edef#2{\the\toks\tw@\the\toks@}%
}
%    \end{macrocode}
%
% \end{macro}
%
% \begin{macro}{\mdw@prepend}
%
% Saying \syntax{"\\mdw@prepend"<item>"\\to"<list>} adds the \<item> to the
% beginning of the \<list>.
%
%    \begin{macrocode}
\def\mdw@prepend#1\to#2{%
  \toks@{\do{#1}}%
  \toks\tw@\expandafter{#2}%
  \edef#2{\the\toks@\the\toks\tw@}%
}
%    \end{macrocode}
%
% \end{macro}
%
% \begin{macro}{\mdw@add}
%
% Finally, saying \syntax{"\\mdw@add"<item>"\\to"<list>} adds the \<item>
% to the list only if it isn't there already.
%
%    \begin{macrocode}
\def\mdw@add#1\to#2{\mdw@ifitem#1\in#2{}{\mdw@append#1\to#2}}
%    \end{macrocode}
%
% \end{macro}
%
%
% \subsection{Described file handling}
%
% I'l maintain lists of packages, document classes, and other files
% described by the current documentation file.
%
% First of all, I'll declare the various list macros.
%
%    \begin{macrocode}
\def\dopackages{}
\def\doclasses{}
\def\dootherfiles{}
%    \end{macrocode}
%
% \begin{macro}{\describespackage}
%
% A document file can declare that it describes a package by saying
% \syntax{"\\describespackage{"<package-name>"}"}.  I add the package to
% my list, read the package into memory (so that the documentation can
% offer demonstrations of it) and read the version information.
%
%    \begin{macrocode}
\def\describespackage#1{%
  \mdw@ifitem#1\in\dopackages{}{%
    \mdw@append#1\to\dopackages%
    \usepackage{#1}%
    \mdwpkginfo{#1.sty}%
  }%
}
%    \end{macrocode}
%
% \end{macro}
%
% \begin{macro}{\describesclass}
%
% By saying \syntax{"\\describesclass{"<class-name>"}"}, a document file
% can declare that it describes a document class.  I'll assume that the
% document class is already loaded, because it's much too late to load
% it now.
%
%    \begin{macrocode}
\def\describesclass#1{\mdw@add#1\to\doclasses\mdwpkginfo{#1.cls}}
%    \end{macrocode}
%
% \end{macro}
%
% \begin{macro}{\describesfile}
%
% Finally, other `random' files, which don't have the status of real \LaTeX\
% packages or document classes, can be described by saying \syntax{^^A
% "\\describesfile{"<file-name>"}" or "\\describesfile*{"<file-name>"}"}.
% The difference is that the starred version will not |\input| the file.
%
%    \begin{macrocode}
\def\describesfile{%
  \@ifstar{\describesfile@i\@gobble}{\describesfile@i\input}%
}
\def\describesfile@i#1#2{%
  \mdw@ifitem#2\in\dootherfiles{}{%
    \mdw@add#2\to\dootherfiles%
    #1{#2}%
    \mdwpkginfo{#2}%
  }%
}
%    \end{macrocode}
%
% \end{macro}
%
%
% \subsection{Author and title handling}
%
% I'll redefine the |\author| and |\title| commands so that I get told
% whether I need to do it myself.
%
% \begin{macro}{\author}
%
% This is easy: I'll save the old meaning, and then redefine |\author| to
% do the old thing and redefine itself to then do nothing.
%
%    \begin{macrocode}
\let\mdw@author\author
\def\author{\let\author\@gobble\mdw@author}
%    \end{macrocode}
%
% \end{macro}
%
% \begin{macro}{\title}
%
% And oddly enough, I'll do exactly the same thing for the title, except
% that I'll also disable the |\mdw@buildtitle| command, which constructs
% the title automatically.
%
%    \begin{macrocode}
\let\mdw@title\title
\def\title{\let\title\@gobble\let\mdw@buildtitle\relax\mdw@title}
%    \end{macrocode}
%
% \end{macro}
%
% \begin{macro}{\date}
%
% This works in a very similar sort of way.
%
%    \begin{macrocode}
\def\date#1{\let\date\@gobble\def\today{#1}}
%    \end{macrocode}
%
% \end{macro}
%
% \begin{macro}{\datefrom}
%
% Saying \syntax{"\\datefrom{"<file-name>"}"} sets the document date from
% the given filename.
%
%    \begin{macrocode}
\def\datefrom#1{%
  \protected@edef\@tempa{\noexpand\date{\csname #1date\endcsname}}%
  \@tempa%
}
%    \end{macrocode}
%
% \end{macro}
%
% \begin{macro}{\docfile}
%
% Saying \syntax{"\\docfile{"<file-name>"}"} sets up the file name from which
% documentation will be read.
%
%    \begin{macrocode}
\def\docfile#1{%
  \def\@tempa##1.##2\@@{\def\@basefile{##1.##2}\def\@basename{##1}}%
  \edef\@tempb{\noexpand\@tempa#1\noexpand\@@}%
  \@tempb%
}
%    \end{macrocode}
%
% I'll set up a default value as well.
%
%    \begin{macrocode}
\docfile{\jobname.dtx}
%    \end{macrocode}
%
% \end{macro}
%
%
% \subsection{Building title strings}
%
% This is rather tricky.  For each list, I need to build a legible looking
% string.
%
% \begin{macro}{\mdw@addtotitle}
%
% By saying
%\syntax{"\\mdw@addtotitle{"<list>"}{"<command>"}{"<singular>"}{"<plural>"}"}
% I can add the contents of a list to the current title string in the
% |\mdw@title| macro.
%
%    \begin{macrocode}
\tdef\mdw@addtotitle#1#2#3#4{%
%    \end{macrocode}
%
% Now to get to work.  I need to keep one `lookahead' list item, and a count
% of the number of items read so far.  I'll keep the lookahead item in
% |\@nextitem| and the counter in |\count@|.
%
%    \begin{macrocode}
  \count@\z@%
%    \end{macrocode}
%
% Now I'll define what to do for each list item.  The |\protect| command is
% already set up appropriately for playing with |\edef| commands.
%
%    \begin{macrocode}
  \def\do##1{%
%    \end{macrocode}
%
% The first job is to add the previous item to the title string.  If this
% is the first item, though, I'll just add the appropriate \lit{The } or
% \lit{ and the } string to the title (this is stored in the |\@prefix|
% macro).
%
%    \begin{macrocode}
    \edef\mdw@title{%
      \mdw@title%
      \ifcase\count@\@prefix%
      \or\@nextitem%
      \else, \@nextitem%
      \fi%
    }%
%    \end{macrocode}
%
% That was rather easy.  Now I'll set up the |\@nextitem| macro for the
% next time around the loop.
%
%    \begin{macrocode}
    \edef\@nextitem{%
      \protect#2{##1}%
      \protect\footnote{%
        The \protect#2{##1} #3 is currently at version %
        \mdwfileinfo{##1}{version}, dated \mdwfileinfo{##1}{date}.%
      }\space%
    }%
%    \end{macrocode}
%
% Finally, I need to increment the counter.
%
%    \begin{macrocode}
    \advance\count@\@ne%
  }%
%    \end{macrocode}
%
% Now execute the list.
%
%    \begin{macrocode}
  #1%
%    \end{macrocode}
%
% I still have one item left over, unless the list was empty.  I'll add
% that now.
%
%    \begin{macrocode}
  \edef\mdw@title{%
    \mdw@title%
    \ifcase\count@%
    \or\@nextitem\space#3%
    \or\ and \@nextitem\space#4%
    \fi%
  }%
%    \end{macrocode}
%
% Finally, if $|\count@| \ne 0$, I must set |\@prefix| to \lit{ and the }.
%
%    \begin{macrocode}
  \ifnum\count@>\z@\def\@prefix{ and the }\fi%
}
%    \end{macrocode}
%
% \end{macro}
%
% \begin{macro}{\mdw@buildtitle}
%
% This macro will actually do the job of building the title string.
%
%    \begin{macrocode}
\tdef\mdw@buildtitle{%
%    \end{macrocode}
%
% First of all, I'll open a group to avoid polluting the namespace with
% my gubbins (although the code is now much tidier than it has been in
% earlier releases).
%
%    \begin{macrocode}
  \begingroup%
%    \end{macrocode}
%
% The title building stuff makes extensive use of |\edef|.  I'll set
% |\protect| appropriately.  (For those not in the know,
% |\@unexpandable@protect| expands to `|\noexpand\protect\noexpand|',
% which prevents expansion of the following macro, and inserts a |\protect|
% in front of it ready for the next |\edef|.)
%
%    \begin{macrocode}
  \let\@@protect\protect\let\protect\@unexpandable@protect%
%    \end{macrocode}
%
% Set up some simple macros ready for the main code.
%
%    \begin{macrocode}
  \def\mdw@title{}%
  \def\@prefix{The }%
%    \end{macrocode}
%
% Now build the title.  This is fun.
%
%    \begin{macrocode}
  \mdw@addtotitle\dopackages\package{package}{packages}%
  \mdw@addtotitle\doclasses\package{document class}{document classes}%
  \mdw@addtotitle\dootherfiles\texttt{file}{files}%
%    \end{macrocode}
%
% Now I want to end the group and set the title from my string.  The
% following hacking will do this.
%
%    \begin{macrocode}
  \edef\next{\endgroup\noexpand\title{\mdw@title}}%
  \next%
}
%    \end{macrocode}
%
% \end{macro}
%
%
% \subsection{Starting the main document}
%
% \begin{macro}{\mdwdoc}
%
% Once the document preamble has done all of its stuff, it calls the
% |\mdwdoc| command, which takes over and really starts the documentation
% going.
%
%    \begin{macrocode}
\def\mdwdoc{%
%    \end{macrocode}
%
% First, I'll construct the title string.
%
%    \begin{macrocode}
  \mdw@buildtitle%
  \author{Mark Wooding}%
%    \end{macrocode}
%
% Set up the date string based on the date of the package which shares
% the same name as the current file.
%
%    \begin{macrocode}
  \datefrom\@basename%
%    \end{macrocode}
%
% Set up verbatim characters after all the packages have started.
%
%    \begin{macrocode}
  \shortverb\|%
  \shortverb\"%
%    \end{macrocode}
%
% Start the document, and put the title in.
%
%    \begin{macrocode}
  \begin{document}
  \maketitle%
%    \end{macrocode}
%
% This is nasty.  It makes maths displays work properly in demo environments.
% \emph{The \LaTeX\ Companion} exhibits the bug which this hack fixes.  So
% ner.
%
%    \begin{macrocode}
  \abovedisplayskip\z@%
%    \end{macrocode}
%
% Now start the contents tables.  After starting each one, I'll make it
% be multicolumnar.
%
%    \begin{macrocode}
  \def\do##1##2{%
    ##2%
    \ifhave@multicol\addtocontents{##1}{%
      \protect\begin{multicols}{2}%
      \hbadness\@M%
    }\fi%
  }%
  \docontents%
%    \end{macrocode}
%
% Input the main file now.
%
%    \begin{macrocode}
  \DocInput{\@basefile}%
%    \end{macrocode}
%
% That's it.  I'm done.
%
%    \begin{macrocode}
  \end{document}
}
%    \end{macrocode}
%
% \end{macro}
%
%
% \subsection{And finally\dots}
%
% Right at the end I'll put a hook for the configuration file.
%
%    \begin{macrocode}
\ifx\mdwhook\@@undefined\else\expandafter\mdwhook\fi
%    \end{macrocode}
%
% That's all the code done now.  I'll change back to `user' mode, where
% all the magic control sequences aren't allowed any more.
%
%    \begin{macrocode}
\makeatother
%</mdwtools>
%    \end{macrocode}
%
% Oh, wait!  What if I want to typeset this documentation?  Aha.  I'll cope
% with that by comparing |\jobname| with my filename |mdwtools|.  However,
% there's some fun here, because |\jobname| contains category-12 letters,
% while my letters are category-11.  Time to play with |\string| in a messy
% way.
%
%    \begin{macrocode}
%<*driver>
\makeatletter
\edef\@tempa{\expandafter\@gobble\string\mdwtools}
\edef\@tempb{\jobname}
\ifx\@tempa\@tempb
  \describesfile*{mdwtools.tex}
  \docfile{mdwtools.tex}
  \makeatother
  \expandafter\mdwdoc
\fi
\makeatother
%</driver>
%    \end{macrocode}
%
% That's it.  Done!
%
% \hfill Mark Wooding, \today
%
% \Finale
%
\endinput
|
% \<declarations>
% |\mdwdoc|
% \end{listinglist}
% The initial |\input| reads in this file and sets up the various commands
% which may be needed.  The final |\mdwdoc| actually starts the document,
% inserting a title (which is automatically generated), a table of
% contents etc., and reads the documentation file in (using the |\DocInput|
% command from the \package{doc} package.
%
% \subsubsection{Describing packages}
%
% \DescribeMacro{\describespackage}
% \DescribeMacro{\describesclass}
% \DescribeMacro{\describesfile}
% \DescribeMacro{\describesfile*}
% The most important declarations are those which declare what the
% documentation describes.  Saying \syntax{"\\describespackage{<package>}"}
% loads the \<package> (if necessary) and adds it to the auto-generated
% title, along with a footnote containing version information.  Similarly,
% |\describesclass| adds a document class name to the title (without loading
% it -- the document itself must do this, with the |\documentclass| command).
% For files which aren't packages or classes, use the |\describesfile| or
% |\describesfile*| command (the $*$-version won't |\input| the file, which
% is handy for files like |mdwtools.tex|, which are already input).
%
% \DescribeMacro{\author}
% \DescribeMacro{\date}
% \DescribeMacro{\title}
% The |\author|, |\date| and |\title| declarations work slightly differently
% to normal -- they ensure that only the \emph{first} declaration has an
% effect.  (Don't you play with |\author|, please, unless you're using this
% program to document your own packages.)  Using |\title| suppresses the
% automatic title generation.
%
% \DescribeMacro{\docdate}
% The default date is worked out from the version string of the package or
% document class whose name is the same as that of the documentation file.
% You can choose a different `main' file by saying
% \syntax{"\\docdate{"<file>"}"}.
%
% \subsubsection{Contents handling}
%
% \DescribeMacro{\addcontents}
% A documentation file always has a table of contents.  Other
% contents-like lists can be added by saying
% \syntax{"\\addcontents{"<extension>"}{"<command>"}"}.  The \<extension>
% is the file extension of the contents file (e.g., \lit{lot} for the
% list of tables); the \<command> is the command to actually typeset the
% contents file (e.g., |\listoftables|).
%
% \subsubsection{Other declarations}
%
% \DescribeMacro{\implementation}
% The \package{doc} package wants you to say
% \syntax{"\\StopEventually{"<stuff>"}"}' before describing the package
% implementation.  Using |mdwtools.tex|, you just say |\implementation|, and
% everything works.  It will automatically read in the licence text (from
% |gpl.tex|, and wraps some other things up.
%
% 
% \subsection{Other commands}
%
% The |mdwtools.tex| file includes the \package{syntax} and \package{sverb}
% packages so that they can be used in documentation files.  It also defines
% some trivial commands of its own.
%
% \DescribeMacro{\<}
% Saying \syntax{"\\<"<text>">" is the same as "\\synt{"<text>"}"}; this
% is a simple abbreviation.
%
% \DescribeMacro{\smallf}
% Saying \syntax{"\\smallf" <number>"/"<number>} typesets a little fraction,
% like this: \smallf 3/4.  It's useful when you want to say that the default
% value of a length is 2 \smallf 1/2\,pt, or something like that.
%
%
% \subsection{Customisation}
%
% You can customise the way that the package documentation looks by writing
% a file called |mdwtools.cfg|.  You can redefine various commands (before
% they're defined here, even; |mdwtools.tex| checks most of the commands that
% it defines to make sure they haven't been defined already.
%
% \DescribeMacro{\indexing}
% If you don't want the prompt about whether to generate index files, you
% can define the |\indexing| command to either \lit{y} or \lit{n}.  I'd
% recommend that you use |\providecommand| for this, to allow further
% customisation from the command line.
%
% \DescribeMacro{\mdwdateformat}
% If you don't like my date format (maybe you're American or something),
% you can redefine the |\mdwdateformat| command.  It takes three arguments:
% the year, month and date, as numbers; it should expand to something which
% typesets the date nicely.  The default format gives something like
% `10 May 1996'.  You can produce something rather more exotic, like
% `10\textsuperscript{th} May \textsc{\romannumeral 1996}' by saying
%\begin{listing}
%\newcommand{\mdwdateformat}[3]{%
%  \number#3\textsuperscript{\numsuffix{#3}}\ %
%  \monthname{#2}\ %
%  \textsc{\romannumeral #1}%
%}
%\end{listing}
% \DescribeMacro{\monthname}
% \DescribeMacro{\numsuffix}
% Saying \syntax{"\\monthname{"<number>"}"} expands to the name of the
% numbered month (which can be useful when doing date formats).  Saying
% \syntax{"\\numsuffix{"<number>"}"} will expand to the appropriate suffix
% (`th' or `rd' or whatever) for the \<number>.  You'll have to superscript
% it yourself, if this is what you want to do.  Putting the year number
% in roman numerals is just pretentious |;-)|.
%
% \DescribeMacro{\mdwhook}
% After all the declarations in |mdwtools.tex|, the command |\mdwhook| is
% executed, if it exists.  This can be set up by the configuration file
% to do whatever you want.
%
% There are lots of other things you can play with; you should look at the
% implementation section to see what's possible.
%
% \implementation
%
% \section{Implementation}
%
%    \begin{macrocode}
%<*mdwtools>
%    \end{macrocode}
%
% The first thing is that I'm not a \LaTeX\ package or anything official
% like that, so I must enable `|@|' as a letter by hand.
%
%    \begin{macrocode}
\makeatletter
%    \end{macrocode}
%
% Now input the user's configuration file, if it exists.  This is fairly
% simple stuff.
%
%    \begin{macrocode}
\@input{mdwtools.cfg}
%    \end{macrocode}
%
% Well, that's the easy bit done.
%
%
% \subsection{Initialisation}
%
% Obviously the first thing to do is to obtain a document class.  Obviously,
% it would be silly to do this if a document class has already been loaded,
% either by the package documentation or by the configuration file.
%
% The only way I can think of for finding out if a document class is already
% loaded is by seeing if the |\documentclass| command has been redefined
% to raise an error.  This isn't too hard, really.
%
%    \begin{macrocode}
\ifx\documentclass\@twoclasseserror\else
  \documentclass[a4paper]{ltxdoc}
  \ifx\doneclasses\mdw@undefined\else\doneclasses\fi
\fi
%    \end{macrocode}
%
% As part of my standard environment, I'll load some of my more useful
% packages.  If they're already loaded (possibly with different options),
% I'll not try to load them again.
%
%    \begin{macrocode}
\@ifpackageloaded{doc}{}{\usepackage{doc}}
\@ifpackageloaded{syntax}{}{\usepackage[rounded]{syntax}}
\@ifpackageloaded{sverb}{}{\usepackage{sverb}}
%    \end{macrocode}
%
%
% \subsection{Some macros for interaction}
%
% I like the \LaTeX\ star-boxes, although it's a pain having to cope with
% \TeX's space-handling rules.  I'll define a new typing-out macro which
% makes spaces more significant, and has a $*$-version which doesn't put
% a newline on the end, and interacts prettily with |\read|.
%
% First of all, I need to make spaces active, so I can define things about
% active spaces.
%
%    \begin{macrocode}
\begingroup\obeyspaces
%    \end{macrocode}
%
% Now to define the main macro.  This is easy stuff.  Spaces must be
% carefully rationed here, though.
%
% I'll start a group, make spaces active, and make spaces expand to ordinary
% space-like spaces.  Then I'll look for a star, and pass either |\message|
% (which doesn't start a newline, and interacts with |\read| well) or
% |\immediate\write 16| which does a normal write well.
%
%    \begin{macrocode}
\gdef\mdwtype{%
\begingroup\catcode`\ \active\let \space%
\@ifstar{\mdwtype@i{\message}}{\mdwtype@i{\immediate\write\sixt@@n}}%
}
\endgroup
%    \end{macrocode}
%
% Now for the easy bit.  I have the thing to do, and the thing to do it to,
% so do that and end the group.
%
%    \begin{macrocode}
\def\mdwtype@i#1#2{#1{#2}\endgroup}
%    \end{macrocode}
%
%
% \subsection{Decide on indexing}
%
% A configuration file can decide on indexing by defining the |\indexing|
% macro to either \lit{y} or \lit{n}.  If it's not set, then I'll prompt
% the user.
%
% First of all, I want a switch to say whether I'm indexing.
%
%    \begin{macrocode}
\newif\ifcreateindex
%    \end{macrocode}
%
% Right: now I need to decide how to make progress.  If the macro's not set,
% then I want to set it, and start a row of stars.
%
%    \begin{macrocode}
\ifx\indexing\@@undefined
  \mdwtype{*****************************}
  \def\indexing{?}
\fi
%    \end{macrocode}
%
% Now enter a loop, asking the user whether to do indexing, until I get
% a sensible answer.
%
%    \begin{macrocode}
\loop
  \@tempswafalse
  \if y\indexing\@tempswatrue\createindextrue\fi
  \if Y\indexing\@tempswatrue\createindextrue\fi
  \if n\indexing\@tempswatrue\createindexfalse\fi
  \if N\indexing\@tempswatrue\createindexfalse\fi
  \if@tempswa\else
  \mdwtype*{* Create index files? (y/n) *}
  \read\sixt@@n to\indexing%
\repeat
%    \end{macrocode}
%
% Now, based on the results of that, display a message about the indexing.
%
%    \begin{macrocode}
\mdwtype{*****************************}
\ifcreateindex
  \mdwtype{* Creating index files      *}
  \mdwtype{* This may take some time   *}
\else
  \mdwtype{* Not creating index files  *}
\fi
\mdwtype{*****************************}
%    \end{macrocode}
%
% Now I can play with the indexing commands of the \package{doc} package
% to do whatever it is that the user wants.
%
%    \begin{macrocode}
\ifcreateindex
  \CodelineIndex
  \EnableCrossrefs
\else
  \CodelineNumbered
  \DisableCrossrefs
\fi
%    \end{macrocode}
%
% And register lots of plain \TeX\ things which shouldn't be indexed.
% This contains lots of |\if|\dots\ things which don't fit nicely in
% conditionals, which is a shame.  Still, it doesn't matter that much,
% really.
%
%    \begin{macrocode}
\DoNotIndex{\def,\long,\edef,\xdef,\gdef,\let,\global}
\DoNotIndex{\if,\ifnum,\ifdim,\ifcat,\ifmmode,\ifvmode,\ifhmode,%
            \iftrue,\iffalse,\ifvoid,\ifx,\ifeof,\ifcase,\else,\or,\fi}
\DoNotIndex{\box,\copy,\setbox,\unvbox,\unhbox,\hbox,%
            \vbox,\vtop,\vcenter}
\DoNotIndex{\@empty,\immediate,\write}
\DoNotIndex{\egroup,\bgroup,\expandafter,\begingroup,\endgroup}
\DoNotIndex{\divide,\advance,\multiply,\count,\dimen}
\DoNotIndex{\relax,\space,\string}
\DoNotIndex{\csname,\endcsname,\@spaces,\openin,\openout,%
            \closein,\closeout}
\DoNotIndex{\catcode,\endinput}
\DoNotIndex{\jobname,\message,\read,\the,\m@ne,\noexpand}
\DoNotIndex{\hsize,\vsize,\hskip,\vskip,\kern,\hfil,\hfill,\hss}
\DoNotIndex{\m@ne,\z@,\z@skip,\@ne,\tw@,\p@}
\DoNotIndex{\dp,\wd,\ht,\vss,\unskip}
%    \end{macrocode}
%
% Last bit of indexing stuff, for now: I'll typeset the index in two columns
% (the default is three, which makes them too narrow for my tastes).
%
%    \begin{macrocode}
\setcounter{IndexColumns}{2}
%    \end{macrocode}
%
%
% \subsection{Selectively defining things}
%
% I don't want to tread on anyone's toes if they redefine any of these
% commands and things in a configuration file.  The following definitions
% are fairly evil, but should do the job OK.
%
% \begin{macro}{\@gobbledef}
%
% This macro eats the following |\def|inition, leaving not a trace behind.
%
%    \begin{macrocode}
\def\@gobbledef#1#{\@gobble}
%    \end{macrocode}
%
% \end{macro}
%
% \begin{macro}{\tdef}
% \begin{macro}{\tlet}
%
% The |\tdef| command is a sort of `tentative' definition -- it's like
% |\def| if the control sequence named doesn't already have a definition.
% |\tlet| does the same thing with |\let|.
%
%    \begin{macrocode}
\def\tdef#1{
  \ifx#1\@@undefined%
    \expandafter\def\expandafter#1%
  \else%
    \expandafter\@gobbledef%
  \fi%
}
\def\tlet#1#2{\ifx#1\@@undefined\let#1=#2\fi}
%    \end{macrocode}
%
% \end{macro}
% \end{macro}
%
%
% \subsection{General markup things}
%
% Now for some really simple things.  I'll define how to typeset package
% names and environment names (both in the sans serif font, for now).
%
%    \begin{macrocode}
\tlet\package\textsf
\tlet\env\textsf
%    \end{macrocode}
%
% I'll define the |\<|\dots|>| shortcut for syntax items suggested in the
% \package{syntax} package.
%
%    \begin{macrocode}
\tdef\<#1>{\synt{#1}}
%    \end{macrocode}
%
% And because it's used in a few places (mainly for typesetting lengths),
% here's a command for typesetting fractions in text.
%
%    \begin{macrocode}
\tdef\smallf#1/#2{\ensuremath{^{#1}\!/\!_{#2}}}
%    \end{macrocode}
%
%
% \subsection{A table environment}
%
% \begin{environment}{tab}
%
% Most of the packages don't use the (obviously perfect) \package{mdwtab}
% package, because it's big, and takes a while to load.  Here's an
% environment for typesetting centred tables.  The first (optional) argument
% is some declarations to perform.  The mandatory argument is the table
% preamble (obviously).
%
%    \begin{macrocode}
\@ifundefined{tab}{%
  \newenvironment{tab}[2][\relax]{%
    \par\vskip2ex%
    \centering%
    #1%
    \begin{tabular}{#2}%
  }{%
    \end{tabular}%
    \par\vskip2ex%
  }
}{}
%    \end{macrocode}
%
% \end{environment}
%
%
% \subsection{Commenting out of stuff}
%
% \begin{environment}{meta-comment}
%
% Using |\iffalse|\dots|\fi| isn't much fun.  I'll define a gobbling
% environment using the \package{sverb} stuff.
%
%    \begin{macrocode}
\ignoreenv{meta-comment}
%    \end{macrocode}
%
% \end{environment}
%
%
% \subsection{Float handling}
%
% This gubbins will try to avoid float pages as much as possible, and (with
% any luck) encourage floats to be put on the same pages as text.
%
%    \begin{macrocode}
\def\textfraction{0.1}
\def\topfraction{0.9}
\def\bottomfraction{0.9}
\def\floatpagefraction{0.7}
%    \end{macrocode}
%
% Now redefine the default float-placement parameters to allow `here' floats.
%
%    \begin{macrocode}
\def\fps@figure{htbp}
\def\fps@table{htbp}
%    \end{macrocode}
%
%
% \subsection{Other bits of parameter tweaking}
%
% Make \env{grammar} environments look pretty, by indenting the left hand
% sides by a large amount.
%
%    \begin{macrocode}
\grammarindent1in
%    \end{macrocode}
%
% I don't like being told by \TeX\ that my paragraphs are hard to linebreak:
% I know this already.  This lot should shut \TeX\ up about most problems.
%
%    \begin{macrocode}
\sloppy
\hbadness\@M
\hfuzz10\p@
%    \end{macrocode}
%
% Also make \TeX\ shut up in the index.  The \package{multicol} package
% irritatingly plays with |\hbadness|.  This is the best hook I could find
% for playing with this setting.
%
%    \begin{macrocode}
\expandafter\def\expandafter\IndexParms\expandafter{%
  \IndexParms%
  \hbadness\@M%
}
%    \end{macrocode}
%
% The other thing I really don't like is `Marginpar moved' warnings.  This
% will get rid of them, and lots of other \LaTeX\ warnings at the same time.
%
%    \begin{macrocode}
\let\@latex@warning@no@line\@gobble
%    \end{macrocode}
%
% Put some extra space between table rows, please.
%
%    \begin{macrocode}
\def\arraystretch{1.2}
%    \end{macrocode}
%
% Most of the code is at guard level one, so typeset that in upright text.
%
%    \begin{macrocode}
\setcounter{StandardModuleDepth}{1}
%    \end{macrocode}
%
%
% \subsection{Contents handling}
%
% I use at least one contents file (the main table of contents) although
% I may want more.  I'll keep a list of contents files which I need to
% handle.
%
% There are two things I need to do to contents files here:
% \begin{itemize}
% \item I must typeset the table of contents at the beginning of the
%       document; and
% \item I want to typeset tables of contents in two columns (using the
%       \package{multicol} package).
% \end{itemize}
%
% The list consists of items of the form
% \syntax{"\\do{"<extension>"}{"<command>"}"}, where \<extension> is the
% file extension of the contents file, and \<command> is the command to
% typeset it.
%
% \begin{macro}{\docontents}
%
% This is where I keep the list of contents files.  I'll initialise it to
% just do the standard contents table.
%
%    \begin{macrocode}
\def\docontents{\do{toc}{\tableofcontents}}
%    \end{macrocode}
%
% \end{macro}
%
% \begin{macro}{\addcontents}
%
% By saying \syntax{"\\addcontents{"<extension>"}{"<command>"}"}, a document
% can register a new table of contents which gets given the two-column
% treatment properly.  This is really easy to implement.
%
%    \begin{macrocode}
\def\addcontents#1#2{%
  \toks@\expandafter{\docontents\do{#1}{#2}}%
  \edef\docontents{\the\toks@}%
}
%    \end{macrocode}
%
% \end{macro}
%
%
% \subsection{Finishing it all off}
%
% \begin{macro}{\finalstuff}
%
% The |\finalstuff| macro is a hook for doing things at the end of the
% document.  Currently, it inputs the licence agreement as an appendix.
%
%    \begin{macrocode}
\tdef\finalstuff{\appendix\part*{Appendix}% \iffalse <meta-comment>
%
% $Id: gpl.tex,v 1.1 2000/07/13 09:10:20 michael Exp $
%
% The GNU General Public Licence as a LaTeX section
%
% (c) 1989, 1991 Free Software Foundation, Inc.
%   LaTeX markup and minor formatting changes by Mark Wooding
%

%----- Revision history -----------------------------------------------------
%
% $Log: gpl.tex,v $
% Revision 1.1  2000/07/13 09:10:20  michael
% + Initial import
%
% Revision 1.1  1998/09/21 10:19:01  michael
% Initial implementation
%
% Revision 1.1  1996/11/19 20:51:14  mdw
% Initial revision
%

% --- Chapter heading ---
%
% We don't know whether this ought to be a section or a chapter.  Easy.
% We'll see if chapters are possible.
%
% \fi

\begingroup
\makeatletter

\edef\next#1#2#3{\relax
  \ifx\chapter\@@undefined
    \ifx\documentclass\@notprerr#2\else#3\fi
  \else#1\fi
}

\expandafter\endgroup\next
{
  \let\gpltoplevel\chapter
  \let\gplsec\section
  \let\gplend\endinput
}{
  \let\gpltoplevel\section
  \let\gplsec\subsection
  \let\gplend\endinput
}{
  \documentclass[a4paper]{article}
  \def\gpltoplevel#1{%
    \vspace*{1in}%
    \hbox to\hsize{\hfil\LARGE\bfseries#1\hfil}%
    \vspace{1in}%
  }
  \let\gplsec\section
  \def\gplend{\end{document}}
  \advance\textwidth1in
  \advance\oddsidemargin-.5in
  \sloppy
  \begin{document}
}

%^^A-------------------------------------------------------------------------
\gpltoplevel{The GNU General Public Licence}


The following is the text of the GNU General Public Licence, under the terms
of which this software is distrubuted.

\vspace{12pt}

\begin{center}
\textbf{GNU GENERAL PUBLIC LICENSE} \\
Version 2, June 1991
\end{center}

\begin{center}
Copyright (C) 1989, 1991 Free Software Foundation, Inc. \\
675 Mass Ave, Cambridge, MA 02139, USA

Everyone is permitted to copy and distribute verbatim copies \\
of this license document, but changing it is not allowed.
\end{center}


\gplsec{Preamble}

The licenses for most software are designed to take away your freedom to
share and change it.  By contrast, the GNU General Public License is intended
to guarantee your freedom to share and change free software---to make sure
the software is free for all its users.  This General Public License applies
to most of the Free Software Foundation's software and to any other program
whose authors commit to using it.  (Some other Free Software Foundation
software is covered by the GNU Library General Public License instead.)  You
can apply it to your programs, too.

When we speak of free software, we are referring to freedom, not price.  Our
General Public Licenses are designed to make sure that you have the freedom
to distribute copies of free software (and charge for this service if you
wish), that you receive source code or can get it if you want it, that you
can change the software or use pieces of it in new free programs; and that
you know you can do these things.

To protect your rights, we need to make restrictions that forbid anyone to
deny you these rights or to ask you to surrender the rights.  These
restrictions translate to certain responsibilities for you if you distribute
copies of the software, or if you modify it.

For example, if you distribute copies of such a program, whether gratis or
for a fee, you must give the recipients all the rights that you have.  You
must make sure that they, too, receive or can get the source code.  And you
must show them these terms so they know their rights.

We protect your rights with two steps: (1) copyright the software, and (2)
offer you this license which gives you legal permission to copy, distribute
and/or modify the software.

Also, for each author's protection and ours, we want to make certain that
everyone understands that there is no warranty for this free software.  If
the software is modified by someone else and passed on, we want its
recipients to know that what they have is not the original, so that any
problems introduced by others will not reflect on the original authors'
reputations.

Finally, any free program is threatened constantly by software patents.  We
wish to avoid the danger that redistributors of a free program will
individually obtain patent licenses, in effect making the program
proprietary.  To prevent this, we have made it clear that any patent must be
licensed for everyone's free use or not licensed at all.

The precise terms and conditions for copying, distribution and modification
follow.


\gplsec{Terms and conditions for copying, distribution and modification}

\begin{enumerate}

\makeatletter \setcounter{\@listctr}{-1} \makeatother

\item [0.] This License applies to any program or other work which contains a
      notice placed by the copyright holder saying it may be distributed
      under the terms of this General Public License.  The ``Program'',
      below, refers to any such program or work, and a ``work based on the
      Program'' means either the Program or any derivative work under
      copyright law: that is to say, a work containing the Program or a
      portion of it, either verbatim or with modifications and/or translated
      into another language.  (Hereinafter, translation is included without
      limitation in the term ``modification''.)  Each licensee is addressed
      as ``you''.

      Activities other than copying, distribution and modification are not
      covered by this License; they are outside its scope.  The act of
      running the Program is not restricted, and the output from the Program
      is covered only if its contents constitute a work based on the Program
      (independent of having been made by running the Program).  Whether that
      is true depends on what the Program does.

\item [1.] You may copy and distribute verbatim copies of the Program's
      source code as you receive it, in any medium, provided that you
      conspicuously and appropriately publish on each copy an appropriate
      copyright notice and disclaimer of warranty; keep intact all the
      notices that refer to this License and to the absence of any warranty;
      and give any other recipients of the Program a copy of this License
      along with the Program.

      You may charge a fee for the physical act of transferring a copy, and
      you may at your option offer warranty protection in exchange for a fee.

\item [2.] You may modify your copy or copies of the Program or any portion
      of it, thus forming a work based on the Program, and copy and
      distribute such modifications or work under the terms of Section 1
      above, provided that you also meet all of these conditions:

      \begin{enumerate}

      \item [(a)] You must cause the modified files to carry prominent
            notices stating that you changed the files and the date of any
            change.

      \item [(b)] You must cause any work that you distribute or publish,
            that in whole or in part contains or is derived from the Program
            or any part thereof, to be licensed as a whole at no charge to
            all third parties under the terms of this License.

      \item [(c)] If the modified program normally reads commands
            interactively when run, you must cause it, when started running
            for such interactive use in the most ordinary way, to print or
            display an announcement including an appropriate copyright notice
            and a notice that there is no warranty (or else, saying that you
            provide a warranty) and that users may redistribute the program
            under these conditions, and telling the user how to view a copy
            of this License.  (Exception: if the Program itself is
            interactive but does not normally print such an announcement,
            your work based on the Program is not required to print an
            announcement.)

      \end{enumerate}

      These requirements apply to the modified work as a whole.  If
      identifiable sections of that work are not derived from the Program,
      and can be reasonably considered independent and separate works in
      themselves, then this License, and its terms, do not apply to those
      sections when you distribute them as separate works.  But when you
      distribute the same sections as part of a whole which is a work based
      on the Program, the distribution of the whole must be on the terms of
      this License, whose permissions for other licensees extend to the
      entire whole, and thus to each and every part regardless of who wrote
      it.

      Thus, it is not the intent of this section to claim rights or contest
      your rights to work written entirely by you; rather, the intent is to
      exercise the right to control the distribution of derivative or
      collective works based on the Program.

      In addition, mere aggregation of another work not based on the Program
      with the Program (or with a work based on the Program) on a volume of a
      storage or distribution medium does not bring the other work under the
      scope of this License.

\item [3.] You may copy and distribute the Program (or a work based on it,
      under Section 2) in object code or executable form under the terms of
      Sections 1 and 2 above provided that you also do one of the following:

      \begin{enumerate}

      \item [(a)] Accompany it with the complete corresponding
            machine-readable source code, which must be distributed under the
            terms of Sections 1 and 2 above on a medium customarily used for
            software interchange; or,

      \item [(b)] Accompany it with a written offer, valid for at least three
            years, to give any third party, for a charge no more than your
            cost of physically performing source distribution, a complete
            machine-readable copy of the corresponding source code, to be
            distributed under the terms of Sections 1 and 2 above on a medium
            customarily used for software interchange; or,

      \item [(c)] Accompany it with the information you received as to the
            offer to distribute corresponding source code.  (This alternative
            is allowed only for noncommercial distribution and only if you
            received the program in object code or executable form with such
            an offer, in accord with Subsection b above.)

      \end{enumerate}

      The source code for a work means the preferred form of the work for
      making modifications to it.  For an executable work, complete source
      code means all the source code for all modules it contains, plus any
      associated interface definition files, plus the scripts used to control
      compilation and installation of the executable.  However, as a special
      exception, the source code distributed need not include anything that
      is normally distributed (in either source or binary form) with the
      major components (compiler, kernel, and so on) of the operating system
      on which the executable runs, unless that component itself accompanies
      the executable.

      If distribution of executable or object code is made by offering access
      to copy from a designated place, then offering equivalent access to
      copy the source code from the same place counts as distribution of the
      source code, even though third parties are not compelled to copy the
      source along with the object code.

\item [4.] You may not copy, modify, sublicense, or distribute the Program
      except as expressly provided under this License.  Any attempt otherwise
      to copy, modify, sublicense or distribute the Program is void, and will
      automatically terminate your rights under this License.  However,
      parties who have received copies, or rights, from you under this
      License will not have their licenses terminated so long as such parties
      remain in full compliance.

\item [5.] You are not required to accept this License, since you have not
      signed it.  However, nothing else grants you permission to modify or
      distribute the Program or its derivative works.  These actions are
      prohibited by law if you do not accept this License.  Therefore, by
      modifying or distributing the Program (or any work based on the
      Program), you indicate your acceptance of this License to do so, and
      all its terms and conditions for copying, distributing or modifying the
      Program or works based on it.

\item [6.] Each time you redistribute the Program (or any work based on the
      Program), the recipient automatically receives a license from the
      original licensor to copy, distribute or modify the Program subject to
      these terms and conditions.  You may not impose any further
      restrictions on the recipients' exercise of the rights granted herein.
      You are not responsible for enforcing compliance by third parties to
      this License.

\item [7.] If, as a consequence of a court judgment or allegation of patent
      infringement or for any other reason (not limited to patent issues),
      conditions are imposed on you (whether by court order, agreement or
      otherwise) that contradict the conditions of this License, they do not
      excuse you from the conditions of this License.  If you cannot
      distribute so as to satisfy simultaneously your obligations under this
      License and any other pertinent obligations, then as a consequence you
      may not distribute the Program at all.  For example, if a patent
      license would not permit royalty-free redistribution of the Program by
      all those who receive copies directly or indirectly through you, then
      the only way you could satisfy both it and this License would be to
      refrain entirely from distribution of the Program.

      If any portion of this section is held invalid or unenforceable under
      any particular circumstance, the balance of the section is intended to
      apply and the section as a whole is intended to apply in other
      circumstances.

      It is not the purpose of this section to induce you to infringe any
      patents or other property right claims or to contest validity of any
      such claims; this section has the sole purpose of protecting the
      integrity of the free software distribution system, which is
      implemented by public license practices.  Many people have made
      generous contributions to the wide range of software distributed
      through that system in reliance on consistent application of that
      system; it is up to the author/donor to decide if he or she is willing
      to distribute software through any other system and a licensee cannot
      impose that choice.

      This section is intended to make thoroughly clear what is believed to
      be a consequence of the rest of this License.

\item [8.] If the distribution and/or use of the Program is restricted in
      certain countries either by patents or by copyrighted interfaces, the
      original copyright holder who places the Program under this License may
      add an explicit geographical distribution limitation excluding those
      countries, so that distribution is permitted only in or among countries
      not thus excluded.  In such case, this License incorporates the
      limitation as if written in the body of this License.

\item [9.] The Free Software Foundation may publish revised and/or new
      versions of the General Public License from time to time.  Such new
      versions will be similar in spirit to the present version, but may
      differ in detail to address new problems or concerns.

      Each version is given a distinguishing version number.  If the Program
      specifies a version number of this License which applies to it and
      ``any later version'', you have the option of following the terms and
      conditions either of that version or of any later version published by
      the Free Software Foundation.  If the Program does not specify a
      version number of this License, you may choose any version ever
      published by the Free Software Foundation.

\item [10.] If you wish to incorporate parts of the Program into other free
      programs whose distribution conditions are different, write to the
      author to ask for permission.  For software which is copyrighted by the
      Free Software Foundation, write to the Free Software Foundation; we
      sometimes make exceptions for this.  Our decision will be guided by the
      two goals of preserving the free status of all derivatives of our free
      software and of promoting the sharing and reuse of software generally.

\begin{center}
NO WARRANTY
\end{center}

\bfseries

\item [11.] Because the Program is licensed free of charge, there is no
      warranty for the Program, to the extent permitted by applicable law.
      except when otherwise stated in writing the copyright holders and/or
      other parties provide the program ``as is'' without warranty of any
      kind, either expressed or implied, including, but not limited to, the
      implied warranties of merchantability and fitness for a particular
      purpose.  The entire risk as to the quality and performance of the
      Program is with you.  Should the Program prove defective, you assume
      the cost of all necessary servicing, repair or correction.

\item [12.] In no event unless required by applicable law or agreed to in
      writing will any copyright holder, or any other party who may modify
      and/or redistribute the program as permitted above, be liable to you
      for damages, including any general, special, incidental or
      consequential damages arising out of the use or inability to use the
      program (including but not limited to loss of data or data being
      rendered inaccurate or losses sustained by you or third parties or a
      failure of the Program to operate with any other programs), even if
      such holder or other party has been advised of the possibility of such
      damages.

\end{enumerate}

\begin{center}
\textbf{END OF TERMS AND CONDITIONS}
\end{center}


\gplsec{Appendix: How to Apply These Terms to Your New Programs}

If you develop a new program, and you want it to be of the greatest possible
use to the public, the best way to achieve this is to make it free software
which everyone can redistribute and change under these terms.

To do so, attach the following notices to the program.  It is safest to
attach them to the start of each source file to most effectively convey the
exclusion of warranty; and each file should have at least the ``copyright''
line and a pointer to where the full notice is found.

\begin{verbatim}
<one line to give the program's name and a brief idea of what it does.>
Copyright (C) 19yy  <name of author>

This program is free software; you can redistribute it and/or modify
it under the terms of the GNU General Public License as published by
the Free Software Foundation; either version 2 of the License, or
(at your option) any later version.

This program is distributed in the hope that it will be useful,
but WITHOUT ANY WARRANTY; without even the implied warranty of
MERCHANTABILITY or FITNESS FOR A PARTICULAR PURPOSE.  See the
GNU General Public License for more details.

You should have received a copy of the GNU General Public License
along with this program; if not, write to the Free Software
Foundation, Inc., 675 Mass Ave, Cambridge, MA 02139, USA.
\end{verbatim}

Also add information on how to contact you by electronic and paper mail.

If the program is interactive, make it output a short notice like this when
it starts in an interactive mode:

\begin{verbatim}
Gnomovision version 69, Copyright (C) 19yy name of author
Gnomovision comes with ABSOLUTELY NO WARRANTY; for details type `show w'.
This is free software, and you are welcome to redistribute it
under certain conditions; type `show c' for details.
\end{verbatim}

The hypothetical commands `show w' and `show c' should show the appropriate
parts of the General Public License.  Of course, the commands you use may be
called something other than `show w' and `show c'; they could even be
mouse-clicks or menu items--whatever suits your program.

You should also get your employer (if you work as a programmer) or your
school, if any, to sign a ``copyright disclaimer'' for the program, if
necessary.  Here is a sample; alter the names:

\begin{verbatim}
Yoyodyne, Inc., hereby disclaims all copyright interest in the program
`Gnomovision' (which makes passes at compilers) written by James Hacker.

<signature of Ty Coon>, 1 April 1989
Ty Coon, President of Vice
\end{verbatim}

This General Public License does not permit incorporating your program into
proprietary programs.  If your program is a subroutine library, you may
consider it more useful to permit linking proprietary applications with the
library.  If this is what you want to do, use the GNU Library General Public
License instead of this License.

\gplend
}
%    \end{macrocode}
%
% \end{macro}
%
% \begin{macro}{\implementation}
%
% The |\implementation| macro starts typesetting the implementation of
% the package(s).  If we're not doing the implementation, it just does
% this lot and ends the input file.
%
% I define a macro with arguments inside the |\StopEventually|, which causes
% problems, since the code gets put through an extra level of |\def|fing
% depending on whether the implementation stuff gets typeset or not.  I'll
% store the code I want to do in a separate macro.
%
%    \begin{macrocode}
\def\implementation{\StopEventually{\attheend}}
%    \end{macrocode}
%
% Now for the actual activity.  First, I'll do the |\finalstuff|.  Then, if
% \package{doc}'s managed to find the \package{multicol} package, I'll add
% the end of the environment to the end of each contents file in the list.
% Finally, I'll read the index in from its formatted |.ind| file.
%
%    \begin{macrocode}
\tdef\attheend{%
  \finalstuff%
  \ifhave@multicol%
    \def\do##1##2{\addtocontents{##1}{\protect\end{multicols}}}%
    \docontents%
  \fi%
  \PrintIndex%
}
%    \end{macrocode}
%
% \end{macro}
%
%
% \subsection{File version information}
%
% \begin{macro}{\mdwpkginfo}
%
% For setting up the automatic titles, I'll need to be able to work out
% file versions and things.  This macro will, given a file name, extract
% from \LaTeX\ the version information and format it into a sensible string.
%
% First of all, I'll put the original string (direct from the
% |\Provides|\dots\ command).  Then I'll pass it to another macro which can
% parse up the string into its various bits, along with the original
% filename.
%
%    \begin{macrocode}
\def\mdwpkginfo#1{%
  \edef\@tempa{\csname ver@#1\endcsname}%
  \expandafter\mdwpkginfo@i\@tempa\@@#1\@@%
}
%    \end{macrocode}
%
% Now for the real business.  I'll store the string I build in macros called
% \syntax{"\\"<filename>"date", "\\"<filename>"version" and
% "\\"<filename>"info"}, which store the file's date, version and
% `information string' respectively.  (Note that the file extension isn't
% included in the name.)
%
% This is mainly just tedious playing with |\expandafter|.  The date format
% is defined by a separate macro, which can be modified from the
% configuration file.
%
%    \begin{macrocode}
\def\mdwpkginfo@i#1/#2/#3 #4 #5\@@#6.#7\@@{%
  \expandafter\def\csname #6date\endcsname%
    {\protect\mdwdateformat{#1}{#2}{#3}}%
  \expandafter\def\csname #6version\endcsname{#4}%
  \expandafter\def\csname #6info\endcsname{#5}%
}
%    \end{macrocode}
%
% \end{macro}
%
% \begin{macro}{\mdwdateformat}
%
% Given three arguments, a year, a month and a date (all numeric), build a
% pretty date string.  This is fairly simple really.
%
%    \begin{macrocode}
\tdef\mdwdateformat#1#2#3{\number#3\ \monthname{#2}\ \number#1}
\def\monthname#1{%
  \ifcase#1\or%
     January\or February\or March\or April\or May\or June\or%
     July\or August\or September\or October\or November\or December%
  \fi%
}
\def\numsuffix#1{%
  \ifnum#1=1 st\else%
  \ifnum#1=2 nd\else%
  \ifnum#1=3 rd\else%
  \ifnum#1=21 st\else%
  \ifnum#1=22 nd\else%
  \ifnum#1=23 rd\else%
  \ifnum#1=31 st\else%
  th%
  \fi\fi\fi\fi\fi\fi\fi%
}
%    \end{macrocode}
%
% \end{macro}
%
% \begin{macro}{\mdwfileinfo}
%
% Saying \syntax{"\\mdwfileinfo{"<file-name>"}{"<info>"}"} extracts the
% wanted item of \<info> from the version information for file \<file-name>.
%
%    \begin{macrocode}
\def\mdwfileinfo#1#2{\mdwfileinfo@i{#2}#1.\@@}
\def\mdwfileinfo@i#1#2.#3\@@{\csname#2#1\endcsname}
%    \end{macrocode}
%
% \end{macro}
%
%
% \subsection{List handling}
%
% There are several other lists I need to build.  These macros will do
% the necessary stuff.
%
% \begin{macro}{\mdw@ifitem}
%
% The macro \syntax{"\\mdw@ifitem"<item>"\\in"<list>"{"<true-text>"}"^^A
% "{"<false-text>"}"} does \<true-text> if the \<item> matches any item in
% the \<list>; otherwise it does \<false-text>.
%
%    \begin{macrocode}
\def\mdw@ifitem#1\in#2{%
  \@tempswafalse%
  \def\@tempa{#1}%
  \def\do##1{\def\@tempb{##1}\ifx\@tempa\@tempb\@tempswatrue\fi}%
  #2%
  \if@tempswa\expandafter\@firstoftwo\else\expandafter\@secondoftwo\fi%
}
%    \end{macrocode}
%
% \end{macro}
%
% \begin{macro}{\mdw@append}
%
% Saying \syntax{"\\mdw@append"<item>"\\to"<list>} adds the given \<item>
% to the end of the given \<list>.
%
%    \begin{macrocode}
\def\mdw@append#1\to#2{%
  \toks@{\do{#1}}%
  \toks\tw@\expandafter{#2}%
  \edef#2{\the\toks\tw@\the\toks@}%
}
%    \end{macrocode}
%
% \end{macro}
%
% \begin{macro}{\mdw@prepend}
%
% Saying \syntax{"\\mdw@prepend"<item>"\\to"<list>} adds the \<item> to the
% beginning of the \<list>.
%
%    \begin{macrocode}
\def\mdw@prepend#1\to#2{%
  \toks@{\do{#1}}%
  \toks\tw@\expandafter{#2}%
  \edef#2{\the\toks@\the\toks\tw@}%
}
%    \end{macrocode}
%
% \end{macro}
%
% \begin{macro}{\mdw@add}
%
% Finally, saying \syntax{"\\mdw@add"<item>"\\to"<list>} adds the \<item>
% to the list only if it isn't there already.
%
%    \begin{macrocode}
\def\mdw@add#1\to#2{\mdw@ifitem#1\in#2{}{\mdw@append#1\to#2}}
%    \end{macrocode}
%
% \end{macro}
%
%
% \subsection{Described file handling}
%
% I'l maintain lists of packages, document classes, and other files
% described by the current documentation file.
%
% First of all, I'll declare the various list macros.
%
%    \begin{macrocode}
\def\dopackages{}
\def\doclasses{}
\def\dootherfiles{}
%    \end{macrocode}
%
% \begin{macro}{\describespackage}
%
% A document file can declare that it describes a package by saying
% \syntax{"\\describespackage{"<package-name>"}"}.  I add the package to
% my list, read the package into memory (so that the documentation can
% offer demonstrations of it) and read the version information.
%
%    \begin{macrocode}
\def\describespackage#1{%
  \mdw@ifitem#1\in\dopackages{}{%
    \mdw@append#1\to\dopackages%
    \usepackage{#1}%
    \mdwpkginfo{#1.sty}%
  }%
}
%    \end{macrocode}
%
% \end{macro}
%
% \begin{macro}{\describesclass}
%
% By saying \syntax{"\\describesclass{"<class-name>"}"}, a document file
% can declare that it describes a document class.  I'll assume that the
% document class is already loaded, because it's much too late to load
% it now.
%
%    \begin{macrocode}
\def\describesclass#1{\mdw@add#1\to\doclasses\mdwpkginfo{#1.cls}}
%    \end{macrocode}
%
% \end{macro}
%
% \begin{macro}{\describesfile}
%
% Finally, other `random' files, which don't have the status of real \LaTeX\
% packages or document classes, can be described by saying \syntax{^^A
% "\\describesfile{"<file-name>"}" or "\\describesfile*{"<file-name>"}"}.
% The difference is that the starred version will not |\input| the file.
%
%    \begin{macrocode}
\def\describesfile{%
  \@ifstar{\describesfile@i\@gobble}{\describesfile@i\input}%
}
\def\describesfile@i#1#2{%
  \mdw@ifitem#2\in\dootherfiles{}{%
    \mdw@add#2\to\dootherfiles%
    #1{#2}%
    \mdwpkginfo{#2}%
  }%
}
%    \end{macrocode}
%
% \end{macro}
%
%
% \subsection{Author and title handling}
%
% I'll redefine the |\author| and |\title| commands so that I get told
% whether I need to do it myself.
%
% \begin{macro}{\author}
%
% This is easy: I'll save the old meaning, and then redefine |\author| to
% do the old thing and redefine itself to then do nothing.
%
%    \begin{macrocode}
\let\mdw@author\author
\def\author{\let\author\@gobble\mdw@author}
%    \end{macrocode}
%
% \end{macro}
%
% \begin{macro}{\title}
%
% And oddly enough, I'll do exactly the same thing for the title, except
% that I'll also disable the |\mdw@buildtitle| command, which constructs
% the title automatically.
%
%    \begin{macrocode}
\let\mdw@title\title
\def\title{\let\title\@gobble\let\mdw@buildtitle\relax\mdw@title}
%    \end{macrocode}
%
% \end{macro}
%
% \begin{macro}{\date}
%
% This works in a very similar sort of way.
%
%    \begin{macrocode}
\def\date#1{\let\date\@gobble\def\today{#1}}
%    \end{macrocode}
%
% \end{macro}
%
% \begin{macro}{\datefrom}
%
% Saying \syntax{"\\datefrom{"<file-name>"}"} sets the document date from
% the given filename.
%
%    \begin{macrocode}
\def\datefrom#1{%
  \protected@edef\@tempa{\noexpand\date{\csname #1date\endcsname}}%
  \@tempa%
}
%    \end{macrocode}
%
% \end{macro}
%
% \begin{macro}{\docfile}
%
% Saying \syntax{"\\docfile{"<file-name>"}"} sets up the file name from which
% documentation will be read.
%
%    \begin{macrocode}
\def\docfile#1{%
  \def\@tempa##1.##2\@@{\def\@basefile{##1.##2}\def\@basename{##1}}%
  \edef\@tempb{\noexpand\@tempa#1\noexpand\@@}%
  \@tempb%
}
%    \end{macrocode}
%
% I'll set up a default value as well.
%
%    \begin{macrocode}
\docfile{\jobname.dtx}
%    \end{macrocode}
%
% \end{macro}
%
%
% \subsection{Building title strings}
%
% This is rather tricky.  For each list, I need to build a legible looking
% string.
%
% \begin{macro}{\mdw@addtotitle}
%
% By saying
%\syntax{"\\mdw@addtotitle{"<list>"}{"<command>"}{"<singular>"}{"<plural>"}"}
% I can add the contents of a list to the current title string in the
% |\mdw@title| macro.
%
%    \begin{macrocode}
\tdef\mdw@addtotitle#1#2#3#4{%
%    \end{macrocode}
%
% Now to get to work.  I need to keep one `lookahead' list item, and a count
% of the number of items read so far.  I'll keep the lookahead item in
% |\@nextitem| and the counter in |\count@|.
%
%    \begin{macrocode}
  \count@\z@%
%    \end{macrocode}
%
% Now I'll define what to do for each list item.  The |\protect| command is
% already set up appropriately for playing with |\edef| commands.
%
%    \begin{macrocode}
  \def\do##1{%
%    \end{macrocode}
%
% The first job is to add the previous item to the title string.  If this
% is the first item, though, I'll just add the appropriate \lit{The } or
% \lit{ and the } string to the title (this is stored in the |\@prefix|
% macro).
%
%    \begin{macrocode}
    \edef\mdw@title{%
      \mdw@title%
      \ifcase\count@\@prefix%
      \or\@nextitem%
      \else, \@nextitem%
      \fi%
    }%
%    \end{macrocode}
%
% That was rather easy.  Now I'll set up the |\@nextitem| macro for the
% next time around the loop.
%
%    \begin{macrocode}
    \edef\@nextitem{%
      \protect#2{##1}%
      \protect\footnote{%
        The \protect#2{##1} #3 is currently at version %
        \mdwfileinfo{##1}{version}, dated \mdwfileinfo{##1}{date}.%
      }\space%
    }%
%    \end{macrocode}
%
% Finally, I need to increment the counter.
%
%    \begin{macrocode}
    \advance\count@\@ne%
  }%
%    \end{macrocode}
%
% Now execute the list.
%
%    \begin{macrocode}
  #1%
%    \end{macrocode}
%
% I still have one item left over, unless the list was empty.  I'll add
% that now.
%
%    \begin{macrocode}
  \edef\mdw@title{%
    \mdw@title%
    \ifcase\count@%
    \or\@nextitem\space#3%
    \or\ and \@nextitem\space#4%
    \fi%
  }%
%    \end{macrocode}
%
% Finally, if $|\count@| \ne 0$, I must set |\@prefix| to \lit{ and the }.
%
%    \begin{macrocode}
  \ifnum\count@>\z@\def\@prefix{ and the }\fi%
}
%    \end{macrocode}
%
% \end{macro}
%
% \begin{macro}{\mdw@buildtitle}
%
% This macro will actually do the job of building the title string.
%
%    \begin{macrocode}
\tdef\mdw@buildtitle{%
%    \end{macrocode}
%
% First of all, I'll open a group to avoid polluting the namespace with
% my gubbins (although the code is now much tidier than it has been in
% earlier releases).
%
%    \begin{macrocode}
  \begingroup%
%    \end{macrocode}
%
% The title building stuff makes extensive use of |\edef|.  I'll set
% |\protect| appropriately.  (For those not in the know,
% |\@unexpandable@protect| expands to `|\noexpand\protect\noexpand|',
% which prevents expansion of the following macro, and inserts a |\protect|
% in front of it ready for the next |\edef|.)
%
%    \begin{macrocode}
  \let\@@protect\protect\let\protect\@unexpandable@protect%
%    \end{macrocode}
%
% Set up some simple macros ready for the main code.
%
%    \begin{macrocode}
  \def\mdw@title{}%
  \def\@prefix{The }%
%    \end{macrocode}
%
% Now build the title.  This is fun.
%
%    \begin{macrocode}
  \mdw@addtotitle\dopackages\package{package}{packages}%
  \mdw@addtotitle\doclasses\package{document class}{document classes}%
  \mdw@addtotitle\dootherfiles\texttt{file}{files}%
%    \end{macrocode}
%
% Now I want to end the group and set the title from my string.  The
% following hacking will do this.
%
%    \begin{macrocode}
  \edef\next{\endgroup\noexpand\title{\mdw@title}}%
  \next%
}
%    \end{macrocode}
%
% \end{macro}
%
%
% \subsection{Starting the main document}
%
% \begin{macro}{\mdwdoc}
%
% Once the document preamble has done all of its stuff, it calls the
% |\mdwdoc| command, which takes over and really starts the documentation
% going.
%
%    \begin{macrocode}
\def\mdwdoc{%
%    \end{macrocode}
%
% First, I'll construct the title string.
%
%    \begin{macrocode}
  \mdw@buildtitle%
  \author{Mark Wooding}%
%    \end{macrocode}
%
% Set up the date string based on the date of the package which shares
% the same name as the current file.
%
%    \begin{macrocode}
  \datefrom\@basename%
%    \end{macrocode}
%
% Set up verbatim characters after all the packages have started.
%
%    \begin{macrocode}
  \shortverb\|%
  \shortverb\"%
%    \end{macrocode}
%
% Start the document, and put the title in.
%
%    \begin{macrocode}
  \begin{document}
  \maketitle%
%    \end{macrocode}
%
% This is nasty.  It makes maths displays work properly in demo environments.
% \emph{The \LaTeX\ Companion} exhibits the bug which this hack fixes.  So
% ner.
%
%    \begin{macrocode}
  \abovedisplayskip\z@%
%    \end{macrocode}
%
% Now start the contents tables.  After starting each one, I'll make it
% be multicolumnar.
%
%    \begin{macrocode}
  \def\do##1##2{%
    ##2%
    \ifhave@multicol\addtocontents{##1}{%
      \protect\begin{multicols}{2}%
      \hbadness\@M%
    }\fi%
  }%
  \docontents%
%    \end{macrocode}
%
% Input the main file now.
%
%    \begin{macrocode}
  \DocInput{\@basefile}%
%    \end{macrocode}
%
% That's it.  I'm done.
%
%    \begin{macrocode}
  \end{document}
}
%    \end{macrocode}
%
% \end{macro}
%
%
% \subsection{And finally\dots}
%
% Right at the end I'll put a hook for the configuration file.
%
%    \begin{macrocode}
\ifx\mdwhook\@@undefined\else\expandafter\mdwhook\fi
%    \end{macrocode}
%
% That's all the code done now.  I'll change back to `user' mode, where
% all the magic control sequences aren't allowed any more.
%
%    \begin{macrocode}
\makeatother
%</mdwtools>
%    \end{macrocode}
%
% Oh, wait!  What if I want to typeset this documentation?  Aha.  I'll cope
% with that by comparing |\jobname| with my filename |mdwtools|.  However,
% there's some fun here, because |\jobname| contains category-12 letters,
% while my letters are category-11.  Time to play with |\string| in a messy
% way.
%
%    \begin{macrocode}
%<*driver>
\makeatletter
\edef\@tempa{\expandafter\@gobble\string\mdwtools}
\edef\@tempb{\jobname}
\ifx\@tempa\@tempb
  \describesfile*{mdwtools.tex}
  \docfile{mdwtools.tex}
  \makeatother
  \expandafter\mdwdoc
\fi
\makeatother
%</driver>
%    \end{macrocode}
%
% That's it.  Done!
%
% \hfill Mark Wooding, \today
%
% \Finale
%
\endinput
|
% \<declarations>
% |\mdwdoc|
% \end{listinglist}
% The initial |\input| reads in this file and sets up the various commands
% which may be needed.  The final |\mdwdoc| actually starts the document,
% inserting a title (which is automatically generated), a table of
% contents etc., and reads the documentation file in (using the |\DocInput|
% command from the \package{doc} package.
%
% \subsubsection{Describing packages}
%
% \DescribeMacro{\describespackage}
% \DescribeMacro{\describesclass}
% \DescribeMacro{\describesfile}
% \DescribeMacro{\describesfile*}
% The most important declarations are those which declare what the
% documentation describes.  Saying \syntax{"\\describespackage{<package>}"}
% loads the \<package> (if necessary) and adds it to the auto-generated
% title, along with a footnote containing version information.  Similarly,
% |\describesclass| adds a document class name to the title (without loading
% it -- the document itself must do this, with the |\documentclass| command).
% For files which aren't packages or classes, use the |\describesfile| or
% |\describesfile*| command (the $*$-version won't |\input| the file, which
% is handy for files like |mdwtools.tex|, which are already input).
%
% \DescribeMacro{\author}
% \DescribeMacro{\date}
% \DescribeMacro{\title}
% The |\author|, |\date| and |\title| declarations work slightly differently
% to normal -- they ensure that only the \emph{first} declaration has an
% effect.  (Don't you play with |\author|, please, unless you're using this
% program to document your own packages.)  Using |\title| suppresses the
% automatic title generation.
%
% \DescribeMacro{\docdate}
% The default date is worked out from the version string of the package or
% document class whose name is the same as that of the documentation file.
% You can choose a different `main' file by saying
% \syntax{"\\docdate{"<file>"}"}.
%
% \subsubsection{Contents handling}
%
% \DescribeMacro{\addcontents}
% A documentation file always has a table of contents.  Other
% contents-like lists can be added by saying
% \syntax{"\\addcontents{"<extension>"}{"<command>"}"}.  The \<extension>
% is the file extension of the contents file (e.g., \lit{lot} for the
% list of tables); the \<command> is the command to actually typeset the
% contents file (e.g., |\listoftables|).
%
% \subsubsection{Other declarations}
%
% \DescribeMacro{\implementation}
% The \package{doc} package wants you to say
% \syntax{"\\StopEventually{"<stuff>"}"}' before describing the package
% implementation.  Using |mdwtools.tex|, you just say |\implementation|, and
% everything works.  It will automatically read in the licence text (from
% |gpl.tex|, and wraps some other things up.
%
% 
% \subsection{Other commands}
%
% The |mdwtools.tex| file includes the \package{syntax} and \package{sverb}
% packages so that they can be used in documentation files.  It also defines
% some trivial commands of its own.
%
% \DescribeMacro{\<}
% Saying \syntax{"\\<"<text>">" is the same as "\\synt{"<text>"}"}; this
% is a simple abbreviation.
%
% \DescribeMacro{\smallf}
% Saying \syntax{"\\smallf" <number>"/"<number>} typesets a little fraction,
% like this: \smallf 3/4.  It's useful when you want to say that the default
% value of a length is 2 \smallf 1/2\,pt, or something like that.
%
%
% \subsection{Customisation}
%
% You can customise the way that the package documentation looks by writing
% a file called |mdwtools.cfg|.  You can redefine various commands (before
% they're defined here, even; |mdwtools.tex| checks most of the commands that
% it defines to make sure they haven't been defined already.
%
% \DescribeMacro{\indexing}
% If you don't want the prompt about whether to generate index files, you
% can define the |\indexing| command to either \lit{y} or \lit{n}.  I'd
% recommend that you use |\providecommand| for this, to allow further
% customisation from the command line.
%
% \DescribeMacro{\mdwdateformat}
% If you don't like my date format (maybe you're American or something),
% you can redefine the |\mdwdateformat| command.  It takes three arguments:
% the year, month and date, as numbers; it should expand to something which
% typesets the date nicely.  The default format gives something like
% `10 May 1996'.  You can produce something rather more exotic, like
% `10\textsuperscript{th} May \textsc{\romannumeral 1996}' by saying
%\begin{listing}
%\newcommand{\mdwdateformat}[3]{%
%  \number#3\textsuperscript{\numsuffix{#3}}\ %
%  \monthname{#2}\ %
%  \textsc{\romannumeral #1}%
%}
%\end{listing}
% \DescribeMacro{\monthname}
% \DescribeMacro{\numsuffix}
% Saying \syntax{"\\monthname{"<number>"}"} expands to the name of the
% numbered month (which can be useful when doing date formats).  Saying
% \syntax{"\\numsuffix{"<number>"}"} will expand to the appropriate suffix
% (`th' or `rd' or whatever) for the \<number>.  You'll have to superscript
% it yourself, if this is what you want to do.  Putting the year number
% in roman numerals is just pretentious |;-)|.
%
% \DescribeMacro{\mdwhook}
% After all the declarations in |mdwtools.tex|, the command |\mdwhook| is
% executed, if it exists.  This can be set up by the configuration file
% to do whatever you want.
%
% There are lots of other things you can play with; you should look at the
% implementation section to see what's possible.
%
% \implementation
%
% \section{Implementation}
%
%    \begin{macrocode}
%<*mdwtools>
%    \end{macrocode}
%
% The first thing is that I'm not a \LaTeX\ package or anything official
% like that, so I must enable `|@|' as a letter by hand.
%
%    \begin{macrocode}
\makeatletter
%    \end{macrocode}
%
% Now input the user's configuration file, if it exists.  This is fairly
% simple stuff.
%
%    \begin{macrocode}
\@input{mdwtools.cfg}
%    \end{macrocode}
%
% Well, that's the easy bit done.
%
%
% \subsection{Initialisation}
%
% Obviously the first thing to do is to obtain a document class.  Obviously,
% it would be silly to do this if a document class has already been loaded,
% either by the package documentation or by the configuration file.
%
% The only way I can think of for finding out if a document class is already
% loaded is by seeing if the |\documentclass| command has been redefined
% to raise an error.  This isn't too hard, really.
%
%    \begin{macrocode}
\ifx\documentclass\@twoclasseserror\else
  \documentclass[a4paper]{ltxdoc}
  \ifx\doneclasses\mdw@undefined\else\doneclasses\fi
\fi
%    \end{macrocode}
%
% As part of my standard environment, I'll load some of my more useful
% packages.  If they're already loaded (possibly with different options),
% I'll not try to load them again.
%
%    \begin{macrocode}
\@ifpackageloaded{doc}{}{\usepackage{doc}}
\@ifpackageloaded{syntax}{}{\usepackage[rounded]{syntax}}
\@ifpackageloaded{sverb}{}{\usepackage{sverb}}
%    \end{macrocode}
%
%
% \subsection{Some macros for interaction}
%
% I like the \LaTeX\ star-boxes, although it's a pain having to cope with
% \TeX's space-handling rules.  I'll define a new typing-out macro which
% makes spaces more significant, and has a $*$-version which doesn't put
% a newline on the end, and interacts prettily with |\read|.
%
% First of all, I need to make spaces active, so I can define things about
% active spaces.
%
%    \begin{macrocode}
\begingroup\obeyspaces
%    \end{macrocode}
%
% Now to define the main macro.  This is easy stuff.  Spaces must be
% carefully rationed here, though.
%
% I'll start a group, make spaces active, and make spaces expand to ordinary
% space-like spaces.  Then I'll look for a star, and pass either |\message|
% (which doesn't start a newline, and interacts with |\read| well) or
% |\immediate\write 16| which does a normal write well.
%
%    \begin{macrocode}
\gdef\mdwtype{%
\begingroup\catcode`\ \active\let \space%
\@ifstar{\mdwtype@i{\message}}{\mdwtype@i{\immediate\write\sixt@@n}}%
}
\endgroup
%    \end{macrocode}
%
% Now for the easy bit.  I have the thing to do, and the thing to do it to,
% so do that and end the group.
%
%    \begin{macrocode}
\def\mdwtype@i#1#2{#1{#2}\endgroup}
%    \end{macrocode}
%
%
% \subsection{Decide on indexing}
%
% A configuration file can decide on indexing by defining the |\indexing|
% macro to either \lit{y} or \lit{n}.  If it's not set, then I'll prompt
% the user.
%
% First of all, I want a switch to say whether I'm indexing.
%
%    \begin{macrocode}
\newif\ifcreateindex
%    \end{macrocode}
%
% Right: now I need to decide how to make progress.  If the macro's not set,
% then I want to set it, and start a row of stars.
%
%    \begin{macrocode}
\ifx\indexing\@@undefined
  \mdwtype{*****************************}
  \def\indexing{?}
\fi
%    \end{macrocode}
%
% Now enter a loop, asking the user whether to do indexing, until I get
% a sensible answer.
%
%    \begin{macrocode}
\loop
  \@tempswafalse
  \if y\indexing\@tempswatrue\createindextrue\fi
  \if Y\indexing\@tempswatrue\createindextrue\fi
  \if n\indexing\@tempswatrue\createindexfalse\fi
  \if N\indexing\@tempswatrue\createindexfalse\fi
  \if@tempswa\else
  \mdwtype*{* Create index files? (y/n) *}
  \read\sixt@@n to\indexing%
\repeat
%    \end{macrocode}
%
% Now, based on the results of that, display a message about the indexing.
%
%    \begin{macrocode}
\mdwtype{*****************************}
\ifcreateindex
  \mdwtype{* Creating index files      *}
  \mdwtype{* This may take some time   *}
\else
  \mdwtype{* Not creating index files  *}
\fi
\mdwtype{*****************************}
%    \end{macrocode}
%
% Now I can play with the indexing commands of the \package{doc} package
% to do whatever it is that the user wants.
%
%    \begin{macrocode}
\ifcreateindex
  \CodelineIndex
  \EnableCrossrefs
\else
  \CodelineNumbered
  \DisableCrossrefs
\fi
%    \end{macrocode}
%
% And register lots of plain \TeX\ things which shouldn't be indexed.
% This contains lots of |\if|\dots\ things which don't fit nicely in
% conditionals, which is a shame.  Still, it doesn't matter that much,
% really.
%
%    \begin{macrocode}
\DoNotIndex{\def,\long,\edef,\xdef,\gdef,\let,\global}
\DoNotIndex{\if,\ifnum,\ifdim,\ifcat,\ifmmode,\ifvmode,\ifhmode,%
            \iftrue,\iffalse,\ifvoid,\ifx,\ifeof,\ifcase,\else,\or,\fi}
\DoNotIndex{\box,\copy,\setbox,\unvbox,\unhbox,\hbox,%
            \vbox,\vtop,\vcenter}
\DoNotIndex{\@empty,\immediate,\write}
\DoNotIndex{\egroup,\bgroup,\expandafter,\begingroup,\endgroup}
\DoNotIndex{\divide,\advance,\multiply,\count,\dimen}
\DoNotIndex{\relax,\space,\string}
\DoNotIndex{\csname,\endcsname,\@spaces,\openin,\openout,%
            \closein,\closeout}
\DoNotIndex{\catcode,\endinput}
\DoNotIndex{\jobname,\message,\read,\the,\m@ne,\noexpand}
\DoNotIndex{\hsize,\vsize,\hskip,\vskip,\kern,\hfil,\hfill,\hss}
\DoNotIndex{\m@ne,\z@,\z@skip,\@ne,\tw@,\p@}
\DoNotIndex{\dp,\wd,\ht,\vss,\unskip}
%    \end{macrocode}
%
% Last bit of indexing stuff, for now: I'll typeset the index in two columns
% (the default is three, which makes them too narrow for my tastes).
%
%    \begin{macrocode}
\setcounter{IndexColumns}{2}
%    \end{macrocode}
%
%
% \subsection{Selectively defining things}
%
% I don't want to tread on anyone's toes if they redefine any of these
% commands and things in a configuration file.  The following definitions
% are fairly evil, but should do the job OK.
%
% \begin{macro}{\@gobbledef}
%
% This macro eats the following |\def|inition, leaving not a trace behind.
%
%    \begin{macrocode}
\def\@gobbledef#1#{\@gobble}
%    \end{macrocode}
%
% \end{macro}
%
% \begin{macro}{\tdef}
% \begin{macro}{\tlet}
%
% The |\tdef| command is a sort of `tentative' definition -- it's like
% |\def| if the control sequence named doesn't already have a definition.
% |\tlet| does the same thing with |\let|.
%
%    \begin{macrocode}
\def\tdef#1{
  \ifx#1\@@undefined%
    \expandafter\def\expandafter#1%
  \else%
    \expandafter\@gobbledef%
  \fi%
}
\def\tlet#1#2{\ifx#1\@@undefined\let#1=#2\fi}
%    \end{macrocode}
%
% \end{macro}
% \end{macro}
%
%
% \subsection{General markup things}
%
% Now for some really simple things.  I'll define how to typeset package
% names and environment names (both in the sans serif font, for now).
%
%    \begin{macrocode}
\tlet\package\textsf
\tlet\env\textsf
%    \end{macrocode}
%
% I'll define the |\<|\dots|>| shortcut for syntax items suggested in the
% \package{syntax} package.
%
%    \begin{macrocode}
\tdef\<#1>{\synt{#1}}
%    \end{macrocode}
%
% And because it's used in a few places (mainly for typesetting lengths),
% here's a command for typesetting fractions in text.
%
%    \begin{macrocode}
\tdef\smallf#1/#2{\ensuremath{^{#1}\!/\!_{#2}}}
%    \end{macrocode}
%
%
% \subsection{A table environment}
%
% \begin{environment}{tab}
%
% Most of the packages don't use the (obviously perfect) \package{mdwtab}
% package, because it's big, and takes a while to load.  Here's an
% environment for typesetting centred tables.  The first (optional) argument
% is some declarations to perform.  The mandatory argument is the table
% preamble (obviously).
%
%    \begin{macrocode}
\@ifundefined{tab}{%
  \newenvironment{tab}[2][\relax]{%
    \par\vskip2ex%
    \centering%
    #1%
    \begin{tabular}{#2}%
  }{%
    \end{tabular}%
    \par\vskip2ex%
  }
}{}
%    \end{macrocode}
%
% \end{environment}
%
%
% \subsection{Commenting out of stuff}
%
% \begin{environment}{meta-comment}
%
% Using |\iffalse|\dots|\fi| isn't much fun.  I'll define a gobbling
% environment using the \package{sverb} stuff.
%
%    \begin{macrocode}
\ignoreenv{meta-comment}
%    \end{macrocode}
%
% \end{environment}
%
%
% \subsection{Float handling}
%
% This gubbins will try to avoid float pages as much as possible, and (with
% any luck) encourage floats to be put on the same pages as text.
%
%    \begin{macrocode}
\def\textfraction{0.1}
\def\topfraction{0.9}
\def\bottomfraction{0.9}
\def\floatpagefraction{0.7}
%    \end{macrocode}
%
% Now redefine the default float-placement parameters to allow `here' floats.
%
%    \begin{macrocode}
\def\fps@figure{htbp}
\def\fps@table{htbp}
%    \end{macrocode}
%
%
% \subsection{Other bits of parameter tweaking}
%
% Make \env{grammar} environments look pretty, by indenting the left hand
% sides by a large amount.
%
%    \begin{macrocode}
\grammarindent1in
%    \end{macrocode}
%
% I don't like being told by \TeX\ that my paragraphs are hard to linebreak:
% I know this already.  This lot should shut \TeX\ up about most problems.
%
%    \begin{macrocode}
\sloppy
\hbadness\@M
\hfuzz10\p@
%    \end{macrocode}
%
% Also make \TeX\ shut up in the index.  The \package{multicol} package
% irritatingly plays with |\hbadness|.  This is the best hook I could find
% for playing with this setting.
%
%    \begin{macrocode}
\expandafter\def\expandafter\IndexParms\expandafter{%
  \IndexParms%
  \hbadness\@M%
}
%    \end{macrocode}
%
% The other thing I really don't like is `Marginpar moved' warnings.  This
% will get rid of them, and lots of other \LaTeX\ warnings at the same time.
%
%    \begin{macrocode}
\let\@latex@warning@no@line\@gobble
%    \end{macrocode}
%
% Put some extra space between table rows, please.
%
%    \begin{macrocode}
\def\arraystretch{1.2}
%    \end{macrocode}
%
% Most of the code is at guard level one, so typeset that in upright text.
%
%    \begin{macrocode}
\setcounter{StandardModuleDepth}{1}
%    \end{macrocode}
%
%
% \subsection{Contents handling}
%
% I use at least one contents file (the main table of contents) although
% I may want more.  I'll keep a list of contents files which I need to
% handle.
%
% There are two things I need to do to contents files here:
% \begin{itemize}
% \item I must typeset the table of contents at the beginning of the
%       document; and
% \item I want to typeset tables of contents in two columns (using the
%       \package{multicol} package).
% \end{itemize}
%
% The list consists of items of the form
% \syntax{"\\do{"<extension>"}{"<command>"}"}, where \<extension> is the
% file extension of the contents file, and \<command> is the command to
% typeset it.
%
% \begin{macro}{\docontents}
%
% This is where I keep the list of contents files.  I'll initialise it to
% just do the standard contents table.
%
%    \begin{macrocode}
\def\docontents{\do{toc}{\tableofcontents}}
%    \end{macrocode}
%
% \end{macro}
%
% \begin{macro}{\addcontents}
%
% By saying \syntax{"\\addcontents{"<extension>"}{"<command>"}"}, a document
% can register a new table of contents which gets given the two-column
% treatment properly.  This is really easy to implement.
%
%    \begin{macrocode}
\def\addcontents#1#2{%
  \toks@\expandafter{\docontents\do{#1}{#2}}%
  \edef\docontents{\the\toks@}%
}
%    \end{macrocode}
%
% \end{macro}
%
%
% \subsection{Finishing it all off}
%
% \begin{macro}{\finalstuff}
%
% The |\finalstuff| macro is a hook for doing things at the end of the
% document.  Currently, it inputs the licence agreement as an appendix.
%
%    \begin{macrocode}
\tdef\finalstuff{\appendix\part*{Appendix}% \iffalse <meta-comment>
%
% $Id: gpl.tex,v 1.1 2000/07/13 09:10:20 michael Exp $
%
% The GNU General Public Licence as a LaTeX section
%
% (c) 1989, 1991 Free Software Foundation, Inc.
%   LaTeX markup and minor formatting changes by Mark Wooding
%

%----- Revision history -----------------------------------------------------
%
% $Log: gpl.tex,v $
% Revision 1.1  2000/07/13 09:10:20  michael
% + Initial import
%
% Revision 1.1  1998/09/21 10:19:01  michael
% Initial implementation
%
% Revision 1.1  1996/11/19 20:51:14  mdw
% Initial revision
%

% --- Chapter heading ---
%
% We don't know whether this ought to be a section or a chapter.  Easy.
% We'll see if chapters are possible.
%
% \fi

\begingroup
\makeatletter

\edef\next#1#2#3{\relax
  \ifx\chapter\@@undefined
    \ifx\documentclass\@notprerr#2\else#3\fi
  \else#1\fi
}

\expandafter\endgroup\next
{
  \let\gpltoplevel\chapter
  \let\gplsec\section
  \let\gplend\endinput
}{
  \let\gpltoplevel\section
  \let\gplsec\subsection
  \let\gplend\endinput
}{
  \documentclass[a4paper]{article}
  \def\gpltoplevel#1{%
    \vspace*{1in}%
    \hbox to\hsize{\hfil\LARGE\bfseries#1\hfil}%
    \vspace{1in}%
  }
  \let\gplsec\section
  \def\gplend{\end{document}}
  \advance\textwidth1in
  \advance\oddsidemargin-.5in
  \sloppy
  \begin{document}
}

%^^A-------------------------------------------------------------------------
\gpltoplevel{The GNU General Public Licence}


The following is the text of the GNU General Public Licence, under the terms
of which this software is distrubuted.

\vspace{12pt}

\begin{center}
\textbf{GNU GENERAL PUBLIC LICENSE} \\
Version 2, June 1991
\end{center}

\begin{center}
Copyright (C) 1989, 1991 Free Software Foundation, Inc. \\
675 Mass Ave, Cambridge, MA 02139, USA

Everyone is permitted to copy and distribute verbatim copies \\
of this license document, but changing it is not allowed.
\end{center}


\gplsec{Preamble}

The licenses for most software are designed to take away your freedom to
share and change it.  By contrast, the GNU General Public License is intended
to guarantee your freedom to share and change free software---to make sure
the software is free for all its users.  This General Public License applies
to most of the Free Software Foundation's software and to any other program
whose authors commit to using it.  (Some other Free Software Foundation
software is covered by the GNU Library General Public License instead.)  You
can apply it to your programs, too.

When we speak of free software, we are referring to freedom, not price.  Our
General Public Licenses are designed to make sure that you have the freedom
to distribute copies of free software (and charge for this service if you
wish), that you receive source code or can get it if you want it, that you
can change the software or use pieces of it in new free programs; and that
you know you can do these things.

To protect your rights, we need to make restrictions that forbid anyone to
deny you these rights or to ask you to surrender the rights.  These
restrictions translate to certain responsibilities for you if you distribute
copies of the software, or if you modify it.

For example, if you distribute copies of such a program, whether gratis or
for a fee, you must give the recipients all the rights that you have.  You
must make sure that they, too, receive or can get the source code.  And you
must show them these terms so they know their rights.

We protect your rights with two steps: (1) copyright the software, and (2)
offer you this license which gives you legal permission to copy, distribute
and/or modify the software.

Also, for each author's protection and ours, we want to make certain that
everyone understands that there is no warranty for this free software.  If
the software is modified by someone else and passed on, we want its
recipients to know that what they have is not the original, so that any
problems introduced by others will not reflect on the original authors'
reputations.

Finally, any free program is threatened constantly by software patents.  We
wish to avoid the danger that redistributors of a free program will
individually obtain patent licenses, in effect making the program
proprietary.  To prevent this, we have made it clear that any patent must be
licensed for everyone's free use or not licensed at all.

The precise terms and conditions for copying, distribution and modification
follow.


\gplsec{Terms and conditions for copying, distribution and modification}

\begin{enumerate}

\makeatletter \setcounter{\@listctr}{-1} \makeatother

\item [0.] This License applies to any program or other work which contains a
      notice placed by the copyright holder saying it may be distributed
      under the terms of this General Public License.  The ``Program'',
      below, refers to any such program or work, and a ``work based on the
      Program'' means either the Program or any derivative work under
      copyright law: that is to say, a work containing the Program or a
      portion of it, either verbatim or with modifications and/or translated
      into another language.  (Hereinafter, translation is included without
      limitation in the term ``modification''.)  Each licensee is addressed
      as ``you''.

      Activities other than copying, distribution and modification are not
      covered by this License; they are outside its scope.  The act of
      running the Program is not restricted, and the output from the Program
      is covered only if its contents constitute a work based on the Program
      (independent of having been made by running the Program).  Whether that
      is true depends on what the Program does.

\item [1.] You may copy and distribute verbatim copies of the Program's
      source code as you receive it, in any medium, provided that you
      conspicuously and appropriately publish on each copy an appropriate
      copyright notice and disclaimer of warranty; keep intact all the
      notices that refer to this License and to the absence of any warranty;
      and give any other recipients of the Program a copy of this License
      along with the Program.

      You may charge a fee for the physical act of transferring a copy, and
      you may at your option offer warranty protection in exchange for a fee.

\item [2.] You may modify your copy or copies of the Program or any portion
      of it, thus forming a work based on the Program, and copy and
      distribute such modifications or work under the terms of Section 1
      above, provided that you also meet all of these conditions:

      \begin{enumerate}

      \item [(a)] You must cause the modified files to carry prominent
            notices stating that you changed the files and the date of any
            change.

      \item [(b)] You must cause any work that you distribute or publish,
            that in whole or in part contains or is derived from the Program
            or any part thereof, to be licensed as a whole at no charge to
            all third parties under the terms of this License.

      \item [(c)] If the modified program normally reads commands
            interactively when run, you must cause it, when started running
            for such interactive use in the most ordinary way, to print or
            display an announcement including an appropriate copyright notice
            and a notice that there is no warranty (or else, saying that you
            provide a warranty) and that users may redistribute the program
            under these conditions, and telling the user how to view a copy
            of this License.  (Exception: if the Program itself is
            interactive but does not normally print such an announcement,
            your work based on the Program is not required to print an
            announcement.)

      \end{enumerate}

      These requirements apply to the modified work as a whole.  If
      identifiable sections of that work are not derived from the Program,
      and can be reasonably considered independent and separate works in
      themselves, then this License, and its terms, do not apply to those
      sections when you distribute them as separate works.  But when you
      distribute the same sections as part of a whole which is a work based
      on the Program, the distribution of the whole must be on the terms of
      this License, whose permissions for other licensees extend to the
      entire whole, and thus to each and every part regardless of who wrote
      it.

      Thus, it is not the intent of this section to claim rights or contest
      your rights to work written entirely by you; rather, the intent is to
      exercise the right to control the distribution of derivative or
      collective works based on the Program.

      In addition, mere aggregation of another work not based on the Program
      with the Program (or with a work based on the Program) on a volume of a
      storage or distribution medium does not bring the other work under the
      scope of this License.

\item [3.] You may copy and distribute the Program (or a work based on it,
      under Section 2) in object code or executable form under the terms of
      Sections 1 and 2 above provided that you also do one of the following:

      \begin{enumerate}

      \item [(a)] Accompany it with the complete corresponding
            machine-readable source code, which must be distributed under the
            terms of Sections 1 and 2 above on a medium customarily used for
            software interchange; or,

      \item [(b)] Accompany it with a written offer, valid for at least three
            years, to give any third party, for a charge no more than your
            cost of physically performing source distribution, a complete
            machine-readable copy of the corresponding source code, to be
            distributed under the terms of Sections 1 and 2 above on a medium
            customarily used for software interchange; or,

      \item [(c)] Accompany it with the information you received as to the
            offer to distribute corresponding source code.  (This alternative
            is allowed only for noncommercial distribution and only if you
            received the program in object code or executable form with such
            an offer, in accord with Subsection b above.)

      \end{enumerate}

      The source code for a work means the preferred form of the work for
      making modifications to it.  For an executable work, complete source
      code means all the source code for all modules it contains, plus any
      associated interface definition files, plus the scripts used to control
      compilation and installation of the executable.  However, as a special
      exception, the source code distributed need not include anything that
      is normally distributed (in either source or binary form) with the
      major components (compiler, kernel, and so on) of the operating system
      on which the executable runs, unless that component itself accompanies
      the executable.

      If distribution of executable or object code is made by offering access
      to copy from a designated place, then offering equivalent access to
      copy the source code from the same place counts as distribution of the
      source code, even though third parties are not compelled to copy the
      source along with the object code.

\item [4.] You may not copy, modify, sublicense, or distribute the Program
      except as expressly provided under this License.  Any attempt otherwise
      to copy, modify, sublicense or distribute the Program is void, and will
      automatically terminate your rights under this License.  However,
      parties who have received copies, or rights, from you under this
      License will not have their licenses terminated so long as such parties
      remain in full compliance.

\item [5.] You are not required to accept this License, since you have not
      signed it.  However, nothing else grants you permission to modify or
      distribute the Program or its derivative works.  These actions are
      prohibited by law if you do not accept this License.  Therefore, by
      modifying or distributing the Program (or any work based on the
      Program), you indicate your acceptance of this License to do so, and
      all its terms and conditions for copying, distributing or modifying the
      Program or works based on it.

\item [6.] Each time you redistribute the Program (or any work based on the
      Program), the recipient automatically receives a license from the
      original licensor to copy, distribute or modify the Program subject to
      these terms and conditions.  You may not impose any further
      restrictions on the recipients' exercise of the rights granted herein.
      You are not responsible for enforcing compliance by third parties to
      this License.

\item [7.] If, as a consequence of a court judgment or allegation of patent
      infringement or for any other reason (not limited to patent issues),
      conditions are imposed on you (whether by court order, agreement or
      otherwise) that contradict the conditions of this License, they do not
      excuse you from the conditions of this License.  If you cannot
      distribute so as to satisfy simultaneously your obligations under this
      License and any other pertinent obligations, then as a consequence you
      may not distribute the Program at all.  For example, if a patent
      license would not permit royalty-free redistribution of the Program by
      all those who receive copies directly or indirectly through you, then
      the only way you could satisfy both it and this License would be to
      refrain entirely from distribution of the Program.

      If any portion of this section is held invalid or unenforceable under
      any particular circumstance, the balance of the section is intended to
      apply and the section as a whole is intended to apply in other
      circumstances.

      It is not the purpose of this section to induce you to infringe any
      patents or other property right claims or to contest validity of any
      such claims; this section has the sole purpose of protecting the
      integrity of the free software distribution system, which is
      implemented by public license practices.  Many people have made
      generous contributions to the wide range of software distributed
      through that system in reliance on consistent application of that
      system; it is up to the author/donor to decide if he or she is willing
      to distribute software through any other system and a licensee cannot
      impose that choice.

      This section is intended to make thoroughly clear what is believed to
      be a consequence of the rest of this License.

\item [8.] If the distribution and/or use of the Program is restricted in
      certain countries either by patents or by copyrighted interfaces, the
      original copyright holder who places the Program under this License may
      add an explicit geographical distribution limitation excluding those
      countries, so that distribution is permitted only in or among countries
      not thus excluded.  In such case, this License incorporates the
      limitation as if written in the body of this License.

\item [9.] The Free Software Foundation may publish revised and/or new
      versions of the General Public License from time to time.  Such new
      versions will be similar in spirit to the present version, but may
      differ in detail to address new problems or concerns.

      Each version is given a distinguishing version number.  If the Program
      specifies a version number of this License which applies to it and
      ``any later version'', you have the option of following the terms and
      conditions either of that version or of any later version published by
      the Free Software Foundation.  If the Program does not specify a
      version number of this License, you may choose any version ever
      published by the Free Software Foundation.

\item [10.] If you wish to incorporate parts of the Program into other free
      programs whose distribution conditions are different, write to the
      author to ask for permission.  For software which is copyrighted by the
      Free Software Foundation, write to the Free Software Foundation; we
      sometimes make exceptions for this.  Our decision will be guided by the
      two goals of preserving the free status of all derivatives of our free
      software and of promoting the sharing and reuse of software generally.

\begin{center}
NO WARRANTY
\end{center}

\bfseries

\item [11.] Because the Program is licensed free of charge, there is no
      warranty for the Program, to the extent permitted by applicable law.
      except when otherwise stated in writing the copyright holders and/or
      other parties provide the program ``as is'' without warranty of any
      kind, either expressed or implied, including, but not limited to, the
      implied warranties of merchantability and fitness for a particular
      purpose.  The entire risk as to the quality and performance of the
      Program is with you.  Should the Program prove defective, you assume
      the cost of all necessary servicing, repair or correction.

\item [12.] In no event unless required by applicable law or agreed to in
      writing will any copyright holder, or any other party who may modify
      and/or redistribute the program as permitted above, be liable to you
      for damages, including any general, special, incidental or
      consequential damages arising out of the use or inability to use the
      program (including but not limited to loss of data or data being
      rendered inaccurate or losses sustained by you or third parties or a
      failure of the Program to operate with any other programs), even if
      such holder or other party has been advised of the possibility of such
      damages.

\end{enumerate}

\begin{center}
\textbf{END OF TERMS AND CONDITIONS}
\end{center}


\gplsec{Appendix: How to Apply These Terms to Your New Programs}

If you develop a new program, and you want it to be of the greatest possible
use to the public, the best way to achieve this is to make it free software
which everyone can redistribute and change under these terms.

To do so, attach the following notices to the program.  It is safest to
attach them to the start of each source file to most effectively convey the
exclusion of warranty; and each file should have at least the ``copyright''
line and a pointer to where the full notice is found.

\begin{verbatim}
<one line to give the program's name and a brief idea of what it does.>
Copyright (C) 19yy  <name of author>

This program is free software; you can redistribute it and/or modify
it under the terms of the GNU General Public License as published by
the Free Software Foundation; either version 2 of the License, or
(at your option) any later version.

This program is distributed in the hope that it will be useful,
but WITHOUT ANY WARRANTY; without even the implied warranty of
MERCHANTABILITY or FITNESS FOR A PARTICULAR PURPOSE.  See the
GNU General Public License for more details.

You should have received a copy of the GNU General Public License
along with this program; if not, write to the Free Software
Foundation, Inc., 675 Mass Ave, Cambridge, MA 02139, USA.
\end{verbatim}

Also add information on how to contact you by electronic and paper mail.

If the program is interactive, make it output a short notice like this when
it starts in an interactive mode:

\begin{verbatim}
Gnomovision version 69, Copyright (C) 19yy name of author
Gnomovision comes with ABSOLUTELY NO WARRANTY; for details type `show w'.
This is free software, and you are welcome to redistribute it
under certain conditions; type `show c' for details.
\end{verbatim}

The hypothetical commands `show w' and `show c' should show the appropriate
parts of the General Public License.  Of course, the commands you use may be
called something other than `show w' and `show c'; they could even be
mouse-clicks or menu items--whatever suits your program.

You should also get your employer (if you work as a programmer) or your
school, if any, to sign a ``copyright disclaimer'' for the program, if
necessary.  Here is a sample; alter the names:

\begin{verbatim}
Yoyodyne, Inc., hereby disclaims all copyright interest in the program
`Gnomovision' (which makes passes at compilers) written by James Hacker.

<signature of Ty Coon>, 1 April 1989
Ty Coon, President of Vice
\end{verbatim}

This General Public License does not permit incorporating your program into
proprietary programs.  If your program is a subroutine library, you may
consider it more useful to permit linking proprietary applications with the
library.  If this is what you want to do, use the GNU Library General Public
License instead of this License.

\gplend
}
%    \end{macrocode}
%
% \end{macro}
%
% \begin{macro}{\implementation}
%
% The |\implementation| macro starts typesetting the implementation of
% the package(s).  If we're not doing the implementation, it just does
% this lot and ends the input file.
%
% I define a macro with arguments inside the |\StopEventually|, which causes
% problems, since the code gets put through an extra level of |\def|fing
% depending on whether the implementation stuff gets typeset or not.  I'll
% store the code I want to do in a separate macro.
%
%    \begin{macrocode}
\def\implementation{\StopEventually{\attheend}}
%    \end{macrocode}
%
% Now for the actual activity.  First, I'll do the |\finalstuff|.  Then, if
% \package{doc}'s managed to find the \package{multicol} package, I'll add
% the end of the environment to the end of each contents file in the list.
% Finally, I'll read the index in from its formatted |.ind| file.
%
%    \begin{macrocode}
\tdef\attheend{%
  \finalstuff%
  \ifhave@multicol%
    \def\do##1##2{\addtocontents{##1}{\protect\end{multicols}}}%
    \docontents%
  \fi%
  \PrintIndex%
}
%    \end{macrocode}
%
% \end{macro}
%
%
% \subsection{File version information}
%
% \begin{macro}{\mdwpkginfo}
%
% For setting up the automatic titles, I'll need to be able to work out
% file versions and things.  This macro will, given a file name, extract
% from \LaTeX\ the version information and format it into a sensible string.
%
% First of all, I'll put the original string (direct from the
% |\Provides|\dots\ command).  Then I'll pass it to another macro which can
% parse up the string into its various bits, along with the original
% filename.
%
%    \begin{macrocode}
\def\mdwpkginfo#1{%
  \edef\@tempa{\csname ver@#1\endcsname}%
  \expandafter\mdwpkginfo@i\@tempa\@@#1\@@%
}
%    \end{macrocode}
%
% Now for the real business.  I'll store the string I build in macros called
% \syntax{"\\"<filename>"date", "\\"<filename>"version" and
% "\\"<filename>"info"}, which store the file's date, version and
% `information string' respectively.  (Note that the file extension isn't
% included in the name.)
%
% This is mainly just tedious playing with |\expandafter|.  The date format
% is defined by a separate macro, which can be modified from the
% configuration file.
%
%    \begin{macrocode}
\def\mdwpkginfo@i#1/#2/#3 #4 #5\@@#6.#7\@@{%
  \expandafter\def\csname #6date\endcsname%
    {\protect\mdwdateformat{#1}{#2}{#3}}%
  \expandafter\def\csname #6version\endcsname{#4}%
  \expandafter\def\csname #6info\endcsname{#5}%
}
%    \end{macrocode}
%
% \end{macro}
%
% \begin{macro}{\mdwdateformat}
%
% Given three arguments, a year, a month and a date (all numeric), build a
% pretty date string.  This is fairly simple really.
%
%    \begin{macrocode}
\tdef\mdwdateformat#1#2#3{\number#3\ \monthname{#2}\ \number#1}
\def\monthname#1{%
  \ifcase#1\or%
     January\or February\or March\or April\or May\or June\or%
     July\or August\or September\or October\or November\or December%
  \fi%
}
\def\numsuffix#1{%
  \ifnum#1=1 st\else%
  \ifnum#1=2 nd\else%
  \ifnum#1=3 rd\else%
  \ifnum#1=21 st\else%
  \ifnum#1=22 nd\else%
  \ifnum#1=23 rd\else%
  \ifnum#1=31 st\else%
  th%
  \fi\fi\fi\fi\fi\fi\fi%
}
%    \end{macrocode}
%
% \end{macro}
%
% \begin{macro}{\mdwfileinfo}
%
% Saying \syntax{"\\mdwfileinfo{"<file-name>"}{"<info>"}"} extracts the
% wanted item of \<info> from the version information for file \<file-name>.
%
%    \begin{macrocode}
\def\mdwfileinfo#1#2{\mdwfileinfo@i{#2}#1.\@@}
\def\mdwfileinfo@i#1#2.#3\@@{\csname#2#1\endcsname}
%    \end{macrocode}
%
% \end{macro}
%
%
% \subsection{List handling}
%
% There are several other lists I need to build.  These macros will do
% the necessary stuff.
%
% \begin{macro}{\mdw@ifitem}
%
% The macro \syntax{"\\mdw@ifitem"<item>"\\in"<list>"{"<true-text>"}"^^A
% "{"<false-text>"}"} does \<true-text> if the \<item> matches any item in
% the \<list>; otherwise it does \<false-text>.
%
%    \begin{macrocode}
\def\mdw@ifitem#1\in#2{%
  \@tempswafalse%
  \def\@tempa{#1}%
  \def\do##1{\def\@tempb{##1}\ifx\@tempa\@tempb\@tempswatrue\fi}%
  #2%
  \if@tempswa\expandafter\@firstoftwo\else\expandafter\@secondoftwo\fi%
}
%    \end{macrocode}
%
% \end{macro}
%
% \begin{macro}{\mdw@append}
%
% Saying \syntax{"\\mdw@append"<item>"\\to"<list>} adds the given \<item>
% to the end of the given \<list>.
%
%    \begin{macrocode}
\def\mdw@append#1\to#2{%
  \toks@{\do{#1}}%
  \toks\tw@\expandafter{#2}%
  \edef#2{\the\toks\tw@\the\toks@}%
}
%    \end{macrocode}
%
% \end{macro}
%
% \begin{macro}{\mdw@prepend}
%
% Saying \syntax{"\\mdw@prepend"<item>"\\to"<list>} adds the \<item> to the
% beginning of the \<list>.
%
%    \begin{macrocode}
\def\mdw@prepend#1\to#2{%
  \toks@{\do{#1}}%
  \toks\tw@\expandafter{#2}%
  \edef#2{\the\toks@\the\toks\tw@}%
}
%    \end{macrocode}
%
% \end{macro}
%
% \begin{macro}{\mdw@add}
%
% Finally, saying \syntax{"\\mdw@add"<item>"\\to"<list>} adds the \<item>
% to the list only if it isn't there already.
%
%    \begin{macrocode}
\def\mdw@add#1\to#2{\mdw@ifitem#1\in#2{}{\mdw@append#1\to#2}}
%    \end{macrocode}
%
% \end{macro}
%
%
% \subsection{Described file handling}
%
% I'l maintain lists of packages, document classes, and other files
% described by the current documentation file.
%
% First of all, I'll declare the various list macros.
%
%    \begin{macrocode}
\def\dopackages{}
\def\doclasses{}
\def\dootherfiles{}
%    \end{macrocode}
%
% \begin{macro}{\describespackage}
%
% A document file can declare that it describes a package by saying
% \syntax{"\\describespackage{"<package-name>"}"}.  I add the package to
% my list, read the package into memory (so that the documentation can
% offer demonstrations of it) and read the version information.
%
%    \begin{macrocode}
\def\describespackage#1{%
  \mdw@ifitem#1\in\dopackages{}{%
    \mdw@append#1\to\dopackages%
    \usepackage{#1}%
    \mdwpkginfo{#1.sty}%
  }%
}
%    \end{macrocode}
%
% \end{macro}
%
% \begin{macro}{\describesclass}
%
% By saying \syntax{"\\describesclass{"<class-name>"}"}, a document file
% can declare that it describes a document class.  I'll assume that the
% document class is already loaded, because it's much too late to load
% it now.
%
%    \begin{macrocode}
\def\describesclass#1{\mdw@add#1\to\doclasses\mdwpkginfo{#1.cls}}
%    \end{macrocode}
%
% \end{macro}
%
% \begin{macro}{\describesfile}
%
% Finally, other `random' files, which don't have the status of real \LaTeX\
% packages or document classes, can be described by saying \syntax{^^A
% "\\describesfile{"<file-name>"}" or "\\describesfile*{"<file-name>"}"}.
% The difference is that the starred version will not |\input| the file.
%
%    \begin{macrocode}
\def\describesfile{%
  \@ifstar{\describesfile@i\@gobble}{\describesfile@i\input}%
}
\def\describesfile@i#1#2{%
  \mdw@ifitem#2\in\dootherfiles{}{%
    \mdw@add#2\to\dootherfiles%
    #1{#2}%
    \mdwpkginfo{#2}%
  }%
}
%    \end{macrocode}
%
% \end{macro}
%
%
% \subsection{Author and title handling}
%
% I'll redefine the |\author| and |\title| commands so that I get told
% whether I need to do it myself.
%
% \begin{macro}{\author}
%
% This is easy: I'll save the old meaning, and then redefine |\author| to
% do the old thing and redefine itself to then do nothing.
%
%    \begin{macrocode}
\let\mdw@author\author
\def\author{\let\author\@gobble\mdw@author}
%    \end{macrocode}
%
% \end{macro}
%
% \begin{macro}{\title}
%
% And oddly enough, I'll do exactly the same thing for the title, except
% that I'll also disable the |\mdw@buildtitle| command, which constructs
% the title automatically.
%
%    \begin{macrocode}
\let\mdw@title\title
\def\title{\let\title\@gobble\let\mdw@buildtitle\relax\mdw@title}
%    \end{macrocode}
%
% \end{macro}
%
% \begin{macro}{\date}
%
% This works in a very similar sort of way.
%
%    \begin{macrocode}
\def\date#1{\let\date\@gobble\def\today{#1}}
%    \end{macrocode}
%
% \end{macro}
%
% \begin{macro}{\datefrom}
%
% Saying \syntax{"\\datefrom{"<file-name>"}"} sets the document date from
% the given filename.
%
%    \begin{macrocode}
\def\datefrom#1{%
  \protected@edef\@tempa{\noexpand\date{\csname #1date\endcsname}}%
  \@tempa%
}
%    \end{macrocode}
%
% \end{macro}
%
% \begin{macro}{\docfile}
%
% Saying \syntax{"\\docfile{"<file-name>"}"} sets up the file name from which
% documentation will be read.
%
%    \begin{macrocode}
\def\docfile#1{%
  \def\@tempa##1.##2\@@{\def\@basefile{##1.##2}\def\@basename{##1}}%
  \edef\@tempb{\noexpand\@tempa#1\noexpand\@@}%
  \@tempb%
}
%    \end{macrocode}
%
% I'll set up a default value as well.
%
%    \begin{macrocode}
\docfile{\jobname.dtx}
%    \end{macrocode}
%
% \end{macro}
%
%
% \subsection{Building title strings}
%
% This is rather tricky.  For each list, I need to build a legible looking
% string.
%
% \begin{macro}{\mdw@addtotitle}
%
% By saying
%\syntax{"\\mdw@addtotitle{"<list>"}{"<command>"}{"<singular>"}{"<plural>"}"}
% I can add the contents of a list to the current title string in the
% |\mdw@title| macro.
%
%    \begin{macrocode}
\tdef\mdw@addtotitle#1#2#3#4{%
%    \end{macrocode}
%
% Now to get to work.  I need to keep one `lookahead' list item, and a count
% of the number of items read so far.  I'll keep the lookahead item in
% |\@nextitem| and the counter in |\count@|.
%
%    \begin{macrocode}
  \count@\z@%
%    \end{macrocode}
%
% Now I'll define what to do for each list item.  The |\protect| command is
% already set up appropriately for playing with |\edef| commands.
%
%    \begin{macrocode}
  \def\do##1{%
%    \end{macrocode}
%
% The first job is to add the previous item to the title string.  If this
% is the first item, though, I'll just add the appropriate \lit{The } or
% \lit{ and the } string to the title (this is stored in the |\@prefix|
% macro).
%
%    \begin{macrocode}
    \edef\mdw@title{%
      \mdw@title%
      \ifcase\count@\@prefix%
      \or\@nextitem%
      \else, \@nextitem%
      \fi%
    }%
%    \end{macrocode}
%
% That was rather easy.  Now I'll set up the |\@nextitem| macro for the
% next time around the loop.
%
%    \begin{macrocode}
    \edef\@nextitem{%
      \protect#2{##1}%
      \protect\footnote{%
        The \protect#2{##1} #3 is currently at version %
        \mdwfileinfo{##1}{version}, dated \mdwfileinfo{##1}{date}.%
      }\space%
    }%
%    \end{macrocode}
%
% Finally, I need to increment the counter.
%
%    \begin{macrocode}
    \advance\count@\@ne%
  }%
%    \end{macrocode}
%
% Now execute the list.
%
%    \begin{macrocode}
  #1%
%    \end{macrocode}
%
% I still have one item left over, unless the list was empty.  I'll add
% that now.
%
%    \begin{macrocode}
  \edef\mdw@title{%
    \mdw@title%
    \ifcase\count@%
    \or\@nextitem\space#3%
    \or\ and \@nextitem\space#4%
    \fi%
  }%
%    \end{macrocode}
%
% Finally, if $|\count@| \ne 0$, I must set |\@prefix| to \lit{ and the }.
%
%    \begin{macrocode}
  \ifnum\count@>\z@\def\@prefix{ and the }\fi%
}
%    \end{macrocode}
%
% \end{macro}
%
% \begin{macro}{\mdw@buildtitle}
%
% This macro will actually do the job of building the title string.
%
%    \begin{macrocode}
\tdef\mdw@buildtitle{%
%    \end{macrocode}
%
% First of all, I'll open a group to avoid polluting the namespace with
% my gubbins (although the code is now much tidier than it has been in
% earlier releases).
%
%    \begin{macrocode}
  \begingroup%
%    \end{macrocode}
%
% The title building stuff makes extensive use of |\edef|.  I'll set
% |\protect| appropriately.  (For those not in the know,
% |\@unexpandable@protect| expands to `|\noexpand\protect\noexpand|',
% which prevents expansion of the following macro, and inserts a |\protect|
% in front of it ready for the next |\edef|.)
%
%    \begin{macrocode}
  \let\@@protect\protect\let\protect\@unexpandable@protect%
%    \end{macrocode}
%
% Set up some simple macros ready for the main code.
%
%    \begin{macrocode}
  \def\mdw@title{}%
  \def\@prefix{The }%
%    \end{macrocode}
%
% Now build the title.  This is fun.
%
%    \begin{macrocode}
  \mdw@addtotitle\dopackages\package{package}{packages}%
  \mdw@addtotitle\doclasses\package{document class}{document classes}%
  \mdw@addtotitle\dootherfiles\texttt{file}{files}%
%    \end{macrocode}
%
% Now I want to end the group and set the title from my string.  The
% following hacking will do this.
%
%    \begin{macrocode}
  \edef\next{\endgroup\noexpand\title{\mdw@title}}%
  \next%
}
%    \end{macrocode}
%
% \end{macro}
%
%
% \subsection{Starting the main document}
%
% \begin{macro}{\mdwdoc}
%
% Once the document preamble has done all of its stuff, it calls the
% |\mdwdoc| command, which takes over and really starts the documentation
% going.
%
%    \begin{macrocode}
\def\mdwdoc{%
%    \end{macrocode}
%
% First, I'll construct the title string.
%
%    \begin{macrocode}
  \mdw@buildtitle%
  \author{Mark Wooding}%
%    \end{macrocode}
%
% Set up the date string based on the date of the package which shares
% the same name as the current file.
%
%    \begin{macrocode}
  \datefrom\@basename%
%    \end{macrocode}
%
% Set up verbatim characters after all the packages have started.
%
%    \begin{macrocode}
  \shortverb\|%
  \shortverb\"%
%    \end{macrocode}
%
% Start the document, and put the title in.
%
%    \begin{macrocode}
  \begin{document}
  \maketitle%
%    \end{macrocode}
%
% This is nasty.  It makes maths displays work properly in demo environments.
% \emph{The \LaTeX\ Companion} exhibits the bug which this hack fixes.  So
% ner.
%
%    \begin{macrocode}
  \abovedisplayskip\z@%
%    \end{macrocode}
%
% Now start the contents tables.  After starting each one, I'll make it
% be multicolumnar.
%
%    \begin{macrocode}
  \def\do##1##2{%
    ##2%
    \ifhave@multicol\addtocontents{##1}{%
      \protect\begin{multicols}{2}%
      \hbadness\@M%
    }\fi%
  }%
  \docontents%
%    \end{macrocode}
%
% Input the main file now.
%
%    \begin{macrocode}
  \DocInput{\@basefile}%
%    \end{macrocode}
%
% That's it.  I'm done.
%
%    \begin{macrocode}
  \end{document}
}
%    \end{macrocode}
%
% \end{macro}
%
%
% \subsection{And finally\dots}
%
% Right at the end I'll put a hook for the configuration file.
%
%    \begin{macrocode}
\ifx\mdwhook\@@undefined\else\expandafter\mdwhook\fi
%    \end{macrocode}
%
% That's all the code done now.  I'll change back to `user' mode, where
% all the magic control sequences aren't allowed any more.
%
%    \begin{macrocode}
\makeatother
%</mdwtools>
%    \end{macrocode}
%
% Oh, wait!  What if I want to typeset this documentation?  Aha.  I'll cope
% with that by comparing |\jobname| with my filename |mdwtools|.  However,
% there's some fun here, because |\jobname| contains category-12 letters,
% while my letters are category-11.  Time to play with |\string| in a messy
% way.
%
%    \begin{macrocode}
%<*driver>
\makeatletter
\edef\@tempa{\expandafter\@gobble\string\mdwtools}
\edef\@tempb{\jobname}
\ifx\@tempa\@tempb
  \describesfile*{mdwtools.tex}
  \docfile{mdwtools.tex}
  \makeatother
  \expandafter\mdwdoc
\fi
\makeatother
%</driver>
%    \end{macrocode}
%
% That's it.  Done!
%
% \hfill Mark Wooding, \today
%
% \Finale
%
\endinput

\describespackage{syntax}
\DeclareRobustCommand\syn{\package{syntax}}
\mdwdoc
%</driver>
%
% \end{meta-comment}
%
% \section{User guide}
%
% \subsection{Introduction}
%
% The \syn\ package provides a number of commands and environments which
% extend \LaTeX\ and allow you to typeset good expositions of syntax.
%
% The package provides several different types of features: probably not all
% of these will be required by every document which needs the package:
% \begin{itemize}
% \item A system of abbreviated forms for typesetting syntactic items.
% \item An environment for typesetting BNF-type grammars
% \item A collection of environments for building syntax diagrams.
% \end{itemize}
%
% The package also includes some other features which, while not necessarily
% syntax-related, will probably come in handy for similar types of document:
% \begin{itemize}
% \item An abbreviated notation for verbatim text, similar to the
%       \package{shortvrb} package.
% \item A slightly different underscore character, which works as expected
%       in text and maths modes.
% \end{itemize}
%
% \subsection{The abbreviated verbatim notation}
%
% In documents describing programming languages and libraries, it can become
% tedious to type "\verb|...|" every time.  Like Frank Mittelbach's
% \package{shortvrb} package, \syn\ provides a way of setting up single-^^A
% character abbreviations.  The only real difference between the two is that
% the declarations provided by \syn\ obey \LaTeX's normal scoping rules.
%
% \DescribeMacro\shortverb
% You can set up a character as a `verbatim shorthand' character using the
% |\shortverb| command.  This takes a single argument, which should be a
% single-character control sequence containing the character you want to use.
% So, for example, the command
% \begin{listing}
%\shortverb{\|}
% \end{listing}
% would set up the `"|"' character to act as a verbatim delimiter.  While a
% |\shortverb| declaration is in force, any text surrounded by (in this case)
% vertical bar characters will be typeset as if using the normal |\verb|
% command.
%
% \DescribeEnv{shortverb}
% Since \LaTeX\ allows any declaration to be used as an environment, you can
% use a \env{shortverb} environment to delimit the text over which your
% character is active:
% \begin{listing}
%Some text...
%\begin{shortverb}{\|}
%...
%\end{shortverb}
% \end{listing}
%
% \DescribeMacro\unverb
% If you want to disable a |\shortverb| character without ending the scope
% of other declarations, you can use the |\unverb| command, passing it
% a character as a control sequence, in the same way as above.
%
% The default \TeX/\LaTeX\ underscore character is rather too short for
% use in identifiers.  For example:
%
% \begingroup \let\_=\oldus
% \begin{demo}{Old-style underscores}
%Typing long underscore-filled
%names, like big\_function\_name,
%is normally tedious.  The normal
%positioning of the underscore
%is wrong, too.
% \end{demo}
% \endgroup
%
% The \syn\ package redefines the |\_| command to draw a more attractive
% underscore character.  It also allows you to use the |_|~character
% directly to produce an underscore outside of maths mode: |_|~behaves
% as a subscript character as usual inside maths mode.
%
% \begin{demo}{New \syn\ underscores}
%You can use underscore-filled
%names, like big_function_name,
%simply and naturally.  Of
%course, subscripts still work
%normally in maths mode, e.g.,
%$x_i$.
% \end{demo}
%
% \subsection{Typesetting syntactic items}
% \begin{synshorts}
%
% The \syn\ package provides some simple commands for typesetting syntactic
% items.
%
% \DescribeMacro\synt
% Typing "\\synt{"<text>"}" typesets <text> as a \lq non-terminal',
% in italics and surrounded by angle brackets.  If you use "\\synt" a lot,
% you can use the incantation
% \begin{listing}
%\def\<#1>{\synt{#1}}
% \end{listing}
% to allow you to type "\\<"<text>">" as an alternative to 
% "\\synt{"<text>"}".
%
% \DescribeMacro\lit
% You can also display literal text, which the reader should type directly,
% using the "\\lit" command.
%
% \begin{demo}{Use of \cmd\lit}
%Type \lit{ls} to display a
%list of files.
% \end{demo}
%
% Note that the literal text appears in quotes.  To suppress the quotes,
% use the `*' variant.
%
% The "\\lit" command produces slightly better output than "\\verb" for
% running text, since the spaces are somewhat narrower.  However, "\\verb"
% allows you to type arbitrary characters, which are treated literally,
% whereas you must use commands such as "\\{" to use special characters
% within the argument to "\\lit".  Of course, you can use "\\lit" anywhere
% in the document: "\\verb" mustn't be used inside a command argument.
% \end{synshorts}
%
% \subsection{Abbreviated forms for syntactic items}
%
% It would be very tedious to require the use of commands like |\synt|
% when building syntax descriptions like BNF grammars.  It would also make
% your \LaTeX\ source hard to read.  Therefore, \syn\ provides some
% abbreviated forms which make typesetting syntax quicker and easier.
%
% Since the abbreviated forms use several characters which you may want to
% use in normal text, they aren't enabled by default.  They only work
% with special commands and environments provided by the \syn\ package.
%
% The abbreviated forms are shown in the table below:
%
% \begin{tab}[\synshorts]{ll}                                       \hline
% \bf Input        & \bf Output                                  \\ \hline
% "<some text>"    & <some text>                                 \\
% "`some text'"    & `some text'                                 \\
% "\"some text\""  & "some text"                                 \\ \hline
% \end{tab}
%
% Within one of these abbreviated forms, text is treated more-or-less
% verbatim:
% \begin{itemize}
%
% \item Any |$|, |%|, |^|, |&|, |{|, |}|, |~| or |#| characters are treated
%       literally: their normal special meanings are ignored.
%
% \item Other special characters, with the exception of |\|, are also treated
%       literally: this includes any characters made special by |\shortverb|.
%
% \end{itemize}
%
% However, the |\| character retains its meaning.  Since the brace
% characters are not recognised, most commands can't be used within
% abbreviated forms.  However, you can use special commands to type some
% of the remaining special characters:
%
% \begin{tab}[\synshorts]{ll}                                       \hline
% \bf Command   & \bf Result                                     \\ \hline
% "\\\\"        & A `\\' character                               \\
% "\\>"         & A `>' character                                \\
% "\\'"         & A `\'' character                               \\
% "\\\""        & A `"' character                                \\
% "\\\ "        & A `\ ' character (not a space)                 \\ \hline
% \end{tab}
%
% Note that |\\|, |\>|, |\"| and \verb*|\ | are only useful in a |\tt| font,
% i.e., inside |`...'| and |"..."| forms, since the characters don't exist
% in normal fonts.  The |\>|, |\"| and |\'| commands are only provided so
% you can use these characters within |<...>|, |"..."| and |`...'| forms
% respectively: in the other forms, there is no need to use the special
% command.
%
% In addition, when the above abbreviations are enabled, the character "|"
% is set to typeset a \syntax{|} symbol, which is conventionally used to
% separate alternatives in syntax descriptions.
%
% \DescribeMacro\syntax
% Normally, these abbreviated forms are enabled only within special
% environments, such as \env{grammar} and \env{syntdiag}.  To use them
% in running text, use the |\syntax| command.  The abbreviations are made
% active within the argument of the |\syntax| command.\footnote{^^A
%   The argument of the \cmd\syntax\ command may contain commands such
%   as \cmd\verb, which are normally not allowed within arguments.
% }  Note that you cannot use the |\syntax| command within the argument
% of another command.
%
% \DescribeMacro\synshorts
% \DescribeEnv{synshorts}
% You can also enable the syntax shortcuts using the |\synshorts| declaration
% or the \env{synshorts} environment.  This enables the syntax shortcuts
% until the scope of the declaration ends.
%
% \DescribeMacro\synshortsoff
% If syntax shortcuts are enabled, you can disable them using the
% |\synshortsoff| declaration.
%
% \subsection{The \env{grammar} environment}
%
% \DescribeEnv{grammar}
% For typesetting formal grammars, for example, of programming languages,
% the \syn\ package provides a \env{grammar} environment.  Within this
% environment, the abbreviated forms described above are enabled.
%
% Within the environment, separate production rules should be separated by
% blank lines.  You can use the normal |\\| command to perform line-breaking
% of a production rule.  Note that a production rule must begin with a
% nonterminal name enclosed in angle brackets (|<| \dots |>|), followed by
% whitespace, then some kind of production operator (usually `::=') and then
% some more whitespace.  You can control how this text is actually typeset,
% however.
%
% \DescribeMacro{\[[}
% \DescribeMacro{\]]}
% You can use syntax diagrams (see below) instead of a straight piece of BNF
% by enclosing it in a |\[[| \dots |\]]| pair.  Note that you can't mix 
% syntax diagrams and BNF in a production rule, and you will get something
% which looks very strange if you try.
%
% \DescribeMacro\alt
% In addition, a command |\alt| is provided for splitting long production
% rules over several lines: the |\alt| command starts a new line and places
% a \syntax{|} character slightly in the left margin.  This is useful when
% a symbol has many alternative productions.
%
% \begin{demo}[w]{The \env{grammar} environment}
%\begin{grammar}
%<statement> ::= <ident> `=' <expr>
%  \alt `for' <ident> `=' <expr> `to' <expr> `do' <statement>
%  \alt `{' <stat-list> `}'
%  \alt <empty>
%
%<stat-list> ::= <statement> `;' <stat-list> | <statement>
%\end{grammar}
% \end{demo}
%
% You can modify the appearance of grammars using three length parameters:
%
% \begin{description} \def\makelabel{\hskip\labelsep\cmd}
%
% \item [\grammarparsep] is the amount of space inserted between production
%       rules.  It is a rubber length whose default value is 8\,pt, with
%       1\,pt of stretch and shrink.
%
% \item [\grammarindent] is the amount by which the right hand side of a
%       production rule is indented from the left margin.  It is a rigid
%       length.  Its default value is 2\,em.
%
% \end{description}
%
% \DescribeMacro\grammarlabel
% You can also control how the `label' is typeset by redefining the
% |\grammarlabel| command.  The command is given two arguments: the name of
% the nonterminal (which was enclosed in angle brackets), and the `production
% operator'.  The command is expected to produce the label.  By default, it
% typesets the nonterminal name using |\synt| and the operator at opposite
% ends of the label, separated by an |\hfill|.
%
% \subsection{Syntax diagrams}
%
% A full formal BNF grammar can be somewhat overwhelming for less technical
% readers.  Documents aimed at such readers tend to display grammatical
% structures as \emph{syntax diagrams}.
%
% \DescribeEnv{syntdiag}
% A syntax diagram is always enclosed in a \env{syntdiag} environment.  You
% should think of the environment as enclosing a new sort of \LaTeX\ mode:
% trying to type normal text into a syntax diagram will result in very ugly
% output.  \LaTeX\ ignores spaces and return characters while in syntax
% diagram mode.
%
% The syntax of the environment is very simple:
%
% \begin{grammar}
% <synt-diag-env> ::= \[[
%   "\\begin{syntdiag}"
%   \begin{stack} \\ "[" <decls> "]" \end{stack}
%   <text>
%   "\\end{syntdiag}"
% \]]
% \end{grammar}
%
% The \<decls> contain any declarations you want to insert, to control
% the environment.  The parameters to tweak are described below.
%
% Within a syntax diagram, you can include syntactic items using the
% abbreviated forms described elsewhere.  The output from these forms is
% modified slightly in syntax diagram mode so that the diagram looks
% right.
%
% I probably ought to point out now that the syntax diagram typesetting
% commands produce beautiful-looking diagrams with all the rules and curves
% accurately positioned.  Some device drivers don't position these objects
% correctly in their output.  I've had particular trouble with |dvips|.  I'll
% say it again: it's not my fault!
%
% \DescribeEnv{syntdiag*}
% The \env{syntdiag} environment only works in paragraph mode, and it acts
% rather like a paragraph, splitting over several lines when appropriate.
% If you just want to typeset a snippet of a syntax diagram, you can
% use the starred environment \env{syntdiag$*$}.
%
% \begin{grammar}
% <synt-diag-star-env> ::= \[[
%   "\\begin{syntdiag*}"
%   \begin{stack} \\ "[" <decls> "]" \end{stack}
%   \begin{stack} \\ "[" <width> "]" \end{stack}
%   <text>
%   "\\end{syntdiag*}"
% \]]
% \end{grammar}
%
% When typesetting little demos like this, it's not normal to fully adorn
% the syntax diagram with the full double arrows
% (`\begin{syntdiag*}[\left{>>-}\right{-><}]\tok{$\cdots$}\end{syntdiag*}').
% The two declarations \syntax{"\\left{"<arrow>"}" and "\\right{"<arrow>"}"}
% allow you to choose the arrows on each side of the syntax diagram snippet.
% The possible values of \<arrow> are shown in the table-ette below:
%
% ^^A Time to remember what I learned about tables while writing mdwtab.
% ^^A Just for the embarassment factor, here's the number of attempts I
% ^^A took to get the table below to look right: __6.  Hmm... not as bad
% ^^A as I expected.  Most of them were fine-tuning things.
%
% \medskip					^^A Leave a vertical gap
% \hbox to\columnwidth{\hfil\vbox{\tabskip=0pt	^^A Centre it horizontally
% \sdsize \csname sd@setsize\endcsname		^^A Position syntdiag arrows
% \halign to .5\columnwidth{			^^A Set the table width
%   &\ttfamily\ignorespaces#\unskip\hfil\tabskip=0pt ^^A Typeset the name
%   &\quad\csname sd@arr@#\endcsname\hfil	^^A Typeset the arrow
%   &\setbox0=\hbox{#}\tabskip=0pt plus 1fil\cr	^^A Stretch between columns
%   >>-&>>-&   &>-&>-&   &->&->\cr
%   -><&-><&   &...&...& &-&-\cr
% }}\hfil}					^^A Close the boxing
% \medskip					^^A And leave another gap
%
% These declarations should be used only in the optional argument to the
% \env{syntdiag$*$} command.  The second optional argument to the
% environment, if specified, fixes the width of the syntax diagram snippet;
% if you omit this argument, the diagram is made just wide enough to
% fit everything in.
%
% \begin{figure}
% \begin{demo}[w]{Example of \env{syntdiag$*$}}
%\newcommand{\bs}[2]{%
%  \begin{minipage}{1.6in}%
%  \begin{syntdiag*}[\left{#1}\right{#2}][1.6in]%
%}
%\newcommand{\es}{\end{syntdiag*}\end{minipage}}
%
%\begin{center}
%\begin{tabular}{cl}                                      \\ \hline
%\bf Construction    & \bf Meaning                        \\ \hline
%\bs {>>-} {...} \es & Start of syntax diagram            \\
%\bs {...} {-><} \es & End of syntax diagram              \\
%\bs {>-}  {...} \es & Continued on next line             \\
%\bs {...} {->}  \es & Continued from previous line       \\ \hline
%\bs {...} {...}
%  \begin{stack} <option-a> \\ <option-b> \\ <option-c> \end{stack}
%\es                 & Alternatives: choose any one       \\
%\bs {...} {...}
%  \begin{rep} <repeat-me> \\ <separator> \end{rep}
%\es                 & One or more items, with separators \\ \hline
%\end{tabular}
%\end{center}
% \end{demo}
% \end{figure}
%
% \DescribeMacro\tok
% You can also include text using the |\tok| command.  The argument of this
% command is typeset in \LaTeX's LR~mode and inserted into the diagram. 
% Syntax abbreviations are allowed within the argument, so you can, for
% example, include textual descriptions like
% \begin{listing}
%\tok{any <char> except `"'}
% \end{listing}
%
% \DescribeEnv{stack}
% Within a syntax diagram, a choice between several different items is
% shown by stacking the alternatives vertically.  In \LaTeX, this is done
% by enclosing the items in a \env{stack} environment.  Each individual item
% is separated by |\\| commands, as in the \env{array} and \env{tabular}
% environments.  Each row may contain any syntax diagram material, including
% |\tok| commands and other \env{stack} environments.
%
% Note if you end a \env{stack} environment with a |\\| command, a blank
% row is added to the bottom of the stack, indicating that none of the items
% need be specified.
%
% The commands |\(| and |\)| are abbreviations for `|\begin{stack}|' and
% `|\end{stack}|' respectively.  Also, |\[| is `|\begin{stack}\\|' and
% |\]| is `|\end{stack}|' -- these two are useful for stacks in which the
% first item is blank (i.e., none of the options need be taken).
%
% \DescribeEnv{rep}
% Text which can be repeated is enclosed in a \env{rep} environment: the
% text is displayed with a backwards pointing arrow drawn over it, showing
% that it may be repeated.  Optionally, you can specify text to be
% displayed in the arrow, separating it from the main text with a |\\|
% command.
%
% Note that items on the backwards arrow of a \env{rep} construction should
% be displayed \emph{backwards}.  You must put the individual items in
% reverse order when building this part of your diagrams.  \syn\ will 
% correctly reverse the arrows on \env{rep} structures, but apart from
% this, you must cope on your own.  You are recommended to keep these parts
% of your diagrams as simple as possible to avoid confusing readers.
%
% The commands |\<| and |\>| are abbreviations for `|\begin{rep}|' and
% `|\end{rep}|' respectively.
%
% \begin{demo}[w]{A syntax diagram}
%\begin{syntdiag}
%<ident> `('
%  \begin{rep} \begin{stack} \\
%    <type> \begin{stack} \\ <ident> \end{stack}
%  \end{stack} \\ `,' \end{rep}
%\begin{stack} \\ `...' \end{stack} `)'
%\end{syntdiag}
% \end{demo}
%
% \subsubsection{Line breaking in syntax diagrams}
%
% Syntax diagrams are automatically broken over lines and across pages.
% Lines are only broken between items on the outermost level of the diagram:
% i.e., not within \env{stack} or \env{rep} environments.
%
% You can force a line break at a particular place by using the |\\| command
% as usual.  This supports all the usual \LaTeX\ features: a `|*|' variant
% which prohibits page breaking, and an optional argument specifying the
% extra vertical space between lines.
%
% \subsubsection{Customising syntax diagrams}
%
% There are two basic styles of syntax diagrams supported:
%
% \begin{description}
%
% \item [square] Lines in the syntax diagram join at squared-off corners.
%       This appears to be the standard way of displaying syntax diagrams
%       in IBM manuals, and most other documents I've seen.
%
% \item [rounded] Lines curve around corners.  Also, no arrows are drawn
%       around repeating loops: the curving of the lines provides this
%       information instead.  This style is used in various texts on
%       Pascal, and appears to be more popular in academic circles.
%
% \end{description}
%
% You can specify the style you want to use for syntax diagrams by giving
% the style name as an option on the |\usepackage| command.  For example,
% to force rounded edges to be used, you could say
%
% \begin{listing}
%\usepackage[rounded]{syntax}
% \end{listing}
%
% \DescribeMacro\sdsize
% \DescribeMacro\sdlengths
% The \env{syntdiag} environment takes an option argument, which should
% contain declarations which are obeyed while the environment is set up.
% The default value of this argument is `|\sdsize\sdlengths|'.  The
% |\sdsize| command sets the default type size for the environment: this is
% normally |\small|.  |\sdlengths| sets the values of the length parameters
% used by the environment based on the current text size.  These parameters
% are described below.
%
% For example, if you wanted to reduce the type size of the diagrams still
% further, you could use the command
% \begin{listing}
%\begin{syntdiag}[\tiny\sdlengths]
% \end{listing}
%
% The following length parameters may be altered:
%
% \begin{description} \def\makelabel{\hskip\labelsep\cmd}
%
% \item [\sdstartspace] The length of the rule between the arrows which
%       begin each line of the syntax diagram and the first item on the line.
%       Note that most objects have some space on either side of them as
%       well.  This is a rubber length.  Its default value is 1\,em, although
%       it can shrink by up to 10\,pt.
%
% \item [\sdendspace] The length of the rule between the last item on a
%       line and the arrow at the very end.  Note that the final line also
%       has extra rubber space on the end.  This is a rubber length.  Its
%       default value is 1\,em, although it will shrink by up to 10\,pt.
%
% \item [\sdmidskip] The length of the rule on either side of a large
%       construction (either a \env{stack} or a \env{rep}).  It is a rubber
%       length.  Its default value is \smallf 1/2\,em, with a very small
%       amount of infinite stretch.
%
% \item [\sdtokskip] The length of the rule on either side of a |\tok|
%       item or syntax abbreviation.  It is a rubber length.  Its default
%       value is \smallf 1/4\,em, with a very small amount of inifnite
%       stretch.
%
% \item [\sdfinalskip] The length of the rule which finishes the last line
%       of a syntax diagram.  It is a rubber length.  Its default value is
%       \smallf 1/2\,em, with 10000\,fil of stretch, which will left-align
%       the items on the line.\footnote{^^A
%         This is a little \TeX nical.  The idea is that if a stray 1\,fil
%         of stretch is added to the end of the line, it won't be noticed.
%         However, the alignment of the text on the line can still be
%         modified using \cmd{\sd@rule}\cmd{\hfill}, if you're feeling
%         brave.
%       }
%
% \item [\sdrulewidth] Half the width of the rules used in the diagram.
%       It is a rigid length.  Its default value is 0.2\,pt.
%
% \item [\sdcirclediam] The diameter of the circle from which the quadrants
%       used in rounded-style diagrams are taken.  This must be a multiple
%       of 4\,pt, or else the lines on the diagram won't match up.
%
% \end{description}
%
% In addition, you should call |\sdsetstrut| passing it the total height
% (\({\rm height}+{\rm depth}\)) of a normal line of text at the current
% size.  Normally, the value of |\baselineskip| will be appropriate.
%
% You can also alter the appearance of \env{stack}s and \env{rep}s by using
% their optional positioning arguments.  By default, \env{stack}s descend
% below the main line of the diagram, and \env{rep}s extend above it.  
% Specifying an optional argument of |[b]| for either environment reverses
% this, putting \env{stack}s above and \env{rep}s below the line.
%
% \subsection{Changing the presentation styles}
%
% You can change the way in which the syntax items are typeset by altering
% some simple commands (using |\renewcommand|).  Each item (nonterminals,
% as typeset by |\synt|, and quoted and unquoted terminals, as typeset by
% |\lit| and |\lit*|) has two style commands associated with it, as shown
% in the table below.
%
% \begin{tab}{lll}						   \hline
% \bf Syntax item	& \bf Left command & \bf Right command	\\ \hline
% Nonterminals		& |\syntleft|	   & |\syntright|	\\
% Quoted terminals	& |\litleft|	   & |\litright|	\\
% Unquoted terminals	& |\ulitleft|	   & |\ulitright|	\\ \hline
% \end{tab}
%
% It's not too hard to see how this works.  For example, if you look at
% the implementation for |\syntleft| and |\syntright| in the implementation
% section, you'll notice that they're defined like this:
% \begin{listing}
%\newcommand{\syntleft}{$\langle$\normalfont\itshape}
%\newcommand{\syntright}{$\rangle$}
% \end{listing}
% I think this is fairly simple, if you understand things like font changing.
%
% Note that changing these style commands alters the appearance of all syntax
% objects of the appropriate types, as created by the |\synt| and |\lit|
% commands, in \env{grammar} environments, and in syntax diagrams.
%
%
% \section{Change history}
%
% \subsection*{Version 1.9}
%
% Added abbreviations for syntax diagram constructions.  These clobber some
% common abbreviations for maths, but that's not too worrying really; it's
% not likely for people to do maths in syntax diagrams.
%
% \subsection*{Version 1.8}
%
% Added to RCS (so changed version numbering style).
%
% \subsection*{Version 1.07}
%
% \begin{itemize}
% \item Fixed problem with underscore hacking in a \env{tabbing} environment.
% \end{itemize}
%
% \subsection*{Version 1.06}
%
% \begin{itemize}
% \item Added style hooks for syntax items.
% \item Improved colour handling in syntax diagrams, thanks to the |\doafter|
%       package.
% \item Fixed some nasty bugs in the \env{grammar} environment which confused
%       other lists and ruined the spacing.  The \env{grammar} handling is
%       now much tidier in general.
% \end{itemize}
%
% \subsection*{Version 1.05}
%
% \begin{itemize}
% \item Fixed `the bug' in the syntax diagram typesetting.  It now breaks
%       lines almost psychically, and doesn't break in the wrong places.
% \item Almost rewrote the \env{grammar} environment.  It now does lots of
%       the list handling itself, to allow more versatile typesetting of the
%       left hand sides.  There's lots of evil in there now.
% \item Added some more configurability.  In particular, two new settings
%       have been added to control \env{grammar} environments, and a neat
%       way of adding new syntax diagram structures has been introduced.
% \end{itemize}
%
% \subsection*{Version 1.04}
%
% \begin{itemize}
% \item Changed the vertical positioning of the rules, to make all the text
%       line up properly.  While the old version was elegant and simple, it
%       had the drawback of looking nasty.
% \item Allow line breaks at underscores, but don't if there's another one
%       afterwards.  Also, prevent losing following space if underscore is
%       written to a file.
% \end{itemize}
%
% \subsection*{Version 1.02}
%
% \begin{itemize}
% \item Added support for rounded corners in syntax diagrams.
% \item Changed lots of |\hskip| commands to |\kern|s, to prevent possible
%       line breaks.
% \end{itemize}
%
% \subsection*{Version 1.01}
%
% \begin{itemize}
% \item Allowed disabling of underscore active character, to avoid messing
%       up filenames.
% \item Added |\grammarparsep| and |\grammarindent| length parameters to
%       control the appearance of grammars.
% \end{itemize}
%
% \implementation
%
% \section{Implementation of \syn}
%
%    \begin{macrocode}
%<*package>
%    \end{macrocode}
%
% \subsection{Options handling}
%
% We define all the options we know about, and then see what's been put
% on the usepackage line.
%
% The options we provide currently are as follows:
%
% \begin{description}
% \item [rounded] draws neatly rounded edges on the diagram.
% \item [square] draws squared-off edges on the diagram.  This is the
%       default.
% \item [nounderscore] disables the undescore active character,  The |\_|
%       command still produces the nice version created here.
% \end{description}
%
%    \begin{macrocode}
\DeclareOption{rounded}{\sd@roundtrue}
\DeclareOption{square}{\sd@roundfalse}
\DeclareOption{nounderscore}{\@uscorefalse}
%    \end{macrocode}
%
% Now process the options:
%
%    \begin{macrocode}
\newif\ifsd@round
\newif\if@uscore\@uscoretrue
\ExecuteOptions{square}
\ProcessOptions
%    \end{macrocode}
%
% \subsection{Special character handling}
%
% A lot of the \syn\ package requires the use special active characters.
% These must be added to two lists: |\dospecials|, which is used by |\verb|
% and friends, and |\@sanitize|, which is used by |\index|.  The two macros
% here, |\addspecial| and |\remspecial|, provide these registration
% facilities.
%
% Two similar macros are found in Frank Mittelbach's \package{doc} package:
% these have the disadvantage of global operation.  My macros here are based
% on Frank's, which in turn appear to be based on Donald Knuth's list
% handling code presented in Appendix~D of \textit{The \TeX book}.
%
% Both these macros take a single argument: a single-character control
% sequence containing the special character to be added to or removed from
% the lists.
%
% \begin{macro}{\addspecial}
%
% This is reasonably straightforward.  We remove the sequence from the lists,
% in case it's already there, and add it in in the obvious way.  This
% requires a little bit of fun with |\expandafter|.
%
%    \begin{macrocode}
\def\addspecial#1{%
  \remspecial{#1}%
  \expandafter\def\expandafter\dospecials\expandafter{\dospecials\do#1}%
  \expandafter\def\expandafter\@santize\expandafter{%
    \@sanitize\@makeother#1}%
}
%    \end{macrocode}
%
% \end{macro}
%
% \begin{macro}{\remspecial}
%
% This is the difficult bit.  Since |\dospecials| and |\@sanitize| have the
% form of list macros, we can redefine |\do| and |\@makeother| to do the
% job for us.  We must be careful to put the old meaning of |\@makeother|
% back.  The current implementation assumes it knows what |\@makeother| does.
%
%    \begin{macrocode}
\def\remspecial#1{%
  \def\do##1{\ifnum`#1=`##1 \else\noexpand\do\noexpand##1\fi}%
  \edef\dospecials{\dospecials}%
  \def\@makeother##1{\ifnum`#1=`##1 \else%
    \noexpand\@makeother\noexpand##1\fi}%
  \edef\@sanitize{\@sanitize}%
  \def\@makeother##1{\catcode`##112}%
}
%    \end{macrocode}
%
% \end{macro}
%
% \subsection{Underscore handling}
%
% When typing a lot of identifiers, it can be irksome to have to escape
% all `|_|' characters in the manuscript.  We make the underscore character
% active, so that it typesets an underscore in horizontal mode, and does
% its usual job as a subscript operator in maths mode.  Underscore must
% already be in the special character lists, because of its use as a
% subscript character, so this doesn't cause us a problem.
%
% \begin{macro}{\underscore}
%
% The |\underscore| macro typesets an underline character, using a horizontal
% rule.  This is positioned slightly below the baseline, and is also slightly
% wider than the default \TeX\ underscore.  This code is based on a similar
% implementation found in the \package{lgrind} package.
%
%    \begin{macrocode}
\def\underscore{%
  \leavevmode%
  \kern.06em%
  \vbox{%
    \hrule\@width.6em\@depth.4ex\@height-.34ex%
  }%
  \ifdim\fontdimen\@ne\font=\z@%
    \kern.06em%
  \fi%
}
%    \end{macrocode}
%
% \end{macro}
%
% \begin{macro}{\@foundunderscore}
%
% This macro is called by the `|_|' active character to sort out what to do.
%
% If this is maths mode, we use the |\sb| macro, which is already defined
% to do subscripting.  Otherwise, we call |\textunderscore|, which picks the
% nicest underscore it can find.
%
% There's some extra cunningness here, because I'd like to be able to
% hyphenate after underscores usually, but not when there's another one
% following.  And then, because \env{tabbing} redefines |\_|, there's some
% more yukkiness to handle that: the usual |\@tabacckludge| mechanism doesn't
% cope with this particular case.
%
%    \begin{macrocode}
\let\usc@builtindischyphen\-
\def\@uscore.{%
  \ifmmode%
    \expandafter\@firstoftwo%
  \else%
    \expandafter\@secondoftwo%
  \fi%
  \sb%
  {\textunderscore\@ifnextchar_{}{\usc@builtindischyphen}}%
}
%    \end{macrocode}
%
% \end{macro}
%
% Now we set up the active character.  Note the |\protect|, which makes
% underscores work reasonably well in moving arguments.  Note also the way
% we end with a some funny stuff to prevent spaces being lost if this is
% written to a file.
%
%    \begin{macrocode}
\if@uscore
  \AtBeginDocument{%
    \catcode`\_\active%
    \begingroup%
    \lccode`\~`\_%
    \lowercase{\endgroup\def~{\protect\@uscore.}}%
  }
\fi
%    \end{macrocode}
%
% Finally, we redefine the |\_| macro to use our own |\underscore|, because
% it's prettier.  Actually, we don't: we just redefine the
% |\?\textunderscore| command (funny name, isn't it?).
%
%    \begin{macrocode}
\expandafter\let\csname?\string\textunderscore\endcsname\underscore
%    \end{macrocode}
%
% \subsection{Abbreviated verbatim notation}
%
%  In similar style to the \package{doc} package, we allow the user to set up
% characters which delimit verbatim text.  Unlike \package{doc}, we make
% such changes local to the current group.  This is performed through the
% |\shortverb| and |\unverb| commands.
%
% The implementations of these commands are based upon the |\MakeShortVerb|
% and |\DeleteShortVerb| commands of the \package{doc} package, although
% these versions have effect local to the current grouping level.  This
% prevents their redefinition of |\dospecials| from interfering with the
% grammar shortcuts, which require local changes only.
%
% The command |\shortverb| takes a single argument: a single-character
% control sequence defining which character to make into the verbatim text
% delimiter.  We store the old meaning of the active character in a control
% sequence called |\mn@\|\<char>.  Note that this control sequence
% contains a backslash character, which is a little odd.  We also define a
% command |\cc@\|\<char> which will return everything to normal.  This
% is used by the |\unverb| command.
%
% \begin{macro}{\shortverb}
%
% Here we build the control sequences we need to make everything work nicely.
% The active character is defined via |\lowercase|, using the |~| character:
% this is already made active by \TeX\@.
%
% The actual code requires lots of fiddling with |\expandafter| and friends.
%
%    \begin{macrocode}
\def\shortverb#1{%
%    \end{macrocode}
%
% First, we check to see if the command |\cc@\|\<char> has been defined.
%
%    \begin{macrocode}
  \@ifundefined{cc@\string#1}{%
%    \end{macrocode}
%
% If it hasn't been defined, we add the character to the specials list.
%
%    \begin{macrocode}
    \addspecial#1%
%    \end{macrocode}
%
% Now we set our character to be the lowercase version of |~|, which allows
% us to use it, even though we don't know what it is.
%
%    \begin{macrocode}
    \begingroup%
    \lccode`\~`#1%
%    \end{macrocode}
%
% Finally, we reach the tricky bit.  All of this is lowercased, so any
% occurrences of |~| are replaced by the user's special character.
%
%    \begin{macrocode}
    \lowercase{%
      \endgroup%
%    \end{macrocode}
%
% We remember the current meaning of the character, in case it has one.  We
% have to use |\csname| to build the rather strange name we use for this.
%
%    \begin{macrocode}
      \expandafter\let\csname mn@\string#1\endcsname~%
%    \end{macrocode}
%
% Now we build |\cc@\|\<char>.  This is done with |\edef|, since more
% of this needs to be expanded now than not.  In this way, the actual macros
% we create end up being very short.
%
%    \begin{macrocode}
      \expandafter\edef\csname cc@\string#1\endcsname{%
%    \end{macrocode}
%
% First, add a command to restore the character's old catcode.
%
%    \begin{macrocode}
        \catcode`\noexpand#1\the\catcode`#1%
%    \end{macrocode}
%
% Now we restore the character's old meaning, using the version we saved
% earlier.
%
%    \begin{macrocode}
        \let\noexpand~\expandafter\noexpand%
          \csname mn@\string#1\endcsname%
%    \end{macrocode}
%
% Now we remove the character from the specials lists.
%
%    \begin{macrocode}
        \noexpand\remspecial\noexpand#1%
%    \end{macrocode}
%
% Finally, we delete this macro, so that |\unverb| will generate a warning
% if the character is |\unverb|ed again.
%
%    \begin{macrocode}
        \let\csname cc@\string#1\endcsname\relax%
      }%
%    \end{macrocode}
%
% All of that's over now.  We set up the new definition of the character,
% in terms of |\verb|, and make the character active.  The nasty |\syn@ttspace|
% is there to make the spacing come out right.  It's all right really.  Honest.
%
%    \begin{macrocode}
      \def~{\verb~\syn@ttspace}%
    }%
    \catcode`#1\active%
%    \end{macrocode}
%
% If our magic control sequence already existed, we can assume that the
% character is already a verbatim delimiter, and raise a warning.
%
%    \begin{macrocode}
  }{%
    \PackageWarning{syntax}{Character `\expandafter\@gobble\string#1'
                            is already a verbatim\MessageBreak
                            delimiter}%
  }%
}
%    \end{macrocode}
%
% \end{macro}
%
% \begin{macro}{\unverb}
%
% This is actually terribly easy: we just use the |\cc@\|\<char> command
% we definied earlier, after making sure that it's been defined.
%
%    \begin{macrocode}
\def\unverb#1{%
  \@ifundefined{cc@\string#1}{%
    \PackageWarning{syntax}{Character `\expandafter\@gobble\string#1'
                            is not a verbatim\MessageBreak
                            delimiter}%
  }{%
    \csname cc@\string#1\endcsname%
  }%
}
%    \end{macrocode}
%
% \end{macro}
%
% \subsection{Style hooks for syntax forms}
%
% To allow the appearance of syntax things to be configured, we provide some
% redefinable bits.
%
% The three types of objects (nonterminal symbols, and quoted and unquoted
% terminals) each have two macros associated with them: one which does the
% `left' bit of the typesetting, and one which does the `right' bit.  The
% items are typeset as LR~boxes.  I'll be extra good while defining these
% hooks, so that it's obvious what's going on; macho \TeX\ hacker things
% resume after this section.
%
% \begin{macro}{\syntleft}
% \begin{macro}{\syntright}
%
% I can't see why anyone would want to change the typesetting of
% nonterminals, although I'll provide the hooks for symmetry's sake.
%
%     \begin{macrocode}
\newcommand{\syntleft}{$\langle$\normalfont\itshape}
\newcommand{\syntright}{$\rangle$}
%    \end{macrocode}
%
% \end{macro}
% \end{macro}
%
% \begin{macro}{\ulitleft}
% \begin{macro}{\ulitright}
% \begin{macro}{\litleft}
% \begin{macro}{\litright}
%
% Now we can define the left and right parts of quoted and unquoted
% terminals.  US~readers may want to put double quotes around the quoted
% terminals, for example.
%
%    \begin{macrocode}
\newcommand{\ulitleft}{\normalfont\ttfamily\syn@ttspace\frenchspacing}
\newcommand{\ulitright}{}
\newcommand{\litleft}{`\bgroup\ulitleft}
\newcommand{\litright}{\ulitright\egroup'}
%    \end{macrocode}
%
% \end{macro}
% \end{macro}
% \end{macro}
% \end{macro}
%
% \subsection{Simple syntax typesetting}
%
% In general text, we allow access to our typesetting conventions through
% standard \LaTeX\ commands.
%
% \begin{macro}{\synt}
%
% The |\synt| macro typesets its argument as a syntactic quantity.  It puts
% the text of the argument in italics, and sets angle brackets around it.
% Breaking of a |\synt| object across lines is forbidden.
%
%    \begin{macrocode}
\def\synt#1{\mbox{\syntleft{#1\/}\syntright}}
%    \end{macrocode}
%
% \end{macro}
%
% \begin{macro}{\lit}
%
% The |\lit| macro typesets its argument as literal text, to be typed in.
% Normally, this means setting the text in |\tt| font, and putting quotes
% around it, although the quotes can be suppressed by using the $*$-variant.
%
% The |\syn@ttspace| macro sets up the spacing for the text nicely: |\tt|
% spaces tend to be a little wide.
%
%    \begin{macrocode}
\def\lit{\@ifstar{\lit@i\ulitleft\ulitright}{\lit@i\litleft\litright}}
\def\lit@i#1#2#3{\mbox{#1{#3\/}#2}}
%    \end{macrocode}
%
% \end{macro}
%
% \begin{macro}{\syn@ttspace}
%
% This sets up the |\spaceskip| value for |\tt| text.
%
%    \begin{macrocode}
\def\syn@ttspace@{\spaceskip.35em\@plus.2em\@minus.15em\relax}
%    \end{macrocode}
%
% However, this isn't always the right thing to do.
%
%    \begin{macrocode}
\def\ttthinspace{\let\syn@ttspace\syn@ttspace@}
\def\ttthickspace{\let\syn@ttspace\@empty}
%    \end{macrocode}
%
% I know what I like thoough.
%
%    \begin{macrocode}
\ttthinspace
%    \end{macrocode}
%
% \end{macro}
%
% \subsubsection{The shortcuts}
%
% The easy part is over now.  The next job is to set up the `grammar
% shortcuts' which allow easy changing of styles.
%
% We support four shortcuts:
% \begin{itemize}
% \item |`literal text'| typesets \syntax{`literal text'}
% \item |<non-terminal>| typesets \syntax{<non-terminal>}
% \item |"unquoted text"| typesets \syntax{"unquoted text"}
% \item \verb"|" typesets a \syntax{|} character
% \end{itemize}
% These are all implemented through active characters, which are enabled
% using the |\syntaxShortcuts| macro, described below.
%
% \begin{macro}{\readupto}
%
% \syntax{"\\readupto{"<char>"}{"<decls>"}{"<command>"}"} will read all
% characters up until the next occurrence of \<char>.  Normally, all
% special characters will be deactivated.  However, you can reactivate some
% characters, using the \<decls> argument, which is processed before the
% text is read.
%
% The code is borrowed fairly obviously from the \LaTeXe\ source for the
% |\verb| command.
%
%    \begin{macrocode}
\def\readupto#1#2#3{%
  \bgroup%
  \verb@eol@error%
  \let\do\@makeother\dospecials%
  #2%
  \catcode`#1\active%
  \lccode`\~`#1%
  \gdef\verb@balance@group{\verb@egroup%
     \@latex@error{\noexpand\verb illegal in command argument}\@ehc}%
  \def\@vhook{\verb@egroup#3}%
  \aftergroup\verb@balance@group%
  \lowercase{\let~\@vhook}%
}
%    \end{macrocode}
%
% \end{macro}
%
% \begin{macro}{\syn@assist}
%
% The |\syn@assist| macro is used for defining three of the shortcuts.  It
% is called as
%
% \begin{quote}
% \syntax{"\\syn@assist{"<left-decls>"}{"<actives>"}{"<delimeter>"}" \\
% \null \quad "{"<right-decls>"}{"<end-cmd>"}"}
% \end{quote}
%
% It creates an hbox, sets up the escape sequences for quoting our magic
% characters, and then typesets a box containing
%
% \begin{quote}
% \syntax{<left-decls>"{"<delimited-text>"\\/}"<right-decls>}
% \end{quote}
%
% The \<left-decls> and \<right-decls> can be |\relax| if they're not
% required.
%
% The \<actives> argument is passed to |\readupto|, to allow some special
% characters through.  By default, we re-enable |\|, and make `\verb*" "'
% typeset some space glue, rather than a space character.  A macro
% `\verb*"\ "' is defined to actually print a space character, which yield
% `\verb*" "' in the `|\tt|' font. 
%
% Finally, it defines a |\ch| command, which, given a single-character
% control sequence as its argument, typesets the character.  This is useful,
% since |`| has been made active when we set up these calls, so the
% direct |\char`\|\<char> doesn't work.
%
%    \begin{macrocode}
\def\syn@assist#1#2#3#4#5{%
%    \end{macrocode}
%
% First, we start the box, and open a group.  We use |\mbox| because it
% does all the messing with |\leavevmode| which is needed.
%
%    \begin{macrocode}
  \mbox\bgroup%
%    \end{macrocode}
%
% Next job is to set up the escape sequences.
%
%    \begin{macrocode}
  \chardef\\`\\%
  \chardef\>`\>%
  \chardef\'`\'%
  \chardef\"`\"%
  \chardef\ `\ %
%    \end{macrocode}
%
% Now to define |\ch|.  This is done the obvious way.
%
%    \begin{macrocode}
  \def\ch##1{\char`##1}%
%    \end{macrocode}
%
% For active characters, we do some fiddling with |\lccode|s.
%
%    \begin{macrocode}
  \def\act##1{%
    \catcode`##1\active%
    \begingroup%
    \lccode`\~`##1%
    \lowercase{\endgroup\def~}%
  }%
%    \end{macrocode}
%
% Finally, we do the real work of setting the text.  We use |\readupto| to
% actually find the text we want.
%
%    \begin{macrocode}
  #1%
  \begingroup%
  \readupto#3{%
    \catcode`\\0%
    \catcode`\ 10%
    #2%
  }{%
    \/\endgroup#4\egroup#5%
  }%
}
%    \end{macrocode}
%
% \end{macro}
%
% \begin{macro}{\syn@shorts}
%
% This macro actually defines the expansions for the active characters.
% We have to do this separately because |`| must be active when we use it
% in the |\def|, but we can't do that and use |\catcode| at the same time.
% The arguments are commands to do before and after the actual command.
% These are passed up from |\syntaxShortcuts|.
%
% All of the characters use |\syn@assist| in the obvious way except for
% \verb"|", which drops into maths mode instead.
%
% Note that when changing the catcodes, we must save |`| until last.
%
%    \begin{macrocode}
\begingroup
\catcode`\<\active
\catcode`\|\active
\catcode`\"\active
\catcode`\`\active
%
\gdef\syn@shorts#1#2{%
%    \end{macrocode}
%
% The `|<|' character must typeset its argument in italics.  We make `|_|'
% do the same as the `|\_|' command.
%
%    \begin{macrocode}
  \def<{%
    #1%
    \syn@assist%
      \syntleft%
      {\act_{\@foundunderscore}}%
      >%
      \syntright%
      {#2}%
  }%
%    \end{macrocode}
%
% The `|`|' and `|"|' characters should print its argument in |\tt| font.
% We change the `|\tt|' space glue to provide nicer spacing on the line.
%
%    \begin{macrocode}
  \def`{%
    #1%
    \syn@assist%
      \litleft%
      \relax%
      '%
      \litright%
      {#2}%
  }%
  \def"{%
    #1%
    \syn@assist%
      \ulitleft%
      \relax%
      "%
      \ulitright%
      {#2}%
  }%
%    \end{macrocode}
%
% Finally, the `\verb"|"' character is typeset by using the mysterious
% |\textbar| command.
%
%    \begin{macrocode}
  \def|{\textbar}%
%    \end{macrocode}
%
% We're finished here now.
%
%    \begin{macrocode}
}
%
\endgroup
%    \end{macrocode}
%
% \end{macro}
%
% \begin{macro}{\syntaxShortcuts}
%
% This is a user-level command which enables the use of our shortcuts in the
% current group.  It uses |\addspecial|, defined below, to register the
% active characters, sets up their definitions and activates them.
%
% The two arguments are commands to be performed before and after the
% handling of the abbreviation.  In this way, you can further process the
% output.
%
% This command is not intended to be used directly by users: it should be
% used by other macros and packages which wish to take advantage of the
% facilities offered by this package.  We provide a |\synshorts| declaration
% (which may be used as an environment, of course) which is more `user
% palatable'.
%
%    \begin{macrocode}
\def\syntaxShortcuts#1#2{%
  \syn@shorts{#1}{#2}%
  \addspecial\`%
  \addspecial\<%
  \addspecial\|%
  \addspecial\"%
  \catcode`\|\active%
  \catcode`\<\active% 
  \catcode`\"\active%
  \catcode`\`\active%
}
%
\def\synshorts{\syntaxShortcuts\relax\relax}
%    \end{macrocode}
%
% \end{macro}
%
% \begin{macro}{\synshortsoff}
%
% This macro can be useful occasionally: it disables the syntax shortcuts,
% so you can type normal text for a while.
%
%    \begin{macrocode}
\def\synshortsoff{%
  \catcode`\|12%
  \catcode`\<12%
  \catcode`\"12%
  \catcode`\`12%
}
%    \end{macrocode}
%
% \end{macro}
%
% \begin{macro}{\syntax}
%
% The |\syntax| macro typesets its argument, allowing the use of our
% shortcuts within the argument.
%
% Actually, we go to some trouble to ensure that the argument to |\syntax|
% \emph{isn't} a real argument so we can change catcodes as we go.  We
% use the |\let\@let@token=| trick from \PlainTeX\ to do this.
%
%    \begin{macrocode}
\def\syntax#{\bgroup\syntaxShortcuts\relax\relax\let\@let@token}
%    \end{macrocode}
%
% \end{macro}
%
% \begin{environment}{grammar}
%
% The \env{grammar} environment is the final object we have to define.  It
% allows typesetting of beautiful BNF grammars.
%
% First, we define the length parameters we need:
%
%    \begin{macrocode}
\newskip\grammarparsep
  \grammarparsep8\p@\@plus\p@\@minus\p@
\newdimen\grammarindent
  \grammarindent2em
%    \end{macrocode}
%
% Now define the default label typesetting.  This macro is designed to be
% replaced by a user, so we'll be extra-well-behaved and use genuine \LaTeX\
% commands.  Well, almost \dots
%
%    \begin{macrocode}
\newcommand{\grammarlabel}[2]{%
  \synt{#1} \hfill#2%
}
%    \end{macrocode}
%
% Now for a bit of hacking to make the item stuff work properly.  This gets
% done for every new paragraph that's started without an |\item| command.
%
% First, store the left hand side of the production in a box.  Then I'll
% end the paragraph, and insert some nasty glue to take up all the space,
% so no-one will ever notice that there was a paragraph break there.  The
% strut just makes sure that I know exactly how high the line is.
%
%    \begin{macrocode}
\def\gr@implitem<#1> #2 {%
  \sbox\z@{\hskip\labelsep\grammarlabel{#1}{#2}}%
  \strut\@@par%
  \vskip-\parskip%
  \vskip-\baselineskip%
%    \end{macrocode}
%
% The |\item| command will notice that I've inserted these funny glues and
% try to remove them: I'll stymie its efforts by inserting an invisible
% rule.  Then I'll insert the label using |\item| in the normal way.
%
%    \begin{macrocode}
  \hrule\@height\z@\@depth\z@\relax%
  \item[\unhbox\z@]%
%    \end{macrocode}
%
% Just before I go, I'll make \lit{<} back into an active character.
%
%    \begin{macrocode}
  \catcode`\<\active%
}
%    \end{macrocode}
%
% As an abbreviation for syntax diagrams, I usurp the |\[| and |\]| commands.
% Here are the old versions.
%
%    \begin{macrocode}
\let\gr@leftsq\[
\let\gr@rightsq\]
\def\[{\gr@leftsq}
\def\]{\gr@rightsq}
%    \end{macrocode}
%
% Now for the environment proper.  Deep down, it's a list environment, with
% some nasty tricks to stop anyone from noticing.
%
% The first job is to set up the list from the parameters I'm given.
%
%    \begin{macrocode}
\newenvironment{grammar}{%
  \list{}{%
    \labelwidth\grammarindent%
    \leftmargin\grammarindent%
    \advance\grammarindent\labelsep
    \itemindent\z@%
    \listparindent\z@%
    \parsep\grammarparsep%
  }%
%    \end{macrocode}
%
% We have major problems in |\raggedright| layouts, which try to use |\par|
% to start new lines.  We go back to normal |\\| newlines to try and bodge
% our way around these problems.
%
%    \begin{macrocode}
  \let\\\@normalcr
%    \end{macrocode}
%
% Now to enable the shortcuts.
%
%    \begin{macrocode}
  \syntaxShortcuts\relax\relax%
%    \end{macrocode}
%
% Now a little bit of magic.  The |\alt| macro moves us to a new line, and
% typesets a vertical bar in the margin.  This allows typesetting of
% multiline alternative productions in a pretty way.
%
%    \begin{macrocode}
  \def\alt{\\\llap{\textbar\quad}}%
%    \end{macrocode}
%
% Now for another bit of magic.  We set up some |\par| cleverness to spot
% the start of each production rule and format it in some cunning and
% user-defined way.
%
%    \begin{macrocode}
  \def\gr@setpar{%
    \def\par{%
      \parshape\@ne\@totalleftmargin\linewidth%
      \@@par%
      \catcode`\<12%
      \everypar{%
        \everypar{}%
        \catcode`\<\active%
        \gr@implitem%
      }%
    }%
  }%
  \gr@setpar%
  \par%
%    \end{macrocode}
%
% Now set up the |\[[| and |\]]| commands to do the right thing.  We have
% to check the next character to see if it's correct, otherwise we'll
% open a maths display as usual.
%
%    \begin{macrocode}
  \def\gr@endsyntdiag]{\end{syntdiag}\gr@setpar\par}%
  \def\[{\@ifnextchar[{\begin{syntdiag}\@gobble}\gr@leftsq}%
  \def\]{\@ifnextchar]\gr@endsyntdiag\gr@rightsq}%
%    \end{macrocode}
%
% Well, that's it for this side of the environment.
%
%    \begin{macrocode}
}{%
%    \end{macrocode}
%
% Closing the environment is a simple matter of tidying away the list.
%
%    \begin{macrocode}
  \@newlistfalse%
  \everypar{}%
  \endlist%
}
%    \end{macrocode}
%
% \end{environment}
%
% \subsection{Syntax diagrams}
%
% Now we come to the final and most complicated part of the package.
%
% Syntax diagrams are drawn using arrow characters from \LaTeX's line font,
% used in the \env{picture} environment, and rules.  The horizontal rules
% of the diagram are drawn along the baselines of the lines in which they
% are placed.  The text items in the diagram are placed in boxes and lowered
% below the main baseline.  Struts are added throughout to keep the vertical
% spacing consistent.
%
% The vertical structures (stacks and loops) are all implemented with \TeX's
% primitive |\halign| command.
%
% \subsubsection{User-configurable parameters}
%
% First, we allocate the \<dimen> and \<skip> arguments needed.  Fixed
% lengths, as the \LaTeX book calls them, are allocated as \<dimen>s, to
% take some of the load off of all the \<skip> registers.
%
%    \begin{macrocode}
\newskip\sdstartspace
\newskip\sdendspace
\newskip\sdmidskip
\newskip\sdtokskip
\newskip\sdfinalskip
\newdimen\sdrulewidth
\newdimen\sdcirclediam
\newdimen\sdindent
%    \end{macrocode}
%
% We need some \TeX\ \<dimen>s for our own purposes, to get everything in
% the right places.  We use labels for the `temporary' \TeX\ parameters
% which we use, to avoid wasting registers.
%
%    \begin{macrocode}
\dimendef\sd@lower\z@
\dimendef\sd@upper\tw@
\dimendef\sd@mid4
\dimendef\sd@topcirc6
\dimendef\sd@botcirc8
%    \end{macrocode}
%
% \begin{macro}{\sd@setsize}
% When the text size for syntax diagrams changes, it's necessary to work out
% the height for various rules in the diagram.
%
%    \begin{macrocode}
\def\sd@setsize{%
  \sd@mid\ht\strutbox%
  \advance\sd@mid-\dp\strutbox%
  \sd@mid.5\sd@mid%
  \sd@upper\sdrulewidth%
    \advance\sd@upper\sd@mid%
  \sd@lower\sdrulewidth%
    \advance\sd@lower-\sd@mid%
  \sd@topcirc-.5\sdcirclediam%
    \advance\sd@topcirc\sd@mid%
  \sd@botcirc-.5\sdcirclediam%
    \advance\sd@botcirc-\sd@mid%
}
%    \end{macrocode}
%
% \end{macro}
%
% \begin{macro}{\sdsize}
%
% You can set the default type size used by syntax diagrams by redefining
% the |\sdsize| command, using the |\renewcommand| command.
%
% By default, syntax diagrams are set slightly smaller than the main body
% text.\footnote{^^A
%   I've used pure \LaTeX\ commands for this and the \cmd\sdlengths\ macro,
%   to try and illustrate how these values might be changed by a user.  The
%   rest of the code is almost obfuscted in its use of raw \TeX\ features,
%   in an attempt to dissuade more na\"\i ve users from fiddling with it.
%   I suppose this is what you get when you let assembler hackers loose with
%   something like \LaTeX.
% }
%
%    \begin{macrocode}
\newcommand{\sdsize}{%
  \small%
}
%    \end{macrocode}
%
% \end{macro}
%
% \begin{macro}{\sdlengths}
%
% Finally, the default length parameters are set in the |\sdlengths| command.
% You can redefine the command using |\renewcommand|.
%
% We set up the length parameters here.
%
%    \begin{macrocode}
\newcommand{\sdlengths}{%
  \setlength{\sdstartspace}{1em minus 10pt}%
  \setlength{\sdendspace}{1em minus 10pt}%
  \setlength{\sdmidskip}{0.5em plus 0.0001fil}%
  \setlength{\sdtokskip}{0.25em plus 0.0001fil}%
  \setlength{\sdfinalskip}{0.5em plus 10000fil}%
  \setlength{\sdrulewidth}{0.2pt}%
  \setlength{\sdcirclediam}{8pt}%
  \setlength{\sdindent}{0pt}%
}
%    \end{macrocode}
%
% \end{macro}
%
% \subsubsection{Other declarations}
%
% We define four switches.  The table shows what they're used for.
%
% \begin{table}
% \begin{tab}{lp{3in}}                                              \hline
%
% \bf Switch        & \bf Meaning                                \\ \hline
%
% |\ifsd@base|      & We are at `base level' in the diagram:
%                     i.e., not in any other sorts of
%                     constructions.  This is used to decide
%                     whether to allow line breaking.            \\[2pt]
%
% |\ifsd@top|       & The current loop construct is being
%                     typeset with the loop arrow above the
%                     baseline.                                  \\[2pt]
%
% |\ifsd@toplayer|  & We are typesetting the top layer of
%                     a stack.  This is used to ensure that
%                     the vertical rules on either side are
%                     typeset at the right height.               \\[2pt]
%
% |\ifsd@backwards| & We're typesetting backwards, because
%                     we're in the middle of a loop arrow.
%                     the only difference this makes is that
%                     any subloops have the arrow on the
%                     side.                                      \\ \hline
%
% \end{tab}
% \caption{Syntax diagram switches}
% \end{table}
%
%    \begin{macrocode}
\newif\ifsd@base
\newif\ifsd@top
\newif\ifsd@toplayer
\newif\ifsd@backwards
%    \end{macrocode}
%
% \begin{macro}{\sd@err}
%
% We output our errors through this macro, which saves a little typing.
%
%    \begin{macrocode}
\def\sd@err{\PackageError{syntax}}
%    \end{macrocode}
%
% \end{macro}
%
% \subsubsection{Arrow-drawing}
%
% We need to draw some arrows.  \LaTeX\ tries to make this as awkward as
% possible, so we have to start moving the arrows around in boxes quite a
% lot.
%
% The left and right pointing arrows are fairly simple: we just add some
% horizontal spacing to prevent the width of the arrow looking odd.
%
%    \begin{macrocode}
\def\sd@arrow{%
  \ht\tw@\z@%
  \dp\tw@\z@%
  \raise\sd@mid\box\tw@%
  \egroup%
}
\def\sd@rightarr{%
  \bgroup%
  \setbox\tw@\hbox{\kern-6\p@\@linefnt\char'55}%
  \sd@arrow%
}
\def\sd@leftarr{%
  \bgroup%
  \raise\sd@mid\hbox{\@linefnt\char'33\kern-6\p@}%
  \sd@arrow%
}
%    \end{macrocode}
%
% The up arrow is very strange.  We need to bring the arrow down to base
% level, and smash its height.
%
%    \begin{macrocode}
\def\sd@uparr{%
  \bgroup%
  \setbox\tw@\hb@xt@\z@{\kern-\sdrulewidth\@linefnt\char'66\hss}%
  \setbox\tw@\hbox{\lower10\p@\box\tw@}%
  \sd@arrow%
}
%    \end{macrocode}
%
% The down arrow is similar, although it's already at the right height.
% Thus, we can just smash the box.
%
%    \begin{macrocode}
\def\sd@downarr{%
  \bgroup%
  \setbox\tw@\hb@xt@\z@{\kern-\sdrulewidth\@linefnt\char'77\hss}%
  \sd@arrow%
}
%    \end{macrocode}
%
% \subsubsection{Drawing curves}
%
% If the user has selected curved edges, we use the \LaTeX\ features provided
% to obtain the curves.  These are drawn slightly oddly to make it easier
% to fit them into the diagram.
%
% Some explanation about the \LaTeX\ circle font is probably called for
% before we go any further.  The font consists of sets of four quadrants
% of a particular size (and some other characters, which aren't important
% at the moment).  Each collection of quadrants fit together to form a
% perfect circle of a given diameter.  The individual quadrant characters
% have strange bounding boxes, as described in the files \textit{lcircle.mf}
% and \textit{ltpict.dtx}, and also in Appendix~D of \textit{The \TeX book}.
% Our job here is to make these quadrants useful in the context of
% drawing syntax diagrams.
%
% \begin{macro}{\sd@circ}
% First, we define |\sd@circ|, which performs the common parts of the four
% routines.  Since the characters in the circle font are grouped together,
% we can pick out a particular corner piece by specifying its index into
% the group for the required size.  The |\sd@circ| routine will pick out
% the required character, given this index as an argument, and put it in
% box~2, after fiddling with the sizes a little:
% \begin{itemize}
%
% \item We clear the width to zero.  The individual routines then add a kern
%       of the correct amount, so that the quadrant appears in the right
%       place.
%
% \item The piece is lowered by half the rule width.  This positions the
%       top and bottom pieces of the circle to be half way over the baseline,
%       which is the correct position for the rest of the diagram.
%
% \end{itemize}
%
% Finally, we make sure we're in horizontal mode: horrific results occur
% if this is not the case.  I'm sure I don't need to explain this any more
% graphically.
%
%    \begin{macrocode}
\def\sd@circ#1{%
  \@getcirc\sdcirclediam%
  \advance\@tempcnta#1%
  \setbox\tw@\hbox{\lower\sdrulewidth%
    \hbox{\@circlefnt\char\@tempcnta}}%
  \wd\tw@\z@%
  \leavevmode%
}
%    \end{macrocode}
%
% \end{macro}
%
% \begin{macro}{\sd@tlcirc}
% \begin{macro}{\sd@trcirc}
% \begin{macro}{\sd@blcirc}
% \begin{macro}{\sd@brcirc}
%
% These are the macros which actually draw quadrants of circles.  They all
% call |\sd@circ|, passing an appropriate index, and then fiddle with the
% box sizes and apply kerning specific to the quadrant positioning.
%
% The exact requirements for positioning are as follows:
%
% \begin{itemize}
%
% \item The horizontal parts of the arcs must lie along the baseline (i.e.,
%       half the line must be above the baseline, and half must be below).
%       This is consistent with the horizontal rules used in the diagram.
%
% \item The vertical parts must overlap vertical rules on either side, so
%       that a |\vrule\sd@|\textit{xx}|circ| makes the arc appear to be
%       a real curve in the line.  The requirements are actually somewhat
%       inconsistent; for example, the \env{stack} environment uses curves
%       \emph{before} the |\vrule|s.  Special requirements like this are
%       handled as special cases later.
%
% \item The height and width of the arc are at least roughly correct.
%
% \end{itemize}
%
%    \begin{macrocode}
\def\sd@tlcirc{{%
  \sd@circ3%
  \ht\tw@\sdrulewidth%
  \dp\tw@.5\sdcirclediam%
  \kern-\tw@\sdrulewidth%
  \raise\sd@mid\box\tw@%
  \kern.5\sdcirclediam%
}}
%    \end{macrocode}
%
%    \begin{macrocode}
\def\sd@trcirc{{%
  \sd@circ0%
  \ht\tw@\sdrulewidth%
  \dp\tw@.5\sdcirclediam%
  \kern.5\sdcirclediam%
  \raise\sd@mid\box\tw@%
}}
%    \end{macrocode}
%
%    \begin{macrocode}
\def\sd@blcirc{{%
  \sd@circ2%
  \ht\tw@.5\sdcirclediam%
  \dp\tw@\sdrulewidth%
  \kern-\tw@\sdrulewidth%
  \raise\sd@mid\box\tw@%
  \kern.5\sdcirclediam%
}}
%    \end{macrocode}
%
%    \begin{macrocode}
\def\sd@brcirc{{%
  \sd@circ1%
  \ht\tw@.5\sdcirclediam%
  \dp\tw@\sdrulewidth%
  \kern.5\sdcirclediam%
  \raise\sd@mid\box\tw@%
}}
%    \end{macrocode}
%
% \end{macro}
% \end{macro}
% \end{macro}
% \end{macro}
%
% \begin{macro}{\sd@llc}
% \begin{macro}{\sd@rlc}
%
% In the \env{rep} environment, we need to be able to draw arcs with
% horizontal lines running through them.  The two macros here do the job
% nicely.  |\sd@llc| (which is short for left overlapping circle) is
% analogous to |\llap|: it puts its argument in a box of zero width, sticking
% out to the left.  However, it also draws a rule along the baseline.  This
% is important, as it prevents text from overprinting the arc.  |\sd@rlc|
% is very similar, just the other way around.
%
%    \begin{macrocode}
\def\sd@llc#1{%
  \hb@xt@.5\sdcirclediam{%
    \sd@rule\hskip.5\sdcirclediam%
    \hss%
    #1%
  }%
}
%    \end{macrocode}
%
%    \begin{macrocode}
\def\sd@rlc#1{%
  \hb@xt@.5\sdcirclediam{%
    #1%
    \hss%
    \sd@rule\hskip.5\sdcirclediam%
  }%
}
%    \end{macrocode}
%
% \end{macro}
% \end{macro}
%
% \subsubsection{Drawing rules}
%
% It's important to draw the rules \emph{along} the baseline, rather than
% above it: hence, the depth of the rule must be equal to the height.
%
% \begin{macro}{\sd@rule}
%
% We use rule leaders instead of glue through most of the syntax diagrams.
% The command \syntax{"\\sd@rule"<skip>} draws a rule of the correct
% dimensions, which has the behaviour of an \syntax{"\\hskip"<skip>}.
%
%    \begin{macrocode}
\def\sd@rule{\leaders\hrule\@height\sd@upper\@depth\sd@lower}
%    \end{macrocode}
%
% \end{macro}
%
% \begin{macro}{\sd@gap}
%
% The gap between elements is added using this macro.  It will allow a
% line break if we're at the top level of the diagram, using a rather
% strange discretionary.
%
% This is called as \syntax{"\\sd@gap{"<skip-register>"}"}.
%
%    \begin{macrocode}
\def\sd@gap#1{%
%    \end{macrocode}
%
% First, we see if we're at the top level.  Within constructs, we avoid the
% overhead of a |\discretionary|.  We put half of the width of the skip on
% each side of the discretionary break.
%
%    \begin{macrocode}
  \ifsd@base%
    \skip@#1%
      \divide\skip\z@\tw@%
    \nobreak\sd@rule\hskip\skip@%
    \discretionary{%
      \sd@qarrow{->}%
    }{%
      \hbox{%
        \sd@qarrow{>-}%
        \sd@rule\hskip\sdstartspace%
        \sd@rule\hskip3.5\p@%
      }%
    }{%
    }%
    \nobreak\sd@rule\hskip\skip@%
%    \end{macrocode}
%
% If we're not at the base level, we just put in a rule of the correct
% width.
%
%    \begin{macrocode}
  \else%
    \sd@rule\hskip#1%
  \fi%
}
%    \end{macrocode}
%
% \end{macro}
%
% \subsubsection{The \protect\env{syntdiag} environment}
%
% All syntax diagrams are contained within a \env{syntdiag} environment.
%
% \begin{environment}{syntdiag}
%
% The only argument is a collection of declarations, which by
% default is
%
% \begin{listing}
%\sdsize\sdlengths
% \end{listing}
%
% However, if the optional argument is not specified, \TeX\ reads the first
% character of the environment, which may not be catcoded correctly.  We set
% up the catcodes first, using the |\syntaxShortcuts| command, and then read
% the argument.  We don't use |\newcommand|, because that would involve
% creating yet \emph{another} macro.  Time to fiddle with |\@ifnextchar|
% \dots
%
%    \begin{macrocode}
\def\syntdiag{%
  \syntaxShortcuts\sd@tok@i\sd@tok@ii%
  \@ifnextchar[\syntdiag@i{\syntdiag@i[]}%
}
%    \end{macrocode}
%
% Now we actually do the job we're meant to.
%
%    \begin{macrocode}
\def\syntdiag@i[#1]{%
%    \end{macrocode}
%
% The first thing to do is execute the user's declarations.  We then set
% up things for the font size.
%
%    \begin{macrocode}
  \sdsize\sdlengths%
  #1%
  \sd@setsize%
%    \end{macrocode}
%
% Next, we start a list, to change the text layout.
%
%    \begin{macrocode}
  \list{}{%
    \leftmargin\sdindent%
    \rightmargin\leftmargin%
    \labelsep\z@%
    \labelwidth\z@%
  }%
  \item[]%
%    \end{macrocode}
%
% We reconfigure the paragraph format quite a lot now.  We clear
% |\parfillskip| to avoid any justification at the end of the paragraph.
% We also turn off paragraph indentation.
%
%    \begin{macrocode}
  \parfillskip\z@%
  \noindent%
%    \end{macrocode}
%
% Next, we add in the arrows on the beginning of the line, and a bit of
% glue.
%
%    \begin{macrocode}
  \sd@qarrow{>>-}%
  \nobreak\sd@rule\hskip\sdstartspace%
%    \end{macrocode}
%
% This is the base level of the diagram, so we enable line breaking.
%
%    \begin{macrocode}
  \sd@basetrue%
%    \end{macrocode}
%
% Since the objects being broken are rather large, we enable sloppy line
% breaking.  We also try to avoid page breaks in mid-diagram, by upping the
% |\interlinepenalty|.
%
%    \begin{macrocode}
  \sloppy%
  \interlinepenalty100%
  \hyphenpenalty0%
%    \end{macrocode}
%
% We handle all the spacing within the environment, so we make \TeX\ ignore
% spaces and newlines.
%
%    \begin{macrocode}
  \catcode`\ 9%
  \catcode`\^^M9%
%    \end{macrocode}
%
% The environment names are rather cumbersome.  I'll define some better names
% for them here.
%
%    \begin{macrocode}
  \def\gr@leftsq{\begin{stack}\\}%
  \def\gr@rightsq{\end{stack}}%
  \def\({\begin{stack}}%
  \def\){\end{stack}}%
  \def\<{\begin{rep}}%
  \def\>{\end{rep}}%
%    \end{macrocode}
%
% We now have to change the behaviour of |\\| to line-break syntax diagrams.
%
%    \begin{macrocode}
  \let\\\sd@newline%
  \ignorespaces%
}
%    \end{macrocode}
%
% When we end the diagram, we just have to add in the final fillskip, and
% double arrow.
%
%    \begin{macrocode}
\def\endsyntdiag{%
  \unskip%
  \nobreak\sd@rule\hskip\sdmidskip%
  \sd@rule\hskip\sdfinalskip%
  \sd@qarrow{-><}%
  \endlist%
}
%    \end{macrocode}
%
% \end{environment}
%
% \begin{environment}{syntdiag*}
%
% The starred form of \env{syntdiag} typesets a syntax diagram in LR-mode;
% this is useful if you're describing parts of syntax diagrams, for example.
%
% This is in fact really easy.  The first bit which checks for an optional
% argument is almost identical to the non-$*$ version.
%
%    \begin{macrocode}
\@namedef{syntdiag*}{%
  \syntaxShortcuts\sd@tok@i\sd@tok@ii%
  \@ifnextchar[\syntdiag@s@i{\syntdiag@s@i[]}%
}
%    \end{macrocode}
%
% Handle another optional argument giving the width of the box to fill.
%
%    \begin{macrocode}
\def\syntdiag@s@i[#1]{%
  \@ifnextchar[{\syntdiag@s@ii{#1}}{\syntdiag@s@iii{#1}{\hbox}}%
}
\def\syntdiag@s@ii#1[#2]{\syntdiag@s@iii{#1}{\hb@xt@#2}}
%    \end{macrocode}
%
% Now to actually start the display.  This is mostly simple.  Just to make
% sure about the LR-ness of the typesetting, I'll put everything in an hbox.
%
%    \begin{macrocode}
\def\syntdiag@s@iii#1#2{%
  \leavevmode%
  #2\bgroup%
%    \end{macrocode}
%
% Now configure the typesetting according to the user's wishes.
%
%    \begin{macrocode}
  \let\@@left\left%
  \let\@@right\right%
  \def\left##1{\def\sd@startarr{##1}}%
  \def\right##1{\def\sd@endarr{##1}}%
  \left{>-}\right{->}%
  \sdsize\sdlengths%
  #1%
  \sd@setsize%
  \let\left\@@left%
  \let\right\@@right%
%    \end{macrocode}
%
% Put in the initial double-arrow.
%
%    \begin{macrocode}
  \sd@qarrow\sd@startarr%
  \sd@rule\hskip\sdmidskip%
%    \end{macrocode}
%
% We're in horizontal mode, so don't bother with linebreaking.
%
%    \begin{macrocode}
  \sd@basefalse%
%    \end{macrocode}
%
% Finally, disable spaces and things.
%
%    \begin{macrocode}
  \catcode`\ 9%
  \catcode`\^^M9%
  \ignorespaces%
}
%    \end{macrocode}
%
% Ending the environment is very similar.
%
%    \begin{macrocode}
\@namedef{endsyntdiag*}{%
  \unskip%
  \sd@rule\hskip\sdmidskip%
  \sd@rule\hskip\sdfinalskip%
  \sd@qarrow\sd@endarr%
  \egroup%
}
%    \end{macrocode}
%
% \end{environment}
%
% \begin{macro}{\sd@qarrow}
%
% This typesets the various left and right arrows required in syntax
% diagrams.  The argument is one of \syntax{`>>-', `->', `>-' or `-><'}.
%
%    \begin{macrocode}
\def\sd@qarrow#1{%
  \begingroup%
  \lccode`\~=`\<\lowercase{\def~{<}}%
  \hbox{\csname sd@arr@#1\endcsname}%
  \endgroup%
}
\@namedef{sd@arr@>>-}{\sd@rightarr\kern-.5\p@\sd@rightarr\kern-\p@}
\@namedef{sd@arr@>-}{\sd@rightarr\kern-\p@}
\@namedef{sd@arr@->}{\sd@rightarr}
\@namedef{sd@arr@-><}{\sd@rightarr\kern-\p@\sd@leftarr}
\@namedef{sd@arr@...}{$\cdots$}
\@namedef{sd@arr@-}{}
%    \end{macrocode}
%
% \end{macro}
%
% \begin{macro}{\sd@newline}
%
% The line breaking within a syntax diagram is controlled by the
% |\sd@newline| command, to which |\\| is assigned.
%
% We support all the standard \LaTeX\ features here.  The line breaking
% involves adding a fill skip and arrow, moving to the next line, adding
% an arrow and a rule, and continuing.
%
%    \begin{macrocode}
\def\sd@newline{\@ifstar{\vadjust{\penalty\@M}\sd@nl@i}\sd@nl@i}
\def\sd@nl@i{\@ifnextchar[\sd@nl@ii\sd@nl@iii}
\def\sd@nl@ii[#1]{\vspace{#1}\sd@nl@iii}
\def\sd@nl@iii{%
  \nobreak\sd@rule\hskip\sdmidskip%
  \sd@rule\hskip\sdfinalskip%
  \kern-3\p@%
  \sd@rightarr%
  \newline%
  \sd@rightarr%
  \nobreak\sd@rule\hskip\sdstartspace%
  \sd@rule\hskip3.5\p@%
}
%    \end{macrocode}
%
% \end{macro}
%
% \subsubsection{Putting things in the right place}
%
% Syntax diagrams have fairly stiff requirements on the positioning of text
% relative to the diagram's rules.  To help people (and me) to write
% extensions to the syntax diagram typesetting which automatically put things
% in the right place, I provide some simple macros.
%
% \begin{environment}{sdbox}
%
% By placing some text in the \env{sdbox} environment, it will be read into a
% box and then output at the correct height for the syntax diagram.  Note
% that stuff in the box is set in horizontal (LR) mode, so you'll have to use
% a \env{minipage} if you want formatted text.  The macro also supplies rules
% on either side of the box, with a length given in the environment's
% argument.
%
% Macro writers are given explicit permission to use this environment through
% the |\sdbox| and |\endsdbox| commands if this makes life easier.
%
% The calculation in the |\endsdbox| macro works out how to centre the box
% vertically over the baseline.  If the box's height is~$h$, and its depth
% is~$d$, then its centre-line is $(h+d)/2$ from the bottom of the box.
% Since the baseline is already $d$ from the bottom, we need to lower the box
% by $(h+d)/2 - d$, or $h/2-d/2$.
%
%    \begin{macrocode}
\def\sdbox#1{%
  \@tempskipa#1\relax%
  \sd@gap\@tempskipa%
  \setbox\z@\hbox\bgroup%
    \begingroup%
    \catcode`\ 10%
    \catcode`\^^M5%
    \synshortsoff%
}
\def\endsdbox{%
    \endgroup%
  \egroup%
  \@tempdima\ht\z@%
  \advance\@tempdima-\dp\z@%
  \advance\@tempdima-\tw@\sd@mid%
  \lower.5\@tempdima\box\z@%
  \sd@gap\@tempskipa%
}
%    \end{macrocode}
%
% \end{environment}
%
% \subsubsection{Typesetting syntactic items}
%
% Using the hooks built into the syntax abbreviations above, we typeset
% the text into a box, and write it out, centred over the baseline.  A strut
% helps to keep the actual text baselines level for short pieces of text.
%
% \begin{macro}{\sd@tok@i}
%
% The preamble for a syntax abbreviation.  We start a box, and set the
% space and return characters to work again.  A strut is added to the box to
% ensure correct vertical spacing for normal text.
%
%    \begin{macrocode}
\def\sd@tok@i{%
  \sdbox\sdtokskip%
  \strut%
  \space%
}
%    \end{macrocode}
%
% \end{macro}
%
% \begin{macro}{\sd@tok@ii}
%
%    \begin{macrocode}
\def\sd@tok@ii{%
  \space%
  \endsdbox%
}
%    \end{macrocode}
%
% \end{macro}
%
% \subsubsection{Inserting other pieces of text}
%
% Arbitrary text may be put into a syntax diagram through the use of the
% |\tok| macro.  Its `argument' is typeset in the same way as a syntactic
% item (centred over the baseline).  The implementation goes to some effort
% to ensure that the text is not actually an argument, to allow category
% codes to change while the text is being typeset.
%
% \begin{macro}{\tok}
%
% We start a box, and make space and return do their normal jobs.  We use
% |\aftergroup| to regain control once the box is finished.  |\doafter| is
% used to get control after the group finishes.
%
%    \begin{macrocode}
\def\tok#{%
  \sdbox\sdtokskip%
  \strut%
  \enspace%
  \syntaxShortcuts\relax\relax%
  \doafter\sd@tok%
}
%    \end{macrocode}
%
% The |\sd@tok| macro is similar to |\sd@tok@ii| above.
%
%    \begin{macrocode}
\def\sd@tok{%
  \enspace%
  \endsdbox%
}
%    \end{macrocode}
%
% \end{macro}
%
% \subsubsection{The \protect\env{stack} environment}
%
% The \env{stack} environment is used to present alternatives in a syntax
% diagram.  The alternatives are separated by |\\| commands.
%
% \begin{macro}{\stack}
%
% The optional positioning argument is handled using \LaTeX's |\newcommand|
% mechanism.
%
%    \begin{macrocode}
\newcommand\stack[1][t]{%
%    \end{macrocode}
%
% First, we add some horizontal space.
%
%    \begin{macrocode}
  \sd@gap\sdmidskip%
%    \end{macrocode}
%
% We're within a complex construction, so we need to clear the |\ifsd@base|
% flag.
%
%    \begin{macrocode}
  \begingroup\sd@basefalse%
%    \end{macrocode}
%
% The top and bottom rows of the stack are different to the others, since
% the vertical rules mustn't extend all the way up the side of the item.
% The bottom row is handled separately by |\endstack| below.  The top row
% must be handled via a flag, |\ifsd@toplayer|.
%
% Initially, the flag must be set true.
%
%    \begin{macrocode}
  \sd@toplayertrue%
%    \end{macrocode}
%
% We set the |\\| command to separate the items in the |\halign|.
%
%    \begin{macrocode}
  \let\\\sd@stackcr%
%    \end{macrocode}
%
% The actual structure must be set in vertical mode, so we must place it
% in a box.  The position argument determines whether this must be a
% |\vbox| or a |\vtop|.  We also insert a bit of rounding if the options say
% we must.
%
%    \begin{macrocode}
  \if#1t%
    \let\@tempa\vtop%
    \sd@toptrue%
    \ifsd@round\llap{\sd@trcirc\kern\tw@\sdrulewidth}\fi%
  \else\if#1b%
    \let\@tempa\vbox%
    \sd@topfalse%
    \ifsd@round\llap{\sd@brcirc\kern\tw@\sdrulewidth}\fi%
  \else%
    \sd@err{Bad position argument passed to stack}%
           {The positioning argument must be one of `t' or `b'.  I%
            have^^Jassumed you meant to type `t'.}%
    \let\@tempa\vtop%
  \fi\fi%
%    \end{macrocode}
%
% Now we start the box, which we will complete at the end of the environment.
%
%    \begin{macrocode}
  \@tempa\bgroup%
%    \end{macrocode}
%
% We must remove any extra space between rows of the table, since the rules
% will not join up correctly.  We can use |\offinterlineskip| safely, since
% each individual row contains a strut.
%
%    \begin{macrocode}
  \offinterlineskip%
%    \end{macrocode}
%
% Now we can start the alignment.  We actually use \PlainTeX's |\ialign|
% macro, which also clears |\tabskip| for us.
%
%    \begin{macrocode}
  \ialign\bgroup%
%    \end{macrocode}
%
% The preamble is trivial, since we must do all of the work ourselves
%
%    \begin{macrocode}
    ##\cr%
%    \end{macrocode}
%
% We can now start putting the text into a box ready for typesetting later.
% The strut makes the vertical spacing correct.
%
%    \begin{macrocode}
  \setbox\z@\hbox\bgroup%
    \strut%
}
%    \end{macrocode}
%
% \end{macro}
%
% \begin{macro}{\endstack}
%
% The first part of this is similar to the |\sd@stackcr| macro below, except
% that the vertical rules are different.  We don't support rounded edges
% on single-row stacks, although this isn't a great loss to humanity.
%
%    \begin{macrocode}
\def\endstack{%
  \egroup%
  \ifsd@toplayer%
    \sd@dostack\sd@upper\sd@lower\relax\relax%
  \else%
    \ifsd@round%
      \ifsd@top%
        \sd@dostack{\ht\z@}\sd@botcirc\sd@blcirc\sd@brcirc%
      \else%
        \sd@dostack{\ht\z@}\sd@botcirc\relax\relax%
      \fi%
    \else%
      \sd@dostack{\ht\z@}\sd@lower\relax\relax%
    \fi%
  \fi%
%    \end{macrocode}
%
% We now close the |\halign| and the vbox we created.
%
%    \begin{macrocode}
  \egroup%
  \egroup%
%    \end{macrocode}
%
% Deal with any rounding we started off.
%
%    \begin{macrocode}
  \ifsd@round%
    \ifsd@top
      \rlap{\kern\tw@\sdrulewidth\sd@tlcirc}%
    \else%
      \rlap{\kern\tw@\sdrulewidth\sd@blcirc}%
    \fi%
  \fi%
%    \end{macrocode}
%
% Finally, we add some horizontal glue to space the diagram out.
%
%    \begin{macrocode}
  \endgroup\sd@gap\sdmidskip%
}
%    \end{macrocode}
%
% \end{macro}
%
% \begin{macro}{\sd@stackcr}
%
% The |\\| command is set to this macro during a \env{stack} environment.
%
%    \begin{macrocode}
\def\sd@stackcr{%
%    \end{macrocode}
%
% The first job is to close the box containing the previous item.
%
%    \begin{macrocode}
  \egroup%
%    \end{macrocode}
%
% Now we typeset the vertical rules differently depending on whether this is
% the first item in the stack.  This looks quite terrifying initially, but
% it's just an enumeration of the possible cases for the different values
% of |\ifsd@toplayer|, |\ifsd@top| and |\ifsd@round|, putting in appropriate
% rules and arcs in the right places.
%
%    \begin{macrocode}
  \ifsd@toplayer%
    \ifsd@round%
      \ifsd@top%
        \sd@dostack\sd@topcirc{\dp\z@}\relax\relax%
      \else%
        \sd@dostack\sd@topcirc{\dp\z@}\sd@tlcirc\sd@trcirc%
      \fi%
    \else%
      \sd@dostack\sd@upper{\dp\z@}\relax\relax%
    \fi%
  \else%
    \ifsd@round%
      \ifsd@top%
        \sd@dostack{\ht\z@}{\dp\z@}\sd@blcirc\sd@brcirc%
      \else%
        \sd@dostack{\ht\z@}{\dp\z@}\sd@tlcirc\sd@trcirc%
      \fi%
    \else%
      \sd@dostack{\ht\z@}{\dp\z@}\relax\relax%
    \fi%
  \fi%
%    \end{macrocode}
%
% The next item won't be the first, so we clear the flag.
%
%    \begin{macrocode}
  \sd@toplayerfalse%
%    \end{macrocode}
%
% Now we have to set up the next cell.  We put the text into a box again.
%
%    \begin{macrocode}
  \setbox\z@\hbox\bgroup%
    \strut%
}
%    \end{macrocode}
%
% \end{macro}
%
% \begin{macro}{\sd@dostack}
%
% Actually typesetting the text in a cell is performed here.  The macro is
% called as
% \begin{quote}\synshorts
% "\\sd@dostack{"<height>"}{"<depth>"}{"<left-arc>"}{"<right-arc>"}"
% \end{quote}
% where \<height> and \<depth> are the height and depth of the vertical
% rules to put around the item, and \<left-arc> and \<right-arc> are
% commands to draw rounded edges on the left and right hand sides of the
% item.
%
% The values for the height and depth are quite often going to be the height
% and depth of box~0.  Since we empty box~0 in the course of typesetting the
% row, we need to cache the sizes on entry.
%
%    \begin{macrocode}
\def\sd@dostack#1#2#3#4{%
  \@tempdima#1%
  \@tempdimb#2%
  \kern-\tw@\sdrulewidth%
  \vrule\@height\@tempdima\@depth\@tempdimb\@width\tw@\sdrulewidth%
  #3%
  \sd@rule\hfill%
  \sd@gap\sdtokskip%
  \unhbox\z@%
  \sd@gap\sdtokskip%
  \sd@rule\hfill%
  #4%
  \vrule\@height\@tempdima\@depth\@tempdimb\@width\tw@\sdrulewidth%
  \kern-\tw@\sdrulewidth%
  \cr%
}
%    \end{macrocode}
%
% \end{macro}
%
% \subsubsection{The \protect\env{rep} environment}
%
% The \env{rep} environment is used for typesetting loops in the diagram.
% Again, we use |\halign| for the typesetting.  Loops are simpler than
% stacks, however, since there are always two rows.  We store both rows in
% box registers, and build the loop at the end.
%
% \begin{macro}{\rep}
%
% Again, we use |\newcommand| to process the optional argument.
%
%    \begin{macrocode}
\newcommand\rep[1][t]{%
%    \end{macrocode}
%
% First, leave a gap on the left side.
%
%    \begin{macrocode}
  \sd@gap\sdmidskip%
%    \end{macrocode}
%
% We're not at base level any more, so disable linebreaking.
%
%    \begin{macrocode}
  \begingroup\sd@basefalse%
%    \end{macrocode}
%
% Remember we're going backwards now.
%
%    \begin{macrocode}
  \ifsd@backwards\sd@backwardsfalse\else\sd@backwardstrue\fi%
%    \end{macrocode}
%
% Define |\\| to separate the two parts of the loop.
%
%    \begin{macrocode}
   \let\\\sd@loop%
%    \end{macrocode}
%
% Now check the argument, and use the appropriate type of box.  In addition
% to changing the typesetting, we must remember which way up to typeset the
% loop, since the end code must always put the first argument on the
% baseline, with the loop either above or below.
%
%    \begin{macrocode}
  \if#1t%
    \let\@tempa\vbox%
    \sd@toptrue%
  \else\if#1b%
    \let\@tempa\vtop%
    \sd@topfalse%
  \else%
    \sd@err{Bad position argument passed to loop}%
           {The positioning argument must be `t' or `b'.  I have^^J%
            assumed you meant to type `t'.}%
    \let\@tempa\vbox%
    \sd@toptrue%
  \fi\fi%
%    \end{macrocode}
%
% Now we start the box.
%
%    \begin{macrocode}
  \@tempa\bgroup%
%    \end{macrocode}
%
% The loop is by default empty, apart from a strut.  This is put into box~1.
%
%    \begin{macrocode}
  \setbox\tw@\copy\strutbox%
%    \end{macrocode}
%
% Now start typesetting the main text in box~0.
%
%    \begin{macrocode}
  \setbox\z@\hbox\bgroup\strut%
}
%    \end{macrocode}
%
% \end{macro}
%
% \begin{macro}{\endrep}
%
% The final code must first close whatever box was open.
%
%    \begin{macrocode}
\def\endrep{%
  \egroup%
%    \end{macrocode}
%
% Now we typeset the loop, depending on which way up it was meant to be.
% Again, this terrifying piece of code is a simple list of possibile values
% of our various flags.
%
%    \begin{macrocode}
  \ifsd@top%
    \ifsd@round%
      \sd@doloop\tw@\z@\relax\relax%
        \sd@tlcirc\sd@trcirc{\sd@rlc\sd@blcirc}{\sd@llc\sd@brcirc}%
    \else%
      \sd@doloop\tw@\z@\relax\sd@downarr\relax\relax\relax\relax%
    \fi%
  \else%
    \ifsd@round%
      \sd@doloop\z@\tw@\relax\relax%
        {\sd@rlc\sd@tlcirc}{\sd@llc\sd@trcirc}\sd@blcirc\sd@brcirc%
    \else%
      \sd@doloop\z@\tw@\sd@uparr\relax\relax\relax\relax\relax%
    \fi%
  \fi%
%    \end{macrocode}
%
% Close the vbox we opened.
%
%    \begin{macrocode}
  \egroup%
%    \end{macrocode}
%
% Finally, we leave a gap before the next structure.
%
%    \begin{macrocode}
  \endgroup\sd@gap\sdmidskip%
}
%    \end{macrocode}
%
% \end{macro}
%
% \begin{macro}{\sd@loop}
%
% This macro handles the |\\| command within a loop environment.  We close
% the current box, and start filling in box~1.  We also redefine |\\| to
% raise an error when the |\\| command is used again.
%
%    \begin{macrocode}
\def\sd@loop{%
  \egroup%
  \def\\{\sd@err{Too many \string\\\space commands in loop}\@ehc}%
  \setbox\tw@\hbox\bgroup\strut%
}
%    \end{macrocode}
%
% \end{macro}
%
% \begin{macro}{\sd@doloop}
%
% This is the macro which actually creates the |\halign| for the loop.  It
% is called with four arguments, as:
% \begin{quote}\synshorts
% "\\sd@doloop{"<top-box>"}{"<bottom-box>"}"^^A
%                "{"<top-arrow>"}{"<btm-arrow>"}" \\
% \hbox{}\quad "{"<top-left-arc>"}{"<top-right-arc>"}"^^A
%                "{"<bottom-left-arc>"}{"<btm-right-arc>"}"^^A
% \kern-1in ^^A It may be overfull, but it looks OK to me ;-)
% \end{quote}
%
% The two \<box> arguments give the numbers of boxes to extract in the top
% and bottom rows of the alignment.  The \<arrow> arguments specify
% characters to typeset at the end of the top and bottom rows for arrows.
% The various \<arc> arguments are commands which typeset arcs around the
% various parts of the items.
%
% We calculate the height and depth of the two boxes, and store them in
% \<dimen> registers, because the boxes are emptied before the right-hand
% rules are typeset.
%
% Actually, the two rows of the alignment are typeset in a different macro:
% we just pass the correct information on.
%
%    \begin{macrocode}
\def\sd@doloop#1#2#3#4#5#6#7#8{%
  \@tempdima\dp#1\relax%
  \@tempdimb\ht#2\relax%
  \offinterlineskip%
  \ialign{%
    ##\cr%
    \ifsd@round%
      \sd@doloop@i#1#3\sd@topcirc\@tempdima{#5}{#6}%
      \sd@doloop@i#2#4\@tempdimb\sd@botcirc{#7}{#8}%
    \else%
      \sd@doloop@i#1#3\sd@upper\@tempdima{#5}{#6}%
      \sd@doloop@i#2#4\@tempdimb\sd@lower{#7}{#8}%
    \fi%
  }%
}
%    \end{macrocode}
%
% \end{macro}
%
% \begin{macro}{\sd@doloop@i}
%
% Here we do the actual job of typesetting the rows of a loop alignment.
% The four arguments are:
% \begin{quote}\synshorts
% "\\sd@doloop@i{"<box>"}{"<arrow>"}"^^A
%              "{"<rule-height>"}{"<rule-depth>"}" \\
% \hbox{}\quad "{"<left-arc>"}{"<right-arc>"}"^^A
% \end{quote}
%
% The arrow position is determined by the |\ifsd@backwards| flag.  The rest
% is fairly simple.
%
%    \begin{macrocode}
\def\sd@doloop@i#1#2#3#4#5#6{%
  \ifsd@backwards#2\fi%
  \kern-\tw@\sdrulewidth%
  \vrule\@height#3\@depth#4\@width\tw@\sdrulewidth%
  #5%
  \sd@rule\hfill%
  \sd@gap\sdtokskip%
  \unhbox#1%
  \sd@gap\sdtokskip%
  \sd@rule\hfill%
  #6%
  \vrule\@height#3\@depth#4\@width\tw@\sdrulewidth%
  \ifsd@backwards\else#2\fi%
  \kern-\tw@\sdrulewidth%
  \cr%
}
%    \end{macrocode}
%
% \end{macro}
%
% \subsection{The end}
%
% Phew!  That's all of it completed.  I hope this collection of commands
% and environments is of some help to someone.
%
%    \begin{macrocode}
%</package>
%    \end{macrocode}
%
% \hfill Mark Wooding, \today
%
% \Finale
%
\endinput
