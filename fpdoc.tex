%
%   $Id: fpdoc.tex,v 1.3 2005/04/28 20:52:54 michael Exp $
%   This file is part of the FPC documentation.
%   Copyright (C) 1997, by Michael Van Canneyt
%
%   The FPC documentation is free text; you can redistribute it and/or
%   modify it under the terms of the GNU Library General Public License as
%   published by the Free Software Foundation; either version 2 of the
%   License, or (at your option) any later version.
%
%   The FPC Documentation is distributed in the hope that it will be useful,
%   but WITHOUT ANY WARRANTY; without even the implied warranty of
%   MERCHANTABILITY or FITNESS FOR A PARTICULAR PURPOSE.  See the GNU
%   Library General Public License for more details.
%
%   You should have received a copy of the GNU Library General Public
%   License along with the FPC documentation; see the file COPYING.LIB.  If not,
%   write to the Free Software Foundation, Inc., 59 Temple Place - Suite 330,
%   Boston, MA 02111-1307, USA.
%
%%%%%%%%%%%%%%%%%%%%%%%%%%%%%%%%%%%%%%%%%%%%%%%%%%%%%%%%%%%%%%%%%%%%%%%
% Preamble.
\input{preamble.inc}
\latex{%
  \ifpdf
      \hypersetup{
           pdfauthor={Michael Van Canneyt},
           pdftitle={Users' Guide},
           pdfsubject={FPDoc Users' guide},
           pdfkeywords={Free Pascal}
           }
\fi}
%
% Settings
%
\makeindex
%
% Start of document.
%
\newcommand{\fpdoc}{\textsc{FPDoc}\xspace}
\newcommand{\seesu}[1]{section \ref{suse:#1}, page \pageref{suse:#1}\xspace}
\newcommand{\seetag}[1]{\tag{#1} (\pageref{tag:#1})\xspace}
\newcommand{\tag}[1]{\textbf{#1}}
\newcommand{\attr}[1]{\textit{#1}}
\usepackage{tabularx}
\usepackage{syntax}
\begin{document}
\title{FPDoc\\Free Pascal code documenter: Reference manual}
\docdescription{Reference manual for FPDoc}
\docversion{2.6}
\date{\today}
\author{Micha\"el Van Canneyt}
\maketitle
\tableofcontents

%%%%%%%%%%%%%%%%%%%%%%%%%%%%%%%%%%%%%%%%%%%%%%%%%%%%%%%%%%%%%%%%%%%%%
% Introduction
%%%%%%%%%%%%%%%%%%%%%%%%%%%%%%%%%%%%%%%%%%%%%%%%%%%%%%%%%%%%%%%%%%%%%
\chapter{Introduction}

%%%%%%%%%%%%%%%%%%%%%%%%%%%%%%%%%%%%%%%%%%%%%%%%%%%%%%%%%%%%%%%%%%%%%%%
% About this document
\section{About this document}
This is the reference manual for \fpdoc, a free documentation tool for 
Pascal units. It describes the usage of \fpdoc and how to write 
documentation with it. 

It attempts to be complete, but the tool is under continuous development,
and so there may be some slight differences between the documentation and
the actual program. In case of discrepancy, the sources of the tool are the
final reference. A \file{README} or \file{CHANGES} file may be provided, and
can also give some hints as to things which have changed. In case of doubt,
these files or the sources are authoritative.

%%%%%%%%%%%%%%%%%%%%%%%%%%%%%%%%%%%%%%%%%%%%%%%%%%%%%%%%%%%%%%%%%%%%%%%
% About FPDoc
\section{About \fpdoc}
\fpdoc is a tool that combines a Pascal unit file and a description file
in XML format and produces reference documentation for the unit. The
reference documentation contains documentation for all of the identifiers
found in the unit's interface section. The documentation is fully
cross-referenced, making it easy to navigate. It is also possible to refer
to other documentation sets written with \fpdoc, making it possible to
maintain larger documentation sets for large projects.

Contrary to some other documentation techniques, \fpdoc does not require the
presence of formatted comments in the source code. It takes a source file
and a documentation file (in XML format) and merges these two together to a
full documentation of the source. This means that the code doesn't get
obfuscated with large pieces of comment, making it hard to read and
understand. 

\fpdoc is package-oriented, which means that it considers units as part of a
package. Documentation for all units in a package can be generated in one
run.

At the moment of writing, the documentation can be generated in the
following formats:
\begin{description}
\item[HTML] Plain HTML. Javascript is used to be able to show a small window
with class properties or class methods, but the generated HTML will work
without JavaScript as well. Style sheets are used to do the markup, so the
output can be customised.
\item[XHTML] As HTML, but using a more strict syntax.
\item[LaTeX] LaTeX files, which can be used with the \file{fpc.sty} file
which comes with the \fpc documentation. From this output, PDF documents can
be generated, and with the use of latex2rtf, RTF or Winhelp files. Text
files can also be generated.
\item[Text] plain ascii text files. No cross-referencing exists. Other than
that it resembles the LaTeX output in it's structure.
\item[Man] Unix man pages. Each function/procedure/method identifier is a man
page. Constants are on a separate page, as are types, variables and
resourcestrings.
\item[CHM] HTML files compressed into a .chm file using lzx compression.
\item[RTF] Linear RTF files.
\end{description}

Note that pascal program files cannot be parsed by fpdoc, it is therefor 
impossible to create documentation for a program. Only units are supported.

%%%%%%%%%%%%%%%%%%%%%%%%%%%%%%%%%%%%%%%%%%%%%%%%%%%%%%%%%%%%%%%%%%%%%%%
% Getting more information.
\section{Getting more information.}
If the documentation doesn't give an answer to your questions,
you can obtain more information on the Internet, on the Free Pascal Website: 

\htmladdnormallink{http://www.freepascal.org/}
{http://www.freepascal.org} 

It contains links to download all \fpdoc related material.

Finally, if you think something should be added to this manual
(entirely possible), please do not hesitate and contact me at
\htmladdnormallink{michael@freepascal.org}{mailto:michael@freepascal.org}.

%%%%%%%%%%%%%%%%%%%%%%%%%%%%%%%%%%%%%%%%%%%%%%%%%%%%%%%%%%%%%%%%%%%%%
% Installation
%%%%%%%%%%%%%%%%%%%%%%%%%%%%%%%%%%%%%%%%%%%%%%%%%%%%%%%%%%%%%%%%%%%%%
\chapter{Compiling and Installing \fpdoc}
\label{ch:Installation}

%%%%%%%%%%%%%%%%%%%%%%%%%%%%%%%%%%%%%%%%%%%%%%%%%%%%%%%%%%%%%%%%%%%%%%%
% Compiling
\section{Compiling}
In order to compile \fpdoc, the following things are needed:
\begin{enumerate}
\item The fpdoc sources. These can be downloaded from the \fpdoc website.
\item The \fpc compiler sources. \fpdoc uses the scanner from the \fpc
compiler to scan the source file.
\item The FCL units (or their sources) should be installed. 
\item fpcmake is needed to create the makefile for fpdoc. It comes with
\fpc, so if \fpc is installed, there should be no problem.
\item To make new internationalisation support files, \file{rstconv} must be
installed, and the GNU gettext package.
\end{enumerate}
Links to download all these programs can be found on the \fpdoc website.

When the fpdoc sources have been unzipped, the Makefile must be generated.
Before generating the makefile, the location of the compiler source
directory should be indicated. In the \file{Makefile.fpc} file, which has a 
windows ini file format, locate the \var{fpcdir} entry in the \var{defaults} 
section:
\begin{verbatim}
fpcdir=../..
\end{verbatim}
and change it so it points to the top-level \fpc source directory.

After that, running \file{fpcmake} will produce the \file{Makefile}, and
running \file{make} should produce 2 executables: \file{fpdoc} and
\file{makeskel}.

%%%%%%%%%%%%%%%%%%%%%%%%%%%%%%%%%%%%%%%%%%%%%%%%%%%%%%%%%%%%%%%%%%%%%%%
% Installation
\section{Installation}

When installing from sources, a simple 
\begin{verbatim}
make install
cd intl
make install
\end{verbatim}
should completely install the documentation tool.

When installing from a archive with the binaries, it should be sufficient
to copy the binaries to a directory in the \var{PATH}.

To have fpdoc available in several languages, the language files should be
installed in the following directory on Unix systems:
\begin{verbatim}
/usr/local/share/locale/XX/LC_MESSAGES/
\end{verbatim}
or
\begin{verbatim}
/usr/share/locale/XX/LC_MESSAGES/
\end{verbatim}
Depending on the setup. Here \var{XX} should be replaced by the locale 
identifier.

%%%%%%%%%%%%%%%%%%%%%%%%%%%%%%%%%%%%%%%%%%%%%%%%%%%%%%%%%%%%%%%%%%%%%
% Usage
%%%%%%%%%%%%%%%%%%%%%%%%%%%%%%%%%%%%%%%%%%%%%%%%%%%%%%%%%%%%%%%%%%%%%
\chapter{\fpdoc usage}
\label{ch:usage}

%%%%%%%%%%%%%%%%%%%%%%%%%%%%%%%%%%%%%%%%%%%%%%%%%%%%%%%%%%%%%%%%%%%%%
% Usage
\section{fpdoc}
Using \fpdoc is quite simple. It takes some command-line options, and based
on these options, creates documentation. The command-line options can be
given as long or short options, as is common for most GNU programs.

In principle, only 2 command-line options are needed:
\begin{description}
\item[package] This specifies the name of the package for which
documentation must be created. Exactly one package option can be 
specified.
\item[input] The name of a unit file for which documentation should be
generated. This can be a simple filename, but can also contain some syntax
options as they could be given to the \fpc scanner. More than one
\var{input} option can be given, and documentation will be generated for all
specified input files.
\end{description}

Some examples:
\begin{verbatim}
fpdoc --package=fcl --input=crt.pp
\end{verbatim}
This will scan the \file{crt.pp} file and generate documentation for it
in a directory called \file{fcl}.
\begin{verbatim}
fpdoc --package=fcl --input='-I../inc -S2 -DDebug classes.pp'
\end{verbatim}
This will scan the file \file{classes.pp}, with the \var{DEBUG} symbol
defined, the scanner will look for include files in the \file{../inc}
directory, and \var{OBJFPC}-mode syntax will be accepted.

(for more information about these options, see the \fpc compiler user's
guide)

With the above commands, a set of documentation files will be generated in
HTML format (this is the standard). There will be no description of any of
the identifiers found in the unit's interface section, but all identifiers
declarations will be present in the documentation. 

The actual documentation (i.e. the description of each of the identifiers) 
resides in a description file, which can be specified with the \var{descr} 
option:

\begin{verbatim}
fpdoc --package=fcl --descr=crt.xml --input=crt.pp
\end{verbatim}
This will scan the \file{crt.pp} file and generate documentation for it,
using the descriptions found in the file{crt.xml} file. The documentation
will be written in a directory called \file{fcl}.
\begin{verbatim}
fpdoc --package=fcl --descr=classes.xml \
      --input='-I../inc -S2 -DDebug classes.pp'
\end{verbatim}
All options should be given on one line.
This will scan the file \file{classes.pp}, with the \var{DEBUG} symbol
defined, the scanner will look for include files in the \file{../inc}
directory, and \var{OBJFPC}-mode syntax will be accepted.

More than one input file or description file can be given:
\begin{verbatim}
fpdoc --package=fcl --descr=classes.xml --descr=process.xml \
      --input='-I../inc -S2 -DDebug classes.pp' \
      --input='-I../inc -S2 -DDebug process.pp'
\end{verbatim}
Here, documentation will be generated for 2 units: \file{classes}
and \var{process}

The format of the description file is discussed in the next chapter.

Other formats can be generated, such as latex:
\begin{verbatim}
fpdoc --format=latex --package=fcl \
      --descr=classes.xml --descr=process.xml\
      --input='-I../inc -S2 -DDebug classes.pp' \
      --input='-I../inc -S2 -DDebug process.pp'
\end{verbatim}
This will generate a LaTeX file called \file{fcl.tex}, which contains the
documentation of the units \file{classes} and \var{process}. The latex file
contains no document preamble, it starts with a chapter command. 
It is meant to be included (using the LaTeX include command) in a latex 
document with a preamble.

The output of \fpdoc can be further customised by several command-line
options, which will be explained in the next section.

%%%%%%%%%%%%%%%%%%%%%%%%%%%%%%%%%%%%%%%%%%%%%%%%%%%%%%%%%%%%%%%%%%%%%
% Usage
\section{\fpdoc command-line options reference}
In this section all \fpdoc command-line options are explained.

% auto-index
\subsection{auto-index}
This option generates an index of all the types including objects, 
classes, their methods and enums, sorted alphabetically. Methods 
of classes and objects will appear as subitems of their class. If used
in combination with --index-file this option will be used instead. 
This option only applies to the chm backend.

% auto-toc
\subsection{auto-toc}
This option generates a Table of Contents that displays all classes,
objects and routines in several ways. If used in combination with
--toc-file this option will be used instead. This option only applies 
to the chm backend.

% charset
\subsection{charset}
\label{suse:charset}
This option can be used to specify the character set to be used for
the HTML backend. It is simply inserted in the HTML tree. The default
character set is \var{iso-8859-1}

For example
\begin{verbatim}
--charset=UTF8
\end{verbatim}
will result in an UTF8 specification of the \attr{content} attribute of 
the \tag{meta} tag in the generated HTML file:
\begin{verbatim}
<meta content="text/html; charset=UTF8" http-equiv="Content-Type">
\end{verbatim}

% chm-title
\subsection{chm-title}
\label{suse:chm-title}
This option is to specify a nicer title than the default title which is
the same value as --package=value.

% content
\subsection{content}
\label{suse:content}
This option tells  \fpdoc to generate a content file. 
A content file contains a list of all the
possible anchors (labels) in the generated documentation file, and can be
used to create cross-links in documentation for units in other packages, 
using the counterpart of the content option, the \var{import} option
(\seesu{import}).

% css-file
\subsection{css-file}
\label{suse:css-file}
This option only applies to the chm backend. Use this option to set the css
file used to style the html when the html is rendered by a viewer.

% default-page
\subsection{default-page}
\label{suse:default-page}
This option only applies to the chm backend. The default page to load 
when the chm is opened by a viewer. Using this option does not include a file
but instead specifies a file you have included using --other-files. You only 
need to use this option if you have used the --other-files option and included 
a page you wish to use instead of the default page.

%descr
\subsection{descr}
\label{suse:descr}
This option specifies the name of a description file that contains the 
actual documentation for the unit. This option can be given several 
times, for several description files. The file will be searched relative to
the current directory. No extension is added to the file, it should be a
complete filename.

If the filename starts with an 'at' sign \var{@}, then it is interpreted
as a text file which contains a list of filenames, one per line. 
Each of these files will be added to the list of description files.

The nodes in the description files will be merged into one big tree. This
means that the documentation can be divided over multiple files. When
merging the description files, nodes that occur twice will end up only once
in the big node tree: the last node will always be the node that ends up in
the parse tree. This means that the order of the various input commands or
the ordering of the files in the file list is important.

Examples:
\begin{verbatim}
--descr=crt.xml 
\end{verbatim}
will tell \fpdoc to read documentation from \file{crt.xml}, while
\begin{verbatim}
--descr=@fcl.lst 
\end{verbatim}
will tell \fpdoc to read filenames from \file{fcl.lst}; each of the
filenames found in it will be added to the list of files to be scanned for
descriptions.

%footer
\subsection{emit-notes}
\label{suse:emit-notes}
This option tells the documentation engine to include notes in the text.
By default, notes are not included in the text as they serve as working 
notes for the documenter.

%footer
\subsection{footer}
\label{suse:footer}
This option tells the HTML engine to include the file indicated by this
option as the footer of each generated HTML page. The file is assumed to
contain a valid XHTML fragment. The contents of the file will be inserted 
in the HTML tree.

Example:
\begin{verbatim}
--footer=footer.xml
\end{verbatim}
Where \file{footer.xml} contains for example:
\begin{verbatim}
<hr/>
<i>Date generated : June 1, 2008</i>
\end{verbatim}

This option is mutually exclusive with the \var{footer-date} option.

%footer
\subsection{footer-date}
\label{suse:footerdate}
This option tells the HTML engine to generate a footer for each page
containing the current date. The optional format argument can be used
to specify the format of the date. The format must be a formatting string
as understood by the \var{FormatDateTime} function in the \file{sysutils} 
unit.

For example
\begin{verbatim}
--footer-date='mmm dd yyyy'
\end{verbatim}

This option is mutually exclusive with the \var{footer} option.

%format
\subsection{format}
\label{suse:format}
Specifies the output format in which the documentation will be generated. 
Currently, the following formats are known:
\begin{description}
\item[htm] Plain HTML with 8.3 conforming filenames. 
\item[html] HTML with long filenames.
\item[xhtml] XHTML with long filenames.
\item[latex] LaTex, which uses the \file{fpc.sty} style used by the \fpc
documentation. 
\item[xml-struct] Structured XML.
\item[chm] Compressed HTML.
\end{description}

% help
\subsection{help}
Gives a short copyright notice.

% hide-protected
\subsection{hide-protected}
\label{suse:hideprotected}
By default, the documentation will include descriptions and listings of
protected fields and methods in classes or objects. This option changes this
behaviour; if it is specified, no documentation will be generated for these
methods. Note that public methods or properties that refer to these
protected methods will then have a dangling (i.e. unavailable) link.

% html-search
\subsection{html-search}
\label{suse:htmlsearch}
This option can be used when generating HTML documentation, to specify an
url that can be used to search in the generated documentation. The URL will be
included in the header of each generated page with a \var{Search} caption.
The option is ignored for non-html output formats.

\fpdoc does not generate a search page, this should be made by some external
tool. Only the url to such a page can be specified.

Example:
\begin{verbatim}
--html-search=../search.html
\end{verbatim}

% image-dir
\subsection{image-dir}
\label{suse:imagedir}
This option tells the \latex backend where the images are located. By
default, a directory \file{images} is assumed. A path using forward
slashes must be provided.

Example:
\begin{verbatim}
--image-dir=pictures/
\end{verbatim}

% image-dir
\subsection{image-url}
\label{suse:imageurl}
This option tells the HTML backend where the images are located: the
argument must be an absolute URL. By default, the images are assumed to be
in a \var{images} directory below the package directory. Note that the URL 
must be absolute, i.e. if it is provided, the link to the image will be 
composed from the argument of this option with the filename.

Example:
\begin{verbatim}
--image-url=http://www.mysite.org/docs/pictures/
\end{verbatim}


% import
\subsection{import}
\label{suse:import}
Import a table of contents file, generated by \fpdoc for another package
with the \var{content} option (\seesu{content}). This option can be used 
to refer to documentation nodes in documentation sets for other packages. 
The argument consists of two parts: a filename, and a link prefix. 

The filename is the name of the file that will be imported. The link
prefix is a prefix that will be made to each HTML link; this needs to be
done to be able to place the files in different directories.

Example:
\begin{verbatim}
--import=../fcl.cnt,../fcl
\end{verbatim}
This will read the file fcl.cnt in the parent directory. For HTML
documentation, all links to items in the fcl.cnt file, the link will be
prepended with \file{../fcl}.

This allows a setup where all packages have their own subdirectory of a
common documentation directory, and all content files reside in the main
documentation directory, as e.g. in the following directory tree:
\begin{verbatim}
/docs/fcl
     /fpdoc
     /fpgui
     /fpgfx
     /fpimg
\end{verbatim}
The file fcl.cnt would reside in the \file{docs} directory. Similarly, for
each package a contents file \file{xxx.cnt} could be places in that
directory. Inside the subdirectory, commands as the above could be used to
provide links to other documentation packages.

Example:
\begin{verbatim}
--import=../fcl.cnt,ms-its:fcl.chm::/
\end{verbatim}
As in the previous example this will read the file fcl.cnt in the parent 
directory. But all links to items in the fcl.cnt file, will be prepended 
with \file{ms-its:fcl.chm::/}. This is how chm's are crosslinked.

Note that for Latex documentation, this option is ignored.

% index-colcount
\subsection{index-colcount}
\label{suse:index-file}
For the HTML backend, this option can be used to specify the number of
columns that should be used when generating an identifier index page.
By default, 3 columns are used.

Example:
\begin{verbatim}
--index-colcount=4
\end{verbatim}
Will use 4 columns instead.

% index-file
\subsection{index-file}
\label{suse:index-file}
Use this option to specify a local file to include and use as the index. This 
file is in the sitemap format. You can use the unit chmsitemap in your 
programs to read and write this type of file. Usually this file ends with .hhk. 
This option only applies to the chm backend.

% input
\subsection{input}
\label{suse:input}
This option tells \fpdoc what input file should be used. The argument can 
be just a filename, but can also be a complete compiler command-line with 
options that concern the scanning of the Pascal source: defines, include 
files, syntax options, as they would be specified to the \fpc compiler 
when compiling the file. If a complete command is used, then it should be 
enclosed in single or double quotes, so the shell will not break them in 
parts.

It is possible to specify multiple input commands; they will be treated one
by one, and documentation will be generated for each of the processed files.

% lang
\subsection{lang}
\label{suse:lang}
Select the language for the generated documentation. This will change all
header names to the equivalent in the specified language. The documentation
itself will not be translated, only the captions and headers used in the
text.

Currently, valid choices are
\begin{description}
\item[de] German.
\item[fr] French.
\item[nl] Dutch.
\end{description}

Example:
\begin{verbatim}
--lang=de
\end{verbatim}
Will select German language for headers.

The language files should be installed correctly for this option to work.
See the section on installing to check whether the languages are installed
correctly.

% latex-highlight
\subsection{latex-highlight}
\label{suse:latexhighlight}
Switches on an internal latex syntax highlighter. This is not yet
implemented. By default, syntax highlighting is provided by the syntax
package that comes with \fpc.

% make-searchable
\subsection{make-searchable}
This option generates an index of all the *.htm* files added in the chm, 
including files added with the --other-files option so that a full text search
is possible. This option only applies to the chm backend.

% other-files
\subsection{other-files}
\label{suse:other-files}
This option specifies a text file with one filename per line to be included in
the chm. You can include any type of file you want; it does not have to be a
html file. For instance you can include images that you have linked to in the
xml descr files. The files should be in a subfolder of the working directory 
and within their own folder to avoid naming conflicts with the auto generated
files. This option only applies to the chm backend.

% output
\subsection{output}
\label{suse:output}
This option tells  \fpdoc where the output file should be generated. 
How this option is interpreted depends on the format that is used. 
For latex, this is interpreted as the filename for the tex file. 
For chm, this is interpreted as the output filename.
For all other formats, this is interpreted as the directory where all 
documentation files will be written. The directory will be created if 
it does not yet exist.

The filename or directory name is interpreted as relative to the current
directory.

Example:
\begin{verbatim}
--format=html --output=docs/classes
\end{verbatim}
will generate HTML documentation in the directory \file{docs/classes}.
\begin{verbatim}
--format=latex --output=docs/classes.tex
\end{verbatim}
will generate latex documentation in the file \file{docs/classes}.
\begin{verbatim}
--format=chm --output=docs.chm
\end{verbatim}
will generate chm documentation in the file \file{docs.chm}.

% package
\subsection{package}
\label{suse:package}
This option specifies the name of the package to be used. The package name
will also be used as a default for the \var{output} option (\seesu{output}).

Note that if a project file is specified with multiple packages, this option
will be used to select the package for which to produce documentation.

% package
\subsection{project}
\label{suse:project}
\fpdoc can work with a project file. This is an XML file which can contain
specifies global options, and which contains the definition for one or more
packages, for which it lists input files and description files etc.
The project file is described in more detail in \seec{projectfile}.

This option expects an argument, the name of the project file to use.

Example:
\begin{verbatim}
fpdoc --project=rtl-project.xml --format=html --package=rtl
\end{verbatim}

% show-private
\subsection{show-private}
\label{suse:showprivate}
By default, no documentation is generated for private methods or fields of
classes or objects. This option causes \fpdoc to generate documentation 
for these methods and fields as well.

% toc-file
\subsection{toc-file}
Use this option to specify a local file to include as Table of Contents.
The table of contents is in the sitemap format and usually ends in .hhc. You
can use the unit chmsitemap in your programs to read and write files of this type.
This option only applies to the chm backend of \fpdoc.

% warn-no-node
\subsection{warn-no-node}
\label{suse:warnnonode}
If this option is given, then fpdoc will emit a warning if it cannot find a
documentation node for some identifier. This can be used to see whether the
description files are up-to-date, or whether they must be updated.

% package
\subsection{write-project}
\label{suse:write-project}
\fpdoc can create a project file based on all options given on the command
line. For more information about the project file, \seec{projectfile}.

This option expects an argument, the name of the project file to write.

Example:
\begin{verbatim}
fpdoc --write-project=rtl-project.xml --format=html --package=rtl
--input=file.pp --descr=file.xml
\end{verbatim}


%%%%%%%%%%%%%%%%%%%%%%%%%%%%%%%%%%%%%%%%%%%%%%%%%%%%%%%%%%%%%%%%%%%%%
% Makeskel
\section{makeskel}
\label{se:makeskel}

% Introduction
\subsection{introduction}
\label{suse:makeskelintro}
The makeskel tool can be used to generate an empty description file 
for a unit. The description file will contain an element node for each
identifier in the interface section of the Pascal unit. 

It's usage is quite straightforward: the name of an input file 
(one or more) must be specified (as for \fpdoc), an output file,
and the name of a package:
\begin{verbatim}
makeskel --package=rtl --input=crt.pp --output=crt.xml
\end{verbatim}
This will read the file \file{crt.pp} and will create a file \file{crt.xml}
which contains empty nodes for all identifiers found in \file{crt.pp}, all
in a package named \var{rtl}.

The \var{input} option can be given more than once, as for the \file{fpdoc}
command:
\begin{verbatim}
makeskel --input='-Sn system.pp' --input=crt.pp --output=rtl.xml
\end{verbatim}
As can be seen, the \var{input} option can contain some compiler options,
as is the case for \fpdoc. The above command will process the files
\file{system.pp} and \var{crt.pp}, and will create \tag{element} tags 
for the identifiers in both units in the file \var{rtl.xml}.

The output of \file{makeskel} is a valid, empty description file. It will
contain a \tag{module} tag for each unit specified, and each \tag{module}
will have \tag{element} tags for each identifier in the unit. 

Each \tag{element} tag will by default contain \tag{short}, \tag{descr},
\tag{errors} and \tag{seealso} tags, but this can be customised.

%%%%%%%%%%%%%%%%%%%%%%%%%%%%%%%%%%%%%%%%%%%%%%%%%%%%%%%%%%%%%%%%%%%%%
% Makeskel
\section{Makeskel option reference}
\label{se:makeskeloption}
The output of \file{makeskel} can be customised using several options, which
are discussed below.

% descr
\subsection{descr}
%\subsection{suse:descr}
When in update mode (\seesu{update}), this option can be used to add an
existing documentation file, as for \file{fpdoc}. Nodes that are already 
in one of the existing documentation files will not be written to the 
output file.

% disable-arguments
\subsection{disable-arguments}
\label{suse:msdisablearguments}
By default, for each function or procedure argument, a \tag{element} tag will
be generated. This option disables this behaviour: no \tag{element} tags
will be generated for procedure and function arguments.

% disable-errors
\subsection{disable-errors}
\label{suse:msdisableerrors}
If this option is specified, no \tag{errors} tag will be generated in the
element nodes. By default all element tags contain a \tag{errors} node.

The \tag{errors} tag is ignored when it is not needed; Normally, an 
\tag{errors} tag is only needed for procedure and function elements.

% disable-function-results
\subsection{disable-function-results}
\label{suse:disablefunctionresults}
If this option is specified, then no \tag{element} tag will be generated for
function results. By default, \file{makeskel} will generate a result node
for each function in the interface section. The result node is used in the
documentation to document the return value of the function under a separate
heading in the documentation page. Specifying this option suppresses the
generation of the \tag{element} tag for the function result.

% disable-private
\subsection{disable-override}
\label{suse:disableoverride}
If this option is specified, then no \tag{element} tags will be generated
for methods that override a method in a parent class. This means that the
\file{fpdoc} engine will refer to the parent implementation for the documentation
of the inherited method (provided it can find a parent implementation).


% disable-private
\subsection{disable-private}
\label{suse:disableprivate}
If this option is specified, then no \tag{element} tags will be generated
for private methods or fields of classes or objects. 
The default behaviour is to generate nodes for private methods or fields.
It can be used to generate a skeleton for end-user and developer
documentation.

% disable-protected
\subsection{disable-protected}
\label{suse:disableprotected}
If this option is specified, then no \tag{element} tags will be generated
for protected and private methods or fields of classes or objects. 
The default is to generate nodes for protected methods or fields. If this
option is given, the option \var{--disable-private} is implied. It can be
used to generate end-user-only documentation for classes. 


% disable-seealso
\subsection{disable-seealso}
\label{suse:msdisableseealso}
If this option is specified, no \tag{seealso} tag will be generated in the
element nodes. By default all \tag{element} tags contain a \tag{seealso} tag.


% emitclassseparator
\subsection{emitclassseparator}
\label{suse:msemitclassseparator}
When this option is specified, at the beginning of the elements for the
documentation of a class, a comment tag is emitted which contains a
separator text. This can be useful to separate documentation of different
classes and make the description file more understandable.

% help
\subsection{help}
\label{suse:mshelp}
\file{Makeskel} emits a short copyright notice and exits when this option is
specified.

% input
\subsection{input}
\label{suse:msinput}
This option is identical in meaning and functionality as the \var{input}
option for \fpdoc. (\seesu{input}) It specifies the Pascal unit source 
file that will be scanned and for which a skeleton description file will be
generated. Multiple \var{input} options can be given, and \tag{element}
tags will be written for all the files, in one big output file.

% lang
\subsection{lang}
\label{suse:mslang}
This option is used to specify the language for messages emitted by
\file{makeskel}. The supported languages are identical to the ones 
for \fpdoc:
\begin{description}
\item[de] German.
\item[fr] French.
\item[nl] Dutch.
\end{description}

% output
\subsection{output}
\label{suse:msoutput}
This option specifies the name of the output file. A full filename must be
given, no extension will be added. If this option is omitted, the output
will be sent to standard output.

When using update mode, the output file name should not appear in the list of
existing documentation files. The \file{makeskel} program will do some
elementary checks on this.

% package
\subsection{package}
\label{suse:mspackage}
This option specifies the package name that will be used when generating the
skeleton. It is a mandatory option. 

% update
\subsection{update}
\label{suse:update}
This option tells makeskel to create an update file: it will read
description files (\seesu{descr}) and will only create documentation nodes
for identifiers which do not yet have a documentation node in the read
documentation files. The output file in this case can be merged with one (or
more) of the documentation files: it's name should not appear in the list of
existing documentation files. The \file{makeskel} program will do some
elementary checks on this.

%%%%%%%%%%%%%%%%%%%%%%%%%%%%%%%%%%%%%%%%%%%%%%%%%%%%%%%%%%%%%%%%%%%%%
% The description file.
%%%%%%%%%%%%%%%%%%%%%%%%%%%%%%%%%%%%%%%%%%%%%%%%%%%%%%%%%%%%%%%%%%%%%
\chapter{The project file}
\label{ch:projectfile}

%%%%%%%%%%%%%%%%%%%%%%%%%%%%%%%%%%%%%%%%%%%%%%%%%%%%%%%%%%%%%%%%%%%%%%%
% Introduction
\section{Introduction}
When there are a lot of units to document, the command-line can be rather
long. On some operating systems, this is a problem, as the length of the
command-line easily exceeds the maximum length of command-line options
supported by the OS. Also, command-lines are difficult to read and/or 
parse. fpdoc has always supported reading options from a file, but 
this is not very structured and difficult to handle in e.g. an IDE. 

Therefor it is possible to write a project file. The project file is
an XML file that describes an fpdoc project. The XML file contains only 
a few tag names. 

In general, it looks as follows:

\begin{verbatim}
<?xml version="1.0" encoding="utf-8"?>
<docproject>
  <options>
    <option name="ostarget" value="Linux"/>
    <option name="cputarget" value="x86_64"/>
    <option name="show-private" value="false"/>
    <option name="stop-on-parser-error" value="false"/>
  </options>
  <packages>
    <package name="rtl" output="rtl" content="rtl.xct">
      <units>
        <unit file="../rtl/objpas/objpas.pp" 
              options="-dHASINTF"/>
        <unit file="../rtl/objpas/types.pp" 
              options=""/>
      </units>
      <descriptions>
        <description file="rtl.xml"/>
        <description file="system.xml"/>
      </descriptions>
    </package>
  </packages>
</docproject>
\end{verbatim}

%%%%%%%%%%%%%%%%%%%%%%%%%%%%%%%%%%%%%%%%%%%%%%%%%%%%%%%%%%%%%%%%%%%%%%%
% Introduction
\section{Project file structure}

The top node is always a \tag{docproject} node. This node can contain only
2 nodes: a \tag{options} node and a \tag{packages} node; All other nodes
will be ignored.

% Options
\subsection{The options node}
\label{project:options}
The \tag{options} tag that appears below the \tag{docproject} node specifies
the options for the \fpdoc project. It contains a series of \tag{option} tags. 
No other nodes should appear under the \tag{options} tag.

Example:
\begin{verbatim}
<options>
  <option name="ostarget" value="Linux"/>
  <option name="cputarget" value="x86_64"/>
  <option name="hide-protected" value="true"/>
</options>
\end{verbatim}

% Option
\subsection{The option node}
\label{project:option}
Each \tag{option} tag under the \tag{options} tag specifies an option.

Each \tag{option} tag has 2 attributes: \attr{name} and \var{value}. 
The name correspond to the command-line options that fpdoc supports.
This attribute is mandatory. The \attr{value} attribute contains the 
value for the node. It is also mandatory.

For command-line switches, only 2 values are accepted: \var{true} and
\var{false}.  

Both general options and format-specific options can be specified this way.

Example:
\begin{verbatim}
<option name="ostarget" value="Linux"/>
<option name="cputarget" value="x86_64"/>
<option name="hide-protected" value="true"/>
<option name="warn-no-node" value="false"/>
\end{verbatim}

% Packages
\subsection{The packages node}
\label{project:packages}
The \tag{packages} tag under the \tag{docproject} tag contains a
\tag{package} tag for each package described by the project file.
As many packages as needed can be described by the project file.
All other tags under the packages tag are ignored.

Example:
\begin{verbatim}
<packages>
  <package name="rtl" output="rtl" content="rtl.xct">
</packages>
\end{verbatim}

% Package
\subsection{The package node}
\label{project:package}
Each \tag{package} tag under the \tag{packages} tag describes one package.
It contains at most 2 tags: the \tag{units} tag and the \var{descriptions} tag.

The \tag{package} tag has 3 attributes:
\begin{description}
\item[name] This attribute specifies the name of the package. This attribute is mandatory.
\item[output] This attribute contains the default output name for the package. This attribute is
optional.
\item[content] The \attr{content} attribute specifies the name of the content file to write 
for this package. This attribute is optional. No file will be written if it is omitted.
\end{description}
Example:
\begin{verbatim}
<package name="rtl" output="rtl" content="rtl.xct">
\end{verbatim}

% Units
\subsection{The units node}
\label{project:units}
The \var{units} tag below the \tag{package} tag describes the input files for 
the package. It consists of a series of \var{unit} tags, one tag for each unit.

Example:
\begin{verbatim}
<units>
  <unit file="../rtl/linux/system.pp" options="-dfpdocsystem"/>
  <unit file="../rtl/objpas/objpas.pp" options="-dHASINTF"/>
</units>
\end{verbatim}

% Unit
\subsection{The unit node}
\label{project:unit}
The \var{unit} tag describes one input file for the package. 
This tag corresponds to the \var{--input} command-line option.

The tag has 2 attributes: 
\begin{description}
\item[file] The \attr{file} tag contains the file name. This attribute is
mandatory.
\item[options] The \attr{options} attribute is optional, and specifies 
any options that must be passed on to the parser when parsing the input file.
\end{description}

Example:
\begin{verbatim}
<unit file="cthreads.pp" options="-S2"/>
\end{verbatim}

% Descriptions
\subsection{The descriptions node}
\label{project:descriptions}
The \tag{descriptions} tag below the \tag{package} tag contains one 
\tag{description} tag for each description file belonging to the package.
No other tags should appear below the \tag{descriptions} tag.

Example:
\begin{verbatim}
<descriptions>
  <description file="rtl.xml"/>
  <description file="system.xml"/>
</descriptions>
\end{verbatim}


% Description
\subsection{The description node}
\label{project:description}
The \tag{description} tag describes one description file used in the
documentation for the package. It has one attribute \attr{file}, containing
the filename of the documentation file. This attribute is mandatory. All
other attributes are ignored.

Example:
\begin{verbatim}
<description file="cthreads.xml"/>
\end{verbatim}


%%%%%%%%%%%%%%%%%%%%%%%%%%%%%%%%%%%%%%%%%%%%%%%%%%%%%%%%%%%%%%%%%%%%%
% The description file.
%%%%%%%%%%%%%%%%%%%%%%%%%%%%%%%%%%%%%%%%%%%%%%%%%%%%%%%%%%%%%%%%%%%%%
\chapter{The description file}
\label{ch:descriptionfile}

%%%%%%%%%%%%%%%%%%%%%%%%%%%%%%%%%%%%%%%%%%%%%%%%%%%%%%%%%%%%%%%%%%%%%%%
% Introduction
\section{Introduction}
The description file is a XML document, which means that it is a kind of
HTML or SGML like format, however it is more structured than HTML, making it
easier to parse - and makes it easier to connect or merge it with a Pascal
source file. Since the allowed syntax uses a lot of HTML tags, this makes 
it easy to write code for those that are familiar with writing HTML.

More information about the XML format, SGML and HTML can be found on the 
website of the W3 (World Wide Web) consortium:  
\htmladdnormallink{http://www.w3.org/}{http://www.w3.org} 

The remaining of this chapter assumes a basic knowledge of tags, their
attributes and markup language, so these terms will not be explained here. 

The minimal documentation file would look something like this:
\begin{verbatim}
<?xml version="1.0" encoding="ISO-8859-1"?>
<fpdoc-description>
<package name="fpc">
<module name="Classes">
</module>
</package>
</fpdoc-description>
\end{verbatim}
The header \tag{xml} tag is mandatory, and indicates that the file contains a
documentation XML document.

Inside the document, one or more top-level \tag{fpdoc-descriptions} 
tags may appear. Each of these tags can contain one or more \tag{package}
tags, which must have a \attr{name} attribute. The name attribute will be
used by fpdoc to select the documentation nodes.

Inside a \tag{package} tag, one or more \tag{module} tags may appear. There
should be one \tag{module} tag per unit that should be documented. The value
of the \attr{name} attribute of the \var{module} should be the name of the
unit for which the \tag{module} tag contains the documentation. The value
of the \var{name} attribute is case insensitive, i.e. 
\begin{verbatim}
<module name="CRT">
\end{verbatim}
can be used for the documentation of the \file{crt} unit.

As it is above, the documentation description does not do much. To write
real documentation, the \tag{module} tag must be filled with the
documentation for each identifier that appears in the unit interface header.

For each identifier in the unit interface header, the \tag{module} should
contain a tag that documents the identifier: this is the \tag{element} tag.
The name attribute of the element tag links the documentation to the
identifier: the \attr{name} attribute should have as value the fully
qualified name of the identifier in the unit.

For example, to document the type
\begin{verbatim}
Type
  MyEnum = (meOne,meTwo,meThree);
\end{verbatim}
an \tag{element} tag called \var{myenum} should exist:
\begin{verbatim}
<element name="myenum">
</element>
\end{verbatim}
But also for each of the three enumerated values an \tag{element} tag should 
exist:
\begin{verbatim}
<element name="myenum.meOne">
</element>
<element name="myenum.meTwo">
</element>
<element name="myenum.meThree">
</element>
\end{verbatim}
As it can be seen, the names of the identifiers follow a hierarchical
structure. More about this in the next section.

Now the tags for the types are present, all that should be done is to fill
it with the actual description.  For this, several tags can be placed inside
a \tag{element} tag. The most important tag is the \tag{descr} tag. The
contents of the \tag{descr} tag will be used to describe a type, function, 
constant or variable:

\begin{verbatim}
<element name="myenum">
<descr>
The MyEnum type is a simple enumeration type which is not
really useful, except for demonstration purposes.
</descr>
</element>
\end{verbatim}

A second important tag is the \tag{short} tag. It should contain a 
short description of the identifier, preferably a description that fits on
one line. The \tag{short} tag will be used in
various overviews, at the top of a page in the HTML documentation (a
synopsis) or will be used instead of the \tag{descr} tag if that one 
is not available. It can also be used in different other cases: For 
instance the different values of an enumeration type will be laid 
out in a table, using the \tag{short} description:
\begin{verbatim}
<element name="myenum.meOne">
<short>The first enumeration value</short>
</element>
<element name="myenum.meTwo">
<short>The second enumeration value</short>
</element>
<element name="myenum.meThree">
<short>The third enumeration value</short>
</element>
\end{verbatim}
This will be converted to a table looking more or less like this:

\begin{tabularx}{\textwidth}{lX}
meOne & The first enumeration value \\
meTwo & The second enumeration value \\
meThree & The third enumeration value \\
\end{tabularx}

For functions and procedures, a list of possible error conditions can be
documented inside a \tag{errors} tag. This tag is equivalent to the
\tag{descr} tag, but is placed under a different heading in the generated
documentation.

Finally, to cross-reference between related functions, types or classes, a
\tag{seealso} tag is also introduced. This will be used to lay out a series
of links to related information. This tag can only have sub-tags which are
\var{link} tags. For more about the \tag{link} tag, see \seetag{link}.

If a certain identifier has appeared only in a certain version of a library,
then this information can be specified in a \tag{version} tag, see
\seetag{version} for more information.

The following example illustrates the use:
\begin{verbatim}
<element name="myenum">
<descr>
The MyEnum type is a simple enumeration type which is not
really useful, except for demonstration purposes.
</descr>
<version>The myenum type appeared in version 2.1</version>
</element>
\end{verbatim}

When documenting methods or properties, it is possible to let the \var{fpdoc} engine
refer to the parent method when it generates an overview of methods or
properties. There are 2 ways of achieving this: 
\begin{enumerate}
\item Do not include an \tag{element} tag
\item Specify the \attr{link} attribute, with the name of the parent
method which contains the documentation, as in the following example:
\begin{verbatim}
<element name="TParent.SomeMethod">
</element

<element name="TChild.SomeMethod" link="TParent.SomeMethod"/>
\end{verbatim}
this can be used to speed up the search for the parent implementation,
or to skip several parent classes.
\end{enumerate}

To add a section or page of documentation that is not directly related to a
single identifier in a unit, a \tag{topic} tag can be used. This tag can
appear inside a \var{package} or \var{module} tag, and can contain a
\tag{short} or \tag{descr} tag. The contents of the \tag{short} tag will be
used for the title of the section or page. In on-line formats, a short list
of related topics will appear in the package or module page, with links to
the pages that contain the text of the topics. In a linear format, the topic
sections will be inserted before the description of all identifiers.

If the topic appears in a \tag{package} tag, then it can be nested: 2 levels
of topics may be used. This is not so for topics inside a module: they can
not be nested (the level of nesting in a linear documentation format is limited).

The following is an example of a valid topic tag:
\begin{verbatim}
<module name="CRT">
<topic name="UsingKeyboard">
<short>Using the keyboard functions</short>
<descr>
To use the keyboard functions of the CRT unit, one...
</descr>
</topic>
\end{verbatim}

Topic nodes can be useful to add 'how to' sections to a unit, or to provide
general IDE help.

%%%%%%%%%%%%%%%%%%%%%%%%%%%%%%%%%%%%%%%%%%%%%%%%%%%%%%%%%%%%%%%%%%%%%%%
% cross-referencing
\section{Element names and cross-referencing}

\subsection{Element name conventions}
As mentioned in the previous section, the \var{element} tag's \attr{name}
attribute is hierarchical. All levels in the hierarchy are denoted by a dot
(.) in the name attribute. 

As shown in the previous example, for an enumerated type, the various
enumeration constants can be documented by specifying their name as
\var{enumname.constname}. For example, given the type
\begin{verbatim}
Type
  MyEnum = (meOne,meTwo,meThree);
\end{verbatim}
The various enumeration values can be documented using the element names
\var{MyEnum.meOne}, \var{MyEnum.meTwo} and \var{MyEnum.meThree}:
\begin{verbatim}
<element name="myenum.meOne">
</element>
<element name="myenum.meTwo">
</element>
<element name="myenum.meThree">
</element>
\end{verbatim}
Note that the casing of the name attribute need not be the same as the
casing of the declaration.

This hierarchical structure can be used for all non-simple types: 
\begin{itemize}
\item Enumeration type values.
\item Fields in records, objects, classes. For nested record definitions,
multiple levels are possible in the name.
\item Methods of classes and objects.
\item Properties of classes.
\item Function and procedure arguments. 
\item Function results. The name is always the function name followed by
\var{Result}.
\end{itemize}

\subsection{Cross referencing: the \var{link} tag}
To refer to another point in the documentation (a related function, class or
whatever), a \tag{link} tag exists. The \var{link} tag takes as a sole
attribute a target \attr{id} attribute. The link tag usually encloses a
piece of text. In the HTML version of the documentation, this piece of text
will function as a hyperlink. In the latex version, a page number reference
will be printed. 

The \attr{id} attribute contains the name of an element to which the link
refers. The name is not case sensitive, but it must be a fully qualified
name.

An example of the link type would be:
\begin{verbatim}
The <link id="MyEnum">MyEnum</link> type is a simple type.
\end{verbatim}

The link attribute also has a short form: 
\begin{verbatim}
The <link id="MyEnum"/> type is a simple type.
\end{verbatim}
In the short form, the value of the \attr{id} attribute will be used as the 
text which will be hyperlinked. This is especially useful in the \tag{seealso}
tag.

To refer to a type in another unit, the unit name must be prepended to the
\attr{id} attribute:
\begin{verbatim}
<link id="myunit.myenum"/>
\end{verbatim}
will link to the \var{myenum} type in a unit named \file{myunit}.

To refer to a node in the documentation of another package, the package name
should be prepended to the \attr{id} attribute, and it should be prepended
with the hash symbol (\#):
\begin{verbatim}
<link id="#fcl.classes.sofrombeginning"/>
\end{verbatim}
will link to the constant \var{sofrombeginning} in the \file{classes} unit
of the FCL reference documentation. Note that for this to work correctly,
the contents file which was created when generating the documentation of 
the type must be imported correctly (see the \var{import} option).

%%%%%%%%%%%%%%%%%%%%%%%%%%%%%%%%%%%%%%%%%%%%%%%%%%%%%%%%%%%%%%%%%%%%%%%
% Tag reference
\section{Tag reference}
\subsection{Overview}
The tags can roughly be divided in 2 groups:
\begin{enumerate}
\item Documentation structure tags. These are needed for fpdoc to do it's
work. They determine what elements are documented. See \seet{structtags}
\item Text structure and formatting tags. These tags indicate blocks of text, such as
paragraphs, tables, lists and remarks, but also specify formatting: apply formatting (make-up) 
to the text, or to provide links to other parts of the text. These mostly occur in
text structure tags. See \seet{formattags}
\end{enumerate}

\begin{table}[ht]
\caption{Documentation structure tags}\label{tab:structtags}
\begin{tabularx}{\textwidth}{lXr}
Tag & Description & Page \\ \hline
descr & Element description & \pageref{tag:descr} \\
element & Identifier documentation & \pageref{tag:element} \\
errors  & Error section & \pageref{tag:errors} \\
fpdoc-description & Global tag & \pageref{tag:fpdocdescription} \\
module & Unit tag & \pageref{tag:module} \\
package & Package global tab & \pageref{tag:package} \\
seealso & Cross-reference section & \pageref{tag:seealso} \\
short & Short description & \pageref{tag:short} \\ 
topic & Topic page & \pageref{tag:topic} \\ \hline
\end{tabularx}
\end{table}

\begin{table}[ht]
\caption{Text formatting tags}\label{tab:formattags}
\begin{tabularx}{\textwidth}{lXr}
Tag & Description & Page \\ \hline
b & Format bold & \pageref{tag:b} \\
caption & Specify table caption & \pageref{tag:caption} \\
code & Syntax highlight code & \pageref{tag:code} \\
dd & definition data & \pageref{tag:dd} \\
dl & definition list & \pageref{tag:dl} \\
dt & Definition term & \pageref{tag:dt} \\
i & format italics & \pageref{tag:i} \\
img & include image & \pageref{tag:img} \\
li & list element & \pageref{tag:li} \\
link & Cross-reference & \pageref{tag:link} \\
ol & numbered list & \pageref{tag:ol} \\
p & paragraph & \pageref{tag:p} \\
pre & Preformatted text & \pageref{tag:pre} \\
remark & remark paragraph & \pageref{tag:remark} \\
table & Table & \pageref{tag:table} \\
td & Table cell & \pageref{tag:td} \\
th & Table header & \pageref{tag:th} \\
tr & Table row & \pageref{tag:tr} \\
u & format underlined & \pageref{tag:u} \\
ul & bulleted list & \pageref{tag:ul} \\
var & format as variable & \pageref{tag:var} \\ \hline
\end{tabularx}
\end{table}

The nodes for formatting a text resemble closely the basic HTML formatting tags with the 
following exceptions:
\begin{itemize}
\item Each opening tag must have a corresponding closing tag.
\item Tags are case sensitive. 
\item Tables and paragraphs are at the same level, i.e. a table cannot occur
inside a paragraph. The same is true for all 'structural' tags such as
lists, 
\end{itemize}

Also, if special formatting tags such as a table or lists are inserted, then
the remaining text must be inside a paragraph tag. This means that the
following is wrong:
\begin{verbatim}
<descr>
Some beginning text
<ol>
<li>A list item</li>
</ol>
some ending text
</descr>
\end{verbatim}
Instead, the correct XML should be
\begin{verbatim}
<descr>
<p>Some beginning text</p>
<ol>
<li>A list item</li>
</ol>
<p>some ending text</p>
</descr>
\end{verbatim}

% Tag B
\subsection{\tag{b} : format bold}
\label{tag:b}
This tag will cause the text inside it to be formatted using a bold font.

Example:
\begin{verbatim}
Normal text <b>Bold text</b> normal text.
\end{verbatim}
will be formatted as:\\
Normal text \textbf{Bold text} normal text.

See also: \seetag{i}, \seetag{u}.

% Tag caption
\subsection{\tag{caption} : Specify table caption}
\label{tag:caption}
This tag can occur inside a \tag{table} tag and serves to set the table 
caption.

Example
\begin{verbatim}
<table>
<caption>This caption will end up above the table</caption>
<th><td>Column 1</td><td>Column 2</td></th>
</table>
\end{verbatim}

See also: \seetag{table}

% Tag code
\subsection{\tag{code} : format as pascal code}
\label{tag:code}
The \tag{code} tag serves to insert Pascal code into the text. When possible
this code will be highlighted in the output. It can be used to put
highlighted Pascal code in the documentation of some identifier. It can
occur inside \tag{descr} or \tag{errors} tags.

Note that any text surrounding the \tag{code} tag should be placed in
paragraph tags \tag{p}.

Example:
\begin{verbatim}
<code>
With Strings do
  For i:=Count-1 downto 0 do
    Delete(i);
</code>
\end{verbatim}

Seealso: \seetag{pre}, \seetag{p}

% Tag descr
\subsection{\tag{descr} : Descriptions}
\label{tag:descr}
This is the actual documentation tag. The contents of this tag will be 
written as the documentation of the element. It can contain any mixture of
text and markup tags. The \tag{descr} tag can only occur inside a
\tag{element} or \tag{module}.

Example:
\begin{verbatim}
<element name="MyEnym">
<descr>Myenum is a simple enumeration type</descr>
</element>
\end{verbatim}

See also: \seetag{element}, \seetag{short}, \seetag{errors}, \seetag{seealso}

% Tag dd
\subsection{\tag{dd} : definition data.}
\label{tag:dd}
The \tag{dd} tag is used to denote the definition of a term in a definition
list. It can occur only inside a definition list tag (\tag{dl}), after a
definition term (\tag{dt}) tag. It's usage is identical to the one in HTML.

Example:
\begin{verbatim}
<dl>
<dt>FPC</dt><dd>stands for Free Pascal Compiler.</dd>
</dl>
\end{verbatim}
Will be typeset as
\begin{description}
\item[FPC] stands for Free Pascal Compiler.
\end{description}

See also: \seetag{dl}, \seetag{dt}, \seetag{ol}, \seetag{ul}

% Tag dl
\subsection{\tag{dl} : definition list.}
\label{tag:dl}
Definition lists are meant to type a set of terms together with their
explanation. It's usage is identical to the one in HTML, with the exception
that it cannot occur inside ordinary text: surrounding text should always be
enclosed in paragraph tags.

Example:
\begin{verbatim}
<dl>
<dt>meOne</dt><dd>First element of the enumeration type.</dd>
<dt>meTwo</dt><dd>Second element of the enumeration type.</dd>
<dt>meThree</dt><dd>Thir element of the enumeration type.</dd>
</dl>
\end{verbatim}
Will be typeset as
\begin{description}
\item[meOne] First element of the enumeration type.
\item[meTwo] Second element of the enumeration type.
\item[meThree] Third element of the enumeration type.
\end{description}

See also: \seetag{dt}, \seetag{dd}, \seetag{ol}, \seetag{ul}

% Tag  dt
\subsection{\tag{dt} : definition term.}
\label{tag:dt}
The \tag{dt} tag is used in definition lists to enclose the term for which a
definition is presented. It's usage is identical to the usage in HTML.

Example:
\begin{verbatim}
<dl>
<dt>FPC</dt><dd>stands for Free Pascal Compiler.</dd>
</dl>
\end{verbatim}
Will be typeset as
\begin{description}
\item[FPC] stands for Free Pascal Compiler.
\end{description}

See also: \seetag{dl}, \seetag{dd}, \seetag{ol}, \seetag{ul}


% Tag element
\subsection{\tag{element} : Identifier documentation}
\label{tag:element}
The \tag{element} contains the documentation for an identifier in a unit. It
should occur inside a \var{module} tag. It can contain 4 tags:
\begin{description}
\item[short] For a one-line description of the identifier. Is used as a
header or is used in overviews of constants, types, variables or classes.
\item[descr] Contains the actual description of the identifier.
\item[errors] For functions an procedures this can be used to describe error
conditions. It will be put in a separate section below the description
section.
\item[seealso] Used to refer to other nodes. will be typeset in a separate
section.
\end{description}

The \tag{element} tag should have at least the \attr{name} attribute, it 
is used to link the element node to the actual identifier in the Pascal unit.
Other attributes may be added in future.

Example:
\begin{verbatim}
<element name="MyEnym">
<descr>Myenum is a simple enumeration type</descr>
</element>
\end{verbatim}

The following attributes are supported:
\begin{description}
\item [name] this attribute is required and should equal the identifier name.
\item [opaque] this attribute is used for opaque types: if set to 1, the declaration will be replaced with the term ``opaque''.
\item [link] this should contain the name of an element. A link to the current element will be referred to this identifier instead. 
This can be used in overridden methods/properties to refer to the parent documentation instead.
\item [alwaysvisible] if set to 1, documentation for the element will be shown anyway, even if the current visibility is not shown. 
This can be used to force generation of documentation for private or protected members of a class even when they are not supposed to be shown. 
(for example, \var{--hide-protected} is given, and one wishes to show a protected method.)
\end{description}


See also: \seetag{descr}, \seetag{short}, \seetag{errors}, \seetag{seealso}

% Tag errors
\subsection{\tag{errors} : Error section.}
\label{tag:errors}
The \tag{errors} tag is used to document any errors that can occur when
calling a function or procedure. it is placed in a different section in the
generated documentation. It occurs inside a \tag{element} tag, at the same
level as a \tag{descr} or \tag{short} tag. It's contents can be any
text or formatting tag.

Example:
\begin{verbatim}
<element name="MyDangerousFunction">
<descr>MyDangerousFunction is a dangerous function</descr>
<errors>When MyDangerousFunction fails, all is lost</errors>
</element>
\end{verbatim}

See also: \seetag{descr}, \seetag{short}, \seetag{element}, \seetag{seealso}

% Tag example
\subsection{\tag{example} : Include an example source file.}
\label{tag:example}
The \tag{example} tag is used to include a sample code fragment in the
documentation. It must appear below an \tag{element} tag. Multiple example
tags may appear in an element tag.

The \attr{file} attribute of the is used to specify the file name containing
the example. No extension is required in the filename:  if it is missing, 
\file{.pp} is assumed.

Example:
\begin{verbatim}
<element name="MyDangerousFunction">
<descr>MyDangerousFunction is a dangerous function</descr>
<errors>When MyDangerousFunction fails, all is lost</errors>
<example file="ex/ex1"/>
</element>
\end{verbatim}

See also: \seetag{descr}, \seetag{element}, \seetag{seealso}

% Tag fpdoc-description
\subsection{\tag{fpdoc-description} : Global tag}
\label{tag:fpdocdescription}
The \tag{fpdoc-description} tag is the topmost tag in a description file. It
contains a series of \tag{package} tags, one for each package that is
described in the file.

See also: \seetag{package}

% Tag i
\subsection{\tag{i} : Format italics}
\label{tag:i}
The \tag{i} tag will cause its contents to be typeset in italics. It can
occur mixed with any text.

Example:
\begin{verbatim}
Normal text <i>italic text</i> normal text.
\end{verbatim}
will be formatted as:\\
Normal text \textit{italic text} normal text.

See also: \seetag{b}, \seetag{u}

% Tag img
\subsection{\tag{img} : include image}
\label{tag:img} 
The \tag{img} tag will include an image in the text. It is considered
equivalent to a paragraph tag. The required attribute \attr{file} can 
be used to search to specify the filename. What kind of file is used
depends on the output format. Normally, using \file{png} images should
work with all output formats.

The \attr{caption} attribute is optional and can be used to specify a
caption text which will be displayed above or under the figure. 
Lastly, the \attr{name} attribute can be used to attach
a name to the figure (a label to which can be referred).

For example:
\begin{verbatim}
<p>
some text
</p>
<img file="myfile" caption="A nice caption"/>
<p>
Some more text.
</p>
\end{verbatim}

See also: \seetag{p}.

% Tag li
\subsection{\tag{li} : list element}
\label{tag:li}
The tag \tag{li} denotes an element in a \tag{ol} or \tag{ul} list.
The usage is the same as for it's HTML counterpart: It can occur 
only inside one of the  \tag{ol} or \tag{ul} list tags.
It's contents may be arbitrary text and formatting tags, contrary to HTML
tags, the \tag{li} tag always must have a closing tag. Note that it cannot be used in a
definition list (\seetag{dl}).

Example:
\begin{verbatim}
<ul>
<li>First item in the list</li>
<li>Second item in the list</li>
</ul>
\end{verbatim}
Will be typeset as
\begin{itemize}
\item First item in the list.
\item Second item in the list.
\end{itemize}

See also: \seetag{ol}, \seetag{ul}.

% Tag link
\subsection{\tag{link} : Cross-reference}
\label{tag:link}
The \tag{link} tag is used to insert a reference to an element inside some
piece of text or inside the \tag{seealso} section. It is similar in
functionality to the 
\begin{verbatim}
<A HREF="SomeAnchor">some linked text</A>
\end{verbatim}
construct in HTML.

The mandatory \attr{id} attribute of the \tag{link} tag should have the 
name of an element tag in it. This name is not case sensitive. \fpdoc
will issue a warning if it cannot find a matching name. It will look for
matching names in the current file, and in all content files that have been
specified with the \var{import} command-line option.

The link tag can exist in 2 forms: one with separate closing tag,
surrounding a piece of text, one without separate closing tag. If a piece of
text is surrounded by the link tag, then the text will be converted to a
hyperlink in the HTML documentation. If there is no surrounded text, then
the value of the \attr{id} attribute will be used as the text. This means
that
\begin{verbatim}
<link id="TStream">TStream</link>
\end{verbatim}
and
\begin{verbatim}
<link id="TStream"/>
\end{verbatim}
are completely equivalent.

Example:
\begin{verbatim}
The <link id="TStringlist">stringlist</link> class is a descendent of the
<link id="TStrings"/> class.
\end{verbatim}

See also: \seetag{element}, \seetag{url} and the \var{import} option (\seesu{import}).

% Tag module 
\subsection{\tag{module} : Unit reference}
\label{tag:module}
The \tag{module} tag encloses all \tag{element} tags for a unit. It can
contain only \tag{element} tags for all identifiers in the unit and 
a \tag{descr} tag describing the unit itself. The \tag{module} tag should
occur inside a \tag{package} tag.

The \attr{name} attribute should have as a value the name of the unit which
is described by the module. This name is not case sensitive.

Example:
\begin{verbatim}
<module name="classes">
<descr>
The classes unit contains basic class definitions for the FCL.
</descr>
</module>
\end{verbatim}

See also: \seetag{package}, \seetag{descr}, \seetag{element}
% Tag notes
\subsection{\tag{notes} : Documentation notes (annotations).}
\label{tag:notes}
The \tag{notes} tag indicates that a particular documentation topic contains annotations. 
It consists of one or more \tag{note} tags, no other tags are allowed
They serve as annotations to the documentation, usually for the benefit of the documentation writer.

As such, notes are by default not included in the final documentation, but they can be added to 
them with the --emit-notes command-line option.

Notes can appear below the \tag{module}, \tag{topic} and \tag{element} tags.

For an explanation of what can appear inside the note tag itself, see \seetag{note}.

Example:
\begin{verbatim}
<module name="classes">
<descr>
The classes unit contains basic class definitions for the FCL.
</descr>
<notes>
<note>MVC. This unit needs more documenting</note>
<note>FK The new class MyVeryImportantClass needs documenting</note>
<notes>
</module>
\end{verbatim}

See also: \seetag{note}, \seetag{module}, \seetag{topic} and \seetag{element}.

% Tag notes
\subsection{\tag{note} : Notes/Annotation entry.}
\label{tag:notes}
The \tag{note} tag contains the text of one note. It must appear below a tag \tag{notes} 
collection of annotations. The notes will be typeset as a unordered list below the documentation. 
(usually the \var{descr} node).

This means there is a limitation to the markup that a note can contain:
It can just contain plain text with some markup, similar to what is allowed in \tag{li} tags.
No tables, lists or other structures are supported inside.
Example:
\begin{verbatim}
<element name="TDuplicates">
<short>How to handle duplicates in a list</short>
<descr>
Some text
</descr>
<notes>
<note>Be more explicit. Maybe refer to <link id="TStringList"/></note>
<notes>
</module>
\end{verbatim}

See also: \seetag{notes}, \seetag{li}

% Tag ol
\subsection{\tag{ol} : Numbered list.}
\label{tag:ol}
The \tag{ol} tag starts a numbered list. It can contain only \seetag{li}
tags, which denote the various elements in the list. Each item will be
preceded by a number. The \tag{ol} tag can
occur inside a text, but surrounding text should always be enclosed in a
\seetag{p} paragraph tag, i.e. an \tag{ol} tag should occur always at the
same level as a \tag{p} tag.

Example:
\begin{verbatim}
<p>some text before</p>
<ol>
<li>First item in the list</li>
<li>Second item in the list</li>
</ol>
\end{verbatim}
Will be typeset as:

some text before
\begin{enumerate}
\item First item in the list.
\item Second item in the list.
\end{enumerate}

See also: \seetag{li}, \seetag{ul}.


% Tag paragraph
\subsection{\tag{p} : Paragraph}
\label{tag:p}
The \tag{p} tag is the paragraph tag. Every description text should be 
enclosed in a \tag{p} tag. Only when there is only one paragraph (and no
lists or tables or remarks) in a description node, then the \tag{p} tag  may
be skipped.

Note that if a description node contains a \tag{table}, \tag{pre}, \tag{code}
or any list tag, then the text surrounding these tags {\em must} be  put 
inside a \tag{p} paragraph tag. This is different from the behaviour in HTML.

The paragraph tag must always have an opening tag and a closing tag, unlike
html where only the opening tag may be present.

Example:
\begin{verbatim}
<descr>
This is a paragraph which need not be surrounded by paragraph tags.
</descr>
\end{verbatim}
\begin{verbatim}
<descr>
<p>
This is the first paragraph.
</p>
<p>
This is the second paragraph.
</p>
</descr>
\end{verbatim}

See also: \seetag{table}, \seetag{dl}, \seetag{remark},\seetag{code},
\seetag{ol},\seetag{ul},\seetag{ol}

% Tag package
\subsection{\tag{package} : Package reference}
\label{tag:package}
The \tag{package} tag indicates the package for which the description file
contains documentation. A package is a collection of units which are
logically grouped together (for a library, program, component suites). The
\attr{name} attribute of the \tag{package} tag will be used to select the
package node in the description file: Only the \tag{package} node with name as
specified by the \var{package} command-line option will be used when
generating documentation. All other package nodes will be ignored.

The \tag{package} node must always reside in a \tag{fpdoc-description} node.
It can contain a \tag{descr} node, and various \tag{module} nodes, one node
per unit in the package.

See also: \seetag{fpdocdescription}, \seetag{module}, \seetag{descr}

% Tag pre
\subsection{\tag{pre} : Insert text as-is}
\label{tag:pre}
The \tag{pre} tag can be used to insert arbitrary text in the documentation.
The text will not be formatted in any way, and will be displayed as it is
encountered in the description node. It is functionally equivalent to the
\tag{pre} tag in HTML.

Note that if there is text surrounding the \tag{pre} tag, it should be placed
inside a \tag{p} paragraph tag.

Example:
\begin{verbatim}
<pre>
This is some text.
  This is some more text

    And yet more text...
</pre>
\end{verbatim}
This will be typeset as:
\begin{verbatim}
This is some text.
  This is some more text

    And yet more text...
\end{verbatim}

See also: \seetag{code}, \seetag{p}

% Tag printshort
\subsection{\tag{printshort} : insert short description}
\label{tag:printshort}
A \tag{printshort} tag can be used to insert the short description of an 
element in the current text. The name of the element whose short description
must be printed must be given in the \attr{id} attribute. Typical use for
this is to repeat the elements of an enumerated type when discussion
function results or possible parameter values.

Example:
\begin{verbatim}
<dl>
<dt>float_round_nearest_even</dt><dd><printshort id="float_round_nearest_even"/></dd>
<dt>float_round_down</dt><dd><printshort id="float_round_down"/></dd>
<dt>float_round_up</dt><dd><printshort id="float_round_up"/></dd>
<dt>float_round_to_zero</dt><dd><printshort id="float_round_to_zero"/></dd>
\end{verbatim}

% Tag remark
\subsection{\tag{remark} : format as remark}
\label{tag:remark}
A \tag{remark} tag can be used to make a paragraph stand out. The
\tag{remark} is equivalent to the \tag{p} tag, but it's contents is
formatted in a way that makes it stand out from the rest of the text.

Note that any text before or after the \tag{remark} tag must be enclosed in
paragraph (\tag{p}) tags.

Example:
\begin{verbatim}
<p>Normal text.</p>
<remark>
This text will stand out.
</remark>
<p>Again normal text.</p>
\end{verbatim}
Will be formatted as

Normal text.
\begin{remark}
This text will stand out.
\end{remark}
Again normal text.

See also: \seetag{p}, \seetag{code}, \seetag{pre}

% Tag seealso
\subsection{\tag{seealso} : Cross-reference section}
\label{tag:seealso}
The \tag{seealso} section can occur inside any \tag{element} tag, and will
be used to create a list of cross-references. The contents of the
\tag{seealso} tag is a list of \tag{link} tags. No other text is allowed
inside this tag. Note that both the long and short form if the link tag may
be used.

Example:
\begin{verbatim}
<seealso>
<link id="TStrings"/>
<link id="TStringList.Create">Create</link>
</seealso>
\end{verbatim}

See also: \seetag{link}, \seetag{element}

% Tag short
\subsection{\tag{short} : Short description}
\label{tag:short}
The \var{short} description is used to give a short description of an
identifier. If possible, the description should fit on a single line of
text. The contents of this tag will be used for the following purposes:
\begin{itemize}
\item Used as the synopsis on a page that describes an identifier.
\item Used in overviews of constants, types, variables, record fields, 
functions and procedures, classes, and for method and property listings of classes.
\item Replaces the \tag{descr} tag in an \tag{element} if no \tag{descr} tag is present.
\item In the description of an enumerated type, the enumeration values are
typeset in a table, each row containing the name of the value and the short
description.
\item In the description of a function or procedure that accepts arguments,
the arguments are followed by their short description.
\item In the declaration of a class or record, each method, field or property
is followed by the short description.
\end{itemize}

Example:
\begin{verbatim}
<element name="MyEnum.meOne">
<short>First element of MyEnum</short>
</element>
\end{verbatim}

See also: \seetag{element}, \seetag{descr}

% Tag table
\subsection{\tag{table} : Table start}
\label{tag:table}
The \var{table} tag starts a table, as in HTML. A table can contain
\tag{tr} (table row), \tag{th} (table header row) or \tag{caption} tags.
Any text surrounding the table must be enclosed in paragraph tags (\tag{p}).

Example:
\begin{verbatim}
<table>
<caption>
<var>TALignment</var> values and their meanings.
</caption>
<th><td>Value</td><td>Meaning</td></th>
<tr>
  <td><var>taLeftJustify</var></td>
  <td>Text is displayed aligned to the left.</td>
</tr>
<tr>
  <td><var>taRightJustify</var></td>
  <td>Text is displayed aligned to the right</td>
</tr>
<tr>
  <td><var>taCenter</var></td>
  <td>Text is displayed centred.</td>
</tr>
</table>
\end{verbatim}
Will be formatted approximately as follows:

\begin{tabularx}{\textwidth}{lX}
Value & Meaning \\ \hline
\var{taLeftJustify} & Text is displayed aligned to the left. \\
\var{taRightJustify} & Text is displayed aligned to the right \\
\var{taCenter} & Text is displayed centred.
\end{tabularx}

See also: \seetag{th}, \seetag{caption}, \seetag{tr}, \seetag{p}

% Tag td
\subsection{\tag{td} : Table cell}
\label{tag:td}
The \tag{td} tag is used to denote one cell in a table. It occurs
inside a \tag{tr} or \tag{th} tag, and can contain any text and formatting
tags.

Example:
\begin{verbatim}
<table>
<tr><td>Cell (1,1)</td><td>Cell (2,1)</td></tr>
<tr><td>Cell (1,2)</td><td>Cell (2,2)</td></tr>
</table>
\end{verbatim}
Will be formatted approximately as

\begin{tabular}{ll}
Cell (1,1) & Cell (2,1) \\
Cell (1,2) & Cell (2,2)
\end{tabular}

See also: \seetag{table}, \seetag{th}, \seetag{tr}

% Tag th
\subsection{\tag{th} : Table header}
\label{tag:th}
The \var{th} table header tag is used to denote the first row(s) of a table: 
It (they) will be made up differently from the other rows in the table.
Exactly how it is made up depends on the format. In printed documentation
(latex) a line will be drawn under the row. In HTML, the font and background
may be formatted differently.

The \tag{th} tag can only occur inside a \tag{table} tag, and can contain
only \tag{td} tags.

Example:
\begin{verbatim}
<table>
<th><td>Cell (1,1)</td><td>Cell (2,1)</td></th>
<tr><td>Cell (1,2)</td><td>Cell (2,2)</td></tr>
</table>
\end{verbatim}
Will be formatted approximately as

\begin{tabular}{ll}
Cell (1,1) & Cell (2,1) \\ \hline
Cell (1,2) & Cell (2,2)
\end{tabular}
 
See also: \seetag{tr}, \seetag{table}

\subsection{\tag{topic} : Topic section}
\label{tag:topic}
The topic tag starts a topic page or section. A topic page or section is
not linked to an identifier in some unit: it exists by itself. It must be 
inside a \tag{package} or \tag{module} tag. It must have a \attr{name}
attribute (for cross-referencing). If it appears inside a \tag{package}, 
it can be nested: a \tag{topic} may be inside another \tag{topic} tag.

\begin{verbatim}
<module name="CRT">
<topic name="UsingKeyboard">
<short>Using the keyboard functions</short>
<descr>
To use the keyboard functions of the CRT unit, one...
</descr>
</topic>
\end{verbatim}

% Tag tr
\subsection{\tag{tr} : table row}
\label{tag:tr}
The \tag{tr} tag denotes a row in a table. It works the same as in HTML.
It can occur only in a \tag{table} tag, and should contain only \tag{td}
table data tags. 

Example:
\begin{verbatim}
<table>
<tr><td>Cell (1,1)</td><td>Cell (2,1)</td></tr>
<tr><td>Cell (1,2)</td><td>Cell (2,2)</td></tr>
</table>
\end{verbatim}
Will be formatted approximately as

\begin{tabular}{ll}
Cell (1,1) & Cell (2,1) \\
Cell (1,2) & Cell (2,2)
\end{tabular}

See also: \seetag{table}, \seetag{th}, \seetag{td}

% Tag u
\subsection{\tag{u} : Format underlined}
\label{tag:u}

Example:
\begin{verbatim}
Normal text <u>underlined text</u> normal text.
\end{verbatim}
will be formatted as:\\
Normal text \underline{underlined text} normal text.

See also: \seetag{i}, \seetag{b}.

% Tag ul
\subsection{\tag{ul} : bulleted list}
\label{tag:ul}
The \var{ul} tag starts a bulleted list. This works as under HTML, with the
exception that any text surrounding the list must be enclosed in paragraph
tags (\tag{p}). The list elements should be enclosed in \tag{li} tags.

Example:
\begin{verbatim}
<p>some text before</p>
<ul>
<li>First item in the list</li>
<li>Second item in the list</li>
</ul>
\end{verbatim}
Will be typeset as:

some text before
\begin{itemize}
\item First item in the list.
\item Second item in the list.
\end{itemize}

See also: \seetag{li}, \seetag{ol}.


% Tag url
\subsection{\tag{url} : Hyperlink}
\label{tag:url}
The \var{url} tag is meant to include an URL to an arbitrary web page. It
works much like the \tag{link} tag, except that any URL may be specified.
The attribute \attr{href} specifies the URL to link to. In output formats
that support links, the text in the \tag{url} tag will be the link's text.
For other formats, the URL will be printed after the text.

If the \tag{url} tag does not enclose any text, then the URL itself will
be printed.

Examples:

The normal usage is:
\begin{verbatim}
<url href="http://www.freepascal.org/">Here</url> 
is the website of Free Pascal.
\end{verbatim}
It will be typeset as:

\htmladdnormallink{Here}{http://www.freepascal.org/} is the website of Free Pascal.

The short usage is:
\begin{verbatim}
Check out the following site: <url href="http://www.freepascal.org/"/>.
\end{verbatim}
Which will be typeset as:

Check out the following site: \htmladdnormallink{http://www.freepascal.org/}{http://www.freepascal.org/}.

See also: \seetag{link}.

% Tag var
\subsection{\tag{var} : variable}
\label{tag:var}
The \tag{var} tag is used to mark a piece of text as a variable (or, more
general, as an identifier). It will be typeset differently from the 
surrounding text. Exactly how this is done depends on the output format.

Example:
\begin{verbatim}
The <var>Items</var> property gives indexed access to...
\end{verbatim}
Will be typeset as

The \var{Items} property gives indexed access to...

See also: \seetag{b}, \seetag{u}, \seetag{i}, \seetag{code}

% Tag var
\subsection{\tag{version} : version info}
\label{tag:version}
The \tag{version} can be used to give version information: as of what
version an identifier appeared or was deprecated. It should be specified
below the \seetag{element} tag, and will be output in the documentation.
It can contain the same tags as the \seetag{descr} tag, and is typeset
in the same way.

The following is an example:
\begin{verbatim}
<element name="myenum">
<descr>
The MyEnum type is a simple enumeration type which is not
really useful, except for demonstration purposes.
</descr>
<version>The myenum type appeared in version 2.1</version>
</element>
\end{verbatim}
See also: \seetag{element}, \seetag{descr}, \seetag{errors}, \seetag{seealso}


%%%%%%%%%%%%%%%%%%%%%%%%%%%%%%%%%%%%%%%%%%%%%%%%%%%%%%%%%%%%%%%%%%%%%
% Generated output files.
%%%%%%%%%%%%%%%%%%%%%%%%%%%%%%%%%%%%%%%%%%%%%%%%%%%%%%%%%%%%%%%%%%%%%
\chapter{Generated output files.}
\label{ch:outputfiles}

%%%%%%%%%%%%%%%%%%%%%%%%%%%%%%%%%%%%%%%%%%%%%%%%%%%%%%%%%%%%%%%%%%%%%
% HTML output
\section{HTML output}
The HTML output consists of the following files, per unit:

\begin{enumerate}
\item A general unit description with the contents of the module's
 \tag{descr} tag. The \var{uses} clause is documented here as well. 
All units in the \var{uses} clause together with their \tag{short}
 description tags are typeset in a table.
\item A listing of all constants in the unit.
\item A listing of all types in the unit (except classes).
\item A listing of all variables in the unit.
\item A listing of all functions/procedures in the unit.
\item A listing of all classes in the unit.
\item An index page per unit and per package. This is an index page, listing
all top-level identifiers in the current unit or all units of the package.
\end{enumerate}
All these overviews are hyperlinked to pages which contain the documentation
of each identifier. Each page starts with the name of the identifier,
plus a synopsis (made from the \tag{short} tag's contents).  After that
follows the declaration, and the description. The description is filled with
the \tag{descr} node of the identifiers \tag{element} tag.

If an \var{errors} tag was present, an 'Errors' section follows the
description. Similarly, if there is a \tag{seealso} tag, a 'See also'
section with cross-reference links is made.

For classes, the declaration contains hyperlinks to separate pages which 
document all the members of the class. Each member in the declaration is 
followed by the \tag{short} tag of the member's \tag{element} tag, if one
exists. As an extra, the class hierarchy is given, plus links to pop-up pages
(if JavaScript is available, otherwise they are normal links) which contain
alphabetical or hierarchical listings of the methods, fields or properties
of the class.

For functions and procedures, the declaration will be typeset in such a way
that all function arguments (if they are present) are in tabular format, 
followed by the short description of the argument. If it concerns a
function, and a result element exists, the result description will be
provided in a separate section, before the actual description.

The declaration of an enumerated type will be laid out in a table, with the
enumeration value at the left, and the short description node of the value's 
element.


%%%%%%%%%%%%%%%%%%%%%%%%%%%%%%%%%%%%%%%%%%%%%%%%%%%%%%%%%%%%%%%%%%%%%
% Latex output
\section{Latex output}

The latex output is in one big file with the name of the package as
specified on the command line. in this file, one chapter is made per 
unit.

Per unit the following sections are made:

\begin{enumerate}
\item A section with all constants.
\item A section with all types.
\item A section with all variables.
\item A section with all functions and procedures.
\item A section per declared class.
\end{enumerate}

For the constants, types and variables, the declaration is given, followed
by the \tag{descr} node of the element corresponding to the identifier.
All other nodes are ignored. 

For functions and procedures, a subsection is made per procedure or
function. This subsection consists of a list with the following entries:
\begin{description}
\item[Synopsis] filled with the contents of the \tag{short} tag.
\item[Declaration] Filled with the declaration of the function.
\item[Arguments] A tabular description of all arguments, if \tag{short} tags
are found for them.
\item[Description] Description of the function. Filled with the contents of
the \tag{descr} tag.
\item[Errors] Description of any error conditions. Filled with the contents
of the \tag{errors} tag.
\item[See Also] Cross-references to other functions. Filled with the
contents of the \tag{seealso} tag.
\end{description}

For classes, a subsection is made with an overview of implemented methods.
Then a subsection is presented with available properties. 

Then follows a subsection per method. These are formatted as a function, with an
additional \textbf{Visibility} list element, giving the visibility of the
function.

After the methods, a list of properties is given , formatted as a method,
with an additional \var{Access} list element, specifying whether the
property is read/write or not.

% CHM output
\section{CHM output}

The chm output in \fpdoc is actually inherited from the HTML backend so 
everything that applies to the HTML backend applies to the chm backend, except
that all generated HTML files are written directly to a stream. After all the
HTML files are generated the compression is begun. Once all the auto generated
files are compressed, if the --other-files option is used these files are 
collected and compressed. Now if the --auto-index or --auto-toc are used
the Index and Table of Contents are created and the compression is finished.


%%%%%%%%%%%%%%%%%%%%%%%%%%%%%%%%%%%%%%%%%%%%%%%%%%%%%%%%%%%%%%%%%%%%%%%
% End of file
\end{document}
