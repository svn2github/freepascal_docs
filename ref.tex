%
%   $Id: ref.tex,v 1.52 2005/03/13 00:31:55 michael Exp $
%   This file is part of the FPC documentation.
%   Copyright (C) 1997, by Michael Van Canneyt
%
%   The FPC documentation is free text; you can redistribute it and/or
%   modify it under the terms of the GNU Library General Public License as
%   published by the Free Software Foundation; either version 2 of the
%   License, or (at your option) any later version.
%
%   The FPC Documentation is distributed in the hope that it will be useful,
%   but WITHOUT ANY WARRANTY; without even the implied warranty of
%   MERCHANTABILITY or FITNESS FOR A PARTICULAR PURPOSE.  See the GNU
%   Library General Public License for more details.
%
%   You should have received a copy of the GNU Library General Public
%   License along with the FPC documentation; see the file COPYING.LIB.  If not,
%   write to the Free Software Foundation, Inc., 59 Temple Place - Suite 330,
%   Boston, MA 02111-1307, USA.
%
%%%%%%%%%%%%%%%%%%%%%%%%%%%%%%%%%%%%%%%%%%%%%%%%%%%%%%%%%%%%%%%%%%%%%%%
% Preamble.
\input{preamble.inc}
\begin{latexonly}
  \ifpdf
  \pdfinfo{/Author(Michael Van Canneyt)
           /Title(Free Pascal Language Reference Guide)
           /Subject(Free Pascal Reference guide)
           /Keywords(Free Pascal, Language)
           }
  \fi
\end{latexonly}

%
% Settings
%
\makeindex
%
% Syntax style
%
\usepackage{syntax}
%
% Here we determine the style of the syntax diagrams.
%
% Define a 'boxing' environment
\newenvironment{diagram}[2]%
{\begin{quote}\rule{0.5pt}{1ex}%
\rule[1ex]{\linewidth}{0.5pt}%
\rule{0.5pt}{1ex}\\[-0.5ex]%
\textbf{#1}\\[-0.5ex]}%
{\rule{0.5pt}{1ex}%
\rule{\linewidth}{0.5pt}%
\rule{0.5pt}{1ex}\end{quote}}
%\newenvironment{diagram}[2]{}{}
% Define mysyntdiag for my style of diagrams
\makeatletter
% Under Tex4HT, the diagrams are rendered as pictures.
\@ifpackageloaded{tex4ht}{%
\newenvironment{mysyntdiag}%
{\HCode{<BR/>}\Picture*{}\begin{syntdiag}\setlength{\sdmidskip}{.5em}\sffamily\sloppy}%
{\end{syntdiag}\EndPicture\HCode{<BR/>}}%
}{%
\newenvironment{mysyntdiag}%
{\begin{syntdiag}\setlength{\sdmidskip}{.5em}\sffamily\sloppy}%
{\end{syntdiag}}%
}% 
\makeatother
% Finally, define a combination of the above two.
\newenvironment{psyntax}[2]{\begin{diagram}{#1}{#2}\begin{mysyntdiag}}%
{\end{mysyntdiag}\end{diagram}}
% Redefine the styles used in the diagram.
\latex{\renewcommand{\litleft}{\bfseries\ }
\renewcommand{\ulitleft}{\bfseries\ }
\renewcommand{\syntleft}{\ }
\renewcommand{\litright}{\ \rule[.5ex]{.5em}{2\sdrulewidth}}
\renewcommand{\ulitright}{\ \rule[.5ex]{.5em}{2\sdrulewidth}}
\renewcommand{\syntright}{\ \rule[.5ex]{.5em}{2\sdrulewidth}}
}
% Finally, a referencing command.
\newcommand{\seesy}[1]{see diagram}
%
%
% Start of document.
%
\begin{document}
\renewcommand{\hline}{\xspace}
\title{Free Pascal :\\ Reference guide.}
\docdescription{Reference guide for Free Pascal, version \fpcversion}
\docversion{2.0}
\input{date.inc}
\author{Micha\"el Van Canneyt}
\maketitle
\tableofcontents
\newpage
\listoftables
\newpage


%%%%%%%%%%%%%%%%%%%%%%%%%%%%%%%%%%%%%%%%%%%%%%%%%%%%%%%%%%%%%%%%%%%%%
% Introduction
%%%%%%%%%%%%%%%%%%%%%%%%%%%%%%%%%%%%%%%%%%%%%%%%%%%%%%%%%%%%%%%%%%%%%


%%%%%%%%%%%%%%%%%%%%%%%%%%%%%%%%%%%%%%%%%%%%%%%%%%%%%%%%%%%%%%%%%%%%%%%
% About this guide
\section*{About this guide}
This document serves as the reference for the Pascal langauge as implemented
by the \fpc compiler. It describes all Pascal constructs supported by 
\fpc, and lists all supported data types. It does not, however, give a 
detailed explanation of the pascal language. The aim is to list which 
Pascal constructs are supported, and to show where the \fpc implementation 
differs from the Turbo Pascal or Delphi implementations.

Earlier versions of this document also contained the reference documentation
of the \file{system} unit and \file{objpas} unit. This has been moved to the 
RTL reference guide.

\subsection*{Notations}
Throughout this document, we will refer to functions, types and variables
with \var{typewriter} font. Functions and procedures have their own
subsections, and for each function or procedure we have the following
topics:
\begin{description}
\item [Declaration] The exact declaration of the function.
\item [Description] What does the procedure exactly do ?
\item [Errors] What errors can occur.
\item [See Also] Cross references to other related functions/commands.
\end{description}
The cross-references come in two flavours:
\begin{itemize}
\item References to other functions in this manual. In the printed copy, a
number will appear after this reference. It refers to the page where this
function is explained. In the on-line help pages, this is a hyperlink,
which can be clicked to jump to the declaration.
\item References to Unix manual pages. (For linux and unix related things only) they
are printed in \var{typewriter} font, and the number after it is the Unix
manual section.
\end{itemize}
\subsection*{Syntax diagrams}
All elements of the pascal language are explained in \index{Syntax diagrams}syntax diagrams.
Syntax diagrams are like flow charts. Reading a syntax diagram means getting
from the left side to the right side, following the arrows.
When the right side of a syntax diagram is reached, and it ends with a single
arrow, this means the syntax diagram is continued on the next line. If
the line ends on 2 arrows pointing to each other, then the diagram is
ended.

Syntactical elements are written like this
\begin{mysyntdiag}
\synt{syntactical\ elements\ are\ like\ this}
\end{mysyntdiag}
Keywords which must be typed exactly as in the diagram:
\begin{mysyntdiag}
\lit*{keywords\ are\ like\ this}
\end{mysyntdiag}
When something can be repeated, there is an arrow around it:
\begin{mysyntdiag}
\begin{rep}[b] \synt{this\ can\ be\ repeated} \\ \end{rep}
\end{mysyntdiag}
When there are different possibilities, they are listed in columns:
\begin{mysyntdiag}
\begin{stack}
\synt{First\ possibility} \\
\synt{Second\ possibility}
\end{stack}
\end{mysyntdiag}
Note, that one of the possibilities can be empty:
\begin{mysyntdiag}
\begin{stack}\\
\synt{First\ possibility} \\
\synt{Second\ possibility}
\end{stack}
\end{mysyntdiag}
This means that both the first or second possibility are optional.
Of course, all these elements can be combined and nested.

%%%%%%%%%%%%%%%%%%%%%%%%%%%%%%%%%%%%%%%%%%%%%%%%%%%%%%%%%%%%%%%%%%%%%%%
% The Pascal language

%%%%%%%%%%%%%%%%%%%%%%%%%%%%%%%%%%%%%%%%%%%%%%%%%%%%%%%%%%%%%%%%%%%%%%%
\chapter{Pascal Tokens}
\index{Tokens}
In this chapter we describe all the pascal reserved words, as well as the
various ways to denote strings, numbers, identifiers etc.

%%%%%%%%%%%%%%%%%%%%%%%%%%%%%%%%%%%%%%%%%%%%%%%%%%%%%%%%%%%%%%%%%%%%%%%
% Symbols
\section{Symbols}
\index{Symbols}\index{Tokens!Symbols}
Free Pascal allows all characters, digits and some special character symbols
in a Pascal source file.
\input{syntax/symbol.syn}
The following characters have a special meaning:
\begin{verbatim}
 + - * / = < > [ ] . , ( ) : ^ @ { } $ #
\end{verbatim}
and the following character pairs too:
\begin{verbatim}
<= >= := += -= *= /= (* *) (. .) //
\end{verbatim}
When used in a range specifier, the character pair \var{(.} is equivalent to
the left square bracket \var{[}. Likewise, the character pair \var{.)} is
equivalent to the right square bracket \var{]}.
When used for comment delimiters, the character pair \var{(*} is equivalent
to the  left brace \var{\{} and the character pair \var{*)} is equivalent
to the right brace \var{\}}.
These character pairs retain their normal meaning in string expressions.


%%%%%%%%%%%%%%%%%%%%%%%%%%%%%%%%%%%%%%%%%%%%%%%%%%%%%%%%%%%%%%%%%%%%%%%
% Comments
\section{Comments}
\index{Comments}\index{Tokens!Symbols}\keywordlink{Comment}
\fpc supports the use of nested comments. The following constructs are valid
comments:
\begin{verbatim}
(* This is an old style comment *)
{  This is a Turbo Pascal comment }
// This is a Delphi comment. All is ignored till the end of the line.
\end{verbatim}
The following are valid ways of nesting comments:
\begin{verbatim}
{ Comment 1 (* comment 2 *) }
(* Comment 1 { comment 2 } *)
{ comment 1 // Comment 2 }
(* comment 1 // Comment 2 *)
// comment 1 (* comment 2 *)
// comment 1 { comment 2 }
\end{verbatim}
The last two comments {\em must} be on one line. The following two will give
errors:
\begin{verbatim}
 // Valid comment { No longer valid comment !!
    }
\end{verbatim}
and
\begin{verbatim}
 // Valid comment (* No longer valid comment !!
    *)
\end{verbatim}
The compiler will react with a 'invalid character' error when it encounters
such constructs, regardless of the \var{-So} switch.

%%%%%%%%%%%%%%%%%%%%%%%%%%%%%%%%%%%%%%%%%%%%%%%%%%%%%%%%%%%%%%%%%%%%%%%
% Reserved words
\section{Reserved words}
\index{Reserved words}\index{Tokens!Reserved words}
Reserved words are part of the Pascal language, and cannot be redefined.
They will be denoted as {\sffamily\bfseries this} throughout the syntax
diagrams. Reserved words can be typed regardless of case, i.e. Pascal is
case insensitive.
We make a distinction between Turbo Pascal and Delphi reserved words, since
with the \var{-So} switch, only the Turbo Pascal reserved words are
recognised, and the Delphi ones can be redefined. By default, \fpc
recognises the Delphi reserved words.
\subsection{Turbo Pascal reserved words}
\index{Reserved words!Turbo Pascal}
The following keywords exist in Turbo Pascal mode
\begin{multicols}{4}
\begin{verbatim}
absolute
and
array
asm
begin
case
const
constructor
destructor
div
do
downto
else
end
file
for
function
goto
if
implementation
in
inherited
inline
interface
label
mod
nil
not
object
of
on
operator
or
packed
procedure
program
record
reintroduce
repeat
self
set
shl
shr
string
then
to
type
unit
until
uses
var
while
with
xor
\end{verbatim}
\end{multicols}
\subsection{Delphi reserved words}
\index{Reserved words!Delphi}
The Delphi (II) reserved words are the same as the turbo pascal ones, plus the
following ones:
\begin{multicols}{4}
\begin{verbatim}
as
class
except
exports
finalization
finally
initialization
is
library
on
out
property
raise
threadvar
try
\end{verbatim}
\end{multicols}
\subsection{\fpc reserved words}
\index{Reserved words!Free Pascal}
On top of the Turbo Pascal and Delphi reserved words, \fpc also considers
the following as reserved words:
\begin{multicols}{4}
\begin{verbatim}
dispose
exit
false
new
true
\end{verbatim}
\end{multicols}
\subsection{Modifiers}
\index{Modifiers}\index{Reserved words!Modifiers}
The following is a list of all modifiers. They are not exactly reserved
words in the sense that they can be used as identifiers, but in specific
places, they have a special meaning for the compiler.
\begin{multicols}{4}
\begin{verbatim}
absolute
abstract
alias
assembler
cdecl
cppdecl
default
export
external
far
far16
forward
index
local
name
near
nostackframe
oldfpccall
override
pascal
private
protected
public
published
read
register
safecall
softfloat
stdcall
virtual
write
\end{verbatim}
\end{multicols}
\begin{remark}
Predefined types such as \var{Byte}, \var{Boolean} and constants
such as \var{maxint} are {\em not} reserved words. They are
identifiers, declared in the system unit. This means that these types
can be redefined in other units. The programmer is, however, not
encouraged to do this, as it will cause a lot of confusion.
\end{remark}

%%%%%%%%%%%%%%%%%%%%%%%%%%%%%%%%%%%%%%%%%%%%%%%%%%%%%%%%%%%%%%%%%%%%%%%
% Identifiers
\section{Identifiers}
\index{Identifiers}\index{Tokens!Identifiers}
Identifiers denote constants, types, variables, procedures and functions,
units, and programs. All names of things that are defined are identifiers.
An identifier consists of 255 significant characters (letters, digits and
the underscore character), from which the first must be an alphanumeric
character, or an underscore (\var{\_})
The following diagram gives the basic syntax for identifiers.
\input{syntax/identifier.syn}

%%%%%%%%%%%%%%%%%%%%%%%%%%%%%%%%%%%%%%%%%%%%%%%%%%%%%%%%%%%%%%%%%%%%%%%
% Numbers
\section{Numbers}
\index{Numbers}\index{Tokens!Numbers}\index{Numbers!Real}\keywordlink{numbers}
Numbers are by default denoted in decimal notation.
Real (or decimal) numbers are written using engineering or scientific
notation (e.g. \var{0.314E1}).

For integer type constants, \fpc supports 4 formats:
\begin{enumerate}
\item Normal, decimal format (base 10). This is the standard
format.\index{Numbers!Decimal}
\item Hexadecimal format (base 16), in the same way as Turbo Pascal does.
To specify a constant value in hexadecimal format, prepend it with a dollar
sign (\var{\$}). Thus, the hexadecimal \var{\$FF} equals 255 decimal.
Note that case is insignificant when using hexadecimal constants.
\item As of version 1.0.7, Octal format (base 8) is also supported.
To specify a constant in octal format, prepend it with a ampersand (\&).
For instance 15 is specified in octal notation as
\var{\&17}.\index{Numbers!Hexadecimal}
\item Binary notation (base 2). A binary number can be specified
by preceding it with a percent sign (\var{\%}). Thus, \var{255} can be
specified in binary notation as \var{\%11111111}.\index{Numbers!Binary}
\end{enumerate}
The following diagrams show the syntax for numbers.
\input{syntax/numbers.syn}

\begin{remark}
It is to note that all decimal constants which do no fit within
the -2147483648..2147483647 range, are silently and automatically
parsed as 64-bit integer constants as of version 1.9.0. Earliers 
versions would convert it to a real-typed constant.
\end{remark}

\begin{remark}
Note that Octal and Binary notation are not supported in TP or Delphi compatibility
mode.
\end{remark}

%%%%%%%%%%%%%%%%%%%%%%%%%%%%%%%%%%%%%%%%%%%%%%%%%%%%%%%%%%%%%%%%%%%%%%%
% Labels
\section{Labels}
\index{Labels}
Labels can be digit sequences or identifiers.\keywordlink{label}
\input{syntax/label.syn}
\begin{remark}
Note that the \var{-Sg} or \var{-Mtp} switches must be specified before 
labels can be used. By default, \fpc doesn't support \var{label} and 
\var{goto} statements.
\end{remark}

%%%%%%%%%%%%%%%%%%%%%%%%%%%%%%%%%%%%%%%%%%%%%%%%%%%%%%%%%%%%%%%%%%%%%%%
% Character strings
\section{Character strings}
\index{String}\index{Constants!String}\index{Tokens!Strings}
A character string (or string for short) is a sequence of zero or more
characters (byte sized), enclosed by single quotes, and on 1 line of the 
program source. A character set with nothing between the quotes (\var{'{}'}) is an empty
string.
\input{syntax/string.syn}

%%%%%%%%%%%%%%%%%%%%%%%%%%%%%%%%%%%%%%%%%%%%%%%%%%%%%%%%%%%%%%%%%%%%%%%
% Constants
%%%%%%%%%%%%%%%%%%%%%%%%%%%%%%%%%%%%%%%%%%%%%%%%%%%%%%%%%%%%%%%%%%%%%%%
\chapter{Constants}
\index{Constants}
Just as in Turbo Pascal, \fpc supports both normal and typed constants.

%%%%%%%%%%%%%%%%%%%%%%%%%%%%%%%%%%%%%%%%%%%%%%%%%%%%%%%%%%%%%%%%%%%%%%%
% Ordinary constants
\section{Ordinary constants}
\index{Constants!Ordinary}
Ordinary constants declarations are not different from the Turbo Pascal or
Delphi implementation.
\input{syntax/const.syn}
The compiler must be able to evaluate the expression in a constant
declaration at compile time.  This means that most of the functions
in the Run-Time library cannot be used in a constant
declaration.\index{Operators}
Operators such as \var{+, -, *, /, not, and, or, div, mod, ord, chr,
sizeof, pi, int, trunc, round, frac, odd} can be used, however. For more 
information on expressions, see \seec{Expressions}.
Only constants of the following types can be declared: \var{Ordinal
types}, \var{Real types}, \var{Char}, and \var{String}.\index{Constants!String}
The following are all valid constant declarations:
\begin{verbatim}
Const
  e = 2.7182818;  { Real type constant. }
  a = 2;          { Ordinal (Integer) type constant. }
  c = '4';        { Character type constant. }
  s = 'This is a constant string'; {String type constant.}
  s = chr(32)
  ls = SizeOf(Longint);
\end{verbatim}
Assigning a value to an ordinary constant is not permitted.
Thus, given the previous declaration, the following will result
in a compiler error:
\begin{verbatim}
  s := 'some other string';
\end{verbatim}

Prior to version 1.9, \fpc did not correctly support 64-bit constants. As
of version 1.9, 64-bits constants can be specified.

%%%%%%%%%%%%%%%%%%%%%%%%%%%%%%%%%%%%%%%%%%%%%%%%%%%%%%%%%%%%%%%%%%%%%%%
% Typed constants
\section{Typed constants}
\index{Constants!Typed}\index{Variables!Initialized}
Typed constants serve to provide a program with initialised variables.
Contrary to ordinary constants, they may be assigned to at run-time.
The difference with normal variables is that their value is initialised
when the program starts, whereas normal variables must be initialised
explicitly.\index{Const}
\input{syntax/tconst.syn}
Given the declaration:
\begin{verbatim}
Const
  S : String = 'This is a typed constant string';
\end{verbatim}
The following is a valid assignment:
\begin{verbatim}
 S := 'Result : '+Func;
\end{verbatim}
Where \var{Func} is a function that returns a \var{String}.
Typed constants are often used to initialize arrays and records. For arrays,
the initial elements must be specified, surrounded by round brackets, and
separated by commas. The number of elements must be exactly the same as
the number of elements in the declaration of the type.
As an example:
\begin{verbatim}
Const
  tt : array [1..3] of string[20] = ('ikke', 'gij', 'hij');
  ti : array [1..3] of Longint = (1,2,3);
\end{verbatim}
For constant records, each element of the record should be specified, in
the form \var{Field : Value}, separated by commas, and surrounded by round
brackets.\index{Record!Constant}
As an example:
\begin{verbatim}
Type
  Point = record
    X,Y : Real
    end;
Const
  Origin : Point = (X:0.0; Y:0.0);
\end{verbatim}
The order of the fields in a constant record needs to be the same as in the type declaration,
otherwise a compile-time error will occur.

\begin{remark}
It should be stressed that typed constants are initialized at program start.
This is also true for {\em local} typed constants. Local typed constants are
also initialized at program start. If their value was changed during previous
invocations of the function, they will retain their changed value, i.e. they
are not initialized each time the function is invoked.
\end{remark}

%%%%%%%%%%%%%%%%%%%%%%%%%%%%%%%%%%%%%%%%%%%%%%%%%%%%%%%%%%%%%%%%%%%%%%%
% resource strings
\section{Resource strings}
\label{se:resourcestring}\index{Resourcestring}\index{Const}\index{Const!String}\keywordlink{resourcestring}
A special kind of constant declaration part is the \var{Resourestring}
part. This part is like a \var{Const} section, but it only allows
to declare constant of type string. This part is only available in the
\var{Delphi} or \var{objfpc} mode.

The following is an example of a resourcestring definition:
\begin{verbatim}
Resourcestring

  FileMenu = '&File...';
  EditMenu = '&Edit...';
\end{verbatim}
All string constants defined in the resourcestring section are stored
in special tables, allowing to manipulate the values of the strings
at runtime with some special mechanisms.

Semantically, the strings are like constants; Values can not be assigned  to
them, except through the special mechanisms in the objpas unit. However,
they can be used in assignments or expressions as normal constants.
The main use of the resourcestring section is to provide an easy means
of internationalization.

More on the subject of resourcestrings can be found in the \progref, and
in the chapter on the \file{objpas} later in this manual.

%%%%%%%%%%%%%%%%%%%%%%%%%%%%%%%%%%%%%%%%%%%%%%%%%%%%%%%%%%%%%%%%%%%%%%%
% Types
%%%%%%%%%%%%%%%%%%%%%%%%%%%%%%%%%%%%%%%%%%%%%%%%%%%%%%%%%%%%%%%%%%%%%%%
\chapter{Types}
\index{Types}
All variables have a type. \fpc supports the same basic types as Turbo
Pascal, with some extra types from Delphi.
The programmer can declare his own types, which is in essence defining an identifier
that can be used to denote this custom type when declaring variables further
in the source code.\index{Type}\keywordlink{type}
\input{syntax/typedecl.syn}
There are 7 major type classes :
\input{syntax/type.syn}
The last class, {\sffamily type identifier}, is just a means to give another
name to a type. This presents a way to make types platform independent, by
only using these types, and then defining these types for each platform
individually. The programmer that uses these units doesn't have to worry
about type size and so on. It also allows to use shortcut names for
fully qualified type names. e.g. define \var{system.longint} as
\var{Olongint} and then redefine \var{longint}.

%%%%%%%%%%%%%%%%%%%%%%%%%%%%%%%%%%%%%%%%%%%%%%%%%%%%%%%%%%%%%%%%%%%%%%%
% Base types
\section{Base types}
\index{Types!Base}
The base or simple types of \fpc are the Delphi types.
We will discuss each separate.
\input{syntax/typesim.syn}
\subsection{Ordinal types}
\index{Types!Ordinal}
With the exception of \var{int64}, \var{qword} and Real types, 
all base types are ordinal types. Ordinal types have the following 
characteristics:
\begin{enumerate}
\item Ordinal types are countable and ordered, i.e. it is, in principle,
possible to start counting them one bye one, in a specified order.
This property allows the operation of functions as \var{Inc}, \var{Ord},
\var{Dec}
on ordinal types to be defined.
\item Ordinal values have a smallest possible value. Trying to apply the
\var{Pred} function on the smallest possible value will generate a range
check error if range checking is enabled.
\item Ordinal values have a largest possible value. Trying to apply the
\var{Succ} function on the largest possible value will generate a range
check error if range checking is enabled.
\end{enumerate}
\subsubsection{Integers}
\index{Types!Integer}
A list of pre-defined integer types is presented in \seet{integerstyp}
%\index{Keyword!Integer}
%
\begin{table}[ht]
\caption{Predefined integer types}
\label{tab:integerstyp}
\begin{center}
\begin{tabular}{l}
%\begin{FPCltable}{l}{Predefined integer types}{integerstyp}
Name\\ \hline
Integer \\
Shortint \\
SmallInt \\
Longint \\
Longword \\
Int64 \\
Byte \\
Word \\
Cardinal \\
QWord \\
Boolean \\
ByteBool \\
LongBool \\
Char \\ \hline
\end{tabular}
\end{center}
\end{table}
%\end{FPCltable}
The integer types, and their ranges and sizes, that are predefined in
\fpc are listed in \seet{integersranges}. It is to note that
the \var{qword} and \var{int64} types are not true ordinals, so
some pascal constructs will not work with these two integer types.
\begin{FPCltable}{lcr}{Predefined integer types}{integersranges}
Type & Range & Size in bytes \\ \hline
Byte & 0 .. 255 & 1 \\
Shortint & -128 .. 127 & 1\\
Smallint & -32768 .. 32767 & 2\\
Word & 0 .. 65535 & 2 \\
Integer & either smallint, longint or int64  & size 2,4 or 8 \\
Cardinal & either word, longword or qword  & size 2,4 or 8 \\
Longint & -2147483648 .. 2147483647 & 4\\
Longword & 0..4294967295 & 4 \\
Int64 & -9223372036854775808 .. 9223372036854775807 & 8 \\
QWord & 0 .. 18446744073709551615 & 8 \\ \hline
\end{FPCltable}

The \var{integer} type maps to the smallint type in the default\keywordlink{integer}
\fpc mode. It maps to either a longint or int64 in either Delphi or ObjFPC
mode. The \var{cardinal} type is currently always mapped to the 
longword type. The definition of the \var{cardinal} and \var{integer}
types may change from one architecture to another and from one
compiler mode to another. They usually have the same size as the
underlying target architecture.

%  This IS NOT TRUE, this is a 32-bit compiler, so the integer type
% will always be the same independently the CPU type.
%This is summarized in \seet{integer32type} for 32-bit processors
%(such as Intel 80x86, Motorola 680x0, PowerPC 32-bit, SPARC v7, MIPS32), and
%in \seet{integer64type} for 64-bit processors (such as Alpha AXP,
%SPARC v9 or later, Intel Itanium, MIPS64).

%\begin{FPCltable}{lcr}{\var{Integer} type mapping for 32-bit processors}{integer32type}
%Compiler mode & Range & Size in bytes \\ \hline
%<default> & -32768 .. 32767 & 2\\
%tp & -32768 .. 32767 & 2\\
%Delphi    & -2147483648 .. 2147483647 & 4\\
%ObjFPC    & -2147483648 .. 2147483647 & 4\\
%\end{FPCltable}

%\begin{FPCltable}{lcr}{\var{Integer} type mapping for 64-bit processors}{integer64type}
%Compiler mode & Range & Size in bytes \\ \hline
%<default> & -32768 .. 32767 & 2\\
%tp & -32768 .. 32767 & 2\\
%Delphi    & -9223372036854775808 .. 9223372036854775807 & 8 \\
%ObjFPC   &  -9223372036854775808 .. 9223372036854775807 & 8 \\
%\end{FPCltable}

\fpc does automatic type conversion in expressions where different kinds of
integer types are used.
\subsubsection{Boolean types}
\index{Types!Boolean}\index{Boolean}
\fpc supports the \var{Boolean} type, with its two pre-defined possible
values \var{True} and \var{False}. It also supports the \var{ByteBool},
\var{WordBool} and \var{LongBool} types. These are the only two values that can be
assigned to a \var{Boolean} type. Of course, any expression that resolves
to a \var{boolean} value, can also be assigned to a boolean type.
\begin{FPCltable}{lll}{Boolean types}{booleantypes}
Name & Size & Ord(True) \\ \hline
Boolean & 1 & 1 \\
ByteBool & 1 & Any nonzero value \\
WordBool & 2 & Any nonzero value \\
LongBool & 4 & Any nonzero value \\ \hline
\end{FPCltable}
Assuming \var{B} to be of type \var{Boolean}, the following are valid
assignments:
\begin{verbatim}
 B := True;
 B := False;
 B := 1<>2;  { Results in B := True }
\end{verbatim}
Boolean expressions are also used in conditions.

\begin{remark}
In \fpc, boolean expressions are always evaluated in such a
way that when the result is known, the rest of the expression will no longer
be evaluated (Called short-cut evaluation). In the following example, the function \var{Func} will never
be called, which may have strange side-effects.
\begin{verbatim}
 ...
 B := False;
 A := B and Func;
\end{verbatim}
Here \var{Func} is a function which returns a \var{Boolean} type.
\end{remark}

\subsubsection{Enumeration types}
\index{Types!Enumeration}
Enumeration types are supported in \fpc. On top of the Turbo Pascal
implementation, \fpc allows also a C-style extension of the
enumeration type, where a value is assigned to a particular element of
the enumeration list.
\input{syntax/typeenum.syn}
(see \seec{Expressions} for how to use expressions)
When using assigned enumerated types, the assigned elements must be in
ascending numerical order in the list, or the compiler will complain.
The expressions used in assigned enumerated elements must be known at
compile time.
So the following is a correct enumerated type declaration:
\begin{verbatim}
Type
  Direction = ( North, East, South, West );
\end{verbatim}
The C style enumeration type looks as follows:
\begin{verbatim}
Type
  EnumType = (one, two, three, forty := 40,fortyone);
\end{verbatim}
As a result, the ordinal number of \var{forty} is \var{40}, and not \var{3},
as it would be when the \var{':= 40'} wasn't present.
The ordinal value of \var{fortyone} is then {41}, and not \var{4}, as it
would be when the assignment wasn't present. After an assignment in an
enumerated definition the compiler adds 1 to the assigned value to assign to
the next enumerated value.
When specifying such an enumeration type, it is important to keep in mind
that the enumerated elements should be kept in ascending order. The
following will produce a compiler error:
\begin{verbatim}
Type
  EnumType = (one, two, three, forty := 40, thirty := 30);
\end{verbatim}
It is necessary to keep \var{forty} and \var{thirty} in the correct order.
When using enumeration types it is important to keep the following points
in mind:
\begin{enumerate}
\item The \var{Pred} and \var{Succ} functions cannot be used on
this kind of enumeration types. Trying to do this anyhow will result in a
compiler error.
\item Enumeration types stored using a default size. This behaviour can be changed
with the \var{\{\$PACKENUM n\}} compiler directive, which
tells the compiler the minimal number of bytes to be used for enumeration
types.
For instance
\begin{verbatim}
Type
{$PACKENUM 4}
  LargeEnum = ( BigOne, BigTwo, BigThree );
{$PACKENUM 1}
  SmallEnum = ( one, two, three );
Var S : SmallEnum;
    L : LargeEnum;
begin
  WriteLn ('Small enum : ',SizeOf(S));
  WriteLn ('Large enum : ',SizeOf(L));
end.
\end{verbatim}
will, when run, print the following:
\begin{verbatim}
Small enum : 1
Large enum : 4
\end{verbatim}
\end{enumerate}
More information can be found in the \progref, in the compiler directives
section.
\subsubsection{Subrange types}
\index{Types!Subrange}
A subrange type is a range of values from an ordinal type (the {\em host}
type). To define a subrange type, one must specify it's limiting values: the
highest and lowest value of the type.
\input{syntax/typesubr.syn}
Some of the predefined \var{integer} types are defined as subrange types:
\begin{verbatim}
Type
  Longint  = $80000000..$7fffffff;
  Integer  = -32768..32767;
  shortint = -128..127;
  byte     = 0..255;
  Word     = 0..65535;
\end{verbatim}
Subrange types of enumeration types can also be defined:
\begin{verbatim}
Type
  Days = (monday,tuesday,wednesday,thursday,friday,
          saturday,sunday);
  WorkDays = monday .. friday;
  WeekEnd = Saturday .. Sunday;
\end{verbatim}
%
\subsection{Real types}
\index{Types!Real}
\fpc uses the math coprocessor (or emulation) for all its floating-point
calculations. The Real native type is processor dependant,
but it is either Single or Double. Only the IEEE floating point types are
supported, and these depend on the target processor and emulation options.
The true Turbo Pascal compatible types are listed in
\seet{Reals}.\index{Real}\index{Single}\index{Double}\index{Extended}\index{Comp}\index{Currency}
\begin{FPCltable}{lccr}{Supported Real types}{Reals}
Type & Range & Significant digits & Size \\ \hline
Real & platform dependant & ??? & 4 or 8 \\
Single & 1.5E-45 .. 3.4E38 & 7-8 & 4 \\
Double & 5.0E-324 .. 1.7E308 & 15-16 & 8 \\
Extended & 1.9E-4932 .. 1.1E4932 & 19-20 & 10\\
Comp & -2E64+1 .. 2E63-1 & 19-20 & 8  \\
Currency & -922337203685477.5808 & 922337203685477.5807 & 8 \\
\end{FPCltable}
The \var{Comp} type is, in effect, a 64-bit integer and is not available
on all target platforms. To get more information on the supported types
for each platform, refer to the \progref.

The currency type is a fixed-point real data type which is internally used
as an 64-bit integer type (automatically scaled with a factor 10000), this 
minimalizes rounding errors.

%%%%%%%%%%%%%%%%%%%%%%%%%%%%%%%%%%%%%%%%%%%%%%%%%%%%%%%%%%%%%%%%%%%%%%%
% Character types
\section{Character types}
\index{Types!Char}
\subsection{Char}
\index{Char}
\fpc supports the type \var{Char}. A \var{Char} is exactly 1 byte in
size, and contains one character.
A character constant can be specified by enclosing the character in single
quotes, as follows : 'a' or 'A' are both character constants.
A character can also be specified by its character
value (commonly an ASCII code), by preceding the ordinal value with the 
number symbol (\#). For example specifying \var{\#65} would be the same as \var{'A'}.
Also, the caret character (\verb+^+) can be used in combination with a letter to
specify a character with ASCII value less than 27. Thus \verb+^G+ equals
\var{\#7} (G is the seventh letter in the alphabet.)
When the single quote character must be represented, it should be typed
two times successively, thus \var{''''} represents the single quote character.

\subsection{Strings}
\index{Types!String}
\fpc supports the \var{String} type as it is defined in Turbo Pascal
(A sequence of characters with a specified length) and it
supports ansistrings as in Delphi.
To declare a variable as a string, use the following type specification:
\input{syntax/sstring.syn}

The meaning of a string declaration statement is interpreted differently
depending on the \var{\{\$H\}} switch. The above declaration can declare an
ansistrng or a short string.

Whatever the actual type, ansistrings and short strings can be used
interchangeably. The compiler always takes care of the necessary type
conversions. Note, however, that the result of an expression that contains
ansistrings and short strings will always be an ansistring.

\subsection{Short strings}
\index{Shortstring}
A string declaration declares a short string in the following cases:

\begin{enumerate}
\item If the switch is off: \var{\{\$H-\}}, the string declaration
will always be a short string declaration.
\item If the switch is on \var{\{\$H+\}}, and there is a length
specifier, the declaration is a short string declaration.
\end{enumerate}
The predefined type \var{ShortString} is defined as a string of length 255:
\begin{verbatim}
 ShortString = String[255];
\end{verbatim}

If the size of the string is not specified, \var{255} is taken as a
default. The length of the string can be obtained with the \var{Length}
standard runtime routine.
For example in
\begin{verbatim}
{$H-}

Type
  NameString = String[10];
  StreetString = String;
\end{verbatim}
\var{NameString} can contain a maximum of 10 characters. While
\var{StreetString} can contain up to 255 characters.

\subsection{Ansistrings}
\index{Ansistring}\index{Types!Ansistring}\index{Types!Reference counted}
Ansistrings are strings that have no length limit. They are reference
counted and null terminated. Internally, an ansistring is treated as 
a pointer. This is all handled transparantly, i.e. they can be manipulated 
as a normal short string. Ansistrings can be defined using the predefined
\var{AnsiString} type.

If the \var{\{\$H\}} switch is on, then a string definition using the
regular \var{String} keyword and that doesn't contain a length specifier, 
will be regarded as an ansistring as well. If a length specifier is present,
a short string will be used, regardless of the \var{\{\$H\}} setting.

If the string is empty (\var{''}), then the internal pointer representation
of the string pointer is \var{Nil}. If the string is not empty, then the 
pointer points to a structure in heap memory.

The internal representation as a pointer, and the automatic null-termination
make it possible to typecast\index{Typecast} an ansistring to a pchar. If the string is empty 
(so the pointer is nil) then the compiler makes sure that the typecasted 
pchar will point to a null byte.

Assigning one ansistring to another doesn't involve moving the actual
string. A statement
\begin{verbatim}
  S2:=S1;
\end{verbatim}
results in the reference count of \var{S2} being decreased by one,
The referece count of \var{S1} is increased by one, and finally \var{S1}
(as a pointer) is copied to \var{S2}. This is a significant speed-up in
the code.

If the reference count reaches zero, then the memory occupied by the
string is deallocated automatically, so no memory leaks arise.

When an ansistring is declared, the \fpc compiler initially
allocates just memory for a pointer, not more. This pointer is guaranteed
to be nil, meaning that the string is initially empty. This is
true for local and global ansistrings or anstrings that are part of a 
structure (arrays, records or objects).

This does introduce an overhead. For instance, declaring
\begin{verbatim}
Var
  A : Array[1..100000] of string;
\end{verbatim}
Will copy 100,000 times \var{nil} into \var{A}. When \var{A} goes out of scope
\index{Scope}, then
the reference count of the 100,000 strings will be decreased by 1 for each
of these strings. All this happens
invisibly for the programmer, but when considering performance issues,
this is important.

Memory will be allocated only when the string is assigned a value.
If the string goes out of scope, then its reference count is automatically 
decreased by 1. If the reference count reaches zero, the memory reserved for
the string is released.

If a value is assigned to a character of a string that has a reference count
greater than 1, such as in the following
statements:
\begin{verbatim}
  S:=T;  { reference count for S and T is now 2 }
  S[I]:='@';
\end{verbatim}
then a copy of the string is created before the assignment. This is known
as {\em copy-on-write} semantics.

The \var{Length} function must be used to get the length of an
ansistring.

To set the length of an ansistring, the \var{SetLength} function must be used.
Constant ansistrings have a reference count of -1 and are treated specially.

Ansistrings are converted to short strings by the compiler if needed,
this means that the use of ansistrings and short strings can be mixed
without problems.

Ansistrings can be typecasted\index{Typecast} to \var{PChar} or \var{Pointer} types:
\index{Types!PChar}\index{PChar}
\begin{verbatim}
Var P : Pointer;
    PC : PChar;
    S : AnsiString;

begin
  S :='This is an ansistring';
  PC:=Pchar(S);
  P :=Pointer(S);
\end{verbatim}
There is a difference between the two typecasts. When an empty
ansistring is typecasted to a pointer, the pointer wil be \var{Nil}. If an
empty ansistring is typecasted to a \var{PChar}, then the result will be a pointer to a
zero byte (an empty string).

The result of such a typecast must be used with care. In general, it is best
to consider the result of such a typecast as read-only, i.e. suitable for
passing to a procedure that needs a constant pchar argument.

It is therefore {\em not} advisable to typecast one of the following:
\begin{enumerate}
\item expressions.
\item strings that have reference count larger than 1.
(call uniquestring to ensure a string has reference count 1)
\end{enumerate}

\subsection{WideStrings}
\index{Widestring}\index{Types!Widestring}\index{Types!Reference counted}
Widestrings (used to represent unicode character strings) are implemented in much 
the same way as ansistrings: reference counted, null-terminated arrays, only they 
are implemented as arrays of \var{WideChars} instead of regular \var{Chars}.
A \var{WideChar} is a two-byte character (an element of a DBCS: Double Byte
Character Set). Mostly the same rules apply for \var{WideStrings} as for 
\var{AnsiStrings}. The compiler transparantly converts WideStrings to
AnsiStrings and vice versa. \index{Ansistring}

Similarly to the typecast\index{Typecast} of an Ansistring to a \var{PChar} null-terminated
\index{PChar}\index{PWideChar}%
array of characters, a WideString can be converted to a \var{PWideChar}
null-terminated array of characters. 
Note that the \var{PWideChar} array is terminated by 2 null bytes instead of
1, so a typecast to a pchar is not automatic.

The compiler itself provides no support for any conversion from Unicode to
ansistrings or vice versa; 2 procedural variables are present in the system 
unit which can be set to handle the conversion. For more information, see
the system units reference.

% Constant strings
\subsection{Constant strings}
\index{Constants!String}
To specify a constant string, it must be enclosed in single-quotes, just
as a \var{Char} type, only now more than one character is allowed.
Given that \var{S} is of type \var{String}, the following are valid assignments:
\begin{verbatim}
S := 'This is a string.';
S := 'One'+', Two'+', Three';
S := 'This isn''t difficult !';
S := 'This is a weird character : '#145' !';
\end{verbatim}
As can be seen, the single quote character is represented by 2 single-quote
characters next to each other. Strange characters can be specified by their
character value (usually an ASCII code).
The example shows also that two strings can be added. The resulting string is
just the concatenation of the first with the second string, without spaces in
between them. Strings can not be substracted, however.

Whether the constant string is stored as an ansistring or a short string
depends on the settings of the \var{\{\$H\}} switch.

% PChar
\subsection{PChar - Null terminated strings}
\index{Types!PChar}\index{Types!Pointer}
\fpc supports the Delphi implementation of the \var{PChar} type. \var{PChar}
is defined as a pointer to a \var{Char} type, but allows additional
operations.
The \var{PChar} type can be understood best as the Pascal equivalent of a
C-style null-terminated string, i.e. a variable of type \var{PChar} is a
pointer that points to an array of type \var{Char}, which is ended by a
null-character (\var{\#0}).
\fpc supports initializing of \var{PChar} typed constants, or a direct
assignment. For example, the following pieces of code are equivalent:
\begin{verbatim}
program one;
var p : PChar;
begin
  P := 'This is a null-terminated string.';
  WriteLn (P);
end.
\end{verbatim}
Results in the same as
\begin{verbatim}
program two;
const P : PChar = 'This is a null-terminated string.'
begin
  WriteLn (P);
end.
\end{verbatim}
These examples also show that it is possible to write {\em the contents} of
the string to a file of type \var{Text}.
The \seestrings unit contains procedures and functions that manipulate the
\var{PChar} type as in the standard C library.
Since it is equivalent to a pointer to a type \var{Char} variable, it  is
also possible to do the following:
\begin{verbatim}
Program three;
Var S : String[30];
    P : PChar;
begin
  S := 'This is a null-terminated string.'#0;
  P := @S[1];
  WriteLn (P);
end.
\end{verbatim}
This will have the same result as the previous two examples.
Null-terminated strings cannot be added as normal Pascal
strings. If two \var{PChar} strings mustt be concatenated; the functions from
the unit \seestrings must be used.

However, it is possible to do some pointer arithmetic. The \index{Operators}
operators \var{+} and \var{-} can be used to do operations on \var{PChar} pointers.
In \seet{PCharMath}, \var{P} and \var{Q} are of type \var{PChar}, and
\var{I} is of type \var{Longint}.
\begin{FPCltable}{lr}{\var{PChar} pointer arithmetic}{PCharMath}
Operation & Result \\ \hline
\var{P + I} & Adds \var{I} to the address pointed to by \var{P}. \\
\var{I + P} & Adds \var{I} to the address pointed to by \var{P}. \\
\var{P - I} & Substracts \var{I} from the address pointed to by \var{P}. \\
\var{P - Q} & Returns, as an integer, the distance between 2 addresses \\
 & (or the number of characters between \var{P} and \var{Q}) \\
\hline
\end{FPCltable}

%%%%%%%%%%%%%%%%%%%%%%%%%%%%%%%%%%%%%%%%%%%%%%%%%%%%%%%%%%%%%%%%%%%%%%%
% Structured Types
\section{Structured Types}
\index{Types!Structured}
A structured type is a type that can hold multiple values in one variable.
Stuctured types can be nested to unlimited levels.
\input{syntax/typestru.syn}
Unlike Delphi, \fpc does not support the keyword \var{Packed} for all
structured types, as can be seen in the syntax diagram. It will be mentioned
when a type supports the \var{packed} keyword.
In the following, each of the possible structured types is discussed.
%
\subsubsection{Packed structured types}
When a structured type is declared, no assumptions should be made about
the internal position of the elements in the type. The compiler will lay
out  the elements of the structure in memory as it thinks will be most
suitable. That is, the order of the elements will be kept, but the location
of the elements is not guaranteed, and is partially governed by the \var{\$PACKRECORDS}
directive (this directive is explained in the \progref).

However, \fpc allows to control the layout with the \var{Packed} and
\var{Bitpacked} keywords. The meaning of these words depends on the context:
\begin{description}
\item[Bitpacked] In this case, the compiler will attempt to align ordinal
types on bit boundaries, as explained below.
\item[Packed] the meaning of the \var{Packed} keyword depends on the
situation:
\begin{enumerate}
\item in \var{MACPAS} mode, it is equivalent to the \var{Bitpacked} keyword.
\item In other modes, with the \var{\$BITPACKING} directive set to \var{ON},
it is also equivalent to the \var{Bitpacked} keyword.
\item In other modes, with the \var{\$BITPACKING} directive set to \var{OFF},
it signifies normal packing on byte boundaries.
\end{enumerate}
Packing on byte boundaries means that each new element of a structured type
starts on a byte boundary.
\end{description}

The byte packing mechanism is simple: the compiler aligns each element of
the structure on the first available byte boundary, even if size of the
previous element (small enumerated types, subrange types) is less than a
byte.

When using the bit packing mechanism, the compiler calculates for each
ordinal type how many bits are needed to store it. The next ordinal type
is then stored on the next free bit. Non-ordinal types, which include but
are not limited sets, floats, strings, (bitpacked) records, (bitpacked)
arrays, pointers, classes, objects, and procedural variables, are stored
on the first available byte boundary.

Note that the internals of the bitpacking are opaque: they can change
at any time in the future. What is more: the internal packing depends
on the endianness of the platform for which the compilation is done,
and no conversion between platforms is possible. This makes bitpacked
structures unsuitable for storing on disk or transport over networks.
The format is however the same as the one used by the GNU Pascal
Compiler, and we aim to retain this compatibility in the future.

There are some more restrictions to elements of bitpacked structures:
\begin{itemize}
\item The address cannot be retrieved, unless the bit size is a multiple of
8 and the element happens to be stored on a byte boundary.
\item An element of a bitpacked structure cannot be used as a var parameter,
unless the bit size is a multiple of 8 and the element happens to be stored 
on a byte boundary.
\end{itemize}

To determine the size of an element in a bitpacked structure, there is the 
\var{BitSizeOf} function. It returns the size - in bits - of the element. 
For other types or elements of structures which are not bitpacked, this will 
simply return the size in bytes multiplied by 8, i.e., the return value is 
then the same as \var{8*SizeOf}.

The size of bitpacked records and arrays is limited:
\begin{itemize}
\item On 32 bit systems the maximal size is $2^29$ bytes (512 MB).
\item On 32 bit systems the maximal size is $2^61$ bytes.
\end{itemize}
The reason is that the offset of an element must be calculated with 
the maximum integer size of the system.

%
\subsection{Arrays}
\index{Types!Array}\index{Array}
\fpc supports arrays as in Turbo Pascal, multi-dimensional arrays
and packed arrays are also supported, as well as the dynamic arrays of
Delphi:
\input{syntax/typearr.syn}
Note that arrays can be packed or bitpacked.
%
\subsubsection{Static arrays}
\index{Types!Array}\index{Array!Static}\keywordlink{array}
When the range of the array is included in the array definition, it is
called a static array. Trying to access an element with an index that is
outside the declared range will generate a run-time error (if range checking
is on).  The following is an example of a valid array declaration:
\begin{verbatim}
Type
  RealArray = Array [1..100] of Real;
\end{verbatim}
Valid indexes for accessing an element of the array are between 1 and 100,
where the borders 1 and 100 are included.
As in Turbo Pascal, if the array component type is in itself an array, it is
possible to combine the two arrays into one multi-dimensional array. The
following declaration:
\begin{verbatim}
Type
   APoints = array[1..100] of Array[1..3] of Real;
\end{verbatim}
is equivalent to the following declaration:
\begin{verbatim}
Type
   APoints = array[1..100,1..3] of Real;
\end{verbatim}
The functions \var{High} and \var{Low} return the high and low bounds of
the leftmost index type of the array. In the above case, this would be 100
and 1. You should use them whenever possible, since it improves maintainability
of your code. The use of both functions is just as efficient as using constants.

When static array-type variables are assigned to each other, the contents of the
whole array is copied. This is also true for multi-dimensional arrays:
\begin{verbatim}
program testarray1;

Type
  TA = Array[0..9,0..9] of Integer;
  
var   
  A,B : TA;
  I,J : Integer;
begin
  For I:=0 to 9 do
    For J:=0 to 9 do 
      A[I,J]:=I*J;
  For I:=0 to 9 do
    begin
    For J:=0 to 9 do 
      Write(A[I,J]:2,' ');
    Writeln;
    end;
  B:=A;
  Writeln;
  For I:=0 to 9 do
    For J:=0 to 9 do 
      A[9-I,9-J]:=I*J;
  For I:=0 to 9 do
    begin
    For J:=0 to 9 do 
      Write(B[I,J]:2,' ');
    Writeln;
    end;
end.  
\end{verbatim}
The output will be 2 identical matrices.

\subsubsection{Dynamic arrays}
\index{Types!Array}\index{Array!Dynamic}\index{Types!Reference counted}
As of version 1.1, \fpc also knows dynamic arrays: In that case, the array
range is omitted, as in the following example:
\begin{verbatim}
Type
  TByteArray : Array of Byte;
\end{verbatim}
When declaring a variable of a dynamic array type, the initial length of the
array is zero. The actual length of the array must be set with the standard
\var{SetLength} function, which will allocate the memory to contain the
array elements on the heap. The following example will set the length to
1000:
\begin{verbatim}
Var 
  A : TByteArray;

begin
  SetLength(A,1000);
\end{verbatim}
After a call to \var{SetLength}, valid array indexes are 0 to 999: the array
index is always zero-based.

Note that the length of the array is set in elements, not in bytes of 
allocated memory (although these may be the same). The amount of 
memory allocated is the size of the array multiplied by the size of 
1 element in the array. The memory will be disposed of at the exit of the
current procedure or function. 

It is also possible to resize the array: in that case, as much of the 
elements in the array as will fit in the new size, will be kept. The array
can be resized to zero, which effectively resets the variable.

At all times, trying to access an element of the array that is not in the
current length of the array will generate a run-time error.

Assignment of one dynamic array-type variable to another will let both
variables point to the same array. Contrary to ansistrings, an 
assignment to an element of one array will be reflected in the 
other:
\begin{verbatim}
Var
  A,B : TByteArray;

begin
  SetLength(A,10);
  A[1]:=33;
  B:=A;
  A[1]:=31;
\end{verbatim}
After the second assignment, the first element in B will also contain 31.

It can also be seen from the output of the following example:
\begin{verbatim}
program testarray1;

Type
  TA = Array of array of Integer;
  
var   
  A,B : TA;
  I,J : Integer;
begin
  Setlength(A,10,10);
  For I:=0 to 9 do
    For J:=0 to 9 do 
      A[I,J]:=I*J;
  For I:=0 to 9 do
    begin
    For J:=0 to 9 do 
      Write(A[I,J]:2,' ');
    Writeln;
    end;
  B:=A;
  Writeln;
  For I:=0 to 9 do
    For J:=0 to 9 do 
      A[9-I,9-J]:=I*J;
  For I:=0 to 9 do
    begin
    For J:=0 to 9 do 
      Write(B[I,J]:2,' ');
    Writeln;
    end;
end.  
\end{verbatim}
The output will be a matrix of numbers, and then the same matrix, mirrorred.

\index{Types!Reference counted}
Dynamic arrays are reference counted: if in one of the previous examples A
goes out of \index{Scope} scope and B does not, then the array is not yet disposed of: the
reference count of A (and B) is decreased with 1. As soon as the reference
count reaches zero, the memory is disposed of.

It is also possible to copy and/or resize the array with the standard 
\var{Copy} function, which acts as the copy function for strings:
\begin{verbatim}
program testarray3;

Type
  TA = array of Integer;
  
var   
  A,B : TA;
  I : Integer;

begin
  Setlength(A,10);
  For I:=0 to 9 do
      A[I]:=I;
  B:=Copy(A,3,9);    
  For I:=0 to 5 do
    Writeln(B[I]);
end.  
\end{verbatim}
The \var{Copy} function will copy 9 elements of the array to a new array.
Starting at the element at index 3 (i.e. the fourth element) of the array.

The \var{Low} function on a dynamic array will always return 0, and the
High function will return the value \var{Length-1}, i.e., the value of the
highest allowed array index. The \var{Length} function will return the
number of elements in the array.


\subsubsection{Packing and unpacking an array}
Arrays can be packed and bitpacked. 2 array types which have the same index
type but which are differently packed are not assignment compatible.

However, it is possible to convert a normal array to a bitpacked array with the
\var{pack} routine. The reverse operation is possible as well; a bitpacked
array can be converted to a normally packed array using the \var{unpack}
routine, as in the following example:
\begin{verbatim}
Var
  foo : array [ 'a'..'f' ] of Boolean 
    = ( false, false, true, false, false, false );
  bar : packed array [ 42..47 ] of Boolean;
  baz : array [ '0'..'5' ] of Boolean;

begin
  pack(foo,'a',bar);
  unpack(bar,baz,'0');
end.
\end{verbatim}
More information about the pack and unpack routines can be found in the unit
reference.

%
\subsection{Record types}
\index{Record}\index{Types!Record}\index{Fields}\keywordlink{record}
\fpc supports fixed records and records with variant parts.
The syntax diagram for a record type is
\input{syntax/typerec.syn}
\index{Packed} So the following are valid record types declarations:
\begin{verbatim}
Type
  Point = Record
          X,Y,Z : Real;
          end;
  RPoint = Record
          Case Boolean of
          False : (X,Y,Z : Real);
          True : (R,theta,phi : Real);
          end;
  BetterRPoint = Record
          Case UsePolar : Boolean of
          False : (X,Y,Z : Real);
          True : (R,theta,phi : Real);
          end;
\end{verbatim}
The variant part must be last in the record. The optional identifier in the
case statement serves to access the tag field value, which otherwise would
be invisible to the programmer. It can be used to see which variant is
active at a certain time. In effect, it introduces a new field in the
record.
\begin{remark}
It is possible to nest variant parts, as in:
\begin{verbatim}
Type
  MyRec = Record
          X : Longint;
          Case byte of
            2 : (Y : Longint;
                 case byte of
                 3 : (Z : Longint);
                 );
          end;
\end{verbatim}
\end{remark}
The size of a record is the sum of the sizes of its fields, each size of a
field is rounded up to a power of two. If the record contains a variant part, the size
of the variant part is the size of the biggest variant, plus the size of the
tag field type {\em if an identifier was declared for it}. Here also, the size of
each part is first rounded up to two. So in the above example,
\var{SizeOf} would return 24 for \var{Point}, 24 for \var{RPoint} and
26 for \var{BetterRPoint}. For \var{MyRec}, the value would be 12.
If a typed file with records, produced by a Turbo Pascal program, must be read,
then chances are that attempting to read that file correctly will fail.
The reason for this is that by default, elements of a record are aligned at
2-byte boundaries, for performance reasons. This default behaviour can be
changed with the \var{\{\$PackRecords n\}} switch. Possible values for
\var{n} are 1, 2, 4, 16 or \var{Default}.\index{Packed}
This switch tells the compiler to align elements of a record or object or
class that have size larger than \var{n} on \var{n} byte boundaries.
Elements that have size smaller or equal than \var{n} are aligned on
natural boundaries, i.e. to the first power of two that is larger than or
equal to the size of the record element.
The keyword \var{Default} selects the default value for the platform
that the code is compiled for (currently, this is 2 on all platforms)
Take a look at the following program:
\begin{verbatim}
Program PackRecordsDemo;
type
   {$PackRecords 2}
     Trec1 = Record
       A : byte;
       B : Word;
     end;

     {$PackRecords 1}
     Trec2 = Record
       A : Byte;
       B : Word;
       end;
   {$PackRecords 2}
     Trec3 = Record
       A,B : byte;
     end;

    {$PackRecords 1}
     Trec4 = Record
       A,B : Byte;
       end;
   {$PackRecords 4}
     Trec5 = Record
       A : Byte;
       B : Array[1..3] of byte;
       C : byte;
     end;

     {$PackRecords 8}
     Trec6 = Record
       A : Byte;
       B : Array[1..3] of byte;
       C : byte;
       end;
   {$PackRecords 4}
     Trec7 = Record
       A : Byte;
       B : Array[1..7] of byte;
       C : byte;
     end;

     {$PackRecords 8}
     Trec8 = Record
       A : Byte;
       B : Array[1..7] of byte;
       C : byte;
       end;
Var rec1 : Trec1;
    rec2 : Trec2;
    rec3 : TRec3;
    rec4 : TRec4;
    rec5 : Trec5;
    rec6 : TRec6;
    rec7 : TRec7;
    rec8 : TRec8;

begin
  Write ('Size Trec1 : ',SizeOf(Trec1));
  Writeln (' Offset B : ',Longint(@rec1.B)-Longint(@rec1));
  Write ('Size Trec2 : ',SizeOf(Trec2));
  Writeln (' Offset B : ',Longint(@rec2.B)-Longint(@rec2));
  Write ('Size Trec3 : ',SizeOf(Trec3));
  Writeln (' Offset B : ',Longint(@rec3.B)-Longint(@rec3));
  Write ('Size Trec4 : ',SizeOf(Trec4));
  Writeln (' Offset B : ',Longint(@rec4.B)-Longint(@rec4));
  Write ('Size Trec5 : ',SizeOf(Trec5));
  Writeln (' Offset B : ',Longint(@rec5.B)-Longint(@rec5),
           ' Offset C : ',Longint(@rec5.C)-Longint(@rec5));
  Write ('Size Trec6 : ',SizeOf(Trec6));
  Writeln (' Offset B : ',Longint(@rec6.B)-Longint(@rec6),
           ' Offset C : ',Longint(@rec6.C)-Longint(@rec6));
  Write ('Size Trec7 : ',SizeOf(Trec7));
  Writeln (' Offset B : ',Longint(@rec7.B)-Longint(@rec7),
           ' Offset C : ',Longint(@rec7.C)-Longint(@rec7));
  Write ('Size Trec8 : ',SizeOf(Trec8));
  Writeln (' Offset B : ',Longint(@rec8.B)-Longint(@rec8),
           ' Offset C : ',Longint(@rec8.C)-Longint(@rec8));
end.
\end{verbatim}
The output of this program will be :
\begin{verbatim}
Size Trec1 : 4 Offset B : 2
Size Trec2 : 3 Offset B : 1
Size Trec3 : 2 Offset B : 1
Size Trec4 : 2 Offset B : 1
Size Trec5 : 8 Offset B : 4 Offset C : 7
Size Trec6 : 8 Offset B : 4 Offset C : 7
Size Trec7 : 12 Offset B : 4 Offset C : 11
Size Trec8 : 16 Offset B : 8 Offset C : 15
\end{verbatim}
And this is as expected. In \var{Trec1}, since \var{B} has size 2, it is
aligned on a 2 byte boundary, thus leaving an empty byte between \var{A}
and \var{B}, and making the total size 4. In \var{Trec2}, \var{B} is aligned
on a 1-byte boundary, right after \var{A}, hence, the total size of the
record is 3.
For \var{Trec3}, the sizes of \var{A,B} are 1, and hence they are aligned on 1
byte boundaries. The same is true for \var{Trec4}.
For \var{Trec5}, since the size of B -- 3 -- is smaller than 4, \var{B} will
be on a  4-byte boundary, as this is the first power of two that is
larger than it's size. The same holds for \var{Trec6}.
For \var{Trec7}, \var{B} is aligned on a 4 byte boundary, since it's size --
7 -- is larger than 4. However, in \var{Trec8}, it is aligned on a 8-byte
boundary, since 8 is the first power of two that is greater than 7, thus
making the total size of the record 16.
\fpc supports also the 'packed record', this is a record where all the
elements are byte-aligned.
Thus the two following declarations are equivalent:
\begin{verbatim}
     {$PackRecords 1}
     Trec2 = Record
       A : Byte;
       B : Word;
       end;
     {$PackRecords 2}
\end{verbatim}
and
\begin{verbatim}
     Trec2 = Packed Record
       A : Byte;
       B : Word;
       end;
\end{verbatim}
Note the \var{\{\$PackRecords 2\}} after the first declaration !
%
\subsection{Set types}
\index{Set}\index{Types!Set}\keywordlink{set}
\fpc supports the set types as in Turbo Pascal. The prototype of a set
declaration is:
\input{syntax/typeset.syn}
Each of the elements of \var{SetType} must be of type \var{TargetType}.
\var{TargetType} can be any ordinal type with a range between \var{0} and
\var{255}. A set can contain maximally \var{255} elements.
The following are valid set declaration:
\begin{verbatim}
Type
  Junk = Set of Char;

  Days = (Mon, Tue, Wed, Thu, Fri, Sat, Sun);
  WorkDays : Set of days;
\end{verbatim}
Given this set declarations, the following assignment is legal:
\begin{verbatim}
WorkDays := [ Mon, Tue, Wed, Thu, Fri];
\end{verbatim}
The operators \index{Operators} and functions for manipulations of sets are listed in
\seet{SetOps}.
\begin{FPCltable}{lr}{Set Manipulation operators}{SetOps}
Operation & Operator \\ \hline
Union & + \\
Difference & - \\
Intersection & * \\
Add element & \var{include} \\
Delete element & \var{exclude} \\ \hline
\end{FPCltable}
Two sets can be compared with the \var{<>} and \var{=} operators, but not
(yet) with the \var{<} and \var{>} operators.
The compiler stores small sets (less than 32 elements) in a Longint, if the
type range allows it. This allows for faster processing and decreases
program size. Otherwise, sets are stored in 32 bytes.
%
\subsection{File types}
\index{File}\index{Text}\index{Types!File}\keywordlink{file}
File types are types that store a sequence of some base type, which can be
any type except another file type. It can contain (in principle) an infinite
number of elements.
File types are used commonly to store data on disk. Nothing prevents the programmer,
however, from writing a file driver that stores it's data in memory.
Here is the type declaration for a file type:
\input{syntax/typefil.syn}
If no type identifier is given, then the file is an untyped file; it can be
considered as equivalent to a file of bytes. Untyped files require special
commands to act on them (see \var{Blockread}, \var{Blockwrite}).
The following declaration declares a file of records:
\begin{verbatim}
Type
  Point = Record
    X,Y,Z : real;
    end;
  PointFile = File of Point;
\end{verbatim}
Internally, files are represented by the \var{FileRec} record, which is
declared in the DOS unit.

A special file type is the \var{Text} file type, represented by the
\var{TextRec} record. A file of type \var{Text} uses special input-output
routines.

%%%%%%%%%%%%%%%%%%%%%%%%%%%%%%%%%%%%%%%%%%%%%%%%%%%%%%%%%%%%%%%%%%%%%%%
% Pointers
\section{Pointers}
\index{Pointer}\index{Types!Pointer}\keywordlink{pointer}
\fpc supports the use of pointers. A variable of the pointer type
contains an address in memory, where the data of another variable may be
stored.
\input{syntax/typepoin.syn}
As can be seen from this diagram, pointers are typed, which means that
they point to a particular kind of data. The type of this data must be
known at compile time.
Dereferencing the pointer (denoted by adding \var{\^{}} after the variable
name) behaves then like a variable. This variable has the type declared in
the pointer declaration, and the variable is stored in the address that is
pointed to by the pointer variable.
Consider the following example:
\begin{verbatim}
Program pointers;
type
  Buffer = String[255];
  BufPtr = ^Buffer;
Var B  : Buffer;
    BP : BufPtr;
    PP : Pointer;
etc..
\end{verbatim}
In this example, \var{BP} {\em is a pointer to} a \var{Buffer} type; while \var{B}
{\em is} a variable of type \var{Buffer}. \var{B} takes 256 bytes memory,
and \var{BP} only takes 4 bytes of memory (enough to keep an adress in
memory).

The expression
\begin{verbatim}
 BP^
\end{verbatim}
Known as the dereferencing of \var{BP}, is of type \var{Buffer}, so
\begin{verbatim}
 BP^[23]
\end{verbatim}
Denotes the 23-rd character in the string pointed to by \var{BP}.
\begin{remark} \fpc treats pointers much the same way as C does. This means
that a pointer to some type can be treated as being an array of this type.
The pointer then points to the zeroeth element of this array. Thus the
following pointer declaration
\begin{verbatim}
Var p : ^Longint;
\end{verbatim}
Can be considered equivalent to the following array declaration:
\begin{verbatim}
Var p : array[0..Infinity] of Longint;
\end{verbatim}
The difference is that the former declaration allocates memory for the
pointer only (not for the array), and the second declaration allocates
memory for the entire array. If the former is used, the memory must be
allocated manually, using the \var{Getmem} function.
The reference \var{P\^{}} is then the same as \var{p[0]}. The following program
illustrates this maybe more clear:
\begin{verbatim}
program PointerArray;
var i : Longint;
    p : ^Longint;
    pp : array[0..100] of Longint;
begin
  for i := 0 to 100 do pp[i] := i; { Fill array }
  p := @pp[0];                     { Let p point to pp }
  for i := 0 to 100 do
    if p[i]<>pp[i] then
      WriteLn ('Ohoh, problem !')
end.
\end{verbatim}
\end{remark}
\fpc supports pointer arithmetic as C does. This means that, if \var{P} is a
typed pointer, the instructions
\begin{verbatim}
Inc(P);
Dec(P);
\end{verbatim}
Will increase, respectively decrease the address the pointer points to
with the size of the type \var{P} is a pointer to. For example
\begin{verbatim}
Var P : ^Longint;
...
 Inc (p);
\end{verbatim}
will increase \var{P} with 4.
Normal arithmetic operators \index{Operators} on pointers can also be used, 
that is, the following are valid pointer arithmetic operations:
\begin{verbatim}
var  p1,p2 : ^Longint;
     L : Longint;
begin
  P1 := @P2;
  P2 := @L;
  L := P1-P2;
  P1 := P1-4;
  P2 := P2+4;
end.
\end{verbatim}
Here, the value that is added or substracted {\em is } multiplied by the
size of the type the pointer points to. In the previous
example \var{P1} will be decremented by 16 bytes, and 
\var{P2} will be incremented by 16.

%%%%%%%%%%%%%%%%%%%%%%%%%%%%%%%%%%%%%%%%%%%%%%%%%%%%%%%%%%%%%%%%%%%%%%%
% Forward type declarations
\section{Forward type declarations}
\index{Forward}\index{Types!Forward declaration}
Programs often need to maintain a linked list of records. Each record then
contains a pointer to the next record (and possibly to the previous record
as well). For type safety, it is best to define this pointer as a typed
pointer, so the next record can be allocated on the heap using the \var{New}
call. In order to do so, the record should be defined something like this:
\begin{verbatim}
Type
  TListItem = Record
    Data : Integer;
    Next : ^TListItem;
  end;
\end{verbatim}  
When trying to compile this, the compiler will complain that the
\var{TListItem} type is not yet defined when it encounters the \var{Next}
declaration: This is correct, as the definition is still being parsed.

To be able to have the \var{Next} element as a typed pointer, a 'Forward
type declaration' must be introduced:
\begin{verbatim}
Type
  PListItem = ^TListItem;
  TListItem = Record
    Data : Integer;
    Next : PTListItem;
  end;
\end{verbatim}  
When the compiler encounters a typed pointer declaration where the
referenced type is not yet known, it postpones resolving the reference later
on: The pointer definition is a 'Forward type declaration'. The referenced
type should be introduced later in the same \var{Type} block. No other block
may come between the definition of the pointer type and the referenced type.
Indeed, even the word \var{Type} itself may not re-appear: in effect it
would start a new type-block, causing the compiler to resolve all pending
declarations in the current block. In most cases, the definition of the
referenced type will follow immediatly after the definition of the pointer
type, as shown in the above listing. The forward defined type can be used in
any type definition following its declaration.

Note that a forward type declaration is only possible with pointer types and
classes, not with other types.

%%%%%%%%%%%%%%%%%%%%%%%%%%%%%%%%%%%%%%%%%%%%%%%%%%%%%%%%%%%%%%%%%%%%%%%
% Procedural types
\section{Procedural types}
\index{Procedure}\index{Types!Procedural}\index{Procedural}
\fpc has support for procedural types, although it differs a little from
the Turbo Pascal implementation of them. The type declaration remains the
same, as can be seen in the following syntax diagram:
\input{syntax/typeproc.syn}
For a description of formal parameter lists, see \seec{Procedures}.
The two following examples are valid type declarations:
\begin{verbatim}
Type TOneArg = Procedure (Var X : integer);
     TNoArg = Function : Real;
var proc : TOneArg;
    func : TNoArg;
\end{verbatim}
One can assign the following values to a procedural type variable:
\begin{enumerate}
\item \var{Nil}, for both normal procedure pointers and method pointers.
\item A variable reference of a procedural type, i.e. another variable of
the same type.
\item A global procedure or function address, with matching function or
procedure header and calling convention.
\item A method address.
\end{enumerate}
Given these declarations, the following assignments are valid:
\begin{verbatim}
Procedure printit (Var X : Integer);
begin
  WriteLn (x);
end;
...
Proc := @printit;
Func := @Pi;
\end{verbatim}
From this example, the difference with Turbo Pascal is clear: In Turbo
Pascal it isn't necessary to use the address operator (\var{@})
when assigning a procedural type variable, whereas in \fpc it is required
(unless the \var{-So} switch is used, in which case the address
operator can be dropped.)
\begin{remark} The modifiers concerning the calling conventions
must be the same as the declaration;
i.e. the following code would give an error:
\begin{verbatim}
Type TOneArgCcall = Procedure (Var X : integer);cdecl;
var proc : TOneArgCcall;
Procedure printit (Var X : Integer);
begin
  WriteLn (x);
end;
begin
Proc := @printit;
end.
\end{verbatim}
Because the \var{TOneArgCcall} type is a procedure that uses the cdecl
calling convention.
\end{remark}

%%%%%%%%%%%%%%%%%%%%%%%%%%%%%%%%%%%%%%%%%%%%%%%%%%%%%%%%%%%%%%%%%%%%%%%
% Variant types
\section{Variant types}
\index{Variant}\index{Types!Variant}
%%%%%%%%%%%%%%%%%%%%%%%%%%%%%%%%%%%%%%%%%%%%%%%%%%%%%%%%%%%%%%%%%%%%%%%
% Definition
\subsection{Definition}
As of version 1.1, FPC has support for variants. For variant support to be
enabled, the \file{variants} unit must be included in every unit that uses
variants in some way. Furthermore, the compiler must be in \var{Delphi} or
\var{ObjFPC} mode.

The type of a value stored in a variant is only determined at runtime: 
it depends what has been assigned to the to the variant. Almost any type 
can be assigned to variants: ordinal types, string types, int64 types.
Structured types such as sets, records, arrays, files, objects and classes 
are not assign-compatible with a variant, as well as pointers. Interfaces
and COM or CORBA objects can be assigned to a
variant.\index{Interfaces}\index{COM}\index{CORBA}

This means that the following assignments are valid:
\begin{verbatim}
Type
  TMyEnum = (One,Two,Three);

Var
  V : Variant;
  I : Integer;
  B : Byte;
  W : Word;
  Q : Int64;
  E : Extended;
  D : Double;
  En : TMyEnum;
  AS : AnsiString;
  WS : WideString;

begin
  V:=I;
  V:=B;
  V:=W;
  V:=Q;
  V:=E;
  V:=En;
  V:=D:
  V:=AS;
  V:=WS;
end;
\end{verbatim}
And of course vice-versa as well.
\begin{remark}
The enumerated type assignment is broken in the early 1.1 development series of the
compiler. It is expected that this is fixed soon.
\end{remark}

A variant can hold an an array of values: All elements in the array have
the\index{array} same type (but can be of type 'variant'). For a variant 
that contains an array, the variant can be indexed:
\begin{verbatim}
Program testv;

uses variants;

Var
  A : Variant;
  I : integer;

begin
  A:=VarArrayCreate([1,10],varInteger);
  For I:=1 to 10 do
    A[I]:=I;
end.
\end{verbatim}
(for the explanation of \var{VarArrayCreate}, see \unitsref.)

Note that when the array contains a string, this is not considered an 'array
of characters', and so the variant cannot be indexed to retrieve a character
at a certain position in the string.

\begin{remark}
The array functionality is broken in the early 1.1 development series of the
compiler. It is expected that this is fixed soon.
\end{remark}

\subsection{Variants in assignments and expressions}
As can be seen from the definition above, most simple types can be assigned
to a variant. Likewise, a variant can be assigned to a simple type: If
possible, the value of the variant will be converted to the type that is
being assigned to. This may fail: Assigning a variant containing a string 
to an integer will fail unless the string represents a valid integer. In the
following example, the first assignment will work, the second will fail:
\begin{verbatim}
program testv3;

uses Variants;

Var
  V : Variant;
  I : Integer;

begin
  V:='100';
  I:=V;
  Writeln('I : ',I);
  V:='Something else';
  I:=V;
  Writeln('I : ',I);
end.
\end{verbatim}
The first assignment will work, but the second will not, as \var{Something else}
cannot be converted to a valid integer value. An \var{EConvertError} exception 
will be the result.

The result of an expression involving a variant will be of type variant again, 
but this can be assigned to a variable of a different type - if the result
can be converted to a variable of this type.

Note that expressions involving variants take more time to be evaluated, and
should therefore be used with caution. If a lot of calculations need to be
made, it is best to avoid the use of variants.

When considering implicit type conversions (e.g. byte to integer, integer to
double, char to string) the compiler will ignore variants unless a variant
appears explicitly in the expression. 
 
\subsection{Variants and interfaces}
\index{Interfaces}
\begin{remark}
Dispatch interface support for variants is currently broken in the compiler.
\end{remark}

Variants can contain a reference to an interface - a normal interface
(descending from \var{IInterface}) or a dispatchinterface (descending 
from \var{IDispatch}). Variants containing a reference to a dispatch
interface can be used to control the object behind it: the compiler will use
late binding to perform the call to the dispatch interface: there will be no
run-time checking of the function names and parameters or arguments given to 
the functions. The result type is also not checked. The compiler will simply
insert code to make the dispatch call and retrieve the result. 

This means basically, that you can do the following on Windows:
\begin{verbatim}
Var
  W : Variant;
  V : String;

begin
  W:=CreateOleObject('Word.Application');
  V:=W.Application.Version;
  Writeln('Installed version of MS Word is : ',V);
end;
\end{verbatim}
The line 
\begin{verbatim}
  V:=W.Application.Version;
\end{verbatim}
is executed by inserting the necessary code to query the dispatch interface
stored in the variant \var{W}, and execute the call if the needed dispatch
information is found.

%%%%%%%%%%%%%%%%%%%%%%%%%%%%%%%%%%%%%%%%%%%%%%%%%%%%%%%%%%%%%%%%%%%%%%%
% Variables
%%%%%%%%%%%%%%%%%%%%%%%%%%%%%%%%%%%%%%%%%%%%%%%%%%%%%%%%%%%%%%%%%%%%%%%
\chapter{Variables}
\index{Variables}\index{Variable}\index{Var}\keywordlink{var}
\label{ch:Variables}
\section{Definition}
Variables are explicitly named memory locations with a certain type. When
assigning values to variables, the \fpc compiler generates machine code 
to move the value to the memory location reserved for this variable. Where
this variable is stored depends on where it is declared:

\begin{itemize}
\item Global variables are variables declared in a unit or program, but not
inside a procedure or function. They are stored in fixed memory locations,
and are available during the whole execution time of the program.
\item Local variables are declared inside a procedure or function. Their
value is stored on the program stack, i.e. not at fixed locations.
\end{itemize}

The \fpc compiler handles the allocation of these memory locations
transparantly, although this location can be influenced in the declaration.

The \fpc compiler also handles reading values from or writing values to
the variables transparantly. But even this can be explicitly handled by the
programmer when using properties.

Variables must be explicitly declared when they are needed. No memory is
allocated unless a variable is declared. Using an variable identifier (for
instance, a loop variable) which is not declared first, is an error which
will be reported by the compiler. 

\section{Declaration}
The variables must be declared in a variable declaration section of a unit
or a procedure or function. It looks as follows:\index{Var}
\input{syntax/vardecl.syn}

This means that the following are valid variable declarations:
\begin{verbatim}
Var
  curterm1 : integer;

  curterm2 : integer; cvar;
  curterm3 : integer; cvar; external;

  curterm4 : integer; external name 'curterm3';
  curterm5 : integer; external 'libc' name 'curterm9';

  curterm6 : integer absolute curterm1;

  curterm7 : integer; cvar;  export;
  curterm8 : integer; cvar;  public;
  curterm9 : integer; export name 'me';
  curterm10 : integer; public name 'ma';

  curterm11 : integer = 1 ;
\end{verbatim}
\index{external}

The difference between these declarations is as follows:
\begin{enumerate}
\item The first form (\var{curterm1}) defines a regular variable. The
compiler manages everything by itself.
\item The second form (\var{curterm2}) declares also a regular variable, 
but specifies that the assembler name for this variable equals the name 
of the variable as written in the source.
\item The third form (\var{curterm3}) declares a variable which is located
externally: the compiler will assume memory is located elsewhere, and that
the assembler name of this location is specified by the name of the
variable, as written in the source. The name may not be specified.
\item The fourth form is completely equivalent to the third, it declares a
variable which is stored externally, and explicitly gives the assembler
name of the location. If \var{cvar} is not used, the name must be specified.
\item The fifth form is a variant of the fourth form, only the name of the
library in which the memory is reserved is specified as well.
\item The sixth form declares a variable (\var{curterm6}), and tells the compiler that it is
stored in the same location as another variable (\var{curterm1})
\item The seventh form declares a variable (\var{curterm7}), and tells the
compiler that the assembler label of this variable should be the name of the
variable (case sensitive) and must be made public. (i.e. it can be
referenced from other object files)
\item The eight form (\var{curterm8}) is equivalent to the seventh: 'public'
is an alias for 'export'.
\item The ninth and tenth form are equivalent: they specify the assembler 
name of the variable.
\item the elevents form declares a variable (\var{curterm11}) and
initializes it with a value (1 in the above case).
\end{enumerate}
Note that assembler names must be unique. It's not possible to declare or 
export 2 variables with the same assembler name.

\section{Scope}
\index{Scope}
Variables, just as any identifier, obey the general rules of scope. 
In addition, initialized variables are initialized when they enter scope:
\begin{itemize}
\item Global initialized variables are initialized once, when the program starts.
\item Local initialized variables are initialized each time the procedure is
entered.
\end{itemize}
Note that the behaviour for local initialized variables is different from
the one of a local typed constant. A local typed constant behaves like a
global initialized variable.

\section{Thread Variables}
\index{Thread Variables}\index{Threadvar}\keywordlink{threadvar}
For a program which uses threads, the variables can be really global, i.e. the same for all 
threads, or thread-local: this means that each thread gets a copy of the variable. 
Local variables (defined inside a procedure) are always thread-local. Global 
variables are normally the same for all threads. A global variable can be 
declared thread-local by replacing the \var{var} keyword at the start of the 
variable declaration block with \var{Threadvar}:
\begin{verbatim}
Threadvar
  IOResult : Integer;
\end{verbatim}
If no threads are used, the variable behaves as an ordinary variable. 
If threads are used then a copy is made 
for each thread (including the main thread). Note that the copy is 
made with the original value  of the variable, {\em not} with the 
value of the variable at the time the thread is started.

Threadvars should be used sparingly: There is an overhead for retrieving 
or setting the variable's value. If possible at all, consider using local 
variables; they are always faster than thread variables.

Threads are not enabled by default. For more information about programming 
threads, see the chapter on threads in the \progref.

\section{Properties}
\index{Properties}\keywordlink{property}
A global block can declare properties, just as they could be defined in a
class. The difference is that the global property does not need a class
instance: there is only 1 instance of this property. Other than that, a
global property behaves like a class property. The read/write specifiers for
the global property must also be regular procedures, not methods.
The concept of a global property is specific to \fpc, and does not exist in
Delphi.

The concept of a global property can be used to 'hide' the location of the
value, or to calculate the value on the fly, or to check the values which
are written to the property.

The declaration is as follows:
\input{syntax/propvar.syn}

The following is an example:
\begin{verbatim}
{$mode objfpc}
unit testprop;

Interface

Function GetMyInt : Integer;
Procedure SetMyInt(Value : Integer);

Property
  MyProp : Integer Read GetMyInt Write SetMyInt;
  
Implementation

Uses sysutils;

Var
  FMyInt : Integer;  
  
Function GetMyInt : Integer;

begin
  Result:=FMyInt;
end;

Procedure SetMyInt(Value : Integer);

begin
  If ((Value mod 2)=1) then
    Raise Exception.Create('MyProp can only contain even value');
  FMyInt:=Value;  
end;

end.
\end{verbatim}
The read/write specifiers can be hidden by declaring them in another unit
which must be in the \var{uses} clause of the unit. This can be used to hide
the read/write access specifiers for programmers, just as if they were in a
\var{private} section of a class (discussed below). For the previous
example, this could look as follows:
\begin{verbatim}
{$mode objfpc}
unit testrw;

Interface

Function GetMyInt : Integer;
Procedure SetMyInt(Value : Integer);

Implementation

Uses sysutils;

Var
  FMyInt : Integer;  
  
Function GetMyInt : Integer;

begin
  Result:=FMyInt;
end;

Procedure SetMyInt(Value : Integer);

begin
  If ((Value mod 2)=1) then
    Raise Exception.Create('Only even values are allowed');
  FMyInt:=Value;  
end;

end.
\end{verbatim}
The unit \file{testprop} would then look like:
\begin{verbatim}
{$mode objfpc}
unit testprop;

Interface

uses testrw;

Property
  MyProp : Integer Read GetMyInt Write SetMyInt;

Implementation

end.  
\end{verbatim}

%%%%%%%%%%%%%%%%%%%%%%%%%%%%%%%%%%%%%%%%%%%%%%%%%%%%%%%%%%%%%%%%%%%%%%%
% Objects
%%%%%%%%%%%%%%%%%%%%%%%%%%%%%%%%%%%%%%%%%%%%%%%%%%%%%%%%%%%%%%%%%%%%%%%
\chapter{Objects}
\label{ch:Objects}
\index{Objects}\index{Types!Object}

%%%%%%%%%%%%%%%%%%%%%%%%%%%%%%%%%%%%%%%%%%%%%%%%%%%%%%%%%%%%%%%%%%%%%%%
% Declaration
\section{Declaration}
\fpc supports object oriented programming. In fact, most  of the compiler is
written using objects. Here we present some technical questions regarding
object oriented programming in \fpc.
Objects should be treated as a special kind of record. The record contains
all the fields that are declared in the objects definition, and pointers
to the methods that are associated to the objects' type.

An object is declared just as a record would be declared; except that
now, procedures and functions can be declared as if they were part of the record.
Objects can ''inherit'' fields and methods from ''parent'' objects. This means
that these fields and methods can be used as if they were included in the
objects declared as a ''child'' object.

Furthermore, a concept of visibility \index{Visibility} is introduced: 
fields, procedures and functions can be declared as \var{public} or 
\var{private}. By default, fields and methods are \var{public}, and 
are exported outside the current unit. 
\index{Visibility!Public}\index{Visibility!Private}
Fields or methods that are declared \var{private} are only accessible 
in the current unit: their scope is limited to the implementation of the
current unit.\index{Scope}

The prototype declaration of an object is as follows:
\index{object}\keywordlink{object}
\input{syntax/typeobj.syn}
As can be seen, as many \var{private} and \var{public} blocks as needed can be
declared.\index{private}\index{public}


\begin{remark}\index{Packed}
\fpc also supports the packed object. This is the same as an object, only
the elements (fields) of the object are byte-aligned, just as in the packed
record.
The declaration of a packed object is similar to the declaration
of a packed record :
\begin{verbatim}
Type
  TObj = packed object;
   Constructor init;
   ...
   end;
  Pobj = ^TObj;
Var PP : Pobj;
\end{verbatim}
Similarly, the \var{\{\$PackRecords \}} directive acts on objects as well.
\end{remark}
%%%%%%%%%%%%%%%%%%%%%%%%%%%%%%%%%%%%%%%%%%%%%%%%%%%%%%%%%%%%%%%%%%%%%%%
% Fields
\section{Fields}
\index{Fields}
Object Fields are like record fields. They are accessed in the same way as
a record field  would be accessed : by using a qualified identifier. Given the
following declaration:
\begin{verbatim}
Type TAnObject = Object
       AField : Longint;
       Procedure AMethod;
       end;
Var AnObject : TAnObject;
\end{verbatim}
then the following would be a valid assignment:
\begin{verbatim}
  AnObject.AField := 0;
\end{verbatim}
Inside methods, fields can be accessed using the short identifier:
\begin{verbatim}
Procedure TAnObject.AMethod;
begin
  ...
  AField := 0;
  ...
end;
\end{verbatim}
Or, one can use the \var{self} identifier. The \var{self} identifier refers
to the current instance of the object:
\begin{verbatim}
Procedure TAnObject.AMethod;
begin
  ...
  Self.AField := 0;
  ...
end;
\end{verbatim}
One cannot access fields that are in a private section of an object from
outside the objects' methods. If this is attempted anyway, the compiler will complain about
an unknown identifier.
It is also possible to use the \var{with} statement with an object instance:
\begin{verbatim}
With AnObject do
  begin
  Afield := 12;
  AMethod;
  end;
\end{verbatim}
In this example, between the \var{begin} and \var{end}, it is as if
\var{AnObject} was prepended to the \var{Afield} and \var{Amethod}
identifiers. More about this in \sees{With}

%%%%%%%%%%%%%%%%%%%%%%%%%%%%%%%%%%%%%%%%%%%%%%%%%%%%%%%%%%%%%%%%%%%%%%%
% Constructors and destructors
\section{Constructors and destructors }
\index{Constructor}\index{Destructor}
\index{Virtual}
\label{se:constructdestruct}
As can be seen in the syntax diagram for an object declaration, \fpc supports
constructors and destructors. The programmer is responsible for calling the
constructor and the destructor explicitly when using objects.
The declaration of a constructor or destructor is as follows:
\input{syntax/construct.syn}
A constructor/destructor pair is {\em required} if the object uses virtual methods.\index{Virtual}
In the declaration of the object type, a simple identifier should be used
for the name of the constuctor or destructor. When the constructor or destructor
is implemented, A qualified method identifier should be used,
i.e. an identifier of the form \var{objectidentifier.methodidentifier}.
\fpc supports also the extended syntax of the \var{New} and \var{Dispose}
procedures. In case a dynamic variable of an object type must be allocated
the constructor's name can be specified in the call to \var{New}.
The \var{New} is implemented as a function which returns a pointer to the
instantiated object. Consider the following declarations:
\begin{verbatim}
Type
  TObj = object;
   Constructor init;
   ...
   end;
  Pobj = ^TObj;
Var PP : Pobj;
\end{verbatim}
Then the following 3 calls are equivalent:
\begin{verbatim}
 pp := new (Pobj,Init);
\end{verbatim}
and
\begin{verbatim}
  new(pp,init);
\end{verbatim}
and also
\begin{verbatim}
  new (pp);
  pp^.init;
\end{verbatim}
In the last case, the compiler will issue a warning that the
extended syntax of \var{new} and \var{dispose} must be used to generate instances of an
object. It is possible to ignore this warning, but it's better programming practice to
use the extended syntax to create instances of an object.
Similarly, the \var{Dispose} procedure accepts the name of a destructor. The
destructor will then be called, before removing the object from the heap.

In view of the compiler warning remark, the following chapter presents the
Delphi approach to object-oriented programming, and may be considered a
more natural way of object-oriented programming.

%%%%%%%%%%%%%%%%%%%%%%%%%%%%%%%%%%%%%%%%%%%%%%%%%%%%%%%%%%%%%%%%%%%%%%%
% Methods
\section{Methods}
\index{Methods}
Object methods are just like ordinary procedures or functions, only they
have an implicit extra parameter : \var{self}. Self points to the object
with which the method was invoked.\index{Self}
When implementing methods, the fully qualified identifier must be given
in the function header. When declaring methods, a normal identifier must be
given.

%%%%%%%%%%%%%%%%%%%%%%%%%%%%%%%%%%%%%%%%%%%%%%%%%%%%%%%%%%%%%%%%%%%%%%%
% Method declaration
\subsection{Declaration}
The declaration of a method is much like a normal function or procedure
declaration, with some additional specifiers, as can be seen from the
following diagram, which is part of the object declaration:
\input{syntax/omethods.syn}
from the point of view of declarations, \var{Method definitions} are 
normal function or procedure declarations.
Contrary to TP and Delphi, fields can be declared after methods in the same 
block, i.e. the following will generate an error when compiling with Delphi
or Turbo Pascal, but not with FPC:
\begin{verbatim}
Type 
  MyObj = Object
    Procedure Doit;
    Field : Longint;
  end;
\end{verbatim}


%%%%%%%%%%%%%%%%%%%%%%%%%%%%%%%%%%%%%%%%%%%%%%%%%%%%%%%%%%%%%%%%%%%%%%%
% Method invocation
\subsection{Method invocation}
Methods are called just as normal procedures are called, only they have an
object instance identifier prepended to them (see also \seec{Statements}).
To determine which method is called, it is necessary to know the type of
the method. We treat the different types in what follows.

\subsubsection{Static methods}
\index{Methods!Static}
Static methods are methods that have been declared without a \var{abstract}
or \var{virtual} keyword. When calling a static method, the declared (i.e.
compile time) method of the object is used.
For example, consider the following declarations:
\begin{verbatim}
Type
  TParent = Object
    ...
    procedure Doit;
    ...
    end;
  PParent = ^TParent;
  TChild = Object(TParent)
    ...
    procedure Doit;
    ...
    end;
  PChild = ^TChild;
\end{verbatim}
As it is visible, both the parent and child objects have a method called
\var{Doit}. Consider now the following declarations and calls:
\begin{verbatim}
Var 
  ParentA,ParentB : PParent;
  Child           : PChild;

begin
   ParentA := New(PParent,Init);
   ParentB := New(PChild,Init);
   Child := New(PChild,Init);
   ParentA^.Doit;
   ParentB^.Doit;
   Child^.Doit;
\end{verbatim}
Of the three invocations of \var{Doit}, only the last one will call
\var{TChild.Doit}, the other two calls will call \var{TParent.Doit}.
This is because for static methods, the compiler determines at compile
time which method should be called. Since \var{ParentB} is of type
\var{TParent}, the compiler decides that it must be called with
\var{TParent.Doit}, even though it will be created as a \var{TChild}.
There may be times when the method that is actually called should
depend on the actual type of the object at run-time. If so, the method
cannot be a static method, but must be a virtual method.

\subsubsection{Virtual methods}
\index{Virtual}\index{Methods!Virtual}\keywordlink{virtual}
To remedy the situation in the previous section, \var{virtual} methods are
created. This is simply done by appending the method declaration with the
\var{virtual} modifier.
Going back to the previous example, consider the following alternative
declaration:
\begin{verbatim}
Type
  TParent = Object
    ...
    procedure Doit;virtual;
    ...
    end;
  PParent = ^TParent;
  TChild = Object(TParent)
    ...
    procedure Doit;virtual;
    ...
    end;
  PChild = ^TChild;
\end{verbatim}
As it is visible, both the parent and child objects have a method called
\var{Doit}. Consider now the following declarations and calls :
\begin{verbatim}
Var 
  ParentA,ParentB : PParent;
  Child           : PChild;

begin
   ParentA := New(PParent,Init);
   ParentB := New(PChild,Init);
   Child := New(PChild,Init);
   ParentA^.Doit;
   ParentB^.Doit;
   Child^.Doit;
\end{verbatim}
Now, different methods will be called, depending on the actual run-time type
of the object. For \var{ParentA}, nothing changes, since it is created as
a \var{TParent} instance. For \var{Child}, the situation also doesn't
change: it is again created as an instance of \var{TChild}.
For \var{ParentB} however, the situation does change: Even though it was
declared as a \var{TParent}, it is created as an instance of \var{TChild}.
Now, when the program runs, before calling \var{Doit}, the program
checks what the actual type of \var{ParentB} is, and only then decides which
method must be called. Seeing that \var{ParentB} is of type \var{TChild},
\var{TChild.Doit} will be called.
The code for this run-time checking of the actual type of an object is
inserted by the compiler at compile time.
The \var{TChild.Doit} is said to {\em override} the
\var{TParent.Doit}.\index{override}
It is possible to acces the \var{TParent.Doit} from within the
var{TChild.Doit}, with the \var{inherited} keyword:
\begin{verbatim}
Procedure TChild.Doit;
begin
  inherited Doit;
  ...
end;
\end{verbatim}
In the above example, when \var{TChild.Doit} is called, the first thing it
does is call \var{TParent.Doit}.  The inherited keyword cannot be used in
static methods, only on virtual methods.

To be able to do this, the compiler keeps - per object type - a table with
virtual methods: the VMT (Virtual Method Table). This is simply a table 
with pointers to each of the virtual methods: each virtual method has it's
fixed location in this table (an index). The compiler uses this table to 
look up the actual method that must be used. When a descendent object
overrides a method, the entry of the parent method is overwritten in the
VMT. More information about the VMT can be found in \progref.

%
\subsubsection{Abstract methods}
\index{Abstract}\index{Methods!Abstract}\index{Methods!Virtual}\keywordlink{abstract}
An abstract method is a special kind of virtual method. A method can not be
abstract if it is not virtual (this can be seen from the syntax diagram).
An instance of an object that has an abstract method cannot be created directly.
The reason is obvious: there is no method where the compiler could jump to !
A method that is declared \var{abstract} does not have an implementation for
this method. It is up to inherited objects to override and implement this
method. Continuing our example, take a look at this:
\begin{verbatim}
Type
  TParent = Object
    ...
    procedure Doit;virtual;abstract;
    ...
    end;
  PParent=^TParent;
  TChild = Object(TParent)
    ...
    procedure Doit;virtual;
    ...
    end;
  PChild = ^TChild;
\end{verbatim}
As it is visible, both the parent and child objects have a method called
\var{Doit}. Consider now the following declarations and calls :
\begin{verbatim}
Var 
  ParentA,ParentB : PParent;
  Child           : PChild;

begin
   ParentA := New(PParent,Init);
   ParentB := New(PChild,Init);
   Child := New(PChild,Init);
   ParentA^.Doit;
   ParentB^.Doit;
   Child^.Doit;
\end{verbatim}
First of all, Line 3 will generate a compiler error, stating that one cannot
generate instances of objects with abstract methods: The compiler has
detected that \var{PParent} points to an object which has an abstract
method. Commenting line 3 would allow compilation of the program.
\begin{remark}
If an abstract method is overridden, The parent method cannot be called
with \var{inherited}, since there is no parent method; The compiler
will detect this, and complain about it, like this:
\begin{verbatim}
testo.pp(32,3) Error: Abstract methods can't be called directly
\end{verbatim}
If, through some mechanism, an abstract method is called at run-time,
then a run-time error will occur. (run-time error 211, to be precise)
\end{remark}

%%%%%%%%%%%%%%%%%%%%%%%%%%%%%%%%%%%%%%%%%%%%%%%%%%%%%%%%%%%%%%%%%%%%%%%
% Visibility
\section{Visibility}
\index{Visibility}\index{Scope}\index{Private}\index{Protected}\index{Public}
For objects, 3 visibility specifiers exist : \var{private}, \var{protected} and
\var{public}. If a visibility specifier is not specified, \var{public}
is assumed.
Both methods and fields can be hidden from a programmer by putting them
in a \var{private} section. The exact visibility rule is as follows:
\begin{description}
\item [Private\ ] All fields and methods that are in a \var{private} block,
can  only be accessed in the module (i.e. unit or program) that contains
the object definition.\keywordlink{private}
They can be accessed from inside the object's methods or from outside them
e.g. from other objects' methods, or global functions.
\item [Protected\ ] Is the same as \var{Private}, except that the members of
a \var{Protected} section are also accessible to descendent types, even if
they are implemented in other modules.\keywordlink{private}
\item [Public\ ] sections are always accessible, from everywhere.
Fields and metods in a \var{public} section behave as though they were part
of an ordinary \var{record} type.\keywordlink{public}
\end{description}


%%%%%%%%%%%%%%%%%%%%%%%%%%%%%%%%%%%%%%%%%%%%%%%%%%%%%%%%%%%%%%%%%%%%%%%
% Classes
%%%%%%%%%%%%%%%%%%%%%%%%%%%%%%%%%%%%%%%%%%%%%%%%%%%%%%%%%%%%%%%%%%%%%%%
\chapter{Classes}
\label{ch:Classes}\index{Classes}\index{Types!Class}\keywordlink{class}
In the Delphi approach to Object Oriented Programming, everything revolves
around  the concept of 'Classes'.  A class can be seen as a pointer to an
object, or a pointer to a record, with methods associated with it.

The difference between objects and classes is mainly that an object
is allocated on the stack, as an ordinary record would be, and that
classes are always allocated on the heap. In the following example:
\begin{verbatim}
Var
  A : TSomeObject; // an Object
  B : TSomeClass;  // a Class
\end{verbatim}
The main difference is that the variable \var{A} will take up as much 
space on the stack as the size of the object (\var{TSomeObject}). The
variable \var{B}, on the other hand, will always take just the size of
a pointer on the stack. The actual class data is on the heap.

From this, a second difference follows: a class must always be initialized
through it's constructor, whereas for an object, this is not necessary.
Calling the constructor allocates the necessary memory on the heap for the
class instance data. 

\begin{remark}
In earlier versions of \fpc it was necessary, in order to use classes,
to put the \file{objpas} unit in the uses clause of a unit or program.
{\em This is no longer needed} as of version 0.99.12. As of this version,
the unit will be loaded automatically when the \var{-MObjfpc} or
\var{-MDelphi}  options are specified, or their corresponding directives are
used:
\begin{verbatim}
{$mode objfpc}
{$mode delphi}
\end{verbatim}
\end{remark}

%%%%%%%%%%%%%%%%%%%%%%%%%%%%%%%%%%%%%%%%%%%%%%%%%%%%%%%%%%%%%%%%%%%%%%%
% Class definitions
\section{Class definitions}
The prototype declaration of a class is as follows:\index{Class}
\input{syntax/typeclas.syn}
As many \var{private}, \var{protected}, \var{published}
and \var{public} blocks as needed can be repeated.
Methods are normal function or procedure declarations.
As can be seen, the declaration of a class is almost identical to the
declaration of an object. The real difference between objects and classes
is in the way they are created (see further in this chapter).
The visibility of the different sections is as follows:\index{Scope}
\begin{description}
\item [Private\ ] \index{Private}\index{Visibility!Protected} All fields and methods that are in a \var{private} block, can
only be accessed in the module (i.e. unit) that contains the class definition.
They can be accessed from inside the classes' methods or from outside them
(e.g. from other classes' methods)\keywordlink{private}
\item [Protected\ ] \index{Protected}\index{Visibility!Protected}%
Is the same as \var{Private}, except that the members of
a \var{Protected} section are also accessible to descendent types, even if
they are implemented in other modules.\keywordlink{protected}
\item [Public\ ] \index{Public}\index{Visibility!Public} sections are always
accessible.\keywordlink{public}
\item [Published\ ] \index{Published}\index{Visibility!Published} Is the same as a \var{Public} section, but the compiler
generates also type information that is needed for automatic streaming of
these classes. Fields defined in a \var{published} section must be of class type.
Array properties cannot be in a \var{published}
section.\keywordlink{published}
\end{description}
In the syntax diagram, it can be seen that a class lists implemented
interfaces. This will be discussed in the next chapter.

It is also possible to define class reference types:
\input{syntax/classref.syn}
Class reference types are used to create instances of a certain class, which
is not yet known at compile time, but which is specified at run time. 
Essentially, a variable of a class reference type contains a pointer to the
definition of the speficied class. This can be used to construct an instance 
of the class corresponding to the definition, or to check inheritance. 
The following example shows how it works:
\begin{verbatim}
Type
  TComponentClass = Class of TComponent;

Function CreateComponent(AClass : TComponentClass; AOwner : TComponent) : TComponent;

begin
  // ...
  Result:=AClass.Create(AOwner);
  // ...
end;
\end{verbatim}
This function can be passed a class reference of any class that descends
from \var{TComponent}. The following is a valid call:
\begin{verbatim}
Var
  C : TComponent;

begin
  C:=CreateComponent(TEdit,Form1);
end;
\end{verbatim}
On return of the \var{CreateComponent} function, \var{C} will contain an 
instance of the class \var{TEdit}. Note that the following call will fail to
compile:
\begin{verbatim}
Var
  C : TComponent;

begin
  C:=CreateComponent(TStream,Form1);
end;
\end{verbatim}
because \var{TStream} does not descend from \var{TComponent}, and
\var{AClass} refers to a \var{TComponent} class. The compiler can
(and will) check this at compile time, and will produce an error.

References to classes can also be used to check inheritance:
\begin{verbatim}
  TMinClass = Class of TMyClass;
  TMaxClass = Class of TMyClassChild;

Function CheckObjectBetween(Instance : TObject) : boolean;

begin
  If not (Instance is TMinClass) 
     or ((Instance is TMaxClass) 
          and (Instance.ClassType<>TMaxClass)) then
    Raise Exception.Create(SomeError)
end;
\end{verbatim}
The above example will 
More about instantiating a class can be found in the next section.

%%%%%%%%%%%%%%%%%%%%%%%%%%%%%%%%%%%%%%%%%%%%%%%%%%%%%%%%%%%%%%%%%%%%%%%
% Class instantiation
\section{Class instantiation}
\index{Constructor}\keywordlink{constructor}
Classes must be created using one of their constructors (there can be
multiple constructors). Remember that a class is a pointer to an object on
the heap. When a variable of some class is declared, the compiler just 
allocates a pointer, not the entire object. The constructor of
a class returns a pointer to an initialized instance of the object on the
heap. So, to initialize an instance of some class, one would do the following :
\begin{verbatim}
  ClassVar := ClassType.ConstructorName;
\end{verbatim}
The extended syntax of \var{new} and \var{dispose} can {\em not} be used to
instantiate and destroy class instances.
That construct is reserved for use with objects only.
Calling the constructor will provoke a call to \var{getmem}, to allocate
enough space to hold the class instance data.
After that, the constuctor's code is executed.
The constructor has a pointer to it's data, in \var{self}.

\begin{remark}
\begin{itemize}\index{Packed}
\item The \var{\{\$PackRecords \}} directive also affects classes.
i.e. the alignment in memory of the different fields depends on the
value of  the \var{\{\$PackRecords \}} directive.
\item Just as for objects and records, a packed class can be declared.
This has the same effect as on an object, or record, namely that the
elements are aligned on 1-byte boundaries. i.e. as close as possible.
\item \var{SizeOf(class)} will return 4, since a class is but a pointer to
an object. To get the size of the class instance data, use the
\var{TObject.InstanceSize} method.
\end{itemize}
\end{remark}

%%%%%%%%%%%%%%%%%%%%%%%%%%%%%%%%%%%%%%%%%%%%%%%%%%%%%%%%%%%%%%%%%%%%%%%
% Methods
\section{Methods}
\index{Methods}
\subsection{Declaration}
Declaration of methods in classes follows the same rules as method
declarations in objects:
\input{syntax/cmethods.syn}

\subsection{invocation}
Method invocation for classes is no different than for objects. The
following is a valid method invocation:
\begin{verbatim}
Var  AnObject : TAnObject;
begin
  AnObject := TAnObject.Create;
  ANobject.AMethod;
\end{verbatim}


\subsection{Virtual methods}
\index{Methods!Virtual}\index{Virtual}
Classes have virtual methods, just as objects do. There is however a
difference between the two. For objects, it is sufficient to redeclare the
same method in a descendent object with the keyword \var{virtual} to
override it. For classes, the situation is different: virtual methods 
{\em must} be overridden with the \var{override} keyword. Failing to do so,
will start a {\em new} batch of virtual methods, hiding the previous
one.  The \var{Inherited} keyword will not jump to the inherited method, if
virtual was used.

The following code is {\em wrong}:
\begin{verbatim}
Type 
  ObjParent = Class
    Procedure MyProc; virtual;
  end;
  ObjChild  = Class(ObjPArent)
    Procedure MyProc; virtual;
  end;
\end{verbatim}
The compiler will produce a warning:
\begin{verbatim}
Warning: An inherited method is hidden by OBJCHILD.MYPROC
\end{verbatim}
The compiler will compile it, but using \var{Inherited} can\index{Inherited}
produce strange effects.

The correct declaration is as follows:
\begin{verbatim}
Type 
  ObjParent = Class
    Procedure MyProc; virtual;
  end;
  ObjChild  = Class(ObjPArent)
    Procedure MyProc; override;
  end;
\end{verbatim}
This will compile and run without warnings or errors.\index{Override}

If the virtual method should really be replaced with a method with the 
same name, then the \var{reintroduce} keyword can be used:\index{reintroduce}
\begin{verbatim}
Type 
  ObjParent = Class
    Procedure MyProc; virtual;
  end;
  ObjChild  = Class(ObjPArent)
    Procedure MyProc; reintroduce;
  end;
\end{verbatim}
This new method is no longer virtual.

To be able to do this, the compiler keeps - per class type - a table with
virtual methods: the VMT (Virtual Method Table). This is simply a table 
with pointers to each of the virtual methods: each virtual method has it's
fixed location in this table (an index). The compiler uses this table to 
look up the actual method that must be used. When a descendent object
overrides a method, the entry of the parent method is overwritten in the
VMT. More information about the VMT can be found in \progref.


\subsection{Class methods}
\index{Class}\index{Methods!Class}
Class methods are identified by the keyword \var{Class} in front of the
procedure or function declaration, as in the following example:
\begin{verbatim}
  Class Function ClassName : String;
\end{verbatim}
Class methods are methods that do not have an instance (i.e. Self does not
point to a class instance) but which follow the scoping and inheritance 
rules of a class. They can be used to return information about the current
class, for instance for registration or use in a class factory. Since no 
instance is available, no information available in instances can be used.

Class methods can be called from inside a regular method, but can also be called 
using a class identifier:
\begin{verbatim}
Var
  AClass : TClass;

begin
  ..
  if CompareText(AClass.ClassName,'TCOMPONENT')=0 then
  ...

\end{verbatim}
But calling them from an instance is also possible:
\begin{verbatim}
Var
  MyClass : TObject;

begin
  ..
  if MyClass.ClassNameis('TCOMPONENT') then
  ...

\end{verbatim}
Inside a class method, the <var>self</var> identifier points to the VMT\index{Self}
table of the class. No fields, properties or regular methods are available
inside a class method. Accessing a regular property or method will result in
a compiler error. The reverse is possible: a class method can be called from
a regular method.

Note that class methods can be virtual, and can be overridden.

Class methods cannot be used as read or write specifiers for a
property.\index{Property}



\subsection{Message methods}
\index{Methods!Message}
New in classes are \var{message} methods. Pointers to message methods are
stored in a special table, together with the integer or string cnstant that
they were declared with. They are primarily intended to ease programming of
callback functions in several \var{GUI} toolkits, such as \var{Win32} or
\var{GTK}. In difference with Delphi, \fpc also accepts strings as message
identifiers. Message methods are always
virtual.\index{Virtual}\index{message}

Message methods that are declared with an integer constant can take only one
var argument (typed or not):\index{Message}
\begin{verbatim}
 Procedure TMyObject.MyHandler(Var Msg); Message 1;
\end{verbatim}
The method implementation of a message function is no different from an
ordinary method. It is also possible to call a message method directly,
but this should not be done. Instead, the \var{TObject.Dispatch} method
should be used.\index{Dispatch}

The \var{TOBject.Dispatch} method can be used to call a
\var{message}\index{Message}
handler. It is declared in the \file{system} unit and will accept a var
parameter  which must have at the first position a cardinal with the
message ID that should be called. For example:
\begin{verbatim}
Type
  TMsg = Record
    MSGID : Cardinal
    Data : Pointer;
Var
  Msg : TMSg;

MyObject.Dispatch (Msg);
\end{verbatim}
In this example, the \var{Dispatch} method will look at the object and all
it's ancestors (starting at the object, and searching up the class tree),
to see if a message method with message \var{MSGID} has been
declared. If such a method is found, it is called, and passed the
\var{Msg} parameter.

If no such method is found, \var{DefaultHandler} is called.
\var{DefaultHandler} is a virtual method of \var{TObject} that doesn't do
anything, but which can be overridden to provide any processing that might be
needed. \var{DefaultHandler} is declared as follows:
\begin{verbatim}
   procedure defaulthandler(var message);virtual;
\end{verbatim}

In addition to the message method with a \var{Integer} identifier,
\fpc also supports a message method with a string identifier:
\begin{verbatim}
 Procedure TMyObject.MyStrHandler(Var Msg); Message 'OnClick';
\end{verbatim}

The working of the string message handler is the same as the ordinary
integer message handler:

The \var{TOBject.DispatchStr} \index{DispatchStr} method can be used to call a \var{message}
handler. It is declared in the system unit and will accept one parameter
which must have at the first position a string with the message ID that
should be called. For example:
\begin{verbatim}
Type
  TMsg = Record
    MsgStr : String[10]; // Arbitrary length up to 255 characters.
    Data : Pointer;
Var
  Msg : TMSg;

MyObject.DispatchStr (Msg);
\end{verbatim}
In this example, the \var{DispatchStr} method will look at the object and
all it's ancestors (starting at the object, and searching up the class tree),
to see if a message method with message \var{MsgStr} has been
declared. If such a method is found, it is called, and passed the
\var{Msg} parameter.

If no such method is found, \var{DefaultHandlerStr} is called.
\var{DefaultHandlerStr} is a virtual method of \var{TObject} that doesn't do
anything, but which can be overridden to provide any processing that might be
needed. \var{DefaultHandlerStr} is declared as follows:
\begin{verbatim}
   procedure DefaultHandlerStr(var message);virtual;
\end{verbatim}

In addition to this mechanism, a string message method accepts a \var{self}
parameter:
\begin{verbatim}
  Procedure StrMsgHandler(Data : Pointer; Self : TMyObject);Message 'OnClick';
\end{verbatim}
When encountering such a method, the compiler will generate code that loads
the \var{Self} parameter into the object instance pointer. The result of
this is that it is possible to pass \var{Self} as a parameter to such a
method.

\begin{remark}
The type of the \var{Self} \index{Self}parameter must be of the same class
as the class the method is defined in.
\end{remark}

\subsection{Using inherited}
In an overridden virtual method, it is often necessary to call the parent
class' implementation of the virtual method. This can be  done with the
\var{inherited} keyword. Likewise, the \var{inherited} keyword can be used
to call any method of the parent class.

The first case is the simplest:
\begin{verbatim}
Type
  TMyClass = Class(TComponent)
    Constructor Create(AOwner : TComponent); override;
  end;

Constructor TMyClass.Create(AOwner : TComponent); 

begin
  Inherited;
  // Do more things
end;
\end{verbatim}
In the above example, the \var{Inherited} statement will call \var{Create}
of \var{TComponent}, passing it \var{AOwner} as a parameter: the same
parameters that were passed to the current method will be passed to the
parent's method.

The second case is slightly more complicated:
\begin{verbatim}
Type
  TMyClass = Class(TComponent)
    Constructor Create(AOwner : TComponent); override;
    Constructor CreateNew(AOwner : TComponent; DoExtra : Boolean);
  end;

Constructor TMyClass.Create(AOwner : TComponent); 

begin
  Inherited;
end;

Constructor TMyClass.CreateNew(AOwner : TComponent; DoExtra); 

begin
  Inherited Create(AOwner);
  // Do stuff
end;
\end{verbatim}
The \var{CreateNew} method will first call \var{TComponent.Create} and
will pass it \var{AOwner} as a parameter. It will not call
\var{TMyClass.Create}.

Although the examples were given using constructors, the use of
\var{inherited} is not restricted to constructors, it can be used
for any method or function or destructor as well.

%%%%%%%%%%%%%%%%%%%%%%%%%%%%%%%%%%%%%%%%%%%%%%%%%%%%%%%%%%%%%%%%%%%%%%%
% Properties
\section{Properties}
\index{Properties}\keywordlink{property}
Classes can contain properties as part of their fields list. A property
acts like a normal field, i.e. its value can be retrieved or set, but it
allows to redirect the access of the field through functions and
procedures. They provide a means to associate an action with an assignment
of or a reading from a class 'field'. This allows for e.g. checking that a
value is valid when assigning, or, when reading, it allows to construct the
value on the fly. Moreover, properties can be read-only or write only.
The prototype declaration of a property is as follows:\index{Property}
\input{syntax/property.syn}
A \var{read specifier} \index{Read} is either the name of a field that contains the
property, or the name of a method function that has the same return type as
the property type. In the case of a simple type, this
function must not accept an argument. A \var{read specifier} is optional,
making the property write-only. Note that class methods cannot be used as
read specifiers.
A \var{write specifier} \index{Write} is optional: If there is no \var{write specifier}, the
property is read-only. A write specifier is either the name of a field, or
the name of a method procedure that accepts as a sole argument a variable of
the same type as the property.
The section \index{Private}\index{Published} (\var{private}, \var{published}) 
in which the specified function or procedure resides is irrelevant. Usually, 
however, this will be a protected or private method.
Example:
Given the following declaration:
\begin{verbatim}
Type
  MyClass = Class
    Private
    Field1 : Longint;
    Field2 : Longint;
    Field3 : Longint;
    Procedure  Sety (value : Longint);
    Function Gety : Longint;
    Function Getz : Longint;
    Public
    Property X : Longint Read Field1 write Field2;
    Property Y : Longint Read GetY Write Sety;
    Property Z : Longint Read GetZ;
    end;
Var MyClass : TMyClass;
\end{verbatim}
The following are valid statements:
\begin{verbatim}
WriteLn ('X : ',MyClass.X);
WriteLn ('Y : ',MyClass.Y);
WriteLn ('Z : ',MyClass.Z);
MyClass.X := 0;
MyClass.Y := 0;
\end{verbatim}
But the following would generate an error:
\begin{verbatim}
MyClass.Z := 0;
\end{verbatim}
because Z is a read-only property.
What happens in the above statements is that when a value needs to be read,
the compiler inserts a call to the various \var{getNNN} methods of the
object, and the result of this call is used. When an assignment is made,
the compiler passes the value that must be assigned as a paramater to
the various \var{setNNN} methods.
Because of this mechanism, properties cannot be passed as var arguments to a
function or procedure, since there is no known address of the property (at
least, not always).\index{Parameters!Var}

\index{Properties!Indexed}\keywordlink{index}
If the property definition contains an index, \index{index} then the read and write
specifiers must be a function and a procedure. Moreover, these functions
require an additional parameter : An integer parameter. This allows to read
or write several properties with the same function. For this, the properties
must have the same type.
The following is an example of a property with an index:
\begin{verbatim}
{$mode objfpc}
Type TPoint = Class(TObject)
       Private
       FX,FY : Longint;
       Function GetCoord (Index : Integer): Longint;
       Procedure SetCoord (Index : Integer; Value : longint);
       Public
       Property X : Longint index 1 read GetCoord Write SetCoord;
       Property Y : Longint index 2 read GetCoord Write SetCoord;
       Property Coords[Index : Integer]:Longint Read GetCoord;
       end;

Procedure TPoint.SetCoord (Index : Integer; Value : Longint);
begin
  Case Index of
   1 : FX := Value;
   2 : FY := Value;
  end;
end;

Function TPoint.GetCoord (INdex : Integer) : Longint;
begin
  Case Index of
   1 : Result := FX;
   2 : Result := FY;
  end;
end;

Var P : TPoint;

begin
  P := TPoint.create;
  P.X := 2;
  P.Y := 3;
  With P do
    WriteLn ('X=',X,' Y=',Y);
end.
\end{verbatim}
When the compiler encounters an assignment to \var{X}, then \var{SetCoord}
is called with as first parameter the index (1 in the above case) and with
as a second parameter the value to be set.
Conversely, when reading the value of \var{X}, the compiler calls
\var{GetCoord} and passes it index 1.
Indexes can only be integer values.

Array properties also exist.\index{Properties!Array} These are properties that accept an
index, just as an array does. Only now the index doesn't have to be an
ordinal type, but can be any type.

A \var{read specifier} for an array property is the name method function
that has the same return type as the property type.
The function must accept as a sole arguent a variable of the same type as
the index type. For an array property, one cannot specify fields as \var{read
specifiers}.

A \var{write specifier} for an array property is the name of a method
procedure that accepts two arguments: The first argument has the same
type as the index, and the second argument is a parameter of the same
type as the property type.
As an example, see the following declaration:
\begin{verbatim}
Type TIntList = Class
      Private
      Function GetInt (I : Longint) : longint;
      Function GetAsString (A : String) : String;
      Procedure SetInt (I : Longint; Value : Longint;);
      Procedure SetAsString (A : String; Value : String);
      Public
      Property Items [i : Longint] : Longint Read GetInt
                                             Write SetInt;
      Property StrItems [S : String] : String Read GetAsString
                                              Write SetAsstring;
      end;
Var AIntList : TIntList;
\end{verbatim}
Then the following statements would be valid:
\begin{verbatim}
AIntList.Items[26] := 1;
AIntList.StrItems['twenty-five'] := 'zero';
WriteLn ('Item 26 : ',AIntList.Items[26]);
WriteLn ('Item 25 : ',AIntList.StrItems['twenty-five']);
\end{verbatim}
While the following statements would generate errors:
\begin{verbatim}
AIntList.Items['twenty-five'] := 1;
AIntList.StrItems[26] := 'zero';
\end{verbatim}
Because the index types are wrong.
Array properties can be declared as \var{default} properties. This means that
it is not necessary to specify the property name when assigning or reading
it. If, in the previous example, the definition of the items property would
have been
\begin{verbatim}
 Property Items[i : Longint]: Longint Read GetInt
                                      Write SetInt; Default;
\end{verbatim}
Then the assignment
\begin{verbatim}
AIntList.Items[26] := 1;
\end{verbatim}
Would be equivalent to the following abbreviation.
\begin{verbatim}
AIntList[26] := 1;
\end{verbatim}
Only one default property per class is allowed, and descendent classes
cannot redeclare the default property.

The {\em stored specifier} should be either a boolean constant, a boolean
field of the class, or a parameterless function which returns a boolean
result. This specifier has no result on the class behaviour. It is an aid
for the streaming system: the stored specifier is specified in the RTTI
generated for a class (it can only be streamed if RTTI is generated), 
and is used to determine whether a property should be streamed or not: 
it saves space in a stream. It is not possible to specify the 'Stored'
directive for array properties.

The {\em default specifier} can be specified for ordinal types and sets.
It serves the same purpose as the {\em stored specifier}: Properties that
have as value their default value, will not be written to the stream by the
streaming system. The default value is stored in the RTTI that is generated
for the class. Note that
\begin{enumerate}
\item When the class is instantiated, the default value is not automatically
applied to the property, it is the responsability of the programmer to do
this in the constructor of the class.
\item The value 2147483648 cannot be used as a default value, as it is used
internally to denote 'nodefault'.
\item It is not possible to specify a default for array properties.
\end{enumerate}


%%%%%%%%%%%%%%%%%%%%%%%%%%%%%%%%%%%%%%%%%%%%%%%%%%%%%%%%%%%%%%%%%%%%%%%
% Interfaces
%%%%%%%%%%%%%%%%%%%%%%%%%%%%%%%%%%%%%%%%%%%%%%%%%%%%%%%%%%%%%%%%%%%%%%%
\chapter{Interfaces}
\label{ch:Interfaces}\index{Interfaces}
\section{Definition}
As of version 1.1, FPC supports interfaces. Interfaces are an 
alternative to multiple inheritance (where a class can have multiple
parent classes) as implemented for instance in C++.  An interface is
basically a named set of methods and properties: A class that 
{\em implements} the interface provides {\em all} the methods as 
they are enumerated in the Interface definition. It is not possible for a
class to implement only part of the interface: it is all or nothing.

Interfaces can also be ordered in a hierarchy, exactly as classes:
An interface definition that inherits from another interface definition
contains all the methods from the parent interface, as well as the methods
explicitly named in the interface definition. A class implementing an
interface must then implement all members of the interface as well as the
methods of the parent interface(s).

An interface can be uniquely identified by a GUID (GUID is an acronym for
Globally Unique Identifier, a 128-bit integer guaranteed always to be 
unique\footnote{In theory, of course.}. Especially on Windows systems, the
GUID of an interface can and must be used when using COM.

The definition of an Interface has the following
form:\index{interface}\keywordlink{interface}
\input{syntax/typeintf.syn}
Along with this definition the following must be noted:
\begin{itemize}
\item Interfaces can only be used in \var{DELPHI} mode or in \var{OBJFPC}
mode.
\item There are no visibility specifiers. All members are public (indeed,
it would make little sense to make them private or
protected).\index{Visibility}
\item The properties declared in an interface can only have methods as read and
write specifiers.
\item There are no constructors or destructors. Instances of interfaces
cannot be created directly: instead, an instance of a class implementing 
the interface must be created.
\item Only calling convention modifiers may be present in the definition of
a method. Modifiers as \var{virtual}, \var{abstract} or \var{dynamic}, and
hence also \var{override} cannot be present in the definition of a interface
definition.
\end{itemize}
The following are examples of interfaces:
\begin{verbatim}
IUnknown = interface ['{00000000-0000-0000-C000-000000000046}']
  function QueryInterface(const iid : tguid;out obj) : longint;
  function _AddRef : longint;
  function _Release : longint;
end;
IInterface = IUnknown;

IMyInterface = Interface
  Function MyFunc : Integer;
  Function MySecondFunc : Integer;
end;
\end{verbatim}
As can be seen, the GUID identifying the interface is optional.

\section{Interface identification: A GUID}
An interface can be identified by a GUID. This is a 128-bit number, which is
represented in a text representation (a string literal):
\begin{verbatim}
['{HHHHHHHH-HHHH-HHHH-HHHH-HHHHHHHHHHHH}']
\end{verbatim}
Each \var{H} character represents a hexadecimal number (0-9,A-F). The format
contains 8-4-4-4-12 numbers. A GUID can also be represented by the following
record, defined in the \file{objpas} unit (included automatically when in
\var{DELPHI} or \var{OBJFPC} mode:
\begin{verbatim}
PGuid = ^TGuid;
TGuid = packed record
   case integer of
      1 : (
           Data1 : DWord;
           Data2 : word;
           Data3 : word;
           Data4 : array[0..7] of byte;
          );
      2 : (
           D1 : DWord;
           D2 : word;
           D3 : word;
           D4 : array[0..7] of byte;
          );
end;
\end{verbatim}
A constant of type TGUID can be specified using a string literal:
\begin{verbatim}
{$mode objfpc}
program testuid;

Const
  MyGUID : TGUID = '{10101010-1010-0101-1001-110110110110}';

begin
end.
\end{verbatim}
Normally, the GUIDs are only used in Windows, when using COM interfaces.
More on this in the next section.

\section{Interface implementations}
\index{Interfaces!Implementations}
When a class implements an interface, it should implement all methods of the
interface. If a method of an interface is not implemented, then the compiler
will give an error. For example:
\begin{verbatim}
Type
  IMyInterface = Interface
    Function MyFunc : Integer;
    Function MySecondFunc : Integer;
  end;

  TMyClass = Class(TInterfacedObject,IMyInterface)
    Function MyFunc : Integer;
    Function MyOtherFunc : Integer;
  end;

Function TMyClass.MyFunc : Integer;

begin
  Result:=23;
end;

Function TMyClass.MyOtherFunc : Integer;

begin
  Result:=24;
end;
\end{verbatim}
will result in a compiler error:
\begin{verbatim}
Error: No matching implementation for interface method
"IMyInterface.MySecondFunc:LongInt" found
\end{verbatim}

At the moment of writing, the compiler does not yet support providing
aliases for an interface as in Delphi. i.e. the following will not yet 
compile:
\begin{verbatim}
ype
  IMyInterface = Interface
    Function MyFunc : Integer;
  end;

  TMyClass = Class(TInterfacedObject,IMyInterface)
    Function MyOtherFunction : Integer;
    // The following fails in FPC.
    Function IMyInterface.MyFunc = MyOtherFunction;
  end;
\end{verbatim}
This declaration should tell the compiler that the \var{MyFunc} method of
the \var{IMyInterface} interface is implemented in the \var{MyOtherFunction}
method of the \var{TMyClass} class.

\section{Interfaces and COM}
\index{COM}\index{Interfaces!COM}
When using interfaces on Windows which should be available to the COM
subsystem, the calling convention should be \var{stdcall} - this is not the
default \fpc calling convention, so it should be specified explicitly.

COM does not know properties. It only knows methods. So when specifying
property definitions as part of an interface definition, be aware that the
properties will only be known in the \fpc compiled program: other Windows
programs will not be aware of the property definitions. For this reason,
property definitions must always have interface methods as the read/write
specifiers.

\section{CORBA and other Interfaces}
\index{CORBA}\index{COM}\index{Interfaces!CORBA}
COM is not the only architecture where interfaces are used. CORBA knows
interfaces, UNO (the OpenOffice API) uses interfaces, and Java as well.
These languages do not know the \var{IUnknown} interface used as the basis of
all interfaces in COM. It would therefore be a bad idea if an interface
automatically descended from \var{IUnknown} if no parent interface was
specified. Therefore, a directive \var{\{\$INTERFACES\}} was introduced in
 \fpc: it specifies what the parent interface is of an interface, declared
without parent. More information about this directive can be found in the
\progref.

Note that COM interfaces are by default reference
counted.\index{Types!Reference counted}
CORBA interfaces are not necessarily reference counted.

%%%%%%%%%%%%%%%%%%%%%%%%%%%%%%%%%%%%%%%%%%%%%%%%%%%%%%%%%%%%%%%%%%%%%%%
% Generics
%%%%%%%%%%%%%%%%%%%%%%%%%%%%%%%%%%%%%%%%%%%%%%%%%%%%%%%%%%%%%%%%%%%%%%%
\chapter{Generics}
\label{ch:generics}
\index{Generics}
\section{Introduction}
Generics are templates for generating classes. It is a concept that
comes from C++, where it is deeply integrated in the language. As of
version 2.2, Free Pascal also officially has support for templates or
Generics. They are implemented as a kind of macro which is stored in the
unit files that the compiler generates, and which is replayed as soon
as a generic class is specialized.

Currently, only generic classes can be defined. Later, support for
generic records, functions and arrays may be introduced.

Creating and using generics is a 2-phase process.
\begin{enumerate}
\item The definition of the generic class is defined as a new type: 
this is a code template, a macro which can be replayed by the compiler 
at a later stage.
\item A generic class is specialized: this defines a second class,
which is a specific implementation of the generic class: the compiler
replays the macro which was stored when the generic class was defined.
\end{enumerate}

\section{Generic class definition}
A generic class definition is much like a class definition, with the
exception that it contains a list of placeholders for types, and can 
contain a series of local variable blocks or local type blocks, as can be
seen in the following syntax diagram:
\input{syntax/generic.syn}
The generic class declaration should be followed by a class implementation.
It is the same as a normal class implementation with a single exception,
namely that any identifier with the same name as one of the template
identifiers must be a type identifier.

The generic class declaration is much like a normal class declaration, 
except for the local variable and local type block. The local type block
defines types that are type placeholders: they are not actualized until
the class is specialized.

The local variable block is just an alternate syntax for ordinary class fields.
The reason for introducing is the introduction of the \var{Type} block: just
as in a unit or function declaration, a class declaration can now have a
local type and variable block definition.

The following is a valid generic class definition:
\begin{verbatim}
Type
  generic TList<_T>=class(TObject)
    type public
       TCompareFunc = function(const Item1, Item2: _T): Integer;
    var public
      data : _T;
    procedure Add(item: _T);
    procedure Sort(compare: TCompareFunc);
  end;
\end{verbatim}
This class could be followed by an implementation as follows:
\begin{verbatim}
procedure TList.Add(item: _T);
begin
  data:=item;
end;

procedure TList.Sort(compare: TCompareFunc);
begin
  if compare(data, 20) <= 0 then
    halt(1);
end;
\end{verbatim}
There are some noteworthy things about this declaration and implementation:
\begin{enumerate}
\item There is a single placeholder \var{\_T}. It will be substituted by a
type identifier when the generic class is specialized. The identifier
\var{\_T} may not be used for anything else than a placehoder. This means
that the following would be invalid:
\begin{verbatim}
procedure TList.Sort(compare: TCompareFunc);

Var
  _t : integer;

begin
  // do something.
end;
\end{verbatim}
\item The local type block contains a single type \var{TCompareFunc}. Note
that the actual type is not yet known inside the generic class definition:
the definition contains a reference to the placeholder \var{\_T}. All other
identifier references must be known when the generic class is defined, {\em not}
when the generic class is specialized.
\item The local variable block is equivalent to the following:
\begin{verbatim}
  generic TList<_T>=class(TObject)
    type public
       TCompareFunc = function(const Item1, Item2: _T): Integer;
  Public  
    data : _T;
    procedure Add(item: _T);
    procedure Sort(compare: TCompareFunc);
  end;
\end{verbatim}
\item Both the local variable block and local type block have a visibility
specifier. This is optional; if it is omitted, the current visibility is
used.
\end{enumerate}

\section{Generic class specialization}
Once a generic class is defined, it can be used to generate other classes:
this is like replaying the definition of the class, with the template
placeholders filled in with actual type definitions.

This can be done in any \var{Type} definition block. The specialized type
looks as follows:
\input{syntax/specialize.syn}
Which is a very simple definition. Given the declaration of \var{TList} in
the previous section, the following would be a valid type definition:
\begin{verbatim}
Type
  TPointerList = specialize TList<Pointer>;
  TIntegerList = specialize TList<Integer>;
\end{verbatim}
The following is not allowed:
\begin{verbatim}
Var
  P : specialize TList<Pointer>;  
\end{verbatim}
that is, a variable cannot be directly declared using a specialization.

The type in the specialize statement must be known. Given the 2 generic
class definitions:
\begin{verbatim}
type 
  Generic TMyFirstType<T1> = Class(TMyObject);
  Generic TMySecondType<T2> = Class(TMyOtherObject);
\end{verbatim}
Then the following specialization is not valid:
\begin{verbatim}
type
  TMySpecialType = specialize TMySecondType<TMyFirstType>;
\end{verbatim}
because the type \var{TMyFirstType} is a generic type, and thus
not fully defined. However, the following is allowed:
\begin{verbatim}
type
  TA = specialize TMyFirstType<Atype>;
  TB = specialize TMySecondType<TA>;
\end{verbatim}
because \var{TA} is already fully defined when \var{TB} is specialized.

Note that 2 specializations of a generic type with the same types in a
placeholder are not assignment compatible. In the following example:
\begin{verbatim}
type
  TA = specialize TList<Pointer>;
  TB = specialize TList<Pointer>;
\end{verbatim}
variables of types \var{TA} and \var{TB} cannot be assigned to each other,
i.e the following assignment will be invalid:
\begin{verbatim}
Var
  A : TA;
  B : TB;

begin
  A:=B;
\end{verbatim}

\begin{remark}
It is not possible to make a forward definition of a generic class. The
compiler will generate an error if a forward declaration of a class is
later defined as a generic specialization.
\end{remark}

\section{A word about scope}
It should be stressed that all identifiers other than the template placeholders
should be known when the generic class is declared. This works in 2 ways.
First, all types must be known, that is, a type identifier with the same name
must exist. The following unit will produce an error:
\begin{verbatim}
unit myunit;

interface

type 
  Generic TMyClass<T> = Class(TObject)
    Procedure DoSomething(A : T; B : TSomeType);
  end;

Type
  TSomeType = Integer;
  TSomeTypeClass = specialize TMyClass<TSomeType>;

Implementation

Procedure TMyClass.DoSomething(A : T; B : TSomeType);

begin
  // Some code.
end;

end.
\end{verbatim}
The above code will result in an error, because the type \var{TSomeType} is
not known when the declaration is parsed:
\begin{verbatim}
home: >fpc myunit.pp
myunit.pp(8,47) Error: Identifier not found "TSomeType"
myunit.pp(11,1) Fatal: There were 1 errors compiling module, stopping
\end{verbatim}

The second way in which this is visible, is the following. Assume a unit
\begin{verbatim}
unit mya;

interface

type
  Generic TMyClass<T> = Class(TObject)
    Procedure DoSomething(A : T);
  end;


Implementation

Procedure DoLocalThings;

begin
  Writeln('mya.DoLocalThings');
end;


Procedure TMyClass.DoSomething(A : T);

begin
  DoLocalThings;
end;

end.
\end{verbatim}
and a program 
\begin{verbatim}
program myb;

uses mya;

procedure DoLocalThings;

begin
  Writeln('myb.DoLocalThings');
end;

Type
  TB = specialize TMyClass<Integer>;

Var
  B : TB;

begin
  B:=TB.Create;
  B.DoSomething(1);
end.
\end{verbatim}
Despite the fact that generics act as a macro which is replayed at
specialization time, the reference to \var{DoLocalThings} is resolved
when \var{TMyClass} is defined, not when TB is defined. This means that the
output of the program is:
\begin{verbatim}
home: >fpc -S2 myb.pp
home: >myb
mya.DoLocalThings
\end{verbatim}
This is dictated by safety and necessity:
\begin{enumerate}
\item A programmer specializing a class has no way of knowing which local
procedures are used, so he cannot accidentally 'override' it.
\item A programmer specializing a class has no way of knowing which local
procedures are used, so he cannot implement it either, since he does not
know the parameters.
\item If implementation procedures are used as in the example above, they 
cannot be referenced from outside the unit. They could be in another unit
altogether, and the programmer has no way of knowing he should include them
before specializing his class.
\end{enumerate}

\section{Operator overloading and generics}

%%%%%%%%%%%%%%%%%%%%%%%%%%%%%%%%%%%%%%%%%%%%%%%%%%%%%%%%%%%%%%%%%%%%%%%
% Expressions
%%%%%%%%%%%%%%%%%%%%%%%%%%%%%%%%%%%%%%%%%%%%%%%%%%%%%%%%%%%%%%%%%%%%%%%
\chapter{Expressions}
\label{ch:Expressions}
\index{Expressions}
Expressions occur in assignments or in tests. Expressions produce a value,
of a certain type.
Expressions are built with two components: Operators and their operands.
Usually an operator is binary, i.e. it requires 2 operands. Binary operators
occur always between the operands (as in \var{X/Y}). Sometimes an
operator is unary, i.e. it requires only one argument. A unary operator
occurs always before the operand, as in \var{-X}.

When using multiple operands in an expression, the precedence rules of
\seet{OperatorPrecedence} are used.\index{Operators}
\begin{FPCltable}{lll}{Precedence of operators}{OperatorPrecedence}
Operator & Precedence & Category \\ \hline
\var{Not, @} & Highest (first) & Unary operators\\
\var{* / div mod and shl shr as} & Second & Multiplying operators\\
\var{+ - or xor} & Third & Adding operators \\
\var{< <> < > <= >= in is} & Lowest (Last) & relational operators \\
\hline
\end{FPCltable}
When determining the precedence, the compiler uses the following rules:
\begin{enumerate}
\item In operations with unequal precedences the operands belong to the
operater with the highest precedence. For example, in \var{5*3+7}, the
multiplication is higher in precedence than the addition, so it is
executed first. The result would be 22.
\item If parentheses are used in an expression, their contents is evaluated
first. Thus, \var {5*(3+7)} would result in 50.
\end{enumerate}

\begin{remark}
The order in which expressions of the same precedence are evaluated is not
guaranteed to be left-to-right. In general, no assumptions on which expression
is evaluated first should be made in such a case.
The compiler will decide which expression to evaluate first based on
optimization rules. Thus, in the following expression:
\begin{verbatim}
  a := g(3) + f(2);
\end{verbatim}
\var{f(2)} may be executed before \var{g(3)}. This behaviour is distinctly
different from \delphi or \tp.

If one expression {\em must} be executed before the other, it is necessary
to split up the statement using temporary results:
\begin{verbatim}
  e1 := g(3);
  a  := e1 + f(2);
\end{verbatim}
\end{remark}

%%%%%%%%%%%%%%%%%%%%%%%%%%%%%%%%%%%%%%%%%%%%%%%%%%%%%%%%%%%%%%%%%%%%%%%
% Expression syntax
\section{Expression syntax}
An expression applies relational operators to simple expressions. Simple
expressions are a series of terms (what a term is, is explained below), joined by
adding operators.
\input{syntax/expsimpl.syn}
The following are valid expressions:
\begin{verbatim}
GraphResult<>grError
(DoItToday=Yes) and (DoItTomorrow=No);
Day in Weekend
\end{verbatim}
And here are some simple expressions:
\begin{verbatim}
A + B
-Pi
ToBe or NotToBe
\end{verbatim}
Terms consist of factors, connected by multiplication operators.
\input{syntax/expterm.syn}
Here are some valid terms:
\begin{verbatim}
2 * Pi
A Div B
(DoItToday=Yes) and (DoItTomorrow=No);
\end{verbatim}
Factors are all other constructions:
\input{syntax/expfact.syn}

%%%%%%%%%%%%%%%%%%%%%%%%%%%%%%%%%%%%%%%%%%%%%%%%%%%%%%%%%%%%%%%%%%%%%%%
% Function calls
\section{Function calls}
Function calls are part of expressions (although, using extended syntax,
they can be statements too). They are constructed as follows:
\input{syntax/fcall.syn}
The \synt{variable reference} must be a procedural type variable reference.
A method designator can only be used inside the method of an object. A
qualified method designator can be used outside object methods too.
The function that will get called is the function with a declared parameter
list that matches the actual parameter list. This means that
\begin{enumerate}
\item The number of actual parameters must equal the number of declared
parameters (unless default parameter values are used).
\item The types of the parameters must be compatible. For variable
reference parameters, the parameter types must be exactly the same.
\end{enumerate}
If no matching function is found, then the compiler will generate an error.
Depending on the fact of the function is overloaded (i.e. multiple functions
with the same name, but different parameter lists) the error will be
different.
There are cases when the compiler will not execute the function call in an
expression. This is the case when assigning a value to a procedural
type variable, as in the following example:
\begin{verbatim}
Type
  FuncType = Function: Integer;
Var A : Integer;
Function AddOne : Integer;
begin
  A := A+1;
  AddOne := A;
end;
Var F : FuncType;
    N : Integer;
begin
  A := 0;
  F := AddOne; { Assign AddOne to F, Don't call AddOne}
  N := AddOne; { N := 1 !!}
end.
\end{verbatim}
In the above listing, the assigment to F will not cause the function AddOne
to be called. The assignment to N, however, will call AddOne.
A problem with this syntax is the following construction:
\begin{verbatim}
If F = AddOne Then
  DoSomethingHorrible;
\end{verbatim}
Should the compiler compare the addresses of \var{F} and \var{AddOne},
or should it call both functions, and compare the result ? \fpc solves this
by deciding that a procedural variable is equivalent to a pointer. Thus the
compiler will give a type mismatch error, since AddOne is considered a
call to a function with integer result, and F is a pointer, Hence a type
mismatch occurs.
How then, should one compare whether \var{F} points to the function
\var{AddOne} ? To do this, one should use the address operator \var{@}:
\begin{verbatim}
If F = @AddOne Then
  WriteLn ('Functions are equal');
\end{verbatim}
The left hand side of the boolean expression is an address. The right hand
side also, and so the compiler compares 2 addresses.
How to compare the values that both functions return ? By adding an empty
parameter list:
\begin{verbatim}
  If F()=Addone then
    WriteLn ('Functions return same values ');
\end{verbatim}
Remark that this behaviour is not compatible with Delphi syntax.

%%%%%%%%%%%%%%%%%%%%%%%%%%%%%%%%%%%%%%%%%%%%%%%%%%%%%%%%%%%%%%%%%%%%%%%
% Set constructors
\section{Set constructors}
\index{Constructor}
When a set-type constant must be entered in an expression, a
set constructor must be given. In essence this is the same thing as when a
type is defined, only there is no identifier to identify the set with.
A set constructor is a comma separated list of expressions, enclosed in
square brackets.
\input{syntax/setconst.syn}
All set groups and set elements must be of the same ordinal type.
The empty set is denoted by \var{[]}, and it can be assigned to any type of
set. A set group with a range  \var{[A..Z]} makes all values in the range a
set element. 
The following are valid set constructors:
\begin{verbatim}
[today,tomorrow]
[Monday..Friday,Sunday]
[ 2, 3*2, 6*2, 9*2 ]
['A'..'Z','a'..'z','0'..'9']
\end{verbatim}
\begin{remark}
If the first range specifier has a bigger ordinal value than
the second, the resulting set will be empty, e.g., \var{[Z..A]} 
denotes an empty set. One should be careful when denoting a range.
\end{remark}

%%%%%%%%%%%%%%%%%%%%%%%%%%%%%%%%%%%%%%%%%%%%%%%%%%%%%%%%%%%%%%%%%%%%%%%
% Value typecasts
\section{Value typecasts}
\index{Typecast}\index{Typecast!Value}
Sometimes it is necessary to change the type of an expression, or a part of
the expression, to be able to be assignment compatible. This is done through
a value typecast. The syntax diagram for a value typecast is as follows:
\input{syntax/tcast.syn}
Value typecasts cannot be used on the left side of assignments, as variable
typecasts.
Here are some valid typecasts:
\begin{verbatim}
Byte('A')
Char(48)
boolean(1)
longint(@Buffer)
\end{verbatim}
In general, the type size of the expression and the size of the type cast 
must be the same. However, for ordinal types (byte, char, word, boolean,
enumerateds) this is not so, they can be used interchangeably. 
That is, the following will work, although the sizes do not match.
\begin{verbatim}
Integer('A');
Char(4875);
boolean(100);
Word(@Buffer);
\end{verbatim}
This is compatible w5Bith Delphi or Turbo Pascal behaviour.

Kx%%%%%%%%%%%%%%%%%%%%%%%%%%%%%%%%%%%%%%%%%%%%%%%%%%%%%%%%%%%%%%%%%%%%%%%
% Variable typecasts
\section{Variable typecasts}
\index{Typecast}\index{Typecast!Variable}
A variable can be considered a single factor in an expression. It can
therefore be typecast as well. A variable can be typecast to any type,
provided the type has the same size as the original variable. 

It is a bad idea to typecast integer types to real types and vice versa.
It's better to rely on type assignment compatibility and using some of the
standard type changing functions.

Note that variable typecasts can occur on either side of an assignment,
i.e. the following are both valid typecasts:
\begin{verbatim}
Var
  C : Char;
  B : Byte;

begin
  B:=Byte(C);
  Char(B):=C;
end;
\end{verbatim}
pointer variables can be typecasted to procedural types.

A typecast is an expression of the given type, which means the
typecast can be followed by a qualifier:
\begin{verbatim}
Type 
  TWordRec = Packed Record
    L,H : Byte;
  end;

Var
  P : Pointer;
  W : Word;
  S : String;

begin
  TWordRec(W).L:=$FF;
  TWordRec(W).H:=0;
  S:=TObject(P).ClassName;
\end{verbatim}

%%%%%%%%%%%%%%%%%%%%%%%%%%%%%%%%%%%%%%%%%%%%%%%%%%%%%%%%%%%%%%%%%%%%%%%
% The @ operator
\section{The @ operator}
\index{Operators}\index{Address}
The address operator \var{@} returns the address of a variable, procedure
or function. It is used as follows:
\input{syntax/address.syn}
The \var{@} operator returns a typed pointer if the \var{\$T} switch is on.
If the \var{\$T} switch is off then the address operator returns an untyped
pointer, which is assigment compatible with all pointer types. The type of
the pointer is \var{\^{}T}, where \var{T} is the type of the variable
reference.
For example, the following will compile
\begin{verbatim}
Program tcast;
{$T-} { @ returns untyped pointer }

Type art = Array[1..100] of byte;
Var Buffer : longint;
    PLargeBuffer : ^art;

begin
 PLargeBuffer := @Buffer;
end.
\end{verbatim}
Changing the \var{\{\$T-\}} to \var{\{\$T+\}} will prevent the compiler from
compiling this. It will give a type mismatch error.
By default, the address operator returns an untyped pointer.
Applying the address operator to a function, method, or procedure identifier
will give a pointer to the entry point of that function. The result is an
untyped pointer.
By default, the address operator must be used if a value must be assigned
to a procedural type variable. This behaviour can be avoided by using the
\var{-So} or \var{-S2} switches, which result in a more compatible Delphi or
Turbo Pascal syntax.

%%%%%%%%%%%%%%%%%%%%%%%%%%%%%%%%%%%%%%%%%%%%%%%%%%%%%%%%%%%%%%%%%%%%%%%
% Operators
\section{Operators}
\index{Operators}
Operators can be classified according to the type of expression they
operate on. We will discuss them type by type.
%
\subsection{Arithmetic operators}
\index{Operators!Arithmetic}
Arithmetic operators occur in arithmetic operations, i.e. in expressions
that contain integers or reals. There are 2 kinds of operators : Binary and
unary arithmetic operators.
Binary operators are listed in \seet{binaroperators}, unary operators are
listed in \seet{unaroperators}.
\begin{FPCltable}{ll}{Binary arithmetic operators}{binaroperators}
Operator & Operation \\ \hline
\var{+} & Addition\\
\var{-} & Subtraction\\
\var{*} & Multiplication \\
\var{/} & Division \\
\var{Div} & Integer division \\
\var{Mod} & Remainder \\ \hline
\end{FPCltable}
With the exception of \var{Div} and \var{Mod}, which accept only integer
expressions as operands, all operators accept real and integer expressions as
operands.
For binary operators, the result type will be integer if both operands are
integer type expressions. If one of the operands is a real type expression,
then  the result is real.
As an exception : division (\var{/}) results always in real values.
\index{Operators!Unary}
\begin{FPCltable}{ll}{Unary arithmetic operators}{unaroperators}
Operator & Operation \\ \hline
\var{+} & Sign identity\\
\var{-} & Sign inversion \\ \hline
\end{FPCltable}
For unary operators, the result type is always equal to the expression type.
The division (\var{/}) and \var{Mod} operator will cause run-time errors if
the second argument is zero.
The sign of the result of a \var{Mod} operator is the same as the sign of
the left side operand of the \var{Mod} operator. In fact, the \var{Mod}
operator is equivalent to the following operation :
\begin{verbatim}
  I mod J = I - (I div J) * J
\end{verbatim}
but it executes faster than the right hand side expression.
%
\subsection{Logical operators}
\index{Operators!Logical}
Logical operators act on the individual bits of ordinal expressions.
Logical operators require operands that are of an integer type, and produce
an integer type result. The possible logical operators are listed in
\seet{logicoperations}.
\begin{FPCltable}{ll}{Logical operators}{logicoperations}
Operator & Operation \\ \hline
\var{not} & Bitwise negation (unary) \\
\var{and} & Bitwise and \\
\var{or}  & Bitwise or \\
\var{xor} & Bitwise xor \\
\var{shl} & Bitwise shift to the left \\
\var{shr} & Bitwise shift to the right \\ \hline
\end{FPCltable}
The following are valid logical expressions:
\begin{verbatim}
A shr 1  { same as A div 2, but faster}
Not 1    { equals -2 }
Not 0    { equals -1 }
Not -1   { equals 0  }
B shl 2  { same as B * 4 for integers }
1 or 2   { equals 3 }
3 xor 1  { equals 2 }
\end{verbatim}
%
\subsection{Boolean operators}
\index{Operators!Boolean}
Boolean operators can be considered logical operations on a type with 1 bit
size. Therefore the \var{shl} and \var{shr} operations have little sense.
Boolean operators can only have boolean type operands, and the resulting
type is always boolean. The possible operators are listed in
\seet{booleanoperators}
\begin{FPCltable}{ll}{Boolean operators}{booleanoperators}
Operator & Operation \\ \hline
\var{not} & logical negation (unary) \\
\var{and} & logical and \\
\var{or}  & logical or \\
\var{xor} & logical xor \\ \hline
\end{FPCltable}
\begin{remark} Boolean expressions are always evaluated with short-circuit
evaluation. This means that from the moment the result of the complete
expression is known, evaluation is stopped and the result is returned.
For instance, in the following expression:
\begin{verbatim}
 B := True or MaybeTrue;
\end{verbatim}
The compiler will never look at the value of \var{MaybeTrue}, since it is
obvious that the expression will always be true. As a result of this
strategy, if \var{MaybeTrue} is a function, it will not get called !
(This can have surprising effects when used in conjunction with properties)
\end{remark}
%
\subsection{String operators}
\index{Operators!String}
There is only one string operator : \var{+}. It's action is to concatenate
the contents of the two strings (or characters) it stands between.
One cannot use \var{+} to concatenate null-terminated (\var{PChar}) strings.
The following are valid string operations:
\begin{verbatim}
  'This is ' + 'VERY ' + 'easy !'
  Dirname+'\'
\end{verbatim}
The following is not:
\begin{verbatim}
Var Dirname = Pchar;
...
  Dirname := Dirname+'\';
\end{verbatim}
Because \var{Dirname} is a null-terminated string.
%
\subsection{Set operators}
\index{Operators!Set}
The following operations on sets can be performed with operators:
Union, difference and intersection. The operators needed for this are listed
in \seet{setoperators}.
\begin{FPCltable}{ll}{Set operators}{setoperators}
Operator & Action \\ \hline
\var{+} & Union \\
\var{-} & Difference \\
\var{*} & Intersection \\ \hline
\end{FPCltable}
The set type of the operands must be the same, or an error will be
generated by the compiler.
%
\subsection{Relational operators}
\index{Operators!Relational}
The relational operators are listed in \seet{relationoperators}
\begin{FPCltable}{ll}{Relational operators}{relationoperators}
Operator & Action \\ \hline
\var{=} & Equal \\
\var{<>} & Not equal \\
\var{<} & Stricty less than\\
\var{>} & Strictly greater than\\
\var{<=} & Less than or equal \\
\var{>=} & Greater than or equal \\
\var{in} & Element of \\ \hline
\end{FPCltable}
Normally, left and right operands must be of the same type. There are some
notable exceptions, where the compiler can handle mixed expressions:
\begin{enumerate}
\item Integer and real types can be mixed in relational expressions.
\item If the operator is overloaded, and an overloaded version exists whose
arguments types match the types in the expression.
\item Short-, Ansi- and widestring types can be mixed.
\end{enumerate}
Comparing strings is done on the basis of their character code representation.

When comparing pointers, the addresses to which they point are compared.
This also is true for \var{PChar} type pointers. To compare the strings
the \var{Pchar} point to, the \var{StrComp} function
from the \file{strings} unit must be used.
The \var{in} returns \var{True} if the left operand (which must have the same
ordinal type as the set type, and which must be in the range 0..255) is an 
element of the set which is the right operand, otherwise it returns \var{False}

\subsection{Class operators}
Class operators are slightly different from the operators above in the sense
that they can only be used in class expressions which return a class. There
are only 2 class operators, as can be seen in \seet{classoperators}.
\begin{FPCltable}{ll}{Class operators}{classoperators}
Operator & Action \\ \hline
\var{is} & Checks class type \\
\var{as} & Conditional typecast \\
\end{FPCltable}
An expression containing the \var{is} operator results in a boolean type.
The \var{is} operator can only be used with a class reference or a class
instance. The usage of this operator is as follows:
\begin{verbatim}
 Object is Class
\end{verbatim}
This expression is completely equivalent to
\begin{verbatim}
 Object.InheritsFrom(Class)
\end{verbatim}
If \var{Object} is \var{Nil}, \var{False} will be returned.

The following are examples:
\begin{verbatim}
Var
  A : TObject;
  B : TClass;

begin
  if A is TComponent then ;
  If A is B then; 
end;
\end{verbatim}

The \var{as} operator performs a conditional typecast. It results in an
expression that has the type of the class:
\begin{verbatim}
  Object as Class
\end{verbatim}
This is equivalent to the following statements:
\begin{verbatim}
  If Object=Nil then
    Result:=Nil
  else if Object is Class then
    Result:=Class(Object)
  else
    Raise Exception.Create(SErrInvalidTypeCast);
\end{verbatim}
Note that if the object is \var{nil}, the \var{as} operator does not
generate an exception.

The following are some examples of the use of the \var{as} operator:
\begin{verbatim}
Var
  C : TComponent;
  O : TObject; 

begin
  (C as TEdit).Text:='Some text';
  C:=O as TComponent;
end;
\end{verbatim}

%%%%%%%%%%%%%%%%%%%%%%%%%%%%%%%%%%%%%%%%%%%%%%%%%%%%%%%%%%%%%%%%%%%%%%%
% Statements
%%%%%%%%%%%%%%%%%%%%%%%%%%%%%%%%%%%%%%%%%%%%%%%%%%%%%%%%%%%%%%%%%%%%%%%
\chapter{Statements}
\label{ch:Statements}\index{Statements}
The heart of each algorithm are the actions it takes. These actions are
contained in the statements of a program or unit. Each statement can be
labeled and jumped to (within certain limits) with \var{Goto} statements.
This can be seen in the following syntax diagram:
\input{syntax/statement.syn}
A label can be an identifier or an integer digit.

%%%%%%%%%%%%%%%%%%%%%%%%%%%%%%%%%%%%%%%%%%%%%%%%%%%%%%%%%%%%%%%%%%%%%%%
% Simple statements
\section{Simple statements}
\index{Statements!Simple}
A simple statement cannot be decomposed in separate statements. There are
basically 4 kinds of simple statements:
\input{syntax/simstate.syn}
Of these statements, the {\em raise statement} will be explained in the
chapter on Exceptions (\seec{Exceptions})

\subsection{Assignments}
\index{Statements!Assignment}
Assignments give a value to a variable, replacing any previous value the
variable might have had:
\input{syntax/assign.syn}
In addition to the standard Pascal assignment operator (\var{ := }), which
simply replaces the value of the varable with the value resulting from the
expression on the right of the { := } operator, \fpc
supports some c-style constructions. All available constructs are listed in
\seet{assignments}.
\begin{FPCltable}{lr}{Allowed C constructs in \fpc}{assignments}
Assignment & Result \\ \hline
a += b & Adds \var{b} to \var{a}, and stores the result in \var{a}.\\
a -= b & Substracts \var{b} from \var{a}, and stores the result in
\var{a}. \\
a *= b & Multiplies \var{a} with \var{b}, and stores the result in
\var{a}. \\
a /= b & Divides \var{a} through \var{b}, and stores the result in
\var{a}. \\ \hline
\end{FPCltable}
For these constructs to work, the \var{-Sc} command-line switch must
be specified.

\begin{remark}
These constructions are just for typing convenience, they
don't generate different code.
Here are some examples of valid assignment statements:
\begin{verbatim}
X := X+Y;
X+=Y;      { Same as X := X+Y, needs -Sc command line switch}
X/=2;      { Same as X := X/2, needs -Sc command line switch}
Done := False;
Weather := Good;
MyPi := 4* Tan(1);
\end{verbatim}
\end{remark}

Keeping in mind that the dereferencing of a typed pointer results
in a variable of the type the pointer points to, the following are
also valid assignments:
\begin{verbatim}
Var
  L : ^Longint;
  P : PPChar; 

begin
  L^:=3;
  P^^:='A';
\end{verbatim}
Note the double dereferencing in the second assignment.
\subsection{Procedure statements}
\index{Statements!Procedure}
Procedure statements are calls to subroutines. There are
different possibilities for procedure calls: A normal procedure call, an
object method call (fully qualified or not), or even a call to a procedural
type variable. All types are present in the following diagram.
\input{syntax/procedure.syn}
The \fpc compiler will look for a procedure with the same name as given in
the procedure statement, and with a declared parameter list that matches the
actual parameter list.
The following are valid procedure statements:
\begin{verbatim}
Usage;
WriteLn('Pascal is an easy language !');
Doit();
\end{verbatim}

\subsection{Goto statements}
\index{Statements!Goto}\keywordlink{goto}
\fpc supports the \var{goto} jump statement. Its prototype syntax is
\input{syntax/goto.syn}
When using \var{goto} statements, the following must be kept in mind:
\begin{enumerate}
\item The jump label must be defined in the same block as the \var{Goto}
statement.
\item Jumping from outside a loop to the inside of a loop or vice versa can
 have strange effects.
\item To be able to use the \var{Goto} statement, the \var{-Sg} compiler
switch must be used.
\end{enumerate}
\var{Goto} statements are considered bad practice and should be avoided as
much as possible. It is always possible to replace a \var{goto} statement by a
construction that doesn't need a \var{goto}, although this construction may
not be as clear as a goto statement.
For instance, the following is an allowed goto statement:
\begin{verbatim}
label
  jumpto;
...
Jumpto :
  Statement;
...
Goto jumpto;
...
\end{verbatim}

%%%%%%%%%%%%%%%%%%%%%%%%%%%%%%%%%%%%%%%%%%%%%%%%%%%%%%%%%%%%%%%%%%%%%%%
% Structured statements
\section{Structured statements}
\index{Statements!Structured}
Structured statements can be broken into smaller simple statements, which
should be executed repeatedly, conditionally  or sequentially:
\input{syntax/struct.syn}
Conditional statements come in 2 flavours :
\input{syntax/conditio.syn}
Repetitive statements come in 3 flavours:
\input{syntax/repetiti.syn}
The following sections deal with each of these statements.

\subsection{Compound statements}
\index{Statements!Compound}
Compound statements are a group of statements, separated by semicolons,
that are surrounded by the keywords \var{Begin} and \var{End}. The
Last statement doesn't need to be followed by a semicolon, although it is
allowed. A compound statement is a way of grouping statements together,
executing the statements sequentially. They are treated as one statement
in cases where Pascal syntax expects 1 statement, such as in
\var{if ... then} statements.
\input{syntax/compound.syn}

\subsection{The \var{Case} statement}
\index{Statements!Case}\index{Case}\keywordlink{case}
\fpc supports the \var{case} statement. Its syntax diagram is
\input{syntax/case.syn}
The constants appearing in the various case parts must be known at
compile-time, and can be of the following types : enumeration types,
Ordinal types (except boolean), and chars. The expression must be also of
this type, or a compiler error will occur. All case constants must
have the same type.
The compiler will evaluate the expression. If one of the case constants
values matches the value of the expression, the statement that follows
this constant is executed. After that, the program continues after the final
\var{end}.
If none of the case constants match the expression value, the statement
after the \var{else} \index{else} keyword is executed. This can be an empty statement.
If no else part is present, and no case constant matches the expression
value, program flow continues after the final \var{end}.
The case statements can be compound statements
(i.e. a \var{begin..End} block).

\begin{remark}
Contrary to Turbo Pascal, duplicate case labels are not
allowed in \fpc, so the following code will generate an error when
compiling:
\begin{verbatim}
Var i : integer;
...
Case i of
 3 : DoSomething;
 1..5 : DoSomethingElse;
end;
\end{verbatim}
The compiler will generate a \var{Duplicate case label} error when compiling
this, because the 3 also appears (implicitly) in the range \var{1..5}. This
is similar to Delphi syntax.
\end{remark}
The following are valid case statements:
\begin{verbatim}
Case C of
 'a' : WriteLn ('A pressed');
 'b' : WriteLn ('B pressed');
 'c' : WriteLn ('C pressed');
else
  WriteLn ('unknown letter pressed : ',C);
end;
\end{verbatim}
Or
\begin{verbatim}
Case C of
 'a','e','i','o','u' : WriteLn ('vowel pressed');
 'y' : WriteLn ('This one depends on the language');
else
  WriteLn ('Consonant pressed');
end;
\end{verbatim}
\begin{verbatim}
Case Number of
 1..10   : WriteLn ('Small number');
 11..100 : WriteLn ('Normal, medium number');
else
 WriteLn ('HUGE number');
end;
\end{verbatim}

\subsection{The \var{If..then..else} statement}
\index{Statements!if}\index{If}\index{then}\index{else}\keywordlink{if}\keywordlink{then}\keywordlink{else}
The \var{If .. then .. else..} prototype syntax is
\input{syntax/ifthen.syn}
The expression between the \var{if} and \var{then} keywords must have a
boolean return type. If the expression evaluates to \var{True} then the
statement following \var{then} is executed.

If the expression evaluates to \var{False}, then the statement following
\var{else} is executed, if it is present.

Be aware of the fact that the boolean expression will be short-cut evaluated.
(Meaning that the evaluation will be stopped at the point where the
 outcome is known with certainty)
Also, before the \var {else} keyword,  no semicolon (\var{;}) is allowed,
but all statements can be compound statements.
In nested \var{If.. then .. else} constructs, some ambiguity may araise as
to which  \var{else} statement pairs with which \var{if} statement. The rule
is that the \var{else } keyword matches the first \var{if} keyword
(searching backwards) not already matched by an \var{else} keyword.
For example:
\begin{verbatim}
If exp1 Then
  If exp2 then
    Stat1
else
  stat2;
\end{verbatim}
Despite it's appearance, the statement is syntactically equivalent to
\begin{verbatim}
If exp1 Then
   begin
   If exp2 then
      Stat1
   else
      stat2
   end;
\end{verbatim}
and not to
\begin{verbatim}
{ NOT EQUIVALENT }
If exp1 Then
   begin
   If exp2 then
      Stat1
   end
else
   stat2
\end{verbatim}
If it is this latter construct which is needed, the \var{begin} and \var{end}
keywords must be present. When in doubt, it is better to add them.

The following is a valid statement:
\begin{verbatim}
If Today in [Monday..Friday] then
  WriteLn ('Must work harder')
else
  WriteLn ('Take a day off.');
\end{verbatim}

\subsection{The \var{For..to/downto..do} statement}
\index{Statements!For}\index{Statements!Loop}\index{For}
\fpc supports the \var{For} loop construction. A for loop is used in case
one wants to calculated something a fixed number of times.
The prototype syntax is as follows:
\input{syntax/for.syn}
\var{Statement} can be a compound statement.
When this statement is encountered, the control variable is initialized with
the initial value, and is compared with the final value.
What happens next depends on whether \var{to} or \var{downto} is used:
\begin{enumerate}
\item In the case \var{To} is used, if the initial value is larger than the final
value then \var{Statement} will never be executed.
\item In the case \var{DownTo} is used, if the initial value is less than the final
value then \var{Statement} will never be executed.
\end{enumerate}
After this check, the statement after \var{Do} is executed. After the
execution of the statement, the control variable is increased or decreased
with 1, depending on whether \var{To} or \var{Downto} is used.
The control variable must be an ordinal type, no other
types can be used as counters in a loop.

\begin{remark}
Contrary to ANSI pascal specifications, \fpc first initializes
the counter variable, and only then calculates the upper bound.
\end{remark}

\begin{remark}
It is not allowed to change (i.e. assign a value to) the value of a 
loop variable inside the loop.
\end{remark}

The following are valid loops:
\begin{verbatim}
For Day := Monday to Friday do Work;
For I := 100 downto 1 do
  WriteLn ('Counting down : ',i);
For I := 1 to 7*dwarfs do KissDwarf(i);
\end{verbatim}
The following will generate an error:
\begin{verbatim}
For I:=0 to 100 do
  begin
  DoSomething;
  I:=I*2;
  end;
\end{verbatim}
because the loop variable \var{I} cannot be assigned to.

If the statement is a compound statement, then the \var{Break} and
\var{Continue} reserved words can be used to jump to the end or just
after the end of the \var{For} statement.

\subsection{The \var{Repeat..until} statement}
\index{Statements!Repeat}\index{Statements!Loop}\index{Repeat}
The \var{repeat} statement is used to execute a statement until a certain
condition is reached. The statement will be executed at least once.
The prototype syntax of the \var{Repeat..until} statement is
\input{syntax/repeat.syn}
This will execute the statements between \var{repeat} and \var{until} up to
the moment when \var{Expression}\index{Expression} evaluates to \var{True}.
Since the \var{expression} is evaluated {\em after} the execution of the
statements, they are executed at least once.
Be aware of the fact that the boolean expression \var{Expression} will be
short-cut evaluated. (Meaning that the evaluation will be stopped at the
point where the outcome is known with certainty)
The following are valid \var{repeat} statements
\begin{verbatim}
repeat
  WriteLn ('I =',i);
  I := I+2;
until I>100;
repeat
 X := X/2
until x<10e-3
\end{verbatim}
The \var{Break} and \var{Continue} reserved words can be used to jump to
the end or just after the end of the \var{repeat .. until} statement.

\subsection{The \var{While..do} statement}
\index{Statements!While}\index{Statements!Loop}\index{While}
A \var{while} statement is used to execute a statement as long as a certain
condition holds. This may imply that the statement is never executed.
The prototype syntax of the \var{While..do} statement is
\input{syntax/while.syn}
This will execute \var{Statement} as long as \var{Expression} evaluates
to\index{Expression}
\var{True}. Since \var{Expression} is evaluated {\em before} the execution
of \var{Statement}, it is possible that \var{Statement} isn't executed at
all. \var{Statement} can be a compound statement.
Be aware of the fact that the boolean expression \var{Expression} will be
short-cut evaluated. (Meaning that the evaluation will be stopped at the
point where the outcome is known with certainty)
The following are valid \var{while} statements:
\begin{verbatim}
I := I+2;
while i<=100 do
  begin
  WriteLn ('I =',i);
  I := I+2;
  end;
X := X/2;
while x>=10e-3 do
  X := X/2;
\end{verbatim}
They correspond to the example loops for the \var{repeat} statements.

If the statement is a compound statement, then  the \var{Break} and
\var{Continue} reserved words can be used to jump to the end or just
after the end of the \var{While} statement.

\subsection{The \var{With} statement}
\label{se:With}\index{With}\index{Statements!With}
The \var{with} statement serves to access the elements of a record
or object or class, without having to specify the name of the each time.
The syntax for a \var{with} statement is
\input{syntax/with.syn}
The variable reference must be a variable of a record, object or class type.
In the \var{with} statement, any variable reference, or method reference is
checked to see if it is a field or method of the record or object or class.
If so, then that field is accessed, or that method is called.
Given the declaration:
\begin{verbatim}
Type 
  Passenger = Record
    Name : String[30];
    Flight : String[10];
  end;

Var 
  TheCustomer : Passenger;
\end{verbatim}
The following statements are completely equivalent:
\begin{verbatim}
TheCustomer.Name := 'Michael';
TheCustomer.Flight := 'PS901';
\end{verbatim}
and
\begin{verbatim}
With TheCustomer do
  begin
  Name := 'Michael';
  Flight := 'PS901';
  end;
\end{verbatim}
The statement
\begin{verbatim}
With A,B,C,D do Statement;
\end{verbatim}
is equivalent to
\begin{verbatim}
With A do
 With B do
  With C do
   With D do Statement;
\end{verbatim}
This also is a clear example of the fact that the variables are tried {\em last
to first}, i.e., when the compiler encounters a variable reference, it will
first check if it is a field or method of the last variable. If not, then it
will check the last-but-one, and so on.
The following example shows this;
\begin{verbatim}
Program testw;
Type AR = record
      X,Y : Longint;
     end;
     PAR = Record;

Var S,T : Ar;
begin
  S.X := 1;S.Y := 1;
  T.X := 2;T.Y := 2;
  With S,T do
    WriteLn (X,' ',Y);
end.
\end{verbatim}
The output of this program is
\begin{verbatim}
2 2
\end{verbatim}
Showing thus that the \var{X,Y} in the \var{WriteLn} statement match the
\var{T} record variable.

\begin{remark}
When using a \var{With} statement with a pointer, or a class, it is not
permitted to change the pointer or the class in the \var{With} block.
With the definitions of the previous example, the following illustrates
what it is about:
\begin{verbatim}

Var p : PAR;

begin
  With P^ do
   begin
   // Do some operations
   P:=OtherP;
   X:=0.0;  // Wrong X will be used !!
   end;
\end{verbatim}
The reason the pointer cannot be changed is that the address is stored
by the compiler in a temporary register. Changing the pointer won't change
the temporary address. The same is true for classes.
\end{remark}

\subsection{Exception Statements}
\index{Statements!Exception}
\fpc supports exceptions. Exceptions provide a convenient way to
program error and error-recovery mechanisms, and are
closely related to classes.
Exception support is explained in \seec{Exceptions}

%%%%%%%%%%%%%%%%%%%%%%%%%%%%%%%%%%%%%%%%%%%%%%%%%%%%%%%%%%%%%%%%%%%%%%%
% Assembler statements
\section{Assembler statements}
\index{Statements!Assembler}\index{Asm}\index{Assembler}
An assembler statement allows to insert assembler code right in the
pascal code.
\input{syntax/statasm.syn}
More information about assembler blocks can be found in the \progref.
The register list is used to indicate the registers that are modified by an
assembler statement in the assembler block. The compiler stores certain results in the
registers. If  the registers are modified in an assembler statement, the compiler
should, sometimes, be told about it. The registers are denoted with their
Intel names for the I386 processor, i.e., \var{'EAX'}, \var{'ESI'} etc...
As an example, consider the following assembler code:
\begin{verbatim}
asm
  Movl $1,%ebx
  Movl $0,%eax
  addl %eax,%ebx
end; ['EAX','EBX'];
\end{verbatim}
This will tell the compiler that it should save and restore the contents of
the \var{EAX} and \var{EBX} registers when it encounters this asm statement.

\fpc supports various styles of assembler syntax. By default, \var{AT\&T}
syntax is assumed for the 80386 and compatibles platform.
The default assembler style can be changed with the \var{\{\$asmmode xxx\}}
switch in the code, or the \var{-R} command-line option. More about this can
be found in the \progref.


%%%%%%%%%%%%%%%%%%%%%%%%%%%%%%%%%%%%%%%%%%%%%%%%%%%%%%%%%%%%%%%%%%%%%%%
% Using functions and procedures.
%%%%%%%%%%%%%%%%%%%%%%%%%%%%%%%%%%%%%%%%%%%%%%%%%%%%%%%%%%%%%%%%%%%%%%%
\chapter{Using functions and procedures}
\label{ch:Procedures}\index{Functions}\index{Procedures}
\fpc supports the use of functions and procedures, but with some extras:
Function overloading is supported, as well as \var{Const} parameters and
open arrays.

\begin{remark} In many of the subsequent paragraphs the words \var{procedure}
and \var{function} will be used interchangeably. The statements made are
valid for both, except when indicated otherwise.
\end{remark}

%%%%%%%%%%%%%%%%%%%%%%%%%%%%%%%%%%%%%%%%%%%%%%%%%%%%%%%%%%%%%%%%%%%%%%%
% Procedure declaration
\section{Procedure declaration}
A procedure declaration defines an identifier and associates it with a
block of code. The procedure can then be called with a procedure statement.
\input{syntax/procedur.syn}
See \sees{Parameters} for the list of parameters.\index{Procedure}
A procedure declaration that is followed by a block implements the action of
the procedure in that block.
The following is a valid procedure :
\begin{verbatim}
Procedure DoSomething (Para : String);
begin
  Writeln ('Got parameter : ',Para);
  Writeln ('Parameter in upper case : ',Upper(Para));
end;
\end{verbatim}
Note that it is possible that a procedure calls itself.

%%%%%%%%%%%%%%%%%%%%%%%%%%%%%%%%%%%%%%%%%%%%%%%%%%%%%%%%%%%%%%%%%%%%%%%
% Function declaration
\section{Function declaration}
A function declaration defines an identifier and associates it with a
block of code. The block of code will return a result.
The function can then be called inside an expression, or with a procedure
statement, if extended syntax is on.
\input{syntax/function.syn}
\index{Function}The result type of a function can be any previously declared type.
contrary to Turbo pascal, where only simple types could be returned.

%%%%%%%%%%%%%%%%%%%%%%%%%%%%%%%%%%%%%%%%%%%%%%%%%%%%%%%%%%%%%%%%%%%%%%%
% Parameter lists
\section{Parameter lists}
\label{se:Parameters}\index{Parameters}
When arguments must be passed to a function or procedure, these parameters
must be declared in the formal parameter list of that function or procedure.
The parameter list is a declaration of identifiers that can be referred to
only in that procedure or function's block.
\input{syntax/params.syn}
Constant parameters, out parameters and variable parameters can also be \var{untyped}
parameters if they have no type
identifier.\index{Parameters!Untypes}\index{Parameters!Constant}\index{Parameters!Var}

As of version 1.1, \fpc supports default values for both constant parameters
and value parameters, but only for simple types. The compiler must be in
\var{OBJFPC} or \var{DELPHI} mode to accept default values. 

\subsection{Value parameters}
Value parameters are declared as follows:\index{Parameters!Value}
\input{syntax/paramval.syn}
When parameters are declared as value parameters, the procedure gets {\em
a copy} of the parameters that the calling block passes. Any modifications
to these parameters are purely local to the procedure's block, and do not
propagate back to the calling block.
A block that wishes to call a procedure with value parameters must pass
assignment compatible parameters to the procedure. This means that the types
should not match exactly, but can be converted (conversion code is inserted
by the compiler itself)

Care must be taken when using value parameters: Value parameters makes heavy
use of the stack, especially when using large parameters. The total size of
all parameters in the formal parameter list should be below 32K for
portability's sake (the Intel version limits this to 64K).

Open arrays can be passed as value parameters. See \sees{openarray} for
more information on using open arrays.

For a parameter of a  simple type (i.e. not a structured type), a default
value can be specified. This can be an untyped constant. If the function
call omits the parameter, the default value will be passed on to the
function. For dynamic arrays or other types that can be considered as
equivalent to a pointer, the only possible default value is \var{Nil}.

The following example will print 20 on the screen:
\begin{verbatim}
program testp;

Const
  MyConst = 20;

Procedure MyRealFunc(I : Integer = MyConst);

begin
  Writeln('Function received : ',I);
end;  
  
begin
  MyRealFunc;
end.    
\end{verbatim}

\subsection{Variable parameters}
\label{se:varparams}\index{Parameters!Var}
Variable parameters are declared as follows:
\input{syntax/paramvar.syn}
When parameters are declared as variable parameters, the procedure or
function accesses immediatly the variable that the calling block passed in
its parameter list. The procedure gets a pointer to the variable that was
passed, and uses this pointer to access the variable's value.
From this, it follows that any changes made to the parameter, will
propagate back to the calling block. This mechanism can be used to pass
values back in procedures.
Because of this, the calling block must pass a parameter of {\em exactly}
the same type as the declared parameter's type. If it does not, the compiler
will generate an error.

Variable and constant parameters can be untyped. In that case the variable has no type,
and hence is incompatible with all other types. However, the address operator
can be used on it, or it can be can passed to a function that has also an
untyped parameter. If an untyped parameter is used in an assigment,
or a value must be assigned to it, a typecast must be used.

File type variables must always be passed as variable parameters.

Open arrays can be passed as variable parameters. See \sees{openarray} for
more information on using open arrays.

Note that default values are not supported for variable parameters. This
would make little sense since it defeats the purpose of being able to pass a
value back to the caller.

\subsection{Out parameters}
\label{se:outparams}\index{Parameters!Out}
Out parameters  (output parameters) are declared as follows:
\input{syntax/paramout.syn}
The purpose of an \var{out} parameter is to pass values back to the calling
routine: The variable is passed by reference. The initial value of the 
parameter on function entry is discarded, and should not be used.

If a variable must be used to pass a value to a function and retrieve data
from the function, then a variable parameter must be used. If only a value
must be retrieved, a \var{out} parameter can be used.

Needless to say, default values are not supported for \var{out} parameters.

\begin{remark}
Out parameters are only supported in Delphi and ObjFPC mode. For the other 
modes, out is a valid identifier.
\end{remark}
\subsection{Constant parameters}\index{Parameters!Constant}
In addition to variable parameters and value parameters \fpc also supports
Constant parameters. A constant parameter as can be specified as follows:
\input{syntax/paramcon.syn}
A constant argument is passed by reference if it's size is larger than a
pointer. It is passed by value if the size is equal or is less then the 
size of a native pointer.
This means that the function or procedure receives a pointer to the passed
argument, but it cannot be assigned to, this will result in a
compiler error. Furthermore a const parameter cannot be passed on to another
function that requires a variable parameter.
The main use for this is reducing the stack size, hence improving
performance, and still retaining the semantics of passing by value...

Constant parameters can also be untyped. See \sees{varparams} for more
information about untyped parameters.

As for value parameters, constant parameters can get default values.

Open arrays can be passed as constant parameters. See \sees{openarray} for
more information on using open arrays.

\subsection{Open array parameters}
\index{Parameters!Open Array}\index{Array}
\label{se:openarray}
\fpc supports the passing of open arrays, i.e. a procedure can be declared
with an array of unspecified length as a parameter, as in Delphi.
Open array parameters can be accessed in the procedure or function as an
array that is declared with starting index 0, and last element
index \var{High(paremeter)}.
For example, the parameter
\begin{verbatim}
Row : Array of Integer;
\end{verbatim}
would be equivalent to
\begin{verbatim}
Row : Array[0..N-1] of Integer;
\end{verbatim}
Where  \var{N} would be the actual size of the array that is passed to the
function. \var{N-1} can be calculated as \var{High(Row)}.
Open parameters can be passed by value, by reference or as a constant
parameter. In the latter cases the procedure receives a pointer to the
actual array. In the former case, it receives a copy of the array.
In a function or procedure, open arrays can only be passed to functions which
are also declared with open arrays as parameters, {\em not} to functions or
procedures which accept arrays of fixed length.
The following is an example of a function using an open array:
\begin{verbatim}
Function Average (Row : Array of integer) : Real;
Var I : longint;
    Temp : Real;
begin
  Temp := Row[0];
  For I := 1 to High(Row) do
    Temp := Temp + Row[i];
  Average := Temp / (High(Row)+1);
end;
\end{verbatim}

As of FPC 2.2, it is also possible to pass partial arrays to a function that
accepts an open array. This can be done by specifying the range of the array
which should be passed to the open array.

Given the declaration
\begin{verbatim}
Var
  A : Array[1..100];
\end{verbatim}
the following call will compute and print the average of the 100 numbers:
\begin{verbatim}
  Writeln('Average of 100 numbers: ',Average(A));
\end{verbatim}
But the following will compute and print the average of the first and second
half:
\begin{verbatim}
  Writeln('Average of first 50 numbers: ',Average(A[1..50]));
  Writeln('Average of last  50 numbers: ',Average(A[51..100]));
\end{verbatim} 

%%%%%%%%%%%%%%%%%%%%%%%%%%%%%%%%%%%%%%%%%%%%%%%%%%%%%%%%%%%%%%%%%%%%%%%
% The array of const construct
\subsection{Array of const}
\index{Parameters!Open Array}\index{Array}\index{Array!Of const}
In Object Pascal or Delphi mode, \fpc supports the \var{Array of Const}
construction to pass parameters to a subroutine.

This is a special case of the \var{Open array} construction, where it is
allowed to pass any expression in an array to a function or procedure.

In the procedure, passed the arguments can be examined using a special
record:
\begin{verbatim}
Type
  PVarRec = ^TVarRec;
  TVarRec = record
     case VType : Ptrint of
       vtInteger    : (VInteger: Longint);
       vtBoolean    : (VBoolean: Boolean);
       vtChar       : (VChar: Char);
       vtWideChar   : (VWideChar: WideChar);
       vtExtended   : (VExtended: PExtended);
       vtString     : (VString: PShortString);
       vtPointer    : (VPointer: Pointer);
       vtPChar      : (VPChar: PChar);
       vtObject     : (VObject: TObject);
       vtClass      : (VClass: TClass);
       vtPWideChar  : (VPWideChar: PWideChar);
       vtAnsiString : (VAnsiString: Pointer);
       vtCurrency   : (VCurrency: PCurrency);
       vtVariant    : (VVariant: PVariant);
       vtInterface  : (VInterface: Pointer);
       vtWideString : (VWideString: Pointer);
       vtInt64      : (VInt64: PInt64);
       vtQWord      : (VQWord: PQWord);
   end;
\end{verbatim}
Inside the procedure body, the array of const is equivalent to
an open array of TVarRec:
\begin{verbatim}
Procedure Testit (Args: Array of const);

Var I : longint;

begin
  If High(Args)<0 then
    begin
    Writeln ('No aguments');
    exit;
    end;
  Writeln ('Got ',High(Args)+1,' arguments :');
  For i:=0 to High(Args) do
    begin
    write ('Argument ',i,' has type ');
    case Args[i].vtype of
      vtinteger    :
        Writeln ('Integer, Value :',args[i].vinteger);
      vtboolean    :
        Writeln ('Boolean, Value :',args[i].vboolean);
      vtchar       :
        Writeln ('Char, value : ',args[i].vchar);
      vtextended   :
        Writeln ('Extended, value : ',args[i].VExtended^);
      vtString     :
        Writeln ('ShortString, value :',args[i].VString^);
      vtPointer    :
        Writeln ('Pointer, value : ',Longint(Args[i].VPointer));
      vtPChar      :
        Writeln ('PCHar, value : ',Args[i].VPChar);
      vtObject     :
        Writeln ('Object, name : ',Args[i].VObject.Classname);
      vtClass      :
        Writeln ('Class reference, name :',Args[i].VClass.Classname);
      vtAnsiString :
        Writeln ('AnsiString, value :',AnsiString(Args[I].VAnsiStr
    else
        Writeln ('(Unknown) : ',args[i].vtype);
    end;
    end;
end;
\end{verbatim}
In code, it is possible to pass an arbitrary array of elements
to this procedure:
\begin{verbatim}
  S:='Ansistring 1';
  T:='AnsiString 2';
  Testit ([]);
  Testit ([1,2]);
  Testit (['A','B']);
  Testit ([TRUE,FALSE,TRUE]);
  Testit (['String','Another string']);
  Testit ([S,T])  ;
  Testit ([P1,P2]);
  Testit ([@testit,Nil]);
  Testit ([ObjA,ObjB]);
  Testit ([1.234,1.234]);
  TestIt ([AClass]);
\end{verbatim}

If the procedure is declared with the \var{cdecl} modifier, then the
compiler will pass the array as a C compiler would pass it. This, in effect,
emulates the C construct of a variable number of arguments, as the following
example will show:
\begin{verbatim}
program testaocc;
{$mode objfpc}

Const
  P : Pchar = 'example';
  Fmt : PChar =
        'This %s uses printf to print numbers (%d) and strings.'#10;

// Declaration of standard C function printf:
procedure printf (fm : pchar; args : array of const);cdecl; external 'c';

begin
 printf(Fmt,[P,123]);
end.
\end{verbatim}
Remark that this is not true for Delphi, so code relying on this feature
will not be portable.

%%%%%%%%%%%%%%%%%%%%%%%%%%%%%%%%%%%%%%%%%%%%%%%%%%%%%%%%%%%%%%%%%%%%%%%
% Function overloading
\section{Function overloading}
\index{Functions!Overloaded}
Function overloading simply means that the same function is defined more
than once, but each time with a different formal parameter list.
The parameter lists must differ at least in one of it's elements type.
When the compiler encounters a function call, it will look at the function
parameters to decide which one of the defined functions it should call.
This can be useful when the same function must be defined for different
types. For example, in the RTL, the  \var{Dec} procedure could be
 defined as:
\begin{verbatim}
...
Dec(Var I : Longint;decrement : Longint);
Dec(Var I : Longint);
Dec(Var I : Byte;decrement : Longint);
Dec(Var I : Byte);
...
\end{verbatim}
When the compiler encounters a call to the dec function, it will first search
which function it should use. It therefore checks the parameters in a
function call, and looks if there is a function definition which matches the
specified parameter list. If the compiler finds such a function, a call is
inserted to that function. If no such function is found, a compiler error is
generated.
functions that have a \var{cdecl} modifier cannot be overloaded.
(Technically, because this modifier prevents the mangling of
the function name by the compiler).

Prior to version 1.9 of the compiler, the overloaded functions needed to be
in the same unit. Now the compiler will continue searching in other units if
it doesn't find a matching version of an overloaded function in one unit.

The compiler accepts the presence of the \var{overload} modifier as in
Delphi, but it is not required, unless in Delphi mode.\index{overload}

%%%%%%%%%%%%%%%%%%%%%%%%%%%%%%%%%%%%%%%%%%%%%%%%%%%%%%%%%%%%%%%%%%%%%%%
% forward defined functions
\section{Forward defined functions}
\index{Functions!Forward}\index{Forward}
A function can be declared without having it followed by it's implementation,
by having it followed by the \var{forward} procedure. The effective
implementation of that function must follow later in the module.
The function can be used after a \var{forward} declaration as if it had been
implemented already.
The following is an example of a forward declaration.
\begin{verbatim}
Program testforward;
Procedure First (n : longint); forward;
Procedure Second;
begin
  WriteLn ('In second. Calling first...');
  First (1);
end;
Procedure First (n : longint);
begin
  WriteLn ('First received : ',n);
end;
begin
  Second;
end.
\end{verbatim}
A function can be defined as forward only once.
Likewise, in units, it is not allowed to have a forward declared function
of a function that has been declared in the interface part. The interface
declaration counts as a \var{forward} declaration.
The following unit will give an error when compiled:
\begin{verbatim}
Unit testforward;
interface
Procedure First (n : longint);
Procedure Second;
implementation
Procedure First (n : longint); forward;
Procedure Second;
begin
  WriteLn ('In second. Calling first...');
  First (1);
end;
Procedure First (n : longint);
begin
  WriteLn ('First received : ',n);
end;
end.
\end{verbatim}

%%%%%%%%%%%%%%%%%%%%%%%%%%%%%%%%%%%%%%%%%%%%%%%%%%%%%%%%%%%%%%%%%%%%%%%
% External functions
\section{External functions}
\label{se:external}\index{External}\index{Functions!External}
The \var{external} modifier can be used to declare a function that resides in
an external object file. It allows to use the function in some code, and at
linking time, the object file containing the implementation of the function
or procedure must be linked in.
\input{syntax/external.syn}
It replaces, in effect, the function or procedure code block.
As an example:
\begin{verbatim}
program CmodDemo;
{$Linklib c}
Const P : PChar = 'This is fun !';
Function strlen (P : PChar) : Longint; cdecl; external;
begin
  WriteLn ('Length of (',p,') : ',strlen(p))
end.
\end{verbatim}
\begin{remark}
The parameters in our declaration of the \var{external} function
should match exactly the ones in the declaration in the object file.
\end{remark}
If the \var{external} modifier is followed by a string constant:
\begin{verbatim}
external 'lname';
\end{verbatim}
Then this tells the compiler that the function resides in library
'lname'. The compiler will then automatically link this library to
the program.

The name that the function has in the library can also be specified:
\begin{verbatim}
external 'lname' name 'Fname';
\end{verbatim}
\index{name}This tells the compiler that the function resides in library 'lname',
but with name 'Fname'.The compiler will then automatically link this
library to the program, and use the correct name for the function.
Under \windows and \ostwo, the following form can also be used:
\begin{verbatim}
external 'lname' Index Ind;
\end{verbatim}
\index{index}This tells the compiler that the function resides in library 'lname',
but with index \var{Ind}. The compiler will then automatically
link this library to the program, and use the correct index for the
function.

Finally, the external directive can be used to specify the external name
of the function :
\begin{verbatim}
{$L myfunc.o}
external name 'Fname';
\end{verbatim}
\index{external}%
This tells the compiler that the function has the name 'Fname'. The
correct library or object file (in this case myfunc.o) must still be linked.
so that the function 'Fname' is included in the linking stage.

%%%%%%%%%%%%%%%%%%%%%%%%%%%%%%%%%%%%%%%%%%%%%%%%%%%%%%%%%%%%%%%%%%%%%%%
% Assembler functions
\section{Assembler functions}
\index{Assembler}\index{Functions!Assembler}
Functions and procedures can be completely implemented in assembly
language. To indicate this, use the \var{assembler} keyword:
\input{syntax/asm.syn}
Contrary to Delphi, the assembler keyword must be present to indicate an
assembler function.
For more information about assembler functions, see the chapter on using
assembler in the \progref.


%%%%%%%%%%%%%%%%%%%%%%%%%%%%%%%%%%%%%%%%%%%%%%%%%%%%%%%%%%%%%%%%%%%%%%%
% Modifiers
\section{Modifiers}
\index{Modifiers}\index{Functions!Modifiers}
A function or procedure declaration can contain modifiers. Here we list the
various possibilities:
\input{syntax/modifiers.syn}
\fpc doesn't support all Turbo Pascal modifiers, but
does support a number of additional modifiers. They are used mainly for assembler and
reference to C object files.

\subsection{alias}
\label{se:alias}
\index{Alias}\index{Modifiers!Alias}
The \var{alias} modifier allows the programmer to specify a different name for a
procedure or function. This is mostly useful for referring to this procedure
from assembly language constructs or from another object file. As an example,
consider the following program:
\begin{verbatim}
Program Aliases;

Procedure Printit;alias : 'DOIT';
begin
  WriteLn ('In Printit (alias : "DOIT")');
end;
begin
  asm
  call DOIT
  end;
end.
\end{verbatim}
\begin{remark} the specified alias is inserted straight into the assembly
code, thus it is case sensitive.
\end{remark}
The \var{alias} modifier does not make the symbol public to other modules,
unless the routine is also declared in the interface part of a unit, or
the \var{public} modifier is used to force it as public. Consider the
following:
\begin{verbatim}

unit testalias;

interface

procedure testroutine;

implementation

procedure testroutine;alias:'ARoutine';
begin
  WriteLn('Hello world');
end;

end.
\end{verbatim}

This will make the routine \var{testroutine} available publicly to
external object files uunder the label name \var{ARoutine}.

\subsection{cdecl}
\label{se:cdecl}\index{cdecl}\index{Modifiers!cdecl}
The \var{cdecl} modifier can be used to declare a function that uses a C
type calling convention. This must be used when accessing functions residing in
an object file generated by standard C compilers, but must also be used for
pascal functions that are to be used as callbacks for C libraries. 

The \var{cdecl} modifier allows to use C function in the code. 
For external C functions, the object file containing the \var{C} 
implementation of the function or procedure must be linked in.
As an example:
\begin{verbatim}
program CmodDemo;
{$LINKLIB c}
Const P : PChar = 'This is fun !';
Function strlen (P : PChar) : Longint; cdecl; external name 'strlen';
begin
  WriteLn ('Length of (',p,') : ',strlen(p))
end.
\end{verbatim}
When compiling this, and linking to the C-library, the \var{strlen} function
can be called throughout the program. The \var{external} directive tells
the compiler that the function resides in an external object filebrary
with the 'strlen' name (see \ref{se:external}).
\begin{remark}
The parameters in our declaration of the \var{C} function should
match exactly the ones in the declaration in \var{C}.
\end{remark}

For functions that are not external, but which are declared using
\var{cdecl}, no external linking is needed. These functions have some
restrictions, for instance the \var{array of const} construct can not be
used (due the the way this uses the stack). On the other hand, the
\var{cdecl} modifier allows these functions to be used as callbacks for 
routines written in C, as the latter expect the 'cdecl' calling convention.

\subsection{export}
\index{export}\index{Modifiers!export}
The export modifier is used to export names when creating a shared library
or an executable program. This means that the symbol will be publicly
available, and can be imported from other programs. For more information
on this modifier, consult the section on Programming dynamic libraries
in the \progref.


\subsection{inline}
\index{inline}\index{Modifiers!inline}
\label{se:inline}
Procedures that are declared inline are copied to the places where they
are called. This has the effect that there is no actual procedure call,
the code of the procedure is just copied to where the procedure is needed,
this results in faster execution speed if the function or procedure is
used a lot.

By default, \var{inline} procedures are not allowed. Inline code must be enabled
using the command-line switch \var{-Si} or \var{\{\$inline on\}}
directive.

\begin{enumerate}
\item In old versions of \fpc, inline code was {\em not} exported from a unit. This
meant that when calling an inline procedure from another unit, a normal procedure 
call will be performed. Only inside units, \var{Inline} procedures are really inlined.
As of version 2.0.2, inline works accross units.
\item Recursive inline functions are not allowed. i.e. an inline function
that calls itself is not allowed.
\end{enumerate}

\subsection{interrupt}
\label{se:interrupt}\index{interrupt}\index{Mofidiers!interrupt}
The \var{interrupt} keyword is used to declare a routine which will
be used as an interrupt handler. On entry to this routine, all the registers
will be saved and on exit, all registers will be restored
and an interrupt or trap return will be executed (instead of the normal return
from subroutine instruction).

On platforms where a return from interrupt does not exist, the normal exit
code of routines will be done instead. For more information on the generated
code, consult the \progref.

\subsection{local}
\label{se:local}\index{local}\index{Mofidiers!local}
The local directive allows the compiler to optimize the function: a local
function cannot be in the interface section of a unit: it is always in the
implementation section of the unit. From this it follows that the function
cannot be exported from a library. 

On Linux, the local directive results in some optimizations. On Windows, it
has no effect. It was introduced for Kylix compatibility.

\subsection{nostackframe}
\label{se:nostackframe}\index{nostackframe}\index{Modifiers!nostackframe}
The \var{nostackframe} modifier can be used to tell the compiler it should
not generate a stack frame for this procedure or function. By default, a
stack frame is always generated for each procedure or function.

\subsection{pascal}
\label{se:pascal}\index{pascal}\index{Modifiers!pascal}
The \var{pascal} modifier can be used to declare a function that uses the
classic pascal type calling convention (passing parameters from left to right).
For more information on the pascal calling convention, consult the \progref.

\subsection{public}
\index{Modifiers!public}\index{public}
The \var{Public} keyword is used to declare a function globally in a unit.
This is useful if the function should not be accessible from the unit
file (i.e. another unit/program using the unit doesn't see the function),
but must be accessible from the object file. as an example:
\begin{verbatim}
Unit someunit;
interface
Function First : Real;
Implementation
Function First : Real;
begin
  First := 0;
end;
Function Second : Real; [Public];
begin
  Second := 1;
end;
end.
\end{verbatim}
If another program or unit uses this unit, it will not be able to use the
function \var{Second}, since it isn't declared in the interface part.
However, it will be possible to access the function \var{Second} at the
assembly-language level, by using it's mangled name (see the \progref).

\subsection{register}
\label{se:register}\index{register}\index{Modifiers!register}
The \var{register} keyword is used for compatibility with Delphi. In
version 1.0.x of the compiler, this directive has no effect on the
generated code. As of the 1.9.X versions, this directive is supported. The
first three arguments are passed in registers EAX,ECX and EDX.

\subsection{safecall}
\index{safecall}\index{Modifiers!safecall}
This modifier ressembles closely the \var{stdcall} modifier. It sends
parameters from right to left on the stack. The called procedure saves and
restores all registers.

More information about this modifier can be found in the \progref, in the
section on the calling mechanism and the chapter on linking.

\subsection{saveregisters}
\index{saveregisters}\index{Modifiers!saveregisters}
This modifier tells the compiler that all CPU registers should be 
saved prior to calling this routine. Which CPU registers are saved, depends
entirely on the CPU.

\subsection{softfloat}
\index{softfloat}\index{Modifiers!softfloat}
This modifier makes sense only on the ARM architecture.

\subsection{stdcall}
\index{stdcall}\index{Modifiers!stdcall}
This modifier pushes the parameters from right to left on the stack,
it also aligns all the parameters to a default alignment.

More information about this modifier can be found in the \progref, in the
section on the calling mechanism and the chapter on linking.

\subsection{varargs}
\index{varargs}\index{Modifiers!varargs}
This modifier can only be used together with the \var{cdecl} modifier, for
external C procedures. It indicates that the procedure accepts a variable
number of arguments after the last declared variable. These arguments are
passed on without any type checking. It is equivalent to using the
\var{array of const} construction for \var{cdecl} procedures, without having
to declare the \var{array of const}. The square brackets around the variable
arguments do not need to be used when this form of declaration is used.



%%%%%%%%%%%%%%%%%%%%%%%%%%%%%%%%%%%%%%%%%%%%%%%%%%%%%%%%%%%%%%%%%%%%%%%
% Unsupported Turbo Pascal modifiers
\section{Unsupported Turbo Pascal modifiers}
\index{Modifiers}
The modifiers that exist in Turbo pascal, but aren't supported by \fpc, are
listed in \seet{Modifs}.
\begin{FPCltable}{lr}{Unsupported modifiers}{Modifs}
Modifier & Why not supported ? \\ \hline
Near & \fpc is a 32-bit compiler.\\
Far & \fpc is a 32-bit compiler. \\
\end{FPCltable}

%%%%%%%%%%%%%%%%%%%%%%%%%%%%%%%%%%%%%%%%%%%%%%%%%%%%%%%%%%%%%%%%%%%%%%%
% Operator overloading
%%%%%%%%%%%%%%%%%%%%%%%%%%%%%%%%%%%%%%%%%%%%%%%%%%%%%%%%%%%%%%%%%%%%%%%
\chapter{Operator overloading}
\label{ch:operatoroverloading}
\index{overloading!operators}
\section{Introduction}
\fpc supports operator overloading. This means that it is possible to
define the action of some operators on self-defined types, and thus allow
the use of these types in mathematical expressions.

Defining the action of an operator is much like the definition of a
function or procedure, only there are some restrictions on the possible
definitions, as will be shown in the subsequent.

Operator overloading is, in essence, a powerful notational tool;
but it is also not more than that, since the same results can be
obtained with regular function calls. When using operator overloading,
It is important to keep in mind that some implicit rules may produce
some unexpected results. This will be indicated.

\section{Operator declarations}
\index{operators}
To define the action of an operator is much like defining a function:
\input{syntax/operator.syn}
The parameter list for a comparision operator or an arithmetic operator
must always contain 2 parameters. The result type of the comparision
operator must be \var{Boolean}.

\begin{remark}
When compiling in \var{Delphi} mode or \var{Objfpc} mode, the result
identifier may be dropped. The result can then be accessed through
the standard \var{Result} symbol.

If the result identifier is dropped and the compiler is not in one
of these modes, a syntax error will occur.
\end{remark}

The statement block contains the necessary statements to determine the
result of the operation. It can contain arbitrary large pieces of code;
it is executed whenever the operation is encountered in some expression.
The result of the statement block must always be defined; error conditions
are not checked by the compiler, and the code must take care of all possible
cases, throwing a run-time error if some error condition is encountered.

In the following, the three types of operator definitions will be examined.
As an example, throughout this chapter the following type will be used to
define overloaded operators on :
\begin{verbatim}
type
  complex = record
    re : real;
    im : real;
  end;
\end{verbatim}
this type will be used in all examples.

The sources of the Run-Time Library contain a unit \file{ucomplex},
which contains a complete calculus for complex numbers, based on
operator overloading.

\section{Assignment operators}
\index{Operators!Assignment}
The assignment operator defines the action of a assignent of one type of
variable to another. The result type must match the type of the variable
at the left of the assignment statement, the single parameter to the
assignment operator must have the same type as the expression at the
right of the assignment operator.

This system can be used to declare a new type, and define an assignment for
that type. For instance, to be able to assign a newly defined type 'Complex'
\begin{verbatim}
Var
  C,Z : Complex; // New type complex

begin
  Z:=C;  // assignments between complex types.
end;
\end{verbatim}
The following assignment operator would have to be defined:
\begin{verbatim}
Operator := (C : Complex) z : complex;
\end{verbatim}


To be able to assign a real type to a complex type as follows:
\begin{verbatim}
var
  R : real;
  C : complex;

begin
  C:=R;
end;
\end{verbatim}
the following assignment operator must be defined:
\begin{verbatim}
Operator := (r : real) z : complex;
\end{verbatim}
As can be seen from this statement, it defines the action of the operator
\var{:=} with at the right a real expression, and at the left a complex
expression.

an example implementation of this could be as follows:
\begin{verbatim}
operator := (r : real) z : complex;

begin
  z.re:=r;
  z.im:=0.0;
end;
\end{verbatim}
As can be seen in the example, the result identifier (\var{z} in this case)
is used to store the result of the assignment. When compiling in Delphi mode
or objfpc mode, the use of the special identifier \var{Result} is also
allowed, and can be substituted for the \var{z}, so the above would be
equivalent to
\begin{verbatim}
operator := (r : real) z : complex;

begin
  Result.re:=r;
  Result.im:=0.0;
end;
\end{verbatim}

The assignment operator is also used to convert types from one type to
another. The compiler will consider all overloaded assignment operators
till it finds one that matches the types of the left hand and right hand
expressions. If no such operator is found, a 'type mismatch' error
is given.

\begin{remark}
The assignment operator is not commutative; the compiler will never reverse
the role of the two arguments. in other words, given the above definition of
the assignment operator, the following is {\em not} possible:
\begin{verbatim}
var
  R : real;
  C : complex;

begin
  R:=C;
end;
\end{verbatim}
if the reverse assignment should be possible (this is not so for reals and
complex numbers) then the assigment operator must be defined for that as well.
\end{remark}

\begin{remark}
The assignment operator is also used in implicit type conversions. This can
have unwanted effects. Consider the following definitions:
\begin{verbatim}
operator := (r : real) z : complex;
function exp(c : complex) : complex;
\end{verbatim}
then the following assignment will give a type mismatch:
\begin{verbatim}
Var
  r1,r2 : real;

begin
  r1:=exp(r2);
end;
\end{verbatim}
because the compiler will encounter the definition of the \var{exp} function
with the complex argument. It implicitly converts r2 to a complex, so it can
use the above \var{exp} function. The result of this function is a complex,
which cannot be assigned to r1, so the compiler will give a 'type mismatch'
error. The compiler will not look further for another \var{exp} which has
the correct arguments.

It is possible to avoid this particular problem by specifying
\begin{verbatim}
  r1:=system.exp(r2);
\end{verbatim}
An experimental solution for this problem exists in the compiler, but is
not enabled by default. Maybe someday it will be.
\end{remark}

\section{Arithmetic operators}
\index{Operators!Arithmetic}\index{Operators!Binary}
Arithmetic operators define the action of a binary operator. Possible
operations are:
\begin{description}
\item[multiplication] to multiply two types, the \var{*} multiplication
operator must be overloaded.
\item[division] to divide two types, the \var{/} division
operator must be overloaded.
\item[addition] to add two types, the \var{+} addition
operator must be overloaded.
\item[substraction] to substract two types, the \var{-} substraction
operator must be overloaded.
\item[exponentiation] to exponentiate two types, the \var{**} exponentiation
operator must be overloaded.
\end{description}

The definition of an arithmetic operator takes two parameters. The first
parameter must be of the type that occurs at the left of the operator,
the second parameter must be of the type that is at the right of the
arithmetic operator. The result type must match the type that results
after the arithmetic operation.

To compile an expression as
\begin{verbatim}
var
  R : real;
  C,Z : complex;

begin
  C:=R*Z;
end;
\end{verbatim}
one needs a definition of the multiplication operator as:
\begin{verbatim}
Operator * (r : real; z1 : complex) z : complex;

begin
  z.re := z1.re * r;
  z.im := z1.im * r;
end;
\end{verbatim}
As can be seen, the first operator is a real, and the second is
a complex. The result type is complex.

Multiplication and addition of reals and complexes are commutative
operations. The compiler, however, has no notion of this fact so even
if a multiplication between a real and a complex is defined, the
compiler will not use that definition when it encounters a complex
and a real (in that order). It is necessary to define both operations.

So, given the above definition of the multiplication,
the compiler will not accept the following statement:
\begin{verbatim}
var
  R : real;
  C,Z : complex;

begin
  C:=Z*R;
end;
\end{verbatim}
since the types of \var{Z} and \var{R} don't match the types in the
operator definition.

The reason for this behaviour is that it is possible that a multiplication
is not always commutative. e.g. the multiplication of a \var{(n,m)} with a
\var{(m,n)} matrix will result in a \var{(n,n)} matrix, while the
mutiplication of a \var{(m,n)} with a \var{(n,m)} matrix is a \var{(m,m)}
matrix, which needn't be the same in all cases.

\section{Comparision operator}
\index{Operators!Comparison}
The comparision operator can be overloaded to compare two different types
or to compare two equal types that are not basic types. The result type of
a comparision operator is always a boolean.

The comparision operators that can be overloaded are:
\begin{description}
\item[equal to] (=) to determine if two variables are equal.
\item[less than] ($<$) to determine if one variable is less than another.
\item[greater than] ($>$) to determine if one variable is greater than another.
\item[greater than or equal to] ($>=$) to determine if one variable is greater than
or equal to another.
\item[less than or equal to] ($<=$) to determine if one variable is greater
than or equal to another.
\end{description}
There is no separate operator for {\em unequal to} ($<>$). To evaluate a
statement that contans the {\em unequal to} operator, the compiler uses the
{\em equal to} operator (=), and negates the result.


As an example, the following opetrator allows to compare two complex
numbers:
\begin{verbatim}
operator = (z1, z2 : complex) b : boolean;
\end{verbatim}
the above definition allows comparisions of the following form:
\begin{verbatim}
Var
  C1,C2 : Complex;

begin
  If C1=C2 then
    Writeln('C1 and C2 are equal');
end;
\end{verbatim}

The comparision operator definition needs 2 parameters, with the types that
the operator is meant to compare. Here also, the compiler doesn't apply
commutativity; if the two types are different, then it necessary to
define 2 comparision operators.

In the case of complex numbers, it is, for instance necessary to define
2 comparsions: one with the complex type first, and one with the real type
first.

Given the definitions
\begin{verbatim}
operator = (z1 : complex;r : real) b : boolean;
operator = (r : real; z1 : complex) b : boolean;
\end{verbatim}
the following two comparisions are possible:
\begin{verbatim}
Var
  R,S : Real;
  C : Complex;

begin
  If (C=R) or (S=C) then
   Writeln ('Ok');
end;
\end{verbatim}
Note that the order of the real and complex type in the two comparisions
is reversed.

%%%%%%%%%%%%%%%%%%%%%%%%%%%%%%%%%%%%%%%%%%%%%%%%%%%%%%%%%%%%%%%%%%%%%%%
% Programs, Units, Blocks
%%%%%%%%%%%%%%%%%%%%%%%%%%%%%%%%%%%%%%%%%%%%%%%%%%%%%%%%%%%%%%%%%%%%%%%

\chapter{Programs, units, blocks}
A Pascal program consists of modules called \var{units}. A unit can be used
to group pieces of code together, or to give someone code without giving
the sources.
Both programs and units consist of code blocks, which are mixtures of
statements, procedures, and variable or type declarations.

%%%%%%%%%%%%%%%%%%%%%%%%%%%%%%%%%%%%%%%%%%%%%%%%%%%%%%%%%%%%%%%%%%%%%%%
% Programs
\section{Programs}
\index{program}\index{uses}
A pascal program consists of the program header, followed possibly by a
'uses' clause, and a block.
\input{syntax/program.syn}
The program header is provided for backwards compatibility, and is ignored
by the compiler.
The uses clause serves to identify all units that are needed by the program.
The system unit doesn't have to be in this list, since it is always loaded
by the compiler.
The order in which the units appear is significant, it determines in
which order they are initialized. Units are initialized in the same order
as they appear in the uses clause. Identifiers are searched in the opposite
order, i.e. when the compiler searches for an identifier, then it looks
first in the last unit in the uses clause, then the last but one, and so on.
This is important in case two units declare different types with the same
identifier.
When the compiler looks for unit files, it adds the extension \file{.ppu}
to the name of the unit. On \linux and in operating systems where filenames 
are case sensitive  when looking for a unit, the following mechanism is
used:
\begin{enumerate}
\item The unit is first looked for in the original case.
\item The unit is looked for in all-lowercase letters.
\item The unit is looked for in all-uppercase letters.
\end{enumerate}
Additionally, If a unit name is longer than 8 characters, the compiler 
will first look for a unit name with this length, and then it will 
truncate the name to 8 characters and look for it again. 
For compatibility reasons, this is also true on platforms that 
support long file names.

Note that the above search is performed in each directory in the search
path. 

%%%%%%%%%%%%%%%%%%%%%%%%%%%%%%%%%%%%%%%%%%%%%%%%%%%%%%%%%%%%%%%%%%%%%%%
% Units
\section{Units}
\index{unit}
A unit contains a set of declarations, procedures and functions that can be
used by a program or another unit.
The syntax for a unit is as follows:
\input{syntax/unit.syn}
The interface part declares all identifiers that must be exported from the
unit. This can be constant, type or variable identifiers, and also procedure
or function identifier declarations. Declarations inside the
implementation part are {\em not} accessible outside the unit. The
implementation must contain a function declaration for each function or
procedure that is declared in the interface part. If a function is declared
in the interface part, but no declaration of that function is present in the
implementation part, then the compiler will give an error.

When a program uses a unit (say \file{unitA}) and this units uses a second
unit, say \file{unitB}, then the program depends indirectly also on
\var{unitB}. This means that the compiler must have access to \file{unitB} when
trying to compile the program. If the unit is not present at compile time,
an error occurs.

Note that the identifiers from a unit on which a program depends indirectly,
are not accessible to the program. To have access to the identifiers of a
unit, the unit must be in the uses clause of the program or unit where the
identifiers are needed.

Units can be mutually dependent, that is, they can reference each other in
their uses clauses. This is allowed, on the condition that at least one of
the references is in the implementation section of the unit. This also holds
for indirect mutually dependent units.

If it is possible to start from one interface uses clause of a unit, and to return
there via uses clauses of interfaces only, then there is circular unit
dependence, and the compiler will generate an error.
As and example : the following is not allowed:
\begin{verbatim}
Unit UnitA;
interface
Uses UnitB;
implementation
end.

Unit UnitB
interface
Uses UnitA;
implementation
end.
\end{verbatim}
But this is allowed :
\begin{verbatim}
Unit UnitA;
interface
Uses UnitB;
implementation
end.
Unit UnitB
implementation
Uses UnitA;
end.
\end{verbatim}
Because \file{UnitB} uses \file{UnitA} only in it's implentation section.
In general, it is a bad idea to have circular unit dependencies, even if it is
only in implementation sections.

%%%%%%%%%%%%%%%%%%%%%%%%%%%%%%%%%%%%%%%%%%%%%%%%%%%%%%%%%%%%%%%%%%%%%%%
% Blocks
\section{Blocks}
\index{block}
Units and programs are made of blocks. A block is made of declarations of
labels, constants, types variables and functions or procedures. Blocks can
be nested in certain ways, i.e., a procedure or function declaration can
have blocks in themselves.
A block looks like the following:
\input{syntax/block.syn}
Labels that can be used to identify statements in a block are declared in
the label declaration part of that block. Each label can only identify one
statement.
Constants that are to be used only in one block should be declared in that
block's constant declaration part.
Variables that are to be used only in one block should be declared in that
block's constant declaration part.
Types that are to be used only in one block should be declared in that
block's constant declaration part.
Lastly, functions and procedures that will be used in that block can be
declared in the procedure/function declaration part.
After the different declaration parts comes the statement part. This
contains any actions that the block should execute.
All identifiers declared before the statement part can be used in that
statement part.

%%%%%%%%%%%%%%%%%%%%%%%%%%%%%%%%%%%%%%%%%%%%%%%%%%%%%%%%%%%%%%%%%%%%%%%
% Scope
\section{Scope}
\index{Scope}
Identifiers are valid from the point of their declaration until the end of
the block in which the declaration occurred. The range where the identifier
is known is the {\em scope} of the identifier. The exact scope of an
identifier depends on the way it was defined.
\subsection{Block scope}
\index{Scope!block}
The {\em scope} of a variable declared in the declaration part of a block,
is valid from the point of declaration until the end of the block.
If a block contains a second block, in which the identfier is
redeclared, then inside this block, the second declaration will be valid.
Upon leaving the inner block, the first declaration is valid again.
Consider the following example:
\begin{verbatim}
Program Demo;
Var X : Real;
{ X is real variable }
Procedure NewDeclaration
Var X : Integer;  { Redeclare X as integer}
begin
 // X := 1.234; {would give an error when trying to compile}
 X := 10; { Correct assigment}
end;
{ From here on, X is Real again}
begin
 X := 2.468;
end.
\end{verbatim}
In this example, inside the procedure, X denotes an integer variable.
It has it's own storage space, independent of the variable \var{X} outside
the procedure.

\subsection{Record scope}
\index{Scope!record}
The field identifiers inside a record definition are valid in the following
places:
\begin{enumerate}
\item to the end of the record definition.
\item field designators of a variable of the given record type.
\item identifiers inside a \var{With} statement that operates on a variable
of the given record type.
\end{enumerate}

\subsection{Class scope}
\index{Scope!Class}
A component identifier (one of the items in the class' component list) 
is valid in the following places:
\begin{enumerate}
\item From the point of declaration to the end of the class definition.
\item In all descendent types of this class, unless it is in the private
part of the class declaration.
\item In all method declaration blocks of this class and descendent classes.
\item In a with statement that operators on a variable of the given class's
definition.
\end{enumerate}
Note that method designators are also considered identifiers.
\subsection{Unit scope}
\index{Scope!unit}\index{unit}
All identifiers in the interface part of a unit are valid from the point of
declaration, until the end of the unit. Furthermore, the identifiers are
known in programs or units that have the unit in their uses clause.
Identifiers from indirectly dependent units are {\em not} available.
Identifiers declared in the implementation part of a unit are valid from the
point of declaration to the end of the unit.
The system unit is automatically used in all units and programs.
It's identifiers are therefore always known, in each pascal program, library
or unit.
The rules of unit scope imply that an identifier of a
unit can be redefined. To have access to an identifier of another unit that was redeclared in
the current unit, precede it with that other units name, as in the following
example:
\begin{verbatim}
unit unitA;
interface
Type
  MyType = Real;
implementation
end.
Program prog;
Uses UnitA;

{ Redeclaration of MyType}
Type MyType = Integer;
Var A : Mytype;      { Will be Integer }
    B : UnitA.MyType { Will be real }
begin
end.
\end{verbatim}
This is especially useful when redeclaring the system unit's identifiers.

%%%%%%%%%%%%%%%%%%%%%%%%%%%%%%%%%%%%%%%%%%%%%%%%%%%%%%%%%%%%%%%%%%%%%%%
% Libraries
\section{Libraries}
\index{Libraries}\index{library}
\fpc supports making of dynamic libraries (DLLs under Win32 and \ostwo) trough
the use of the \var{Library} keyword.

A Library is just like a unit or a program:
\input{syntax/library.syn}

By default, functions and procedures that are declared and implemented in
library are not available to a programmer that wishes to use this library.

In order to make functions or procedures available from the library,
they must be exported in an export clause:

\input{syntax/exports.syn}

Under Win32, an index clause can be added to an exports entry.
an index entry must be a positive number larger or equal than 1.

Optionally, an exports entry can have a name specifier. If present, the name
specifier gives the exact name (case sensitive) of the function in the
library.

If neither of these constructs is present, the functions or procedures
are exported with the exact names as specified in the exports clause.



%%%%%%%%%%%%%%%%%%%%%%%%%%%%%%%%%%%%%%%%%%%%%%%%%%%%%%%%%%%%%%%%%%%%%%%
% Exceptions

%%%%%%%%%%%%%%%%%%%%%%%%%%%%%%%%%%%%%%%%%%%%%%%%%%%%%%%%%%%%%%%%%%%%%%%
\chapter{Exceptions}
\label{ch:Exceptions}\index{Exceptions}\index{Exception}
Exceptions provide a convenient way to program error and error-recovery
mechanisms, and are closely related to classes.
Exception support is based on 3 constructs:
\begin{description}
\item [Raise\ ] statements. To raise an exeption. This is usually done to signal an
error condition.\index{Exceptions!Raising}
\item [Try ... Except\ ] blocks. These block serve to catch exceptions
raised within the scope of the block, and to provide exception-recovery
code.\index{Exceptions!Catching}
\item [Try ... Finally\ ] blocks. These block serve to force code to be
executed irrespective of an exception occurrence or not. They generally
serve to clean up memory or close files in case an exception occurs.
The compiler generates many implicit \var{Try ... Finally} blocks around
procedure, to force memory consistence.
\end{description}


%%%%%%%%%%%%%%%%%%%%%%%%%%%%%%%%%%%%%%%%%%%%%%%%%%%%%%%%%%%%%%%%%%%%%%%
% The raise statement
\section{The raise statement}
\index{Raise}\index{Exceptions!Raising}
The \var{raise} statement is as follows:
\input{syntax/raise.syn}
This statement will raise an exception. If it is specified, the exception
instance must be an initialized instance of any class, which is the raise
type. The exception address is optional. If it is not specified, the compiler
will provide the address by itself. If the exception instance is omitted, 
then the current exception is re-raised. This construct can only be used 
in an exception handling block (see further).

\begin{remark} Control {\em never} returns after an exception block. The
control is transferred to the first \var{try...finally} or
\var{try...except} statement that is encountered when unwinding the stack.
If no such statement is found, the \fpc Run-Time Library will generate a
run-time error 217 (see also \sees{exceptclasses}). The exception address
will be printed by the default exception handling routines.
\end{remark}

As an example: The following division checks whether the denominator is
zero, and if so, raises an exception of type \var{EDivException}
\begin{verbatim}
Type EDivException = Class(Exception);
Function DoDiv (X,Y : Longint) : Integer;
begin
  If Y=0 then
    Raise EDivException.Create ('Division by Zero would occur');
  Result := X Div Y;
end;
\end{verbatim}
The class \var{Exception} is defined in the \file{Sysutils} unit of the rtl.
(\sees{exceptclasses})

\begin{remark}
Although the \var{Exception} class is used as the base class for exceptions
throughout the code, this is just an unwritten agreement: the class can
be of any type, and need not be a descendent of the \var{Exception} class.

Of course, most code depends on the unwritten agreement that an exception
class descends from \var{Exception}.
\end{remark}

%%%%%%%%%%%%%%%%%%%%%%%%%%%%%%%%%%%%%%%%%%%%%%%%%%%%%%%%%%%%%%%%%%%%%%%
% The try...except statement
\section{The try...except statement}
\index{except}\index{Exceptions!Catching}
A \var{try...except} exception handling block is of the following form :
\input{syntax/try.syn}
If no exception is raised during the execution of the \var{statement list},
then all statements in the list will be executed sequentially, and the
except block will be skipped, transferring program flow to the statement
after the final \var{end}.

If an exception occurs during the execution of the \var{statement list}, the
program flow will be transferred to the except block. Statements in the
statement list between the place where the exception was raised and the
exception block are ignored.

In the exception handling block, the type of the exception is checked,
and if there is an exception handler where the class type matches the
exception object type, or is a parent type of
the exception object type, then the statement following the corresponding
\var{Do} will be executed. The first matching type is used. After the
\var{Do} block was executed, the program continues after the \var{End}
statement.

The identifier in an exception handling statement is optional, and declares
an exception object. It can be used to manipulate the exception object in
the exception handling code. The scope of this declaration is the statement
block foillowing the \var{Do} keyword.

If none of the \var{On} handlers matches the exception object type, then the
statement list after \var{else} is executed. If no such list is
found, then the exception is automatically re-raised. This process allows
to nest \var{try...except} blocks.

If, on the other hand, the exception was caught, then the exception object is
destroyed at the end of the exception handling block, before program flow
continues. The exception is destroyed through a call to the object's
\var{Destroy} destructor.

As an example, given the previous declaration of the \var{DoDiv} function,
consider the following
\begin{verbatim}
Try
  Z := DoDiv (X,Y);
Except
  On EDivException do Z := 0;
end;
\end{verbatim}
If \var{Y} happens to be zero, then the DoDiv function code will raise an
exception. When this happens, program flow is transferred to the except
statement, where the Exception handler will set the value of \var{Z} to
zero. If no exception is raised, then program flow continues past the last
\var{end} statement.
To allow error recovery, the \var{Try ... Finally} block is supported.
A \var{Try...Finally} block ensures that the statements following the
\var{Finally} keyword are guaranteed to be executed, even if an exception
occurs.


%%%%%%%%%%%%%%%%%%%%%%%%%%%%%%%%%%%%%%%%%%%%%%%%%%%%%%%%%%%%%%%%%%%%%%%
% The try...finally statement
\section{The try...finally statement}
\index{finally}\index{try}\index{Exceptions!Handling}
A \var{Try..Finally} statement has the following form:
\input{syntax/finally.syn}
If no exception occurs inside the \var{statement List}, then the program
runs as if the \var{Try}, \var{Finally} and \var{End} keywords were not
present.

If, however, an exception occurs, the program flow is immediatly
transferred from the point where the excepion was raised to the first
statement of the \var{Finally statements}.

All statements after the finally keyword will be executed, and then
the exception will be automatically re-raised. Any statements between the
place where the exception was raised and the first statement of the
\var{Finally Statements} are skipped.

As an example consider the following routine:
\begin{verbatim}
Procedure Doit (Name : string);
Var F : Text;
begin
  Try
    Assign (F,Name);
    Rewrite (name);
    ... File handling ...
  Finally
    Close(F);
  end;
\end{verbatim}
If during the execution of the file handling an execption occurs, then
program flow will continue at the \var{close(F)} statement, skipping any
file operations that might follow between the place where the exception
was raised, and the \var{Close} statement.
If no exception occurred, all file operations will be executed, and the file
will be closed at the end.


%%%%%%%%%%%%%%%%%%%%%%%%%%%%%%%%%%%%%%%%%%%%%%%%%%%%%%%%%%%%%%%%%%%%%%%
% Exception handling nesting
\section{Exception handling nesting}
\index{finally}\index{except}\index{try}\index{Exceptions!Handling}
It is possible to nest \var{Try...Except} blocks with \var{Try...Finally}
blocks. Program flow will be done according to a \var{lifo} (last in, first
out) principle: The code of the last encountered \var{Try...Except} or
 \var{Try...Finally} block will be executed first. If the exception is not
caught, or it was a finally statement, program flow will be transferred to
the last-but-one block, {\em ad infinitum}.

If an exception occurs, and there is no exception handler present, then a
runerror 217 will be generated. When using the \file{sysutils} unit, a default
handler is installed which will show the exception object message, and the
address where the exception occurred, after which the program will exit with
a \var{Halt} instruction.


%%%%%%%%%%%%%%%%%%%%%%%%%%%%%%%%%%%%%%%%%%%%%%%%%%%%%%%%%%%%%%%%%%%%%%%
% Exception classes
\section{Exception classes}
\index{Exceptions!Classes}
\label{se:exceptclasses}
The \file{sysutils} unit contains a great deal of exception handling.
It defines the following exception types:
\begin{verbatim}
       Exception = class(TObject)
        private
          fmessage : string;
          fhelpcontext : longint;
        public
          constructor create(const msg : string);
          constructor createres(indent : longint);
          property helpcontext : longint read fhelpcontext write fhelpcontext;
          property message : string read fmessage write fmessage;
       end;
       ExceptClass = Class of Exception;
       { mathematical exceptions }
       EIntError = class(Exception);
       EDivByZero = class(EIntError);
       ERangeError = class(EIntError);
       EIntOverflow = class(EIntError);
       EMathError = class(Exception);
\end{verbatim}
The sysutils unit also installs an exception handler. If an exception is
unhandled by any exception handling block, this handler is called by the
Run-Time library. Basically, it prints the exception address, and it prints
the message of the Exception object, and exits with a exit code of 217.
If the exception object is not a descendent object of the \var{Exception}
object, then the class name is printed instead of the exception message.

It is recommended to use the \var{Exception} object or a descendant class for
all \var{raise} statements, since then the message field of the
exception object can be used.

%%%%%%%%%%%%%%%%%%%%%%%%%%%%%%%%%%%%%%%%%%%%%%%%%%%%%%%%%%%%%%%%%%%%%%%
% Using Assembler
%%%%%%%%%%%%%%%%%%%%%%%%%%%%%%%%%%%%%%%%%%%%%%%%%%%%%%%%%%%%%%%%%%%%%%%
\chapter{Using assembler}
\index{Assembler}
\fpc supports the use of assembler in code, but not inline
assembler macros.  To have more information on the processor
specific assembler syntax and its limitations, see the \progref.

%%%%%%%%%%%%%%%%%%%%%%%%%%%%%%%%%%%%%%%%%%%%%%%%%%%%%%%%%%%%%%%%%%%%%%%
% Assembler statements
\section{Assembler statements }
\index{Statements!Assembler}
The following is an example of assembler inclusion in pascal code.
\begin{verbatim}
 ...
 Statements;
 ...
 Asm
   the asm code here
   ...
 end;
 ...
 Statements;
\end{verbatim}
The assembler instructions between the \var{Asm} and \var{end} keywords will
be inserted in the assembler generated by the compiler.
Conditionals can be used ib assembler, the compiler will recognise it,
and treat it as any other conditionals.

%%%%%%%%%%%%%%%%%%%%%%%%%%%%%%%%%%%%%%%%%%%%%%%%%%%%%%%%%%%%%%%%%%%%%%%
% Assembler procedures and functions
\section{Assembler procedures and functions}
\index{Functions!Assembler}
Assembler procedures and functions are declared using the
\var{Assembler} directive.  This permits the code generator to make a number
of code generation optimizations.

The code generator does not generate any stack frame (entry and exit
code for the routine) if it contains no local variables and no
parameters. In the case of functions, ordinal values must be returned
in the accumulator. In the case of floating point values, these depend
on the target processor and emulation options.


%
% The index.
%
\printindex
\end{document}
